\documentclass[colortheme=yellow,txconfig=txmaths.cfg]{textbook}
\stylesetup{ 
  fullwidth-stop = catcode,
  boldemph = false,
}
\Booksetup{
  BookSeries  = 中学经典教材丛书, 
  BookTitle   = 微积分初步(甲种本),
  BookTitle*  = {Textbook for Middle School Calculus},
  SubTitle    = 全一册,
  % SubTitle*   = Volume I,
  BriefIntro    = 
    { 
      本书供六年制中学高中三年级选用。本书内容包括极限;导数和微分;导数的应用;不定积分;定积分及其应用。学完这些内容,约需 84 课时。本书习题共分三类:练习、习题、复习参考题。练习主要供课堂练习用;习题主要供课内外作业用;复习参考题分 A、B 两组。A 组供复习本章知识时使用;B 组题略带综合性、灵活性,仅供学有余力的学生参考使用。练习、习题及复习参考题 A 组题的题量较多,教学时可根据情况选用。本书在编写过程中,曾参考了中小学通用教材数学编写组编写的全日制十年制学校高中课本(试用本)《数学》第四册的有关章节,大部分内容是以原来章节为基础编写的。本书由人民教育出版社数学室编写。参加编写的有方明一、刘远图、曾宪源、于琛等,全书由于琛校订。
    },
  DedicatedTo   = 奔赴高考的莘莘学子,
  % CoverGraph    = graphics/A.pdf,
  AuthorList    = {人民教育出版社数学室},
  ReleaseDate   = 2024-12-12,
  % Url           = https://www.tjad.cn,
  % ISBN          = 978-7-302-11622-6,
  % Publisher     = 同济极客出版社,
  % Logo          = graphics/logo.pdf,
  % Editor        = {张晨南},
  WrittenStyle  = 编,
}
\graphicspath{{figures/CAL/}}
\begin{document}
\frontmatter
\tableofcontents
\mainmatter
\chapter{极限}
\phantomsection\pdfbookmark[1]{极限}{phantomsection1}
\subsection{数列的极限}
我们来考察下面两个数列:
\begin{gather}
  \label{eq:sequence1} 1,\frac{1}{2},\frac{1}{3},\cdots,\frac{1}{n},\cdots\\
  \label{eq:sequence2} \frac{1}{2},\frac{3}{4},\frac{7}{8},\cdots,1-\frac{1}{2^n},\cdots
\end{gather}

为了直观起见,我们把这两个数列中的前几项分别在数轴上表示出来(\cref{fig:1-1}):
\begin{figure}
  \begin{minipage}{\linewidth}\centering
    \subcaption{}\label{fig:1-1a}
  \end{minipage}
  \begin{minipage}{\linewidth}\centering
    \subcaption{}\label{fig:1-1b}
  \end{minipage}
  \caption{}\label{fig:1-1}
\end{figure}

容易看出,当项数 $n$ 无限增大时,数列 \eqref{eq:sequence1} 中的项无限趋近于 0,数列 \eqref{eq:sequence2} 中的项无限趋近于 1。

事实上,在数列 \eqref{eq:sequence1} 中,各项与 0 的差的绝对值如\cref{tab:1-1} 所示。
\begin{table}
  \caption{数列 \eqref{eq:sequence1} 各项与 0 的差的绝对值}\label{tab:1-1}
  \begin{tblr}{colspec={X[c]X[c]X[5,c]},hline{2}={0.8pt},rowsep=2pt}
    项号 & 项 & 这一项与 0 的差的绝对值\\
    1 &  $1$  & $|0-1|=1$ \\
    2 &  $\dfrac{1}{2}$  & $\left|0-\dfrac{1}{2}\right|=\dfrac{1}{2}$ \\
    3 &  $\dfrac{1}{3}$  & $\left|0-\dfrac{1}{3}\right|=\dfrac{1}{3}$ \\
    4 &  $\dfrac{1}{4}$  & $\left|0-\dfrac{1}{4}\right|=\dfrac{1}{4}$ \\
    5 &  $\dfrac{1}{5}$  & $\left|0-\dfrac{1}{5}\right|=\dfrac{1}{5}$ \\
    6 &  $\dfrac{1}{6}$  & $\left|0-\dfrac{1}{6}\right|=\dfrac{1}{6}$ \\
    7 &  $\dfrac{1}{7}$  & $\left|0-\dfrac{1}{7}\right|=\dfrac{1}{7}$ \\
    $\cdots$ &  $\cdots$  & $\cdots$ \\
  \end{tblr}
\end{table}

我们看到,无论预先指定多么小的一个正数 $\varepsilon$,总能在数列 \eqref{eq:sequence1} 中找到这样一项,使得这一项后面的所有项与 0 的差的绝对值都小于 $\varepsilon$。
例如,如果取 $\varepsilon=\dfrac{1}{5}$,那么数列 \eqref{eq:sequence1} 中第 5 项后面所有的项与 0 的差的绝对值都小于 $\varepsilon$。
如果取 $\varepsilon = \dfrac{1}{100}$ ,那么数列 \eqref{eq:sequence1} 中第 100 项后面所有的项与 0 的差的绝对值都小于 $\varepsilon$。
在这种情况下,我们就说数列 \eqref{eq:sequence1} 的极限是 0。

同样,对于数列 \eqref{eq:sequence2},我们也可以列成\cref{tab:1-2}。
\begin{table}
  \caption{数列 \eqref{eq:sequence2} 各项与 0 的差的绝对值}\label{tab:1-2}
  \begin{tblr}{colspec={X[c]X[c]X[5,c]},hline{2}={0.8pt},rowsep=2pt}
    项号 & 项 & 这一项与 1 的差的绝对值\\
    1 &  $\dfrac{1}{2}    $ & $\left|\dfrac{1}{2}    -1\right|=\dfrac{1}{2}=0.5$ \\
    2 &  $\dfrac{3}{4}    $ & $\left|\dfrac{3}{4}    -1\right|=\dfrac{1}{4}=0.25$ \\
    3 &  $\dfrac{7}{8}    $ & $\left|\dfrac{7}{8}    -1\right|=\dfrac{1}{8}=0.125$ \\
    4 &  $\dfrac{15}{16}  $ & $\left|\dfrac{15}{16}  -1\right|=\dfrac{1}{16}=0.0625$ \\
    5 &  $\dfrac{31}{32}  $ & $\left|\dfrac{31}{32}  -1\right|=\dfrac{1}{32}=0.03125$ \\
    6 &  $\dfrac{63}{64}  $ & $\left|\dfrac{63}{64}  -1\right|=\dfrac{1}{64}=0.015625$ \\
    7 &  $\dfrac{127}{128}$ & $\left|\dfrac{127}{128}-1\right|=\dfrac{1}{128}=0.0078125$ \\
    $\cdots$ &  $\cdots$  & $\cdots$ \\
  \end{tblr}
\end{table}

可以看出,如果取 $\varepsilon =0.1$,那么数列 \eqref{eq:sequence2} 中第 3 项后面所有的项与 1 的差的绝对值都小于 $\varepsilon$;如果取 $\varepsilon =0.01$,那么第 6 项后面所有的项与 1 的差的绝对值都小于 $\varepsilon$。
就是说,无论预先指定多么小的一个正数 $\varepsilon$,总能在数列 \eqref{eq:sequence2} 中找到这样一项,使得这一项后面的所有项与 1 的差的绝对值都小于 $\varepsilon$。
这时,我们说数列 \eqref{eq:sequence2} 的极限是 1。

一般地,对于一个无穷数列 $\{a_n\}$,如果存在一个常数 $A$,无论预先指定多么小的正数 $\varepsilon$,都能在数列中找到一项 $a_N$,使得这一项后面所有的项与 $A$ 的差的绝对值都小于 $\varepsilon$(即当 $n>N$ 时,$|a_n-A|<\varepsilon$ 恒成立),就把常数 $A$ 叫做\Concept{数列 $\{a_n\}$ 的极限},记作
\[ \lim_{n\to\infty}a_n=A.\footnotemark[1] \]
\footnotetext[1]{$\lim$ 是拉丁文 limis(极限)一词的前三个字母,一般按英文 limit (极限)一词读音。$\lim\limits_{n\to\infty }a_n=A$ 也可读作 “limit $a_n$ 当 $n$ 趋于无穷大时等于 $A$”。}

这个式子读作 “当 $n$ 趋向于无穷大时,$a_n$ 的极限等于 $A$”。
“$\to$” 表示 “趋向于”,“$\infty$” 表示 “无穷大”,“$n\to\infty$” 表示 “$n$ 趋向于无穷大”,也就是 $n$ 无限增大的意思。

$\lim\limits_{n\to\infty}a_n=A$ 有时也可记作
\[ \text{当}\ n\to\infty\text{ 时,} a_n\to A.\]

从数列极限的定义可以看出,数列 $\{a_n\}$ 以 $A$ 为极限,是指当 $n$ 无限增大时,数列 $\{a_n\}$ 中的项 $a_n$ 无限趋近于常数 $A$。
\begin{example}
  已知数列
  \[1,-\frac{1}{2},\frac{1}{3},-\frac{1}{4},\cdots,(-1)^{n+1}\frac{1}{n},\cdots\]
  \begin{enumerate}
    \item 写出这个数列的各项与 0 的差的绝对值。
    \item 第几项后面所有的项与 0 的差的绝对值都小于 0.1 ?都小于 0.001 ?都小于 0.0003?
    \item 第几项后面所有的项与 0 的差的绝对值都小于任何预先指定的正数 $\varepsilon$?
    \item 0 是不是这个数列的极限?
  \end{enumerate}
\end{example}
\begin{solution}
  这个数列的项在数轴上的表示如\cref{fig:1-2}:
  \begin{figure}
    \caption{}\label{fig:1-2}
  \end{figure}
  \begin{enumerate}
    \item 这个数列的各项与 0 的差的绝对值依次是
    \[ 1,\frac{1}{2},\frac{1}{3},\cdots,\frac{1}{n},\cdots\]
    \item 要使 $\frac{1}{n}<0.1$,只要 $n>10$ 就行了。这就是说,第 10 项后面所有的项与 0 的差的绝对值都小于 0.1。
    
    要使 $\frac{1}{n}<0.001$,只要 $n>1000$ 就行了。这就是说,第 1000 项后面所有的项与 0 的差的绝对值都小于 0.001。

    要使 $\frac{1}{n}<0.0003$,只要 $n>3333\frac{1}{3}$ 就行了。这就是说,第 3333 项后面所有的项与 0 的差的绝对值都小于 0.0003。
    \item 
    \item 
  \end{enumerate}
\end{solution}
\begin{Practice}
  \begin{question}
    \item 
    \item 
  \end{question}
\end{Practice}
\subsection{数列极限的四则运算}
\begin{Practice}
  \begin{question}
    \item 
    \item 
    \item 
  \end{question}
\end{Practice}

\begin{Exercise}
  \begin{question}
    \item 
    \item 
    \item 
    \item 
    \item 
    \item 
    \item 
    \item 
    \item 
    \item 
    \item 
    \item 
    \item 
    \item 
    \item 
  \end{question}
\end{Exercise}

\subsection{函数的极限}

\begin{Practice}
  \begin{question}
    \item 
    \item 
    \item 
    \item 
    \item 
  \end{question}
\end{Practice}
\subsection{函数极限的四则运算法则}
\begin{Practice}
  \begin{question}
    \item 
    \item 
  \end{question}
\end{Practice}

\begin{Exercise}
  \begin{question}
    \item 
    \item 
    \item 
    \item 
  \end{question}
\end{Exercise}

\subsection{函数的连续性}
\begin{Practice}
  \begin{question}
    \item 
    \item 
  \end{question}
\end{Practice}


\begin{Practice}
  \begin{question}
    \item 
    \item 
    \item 
    \item 
  \end{question}
\end{Practice}

\subsection{两个重要的极限}
\begin{Practice}
  \begin{question}
    \item ;
    \item 。
  \end{question}
\end{Practice}

\begin{Exercise}
  \begin{question}
    \item 
    \item 
    \item 
    \item 
    \item 
    \item 
    \item 
  \end{question}
\end{Exercise}

\section*{小结}
\begin{enumerate}[C、,itemindent=4.5em]
  \item 
  \item 
  \item 
  \item 
\end{enumerate}
\chapter*{复习参考题\chinese{chapter}}
\section*{A 组}
\begin{question}
  \item 
  \item 
  \item 
  \item 
  \item 
  \item 
  \item 
  \item 
  \item 
  \item 
  \item 
  \item 
\end{question}
\section*{B 组}
\begin{question}
  \item 
  \item 
  \item 
  \item 
  \item 
  \item 
  \item 
  \item 
  \item 
\end{question}
\chapter{导数和微分}
\section{导数概念}
\subsection{瞬时速度}
\begin{Practice}
  \begin{question}
    \item 
    \item 
  \end{question}
\end{Practice}
\subsection{导数}
\begin{Practice}
  \begin{question}
    \item 
    \item 
    \item 
  \end{question}
\end{Practice}
\subsection{导数的几何意义\texorpdfstring{\quad}{ }切线方程和法线方程}
\begin{Practice}
  \begin{question}
    \item 
    \item 
    \item 
  \end{question}
\end{Practice}
\subsection*{变化率举例}
\begin{Practice}
  \begin{question}
    \item 
    \item 
    \item 
  \end{question}
\end{Practice}
\subsection{函数的可导性与连续性的关系}
\begin{Practice}

\end{Practice}

\begin{Exercise}
  \begin{question}
    \item 
    \item 
    \item 
    \item 
    \item 
    \item 
    \item 
    \item 
    \item 
    \item 
    \item 
  \end{question}
\end{Exercise}

\section{求导方法}
\subsection{几种常见函数的导数}

\begin{Practice}
  (口答)求下列函数的导数:
  \begin{tasks}(2)
    \task $y=x^5$;
    \task $y=x^6$;
    \task $x=\sin t$;
    \task $u=\cos \varphi$。
  \end{tasks}
\end{Practice}

\subsection{函数的和、差、积、商的导数}
\begin{Practice}
  \begin{question}
    \item 
    \item 
    \item 
    \item 
    \item 
    \item 
  \end{question}
\end{Practice}

\subsection{复合函数的导数}
\begin{Practice}
  \begin{question}
    \item 
    \item 
    \item 
    \item 
  \end{question}
\end{Practice}

\begin{Exercise}
  \begin{question}
    \item 
    \item 
    \item 
    \item 
    \item 
    \item 
    \item 
    \item 
    \item 
    \item 
    \item 
    \item 
    \item 
    \item 
    \item 
    \item 
  \end{question}
\end{Exercise}

\subsection{三角函数的导数}
\begin{Practice}
  求下列函数的导数:
  \begin{tasks}(2)
    \task $f(\theta)=\dfrac{1+\cos\theta}{1-\cos\theta}$;
    \task $y=\cos x^2-\sin\sqrt{x}$;
    \task $f(\theta)=\tan\theta-\theta$;
    \task $y=\tan\frac{x}{2}-\cot\frac{x}{2}$。
  \end{tasks}
\end{Practice}

\subsection{反三角函数的导数}
\begin{Practice}
  求下列函数的导数:
  \begin{tasks}(2)
    \task $y=\arcsin\frac{x}{a}$;
    \task $y=x\arcsin x$;
    \task $y=2\arcsin x^2$;
    \task $y=\arccos\frac{x}{2}$;
    \task $y=\arctan\frac{x}{a}$;
    \task $y=(\arccot x)^2$。
  \end{tasks}
\end{Practice}

\subsection{对数函数的导数}
\begin{Practice}
  求下列函数的导数:
  \begin{tasks}(2)
    \task $y=x\ln x$;
    \task $y=\ln\dfrac{1+3x^2}{2-x^2}$;
    \task $y=\log_a(2x^3+3x^2)$;
    \task $y=\ln\sqrt{\dfrac{1+x}{1-x}}$;
    \task $y=\lg(1+\cos x)$;
    \task $y=\ln(\ln x)$。
  \end{tasks}
\end{Practice}

\subsection{指数函数的导数}
\begin{Practice}
  求下列函数的导数:
  \begin{tasks}(2)
    \task $y=e^x\sin x$;
    \task $y=\dfrac{e^x-1}{e^x+1}$;
    \task $y=x^ne^{-x}$;
    \task $y=\frac{a}{2}(e^{\frac{x}{a}}-e^{-\frac{x}{a}})$;
    \task $y=x^3+3^x$;
    \task $y=2^xe^x$;
    \task $y=e^{2x}\ln x$;
    \task $y=e^{x^2+1}$。
  \end{tasks}
\end{Practice}

\subsection{幂函数的导数}
\begin{Practice}
  求下列函数的导数:
  \begin{tasks}(2)
    \task $y=x^{\frac{2}{3}}-2x^{-\frac{1}{2}}+5x^{\frac{7}{6}}$;
    \task $y=\left(\dfrac{1}{\sqrt[3]{x^2}}+\dfrac{1}{\sqrt{x}}\right)^2$;
    \task $y=\sqrt[3]{(4-3x^2)^2}$;
    \task $y=\sqrt[3]{\dfrac{x-a}{x+a}}$。
  \end{tasks}
\end{Practice}

\begin{Exercise}
  \begin{question}
    \item 
    \item 
    \item 
    \item 
    \item 
    \item 
    \item 
    \item 
    \item 
  \end{question}
\end{Exercise}

\subsection{隐函数的导数}
\begin{Practice}
  \begin{question}
    \item 
    \item 
    \item 
    \item 
  \end{question}
\end{Practice}

\subsection{二阶导数}
\begin{Practice}
  \begin{question}
    \item 
    \item 
  \end{question}
\end{Practice}

\begin{Exercise}
  \begin{question}
    \item 
    \item 
    \item 
    \item 
    \item 
    \item 
    \item 
    \item 
  \end{question}
\end{Exercise}

\section{微分}
\subsection{微分概念}
\subsection{微分的运算}
\begin{Practice}
  \begin{question}
    \item 
    \item 
    \item 
    \item 
    \item 
    \item 
  \end{question}
\end{Practice}

\subsection{近似计算}
\begin{Practice}
  \begin{question}
    \item 
    \item 
    \item 
    \item 
  \end{question}
\end{Practice}

\begin{Exercise}
  \begin{question}
    \item 
    \item 
    \item 
    \item 
    \item 
    \item 
    \item 
    \item 
    \item 
  \end{question}
\end{Exercise}

\section*{小结}
\begin{enumerate}[C、,itemindent=4.5em]
  \item 
  \item 
  \item 
  \item 
  \item 
\end{enumerate}

\chapter*{复习参考题\chinese{chapter}}
\section*{A 组}
\begin{question}
  \item 
  \item 
  \item 
  \item 
  \item 
  \item 
  \item 
  \item 
  \item 
  \item 
  \item 
  \item 
  \item 
  \item 
  \item 
  \item 
  \item 
  \item 
  \item 
  \item 
  \item 
  \item 
  \item 
  \item 
\end{question}
\section*{B 组}
\begin{question}
  \item 
  \item 
  \item 
  \item 
  \item 
  \item 
\end{question}
\chapter{导数的应用}
\section{一阶导数的应用}

\subsection{预备知识}
\begin{Practice}
  \begin{question}
    \item 
    \item 
    \item 
  \end{question}
\end{Practice}

\begin{Exercise}
  \begin{question}
    \item 
    \item 
    \item 
    \item 
    \item 
  \end{question}
\end{Exercise}

\subsection{函数的单调性}
\begin{Practice}
  \begin{question}
    \item 
    \item 
    \item 
    \item 
    \item 
  \end{question}
\end{Practice}

\subsection{函数的极大值与极小值}
\begin{Practice}
  \begin{question}
    \item 
    \item 
    \item 
    \item 
  \end{question}
\end{Practice}

\begin{Exercise}
  \begin{question}
    \item 
    \item 
    \item 
    \item 
    \item 
    \item 
    \item 
    \item 
    \item 
    \item 
    \item 
  \end{question}
\end{Exercise}

\subsection{函数的最大值与最小值}
\begin{Practice}
  \begin{question}
    \item 
    \item 
    \item 
    \item 
  \end{question}
\end{Practice}

\begin{Exercise}
  \begin{question}
    \item 
    \item 
    \item 
    \item 
    \item 
    \item 
    \item 
    \item 
    \item 
    \item 
    \item 
    \item 
    \item 
    \item 
    \item 
    \item 
    \item 
    \item 
    \item 
    \item 
    \item 
    \item 
    \item 
    \item 
    \item 
  \end{question}
\end{Exercise}

\section{二阶导数的应用}


\subsection{预备知识}
\subsection{函数极值的判定}
\begin{Practice}
  应用二阶导数求下列函数的极值:
  \begin{tasks}
    \task $f(x)=x^3+3x^2-9x+6$;
    \task $f(x)=x-2\sin x$($0\leqslant x \leqslant 2\uppi $);
    \task $g(x)=ax^2+bx+c$($a\neq 0$)。
  \end{tasks}
\end{Practice}

\subsection{曲线的凸向和拐点}
\begin{Practice}
  \begin{question}
    \item 
    \item 
  \end{question}
\end{Practice}

\subsection{函数的图像}

\begin{Exercise}
  \begin{question}
    \item 
    \item 
    \item 
    \item 
    \item 
    \item 
  \end{question}
\end{Exercise}

\section*{小结}
\begin{enumerate}[C、,itemindent=4.5em]
  \item 
  \item 
  \item 
  \item 
  \item 
\end{enumerate}

\chapter*{复习参考题\chinese{chapter}}
\section*{A 组}
\begin{question}
  \item 
  \item 
  \item 
  \item 
  \item 
  \item 
  \item 
  \item 
  \item 
  \item 
  \item 
  \item 
  \item 
\end{question}
\section*{B 组}
\begin{question}
  \item 
  \item 
  \item 
  \item 
  \item 
  \item 
  \item 
  \item 
  \item 
  \item 
  \item 
  \item 
  \item 
  \item 
\end{question}
\chapter{不定积分}
\phantomsection\pdfbookmark[1]{不定积分}{phantomsection2}
\subsection{原函数}
\subsection{不定积分}
\begin{Practice}
  \begin{question}
    \item 
    \item 
    \item 
    \item 
  \end{question}
\end{Practice}

\subsection{基本积分公式}
\begin{Practice}
  \begin{question}
    \item 
    \item 
  \end{question}
\end{Practice}

\subsection{不定积分的运算法则}
\begin{Practice}
  求不定积分:
  \begin{tasks}(2)
    \task
    \task
    \task
    \task
    \task
    \task
    \task
    \task
  \end{tasks}
\end{Practice}

\subsection{直接积分法}
\begin{Practice}
  求不定积分:
  \begin{tasks}(2)
    \task
    \task
    \task
    \task
    \task
    \task
    \task
    \task
    \task
    \task
  \end{tasks}
\end{Practice}

\begin{Exercise}
  \begin{question}
    \item 
    \item 
    \item 
    \item 
  \end{question}
\end{Exercise}

\subsection{换元积分法}
\begin{Practice}
  \begin{question}
    \item 
    \item 
    \item 
  \end{question}
\end{Practice}

\subsection{分部积分法}
\begin{Practice}
  用分部积分法求不定积分:
  \begin{tasks}(2)
    \task
    \task
    \task
    \task
    \task
    \task
    \task
    \task
  \end{tasks}
\end{Practice}

\subsection{积分表的用法}
\begin{Practice}
  利用积分表求不定积分:
  \begin{tasks}(2)
    \task
    \task
    \task
    \task
    \task
    \task
    \task
    \task
  \end{tasks}
\end{Practice}

\begin{Exercise}
  \begin{question}
    \item 
    \item 
    \item 
    \item 
  \end{question}
\end{Exercise}

\section*{小结}
\begin{enumerate}[C、,itemindent=4.5em]
  \item 
  \item 
  \item 
  \item 
\end{enumerate}

\chapter*{复习参考题\chinese{chapter}}
\section*{A 组}
\begin{question}
  \item 
  \item 
  \item 
\end{question}
\section*{B 组}
\begin{question}
  \item 
  \item 
  \item 
  \item 
  \item 
\end{question}
\chapter{定积分及其应用}
\section{定积分的概念和计算}

\subsection{定积分的概念}
\begin{Practice}
  \begin{question}
    \item 
    \item 
  \end{question}
\end{Practice}

\subsection{微积分基本公式}
\begin{Practice}
  计算定积分:
  \begin{tasks}(2)
    \task
    \task
    \task
    \task
    \task
    \task
    \task
    \task
    \task
    \task
  \end{tasks}
\end{Practice}

\begin{Exercise}
  计算定积分:
  \begin{tasks}(2)
    \task
    \task
    \task
    \task
    \task
    \task
    \task
    \task
    \task
    \task
    \task
    \task
    \task
    \task
  \end{tasks}
\end{Exercise}

\section{定积分的应用}
\subsection{平面图形的面积}
\begin{Practice}
  求下列曲线围成的图形的面积:
  \begin{tasks}
    \task 
    \task 
    \task 
    \task 
    \task 
  \end{tasks}
\end{Practice}

\subsection{旋转体的体积}
\begin{Practice}
  \begin{question}
    \item 
    \item 
    \item 
  \end{question}
\end{Practice}

\subsection*{平面曲线的弧长}
\begin{Practice}
  \begin{question}
    \item 
    \item 
  \end{question}
\end{Practice}

\subsection*{旋转体的侧面积}
\begin{Practice}
  求曲线 $y^2=x$,直线 $x=0$,$x=6$ 所围图形绕 $x$ 轴旋转所得旋转体的侧面积。
\end{Practice}

\begin{Exercise}
  \begin{question}
    \item 
    \item 
    \item 
    \item 
    \item 
    \item 
    \item 
    \item 
  \end{question}
\end{Exercise}

\section*{小结}
\begin{enumerate}[C、,itemindent=4.5em]
  \item 本章主要内容是定积分的概念、计算及其简单应用。
  \item 定积分的概念是从求曲边梯形的面积、变速直线运动的路程等实际问题引入的。解决这类问题都是通过分割,取近似,最后归结为求一种和式的极限:
  \[ \lim_{n \to \infty}\sum_{i=1}^{n} f(\xi_i)\Delta x. \]
  (其中 $f(x)$ 为区间 $[a,b]$ 上的连续函数,把区间 $[a,b]$ $n$ 等分后,$\Delta x= \dfrac{b-a}{n}$,而 $\xi_i$ 是第 $i$ 个小区间上的任意一点)。这个极限叫做函数 $f(x)$ 在区间 $[a,b]$ 上的定积分,记作
  \[\int_a^b f(x)\dif x=\lim_{n\to\infty}\sum_{i=1}^{n} f(\xi_i)\Delta x.\]
  \item 微积分基本公式是
  \[\int_a^b f(x) \dif x= F(b)-F(a)\]
  其中 $F(x)$ 是函数 $f(x)$ 的任一原函数,即 $F'(x)=f(x)$,就是说,函数 $f(x)$ 在区间 $[a,b]$ 上的定积分 $\int_a^b f(x)\dif x$,等于它的任一原函数 $F(x)$ 在区间 $[a,b]$ 上的改变量 $F(b)-F(a)$。
  这个公式是定积分与原函数之问的关系式,它使定积分的计算大为简化。
  \item 定积分的一些简单应用:
  \begin{enumerate}[1.]
    \item 求曲边梯形的面积,公式是
    \[S=\int_a^b f(x)\dif x;\]
    \item 求旋转体的体积,公式是
    \[V=\uppi\int_a^b [f(x)]^2\dif x;\]
    \item 求平面曲线弧长,公式是
    \[l=\int_a^b \sqrt{1+[f'(x)]^2}\dif x;\]
    \item 求旋转体的侧面积,公式是
    \[S=2\uppi\int_a^b f(x)\sqrt{1+[f'(x)]^2}\dif x.\]
  \end{enumerate}
\end{enumerate}

\chapter*{复习参考题\chinese{chapter}}
\section*{A 组}
\begin{question}
  \item 计算定积分:
  \begin{tasks}(2)
    \task $\displaystyle \int_0^a (3x^2-x+1) \dif x$;
    \task $\displaystyle \int_1^2 \left(x^2+\frac{1}{x^4}\right) \dif x$;
    \task $\displaystyle \int_2^4 \frac{x^3-3x^2+5}{x^2} \dif x$;
    \task $\displaystyle \int_1^3 y^2(y-2) \dif y$;
    \task $\displaystyle \int_{-1}^{1} x(x-3) \dif x$;
    \task $\displaystyle \int_{-2}^{2} (6x^3+x+1) \dif x$;
    \task $\displaystyle \int_{-1}^{1} \left(x^2-\frac{x}{x^2+1}\right) \dif x$;
    \task $\displaystyle \int_{0}^{\frac{1}{3}} \frac{1}{4-3x} \dif x$;
    \task $\displaystyle \int_{\frac{1}{2}}^{1} \sqrt{3-2x} \dif x$.
  \end{tasks}
  \item 计算定积分:
  \begin{tasks}(2)
    \task $\displaystyle \int_0^{\uppi} \sqrt{1-\cos2x} \dif x$;
    \task $\displaystyle \int_{\frac{\uppi}{6}}^{\frac{\uppi}{2}} \cos^2u \dif u$;
    \task $\displaystyle \int_0^{\frac{\uppi}{4}} \tan^2\theta \dif \theta$;
    \task $\displaystyle \int_{\frac{\uppi}{3}}^{\frac{2\uppi}{3}} (2\sin x+\cos x) \dif x$;
    \task $\displaystyle \int_0^{\frac{\uppi}{2}} \sin\varphi\cos^2\varphi \dif \varphi$;
    \task $\displaystyle \int_0^4 \frac{1}{1+\sqrt{x}} \dif x$;
    \task $\displaystyle \int_0^{e-1} \ln(x+1) \dif x$;
    \task $\displaystyle \int_0^1 xe^x \dif x$。
  \end{tasks}
  \item 求下列各曲线围成的图形的面积:
  \begin{tasks}
    \task 曲线 $y=x^3$,$y=x^2$,直线 $x=1$,$x=2$;
    \task 曲线 $y=\sin x$,$y=\cos x$,直线 $x=-\frac{\uppi}{4}$,$x=\frac{\uppi}{4}$;
    \task 曲线 $y=\frac{1}{x}$,直线 $y=x$,$x=2$,$y=0$;
    \task 曲线 $y=x^2$,直线 $y=x$,$y=2x$;
    \task 曲线 $y=x^2-4x+5$,直线 $x=3$,$x=5$,$y=0$;
    \task 曲线 $y=3-2x-x^2$,$y=0$。
  \end{tasks}
  \item 求下列曲线所围图形绕 $x$ 轴旋转而成的旋转体体积:
  \begin{tasks}
    \task $y=x^3$,$x=2$,$y=0$;
    \task $y=\cos x$,$x=-\frac{\uppi}{4}$,$x=\frac{\uppi}{4}$,$y=0$;
    \task $xy=4$,$x=1$,$x=4$,$y=0$;
    \task $x^2-y^2=a^2$,$x=a+h$,($a>0,h>0$);
    \task $y=1+\sqrt{x}$,$y=3$,$x=0$。
  \end{tasks}
  \item 求曲线 $y=\dfrac{x^2}{2}-2$ 与 $x$ 轴交点间的曲线弧长。
  \item 将立方抛物线 $a^2y=x^3$ 由 $x=0$ 到 $x=a$ 的一段弧,绕 $x$ 轴旋转一周,求旋转面的面积。
  \item 星形线 $x^{\frac{2}{3}}+y^{\frac{2}{3}}=a^{\frac{2}{3}}$ 绕 $x$ 轴旋转一周,求所得曲面面积。
\end{question}
\section*{B 组}
\begin{question}[resume]
  \item 计算定积分:
  \begin{tasks}(2)
    \task $\displaystyle \int_0^{2a} (x-a)^3 \dif x$;
    \task $\displaystyle \int_{-2}^{0} x^3(x-a)^2 \dif x$;
    \task $\displaystyle \int_{-a}^{0} \left(\frac{x+a}{a}\right)^2 \dif x$;
    \task $\displaystyle \int_{-\uppi}^{\uppi} \sin 2x\sin 4x \dif x$。
  \end{tasks}
  \item 求抛物线 $y=-x^2+4x-3$ 及其在点 $A\,(0,-3)$ 与点 $B\,(3,0)$ 处的切线所围图形的面积。
  \item \label{exec:t-10}如图,已知曲线方程 $y^2=x^2(1-x^2)$,求图中阴影部分的面积。
  \begin{figurehere}
    \begin{minipage}{\linewidth}\centering
      \caption*{第 \ref{exec:t-10} 题}
    \end{minipage}
  \end{figurehere}
  \item 过椭圆 $\frac{x^2}{5}+y^2=1$ 的两个焦点作 $x$ 轴的垂线,将椭圆的夹在这两条垂线间的部分与这两条垂线及 $x$ 轴所围曲边梯形绕 $x$ 轴旋转,求得到的旋转体的体积。
  \item 求曲线 $9ay^2=x(x-3a)^2$ 由 $x=0$ 到 $x=3a$ 的弧长。
  \item 求 $x^2+(y-b)^2=a^2$($b>a$)绕 $x$ 轴旋转所成的旋转体的表面积。
\end{question}
\appendix
\chapter{简易积分表}
\section*{基本积分公式}
\begin{enumerate}[1.,itemsep=5pt]
  \item $\displaystyle \int \dif x=x+C$
  \item $\displaystyle \int x^n\dif x=\frac{x^n}{n+1}+C$
  \item $\displaystyle \int \frac{1}{x}\dif x=\ln|x|+C$
  \item $\displaystyle \int e^x\dif x=e^x+C$
  \item $\displaystyle \int a^x\dif x=\frac{1}{\ln a}a^x+C$
  \item $\displaystyle \int \sin x\dif x=-\cos x+C$
  \item $\displaystyle \int \cos x\dif x=\sin x+C$
  \item $\displaystyle \int \tan x\dif x=-\ln|\cos x|+C$
  \item $\displaystyle \int \cot x\dif x=\ln|\sin x|+C$
  \item $\displaystyle \int \sec^2 x\dif x=\int \frac{1}{\cos^2 x} \dif x=\tan x+C$
  \item $\displaystyle \int \csc^2 x\dif x=\int \frac{1}{\sin^2 x} \dif x=-\cot x+C$
  \item $\displaystyle \int \sec x\dif x=\int \frac{1}{\cos x} \dif x=\ln|\sec x+\tan x|+C =\ln\left|\tan\left(\frac{x}{2}+\frac{\uppi}{4}\right)\right| +C$
  \item $\displaystyle \int \csc x\dif x=\int \frac{1}{\sin x} \dif x=\ln|\csc x-\cot x|+C =\ln\left|\tan\frac{x}{2}\right|+C$
  \item $\displaystyle \int \sec x\tan x\dif x=\sec x+C$
  \item $\displaystyle \int \csc x\cot x\dif x=-\csc x+C$
  \item $\displaystyle \int \frac{1}{\sqrt{a^2-x^2}}\dif x=\arcsin \frac{x}{a}+C = -\arccos\frac{x}{a} +C$
  \item $\displaystyle \int \frac{1}{a^2+x^2}\dif x=\frac{1}{a}\arctan \frac{x}{a}+C$
\end{enumerate}
\section*{有理函数的积分}
\begin{enumerate}[1.,itemsep=5pt,resume]
  \item $\displaystyle \int \frac{1}{a+bx}\dif x=\frac{1}{b}\ln|a+bx|+C$
  \item $\displaystyle \int (a+bx)^n\dif x=\frac{(a+bx)^{n+1}}{b(n+1)}+C\,(n\neq -1)$
  \item $\displaystyle \int \frac{x}{(a+bx)^2}\dif x=\frac{1}{b^2}\left[ \frac{a}{a+bx}+ \ln|a+bx|\right]+C$
  \item $\displaystyle \int \frac{x^2}{(a+bx)^2}\dif x=\frac{1}{b^3}\left[ a+bx-\frac{a^2}{a+bx}-2a\ln|a+bx|\right]+C$
  \item $\displaystyle \int \frac{1}{x(a+bx)}\dif x=-\frac{1}{a}\ln\left|\frac{a+bx}{x}\right|+C$
  \item $\displaystyle \int \frac{1}{(x+a)(x+b)}\dif x=\frac{1}{b-a}\ln\left|\frac{x+a}{x+b}\right|+C$
  \item $\displaystyle \int \frac{1}{x^2(a+bx)}\dif x=-\frac{1}{ax}+\frac{b}{a^2}\ln\left|\frac{a+bx}{x}\right|+C$
  \item $\displaystyle \int \frac{1}{x(a+bx)^2}\dif x=\frac{1}{a(a+bx)}-\frac{1}{a^2}\ln\left|\frac{a+bx}{x}\right|+C$
  \item $\displaystyle \int \frac{1}{x^2(a+bx)^2}\dif x=-\frac{1}{a^3}\left[\frac{a+bx}{x}-2b\ln\left|\frac{a+bx}{x}\right|-\frac{b^2x}{a+bx}\right]+C$
  \item $\displaystyle \int \frac{1}{a+bx^2}\dif x =\frac{1}{\sqrt{ab}}\arctan\frac{x\sqrt{ab}}{a} +C$($a$、$b$ 同号)
  \item $\displaystyle \int \frac{1}{a+bx^2}\dif x =\frac{1}{2\sqrt{-ab}}\ln\left|\frac{a+\sqrt{-ab}x}{a-\sqrt{-ab}x}\right|+C$($a$、$b$ 异号)
  \item $\displaystyle \int \frac{x}{a+bx^2}\dif x =\frac{1}{2b}\ln\left|a+bx^2\right|+C$
  \item $\displaystyle \int \frac{x}{a^2\pm b^2x^2}\dif x =\frac{1}{\pm 2b^2}\ln\left|a^2\pm b^2x^2\right|+C$
  \item $\displaystyle \int \frac{1}{a^2+b^2x^2}\dif x =\frac{1}{ab}\arctan\frac{bx}{a}+C$
  \item $\displaystyle \int \frac{1}{a^2-b^2x^2}\dif x =\frac{1}{2ab}\ln\left|\frac{a+bx}{a-bx}\right|+C$
  \item $\displaystyle \int \frac{1}{x(a^2\pm b^2x^2)}\dif x =\frac{1}{2a^2}\ln\left|\frac{x^2}{a^2\pm b^2x^2}\right|+C$
  \item $\displaystyle \int \frac{1}{x^2(a^2+b^2x^2)}\dif x =-\frac{x}{a^2}-\frac{b}{a^3}\arctan\frac{bx}{a}+C$
  \item $\displaystyle \int \frac{1}{(a^2+b^2x^2)^2}\dif x =\frac{x}{2a^2(a^2+b^2x^2)}+\frac{1}{2a^3b}\arctan\frac{bx}{a}+C$
  \item $\displaystyle \int \frac{1}{(a^2-b^2x^2)^2}\dif x =\frac{x}{2a^2(a^2-b^2x^2)}+\frac{1}{4a^3b}\ln\left|\frac{a+bx}{a-bx}\right|+C$
  \item $\displaystyle \int \frac{1}{a+bx+cx^2}\dif x =\frac{2}{\sqrt{4ac-b^2}}\arctan\left(\frac{2cx+b}{\sqrt{4ac-b^2}} \right) +C$($b^2<4ac$)
  \item $\displaystyle \int \frac{1}{a+bx+cx^2}\dif x =\frac{1}{\sqrt{b^2-4ac}}\ln\left|\frac{2cx+b-\sqrt{b^2-4ac}}{2cx+b+\sqrt{b^2-4ac}}\right|+C$($b^2>4ac$)
\end{enumerate}
\section*{无理函数的积分}
\begin{enumerate}[1.,itemsep=5pt,resume]
  \item $\displaystyle \int x\sqrt{a+bx}\dif x=-\frac{2(2a-3bx)(a+bx)^{\frac{3}{2}}}{15b^2} +C$
  \item $\displaystyle \int x^2\sqrt{a+bx}\dif x=-\frac{2(8a^2-12abx+15b^2x^2)(a+bx)^{\frac{3}{2}}}{105b^3} +C$
  \item $\displaystyle \int \frac{x}{\sqrt{a+bx}}\dif x =-\frac{2(2a-bx)\sqrt{a+bx}}{3b^2} +C$
  \item $\displaystyle \int \frac{x^2}{\sqrt{a+bx}}\dif x =\frac{2(8a^2-4abx+3b^2x^2)\sqrt{a+bx}}{15b^3} +C$
  \item $\displaystyle \int \frac{1}{x\sqrt{a+bx}}\dif x =\frac{1}{\sqrt{a}}\ln\left|\frac{\sqrt{a+bx}-\sqrt{a}}{\sqrt{a+bx}+\sqrt{a}}\right|+C$($a>0$)
  \item $\displaystyle \int \frac{1}{x\sqrt{a+bx}}\dif x =\frac{2}{\sqrt{-a}}\arctan\sqrt{\frac{a+bx}{-a}}+C$($a<0$)
  \item $\displaystyle \int \frac{1}{\sqrt{x^2\pm a^2}}\dif x = \ln\left|x+\sqrt{x^2\pm a^2}\right|+C$
  \item $\displaystyle \int \sqrt{x^2\pm a^2}\dif x =\frac{x}{2}\sqrt{x^2\pm a^2}\pm\frac{a^2}{2}\ln\left|x+\sqrt{x^2\pm a^2}\right| +C$
  \item $\displaystyle \int (x^2\pm a^2)^{\frac{3}{2}}\dif x =\frac{x}{8}(2x^2\pm5a^2)\sqrt{x^2\pm a^2}+\frac{3a^4}{8}\ln\left|x+\sqrt{x^2\pm a^2}\right| +C$
  \item $\displaystyle \int \frac{1}{(x^2\pm a^2)^{\frac{3}{2}}}\dif x =\frac{x}{\pm a^2\sqrt{x^2\pm a^2}} +C$
  \item $\displaystyle \int \frac{x^2}{\sqrt{x^2\pm a^2}}\dif x =\frac{x}{2}\sqrt{x^2+a^2}\mp\frac{a^2}{2}\ln\left|x+\sqrt{x^2\pm a^2}\right| +C$
  \item $\displaystyle \int \frac{1}{x\sqrt{x^2+a^2}}\dif x =-\frac{1}{a}\ln\left|\frac{a+\sqrt{x^2+a^2}}{x}\right| +C$
  \item $\displaystyle \int \frac{1}{x\sqrt{x^2-a^2}}\dif x =\frac{1}{a}\arccos\frac{a}{x} +C$
  \item $\displaystyle \int \frac{1}{x^2\sqrt{x^2\pm a^2}}\dif x =\mp\frac{\sqrt{x^2\pm a^2}}{a^x} +C$
  \item $\displaystyle \int \frac{\sqrt{x^2+a^2}}{x}\dif x =\sqrt{x^2+a^2}-a\ln\left|\frac{a+\sqrt{x^2+a^2}}{x}\right| +C$
  \item $\displaystyle \int \frac{\sqrt{x^2-a^2}}{x}\dif x =\sqrt{x^2-a^2}-a\arccos\frac{a}{x} +C$
  \item $\displaystyle \int \frac{\sqrt{x^2\pm a^2}}{x^2}\dif x =-\frac{\sqrt{x^2\pm a^2}}{x}+\ln\left|x+\sqrt{x^2\pm a^2}\right| +C$
  \item $\displaystyle \int \sqrt{a^2-x^2}\dif x =\frac{x}{2}\sqrt{a^2-x^2}+\frac{a^2}{2}\arcsin\frac{x}{a} +C$
  \item $\displaystyle \int (a^2-x^2)^{\frac{3}{2}}\dif x =\frac{x}{8}(5a^2-2x^2)\sqrt{a^2-x^2}+\frac{3}{8}a^4\arcsin\frac{x}{a} +C$
  \item $\displaystyle \int \frac{1}{\sqrt{a^2-x^2}}\dif x =\arcsin\frac{x}{a} +C$
  \item $\displaystyle \int \frac{1}{(a^2-x^2)^{\frac{3}{2}}}\dif x =\frac{x}{a^2\sqrt{a^2-x^2}} +C$
  \item $\displaystyle \int x^2\sqrt{a^2-x^2}\dif x =\frac{x}{8}(2x^2-a^2)\sqrt{a^2-x^2}+\frac{a^4}{8}\arcsin\frac{x}{a}+C$
  \item $\displaystyle \int \frac{x^2}{\sqrt{a^2-x^2}}\dif x =-\frac{x}{2}\sqrt{a^2-x^2}+\frac{a^2}{2}\arcsin\frac{x}{a} +C$
  \item $\displaystyle \int \frac{1}{x\sqrt{a^2-x^2}}\dif x =-\frac{1}{a}\ln\left|\frac{a+\sqrt{a^2-x^2}}{x}\right| +C$
  \item $\displaystyle \int \frac{1}{x^2\sqrt{a^2-x^2}}\dif x = -\frac{\sqrt{a^2-x^2}}{a^2x}+C$
  \item $\displaystyle \int \frac{\sqrt{a^2-x^2}}{x}\dif x =\sqrt{a^2-x^2}-a\ln\left|\frac{a+\sqrt{a^2-x^2}}{x}\right| +C$
  \item $\displaystyle \int \frac{\sqrt{a^2-x^2}}{x^2}\dif x =-\frac{\sqrt{a^2-x^2}}{x}-\arcsin\frac{x}{a}+C$
  \item $\displaystyle \int \sqrt{2ax-x^2}\dif x =\frac{x-a}{2}\sqrt{2ax-x^2}+\frac{a^2}{2}\arccos\left(1-\frac{x}{a}\right)+C$
  \item $\displaystyle \int x\sqrt{2ax-x^2}\dif x =-\frac{3a^2+ax-2x^2}{6}\sqrt{2ax-x^2}+\frac{a^3}{2}\arccos\left(1-\frac{x}{a}\right)+C$
  \item $\displaystyle \int \frac{\sqrt{2ax-x^2}}{x}\dif x = \sqrt{2ax-x^2}+a\arccos\left(1-\frac{x}{a}\right)+C$
  \item $\displaystyle \int \frac{\sqrt{2ax-x^2}}{x^2}\dif x = -\frac{2\sqrt{2ax-x^2}}{x}-\arccos\left(1-\frac{x}{a}\right)+C$
  \item $\displaystyle \int \frac{1}{\sqrt{2ax-x^2}}\dif x = \arccos\left(1-\frac{x}{a}\right)+C$
  \item $\displaystyle \int \frac{x}{\sqrt{2ax-x^2}}\dif x =-\sqrt{2ax-x^2}+a\arccos\left(1-\frac{x}{a}\right) +C$
  \item $\displaystyle \int \frac{x^2}{\sqrt{2ax-x^2}}\dif x =-\frac{(x+3a)\sqrt{2ax-x^2}}{2}+\frac{3a^2}{2}\arccos\left(1-\frac{x}{a}\right) +C$
  \item $\displaystyle \int \frac{1}{x\sqrt{2ax-x^2}}\dif x = -\frac{\sqrt{2ax-x^2}}{ax} +C$
  \item $\displaystyle \int \frac{1}{\sqrt{2ax+x^2}}\dif x = \ln\left|x+a+\sqrt{2ax+x^2}\right| +C$
  \item $\displaystyle \int \sqrt{\frac{a+x}{b+x}}\dif x = \sqrt{(a+x)(b+x)}+(a-b)\ln(\sqrt{a+x}+\sqrt{b+x})+C$
  \item $\displaystyle \int \sqrt{\frac{a-x}{b+x}}\dif x =\sqrt{(a-x)(b+x)}+(a+b)\arcsin\sqrt{\frac{x+b}{a+b}} +C$
  \item $\displaystyle \int \sqrt{\frac{a+x}{b-x}}\dif x =-\sqrt{(a+x)(b-x)}-(a+b)\arcsin\sqrt{\frac{b-x}{a+b}} +C$
  \item $\displaystyle \int \frac{1}{\sqrt{(x-a)(b-x)}}\dif x =2\arcsin\sqrt{\frac{x-a}{b-a}} +C$
\end{enumerate}
\section*{超越函数的积分}
\begin{enumerate}[1.,itemsep=5pt,resume]
  \item $\displaystyle \int e^{ax}\dif x=\frac{e^{ax}}{a}+C$
  \item $\displaystyle \int b^{ax}\dif x=\frac{b^{ax}}{a\ln b}+C$
  \item $\displaystyle \int \ln x\dif x=x\ln x-x+C$
  \item $\displaystyle \int x^n\ln x\dif x=x^{n+1}\left[ \frac{\ln x}{n+1}-\frac{1}{(n+1)^2}\right]+C$
  \item $\displaystyle \int \sin^2 x\dif x=\frac{1}{2}x-\frac{1}{4}\sin2x+C$
  \item $\displaystyle \int \cos^2 x\dif x=\frac{1}{2}x+\frac{1}{4}\sin2x+C$
  \item $\displaystyle \int \cos^n x\sin x\dif x=-\frac{\cos^{n+1}x}{n+1}+C$
  \item $\displaystyle \int \sin^n x\cos x\dif x=\frac{\sin^{n+1}x}{n+1}+C$
  \item $\displaystyle \int \sin mx\sin nx\dif x=-\frac{\sin(m+n)x}{2(m+n)}+\frac{\sin(m-n)x}{2(m-n)}+C$
  \item $\displaystyle \int \cos mx\cos nx\dif x=\frac{\sin(m+n)x}{2(m+n)}+\frac{\sin(m-n)x}{2(m-n)}+C$
  \item $\displaystyle \int \sin mx\cos nx\dif x=-\frac{\cos(m+n)x}{2(m+n)}-\frac{\cos(m-n)x}{2(m-n)}+C$
  \item $\displaystyle \int \frac{1}{1+\cos x}\dif x=\tan\frac{x}{2}+C$
  \item $\displaystyle \int \frac{1}{1-\cos x}\dif x=-\cot\frac{x}{2}+C$
  \item $\displaystyle \int x\sin nx\dif x=\frac{1}{n^2}\sin nx-\frac{1}{n}x\cos nx+C$
  \item $\displaystyle \int x\cos nx\dif x=\frac{1}{n^2}\cos nx+\frac{1}{n}x\sin nx+C$
  \item $\displaystyle \int x^2\sin nx\dif x=\frac{x}{n^2}(2\sin nx-nx\cos nx)+\frac{2}{n^3}\cos nx+C$
  \item $\displaystyle \int x^2\cos nx\dif x=\frac{x}{n^2}(nx\sin nx+2\cos nx)-\frac{2}{n^3}\sin nx+C$
  \item $\displaystyle \int \arcsin x\dif x=x\arcsin x+\sqrt{1-x^2}+C$
  \item $\displaystyle \int \arccos x\dif x=x\arccos x-\sqrt{1-x^2}+C$
  \item $\displaystyle \int \arctan x\dif x=x\arctan x-\ln\sqrt{1+x^2}+C$
  \item $\displaystyle \int \arccot x\dif x=x\arccot x+\ln\sqrt{1+x^2}+C$
\end{enumerate}
 % ok
\end{document}