\documentclass[colortheme=yellow,txconfig=txmaths.cfg]{textbook}
\stylesetup{ 
  fullwidth-stop = catcode,
  boldemph = false,
}
\Booksetup{
  BookSeries  = 中学经典教材丛书, 
  BookTitle   = 微积分初步(甲种本),
  BookTitle*  = {Textbook for Middle School Calculus},
  SubTitle    = 全一册,
  % SubTitle*   = Volume I,
  BriefIntro    = 
    { 
      本书供六年制中学高中三年级选用。本书内容包括极限;导数和微分;导数的应用;不定积分;定积分及其应用。学完这些内容,约需 84 课时。本书习题共分三类:练习、习题、复习参考题。练习主要供课堂练习用;习题主要供课内外作业用;复习参考题分 A、B 两组。A 组供复习本章知识时使用;B 组题略带综合性、灵活性,仅供学有余力的学生参考使用。练习、习题及复习参考题 A 组题的题量较多,教学时可根据情况选用。本书在编写过程中,曾参考了中小学通用教材数学编写组编写的全日制十年制学校高中课本(试用本)《数学》第四册的有关章节,大部分内容是以原来章节为基础编写的。本书由人民教育出版社数学室编写。参加编写的有方明一、刘远图、曾宪源、于琛等,全书由于琛校订。
    },
  DedicatedTo   = 奔赴高考的莘莘学子,
  CoverGraph    = graphics/CAL.pdf,
  AuthorList    = {人民教育出版社数学室},
  ReleaseDate   = 2025-08-28,
  % Url           = https://www.tjad.cn,
  % ISBN          = 978-7-302-11622-6,
  % Publisher     = 同济极客出版社,
  % Logo          = graphics/logo.pdf,
  % Editor        = {张晨南},
  WrittenStyle  = 编,
}
\graphicspath{{figures/CAL/}}
\ctexset{appendixname={附表}}
\begin{document}
\frontmatter
\tableofcontents
\mainmatter
\chapter{极限}\label{chp:limits}
\phantomsection\pdfbookmark[1]{极限}{phantomsection1}
\subsection{数列的极限}
我们来考察下面两个数列:
\begin{gather}
  \label{eq:sequence1} 1,\frac{1}{2},\frac{1}{3},\cdots,\frac{1}{n},\cdots\\
  \label{eq:sequence2} \frac{1}{2},\frac{3}{4},\frac{7}{8},\cdots,1-\frac{1}{2^n},\cdots
\end{gather}

为了直观起见,我们把这两个数列中的前几项分别在数轴上表示出来(\cref{fig:1-1}):
\begin{figure}
  \begin{minipage}{\linewidth}\centering
    \includegraphics{1-1a.pdf}
    \subcaption{}\label{fig:1-1a}
  \end{minipage}
  \par\medskip
  \begin{minipage}{\linewidth}\centering
    \includegraphics{1-1b.pdf}
    \subcaption{}\label{fig:1-1b}
  \end{minipage}
  \caption{}\label{fig:1-1}
\end{figure}

容易看出,当项数 $n$ 无限增大时,数列 \eqref{eq:sequence1} 中的项无限趋近于 0,数列 \eqref{eq:sequence2} 中的项无限趋近于 1。

事实上,在数列 \eqref{eq:sequence1} 中,各项与 0 的差的绝对值如\cref{tab:1-1} 所示。
\begin{table}
  \caption{数列 \eqref{eq:sequence1} 各项与 0 的差的绝对值}\label{tab:1-1}
  \begin{tblr}{colspec={X[c]X[c]X[5,c]},hline{2}={0.8pt},rowsep=2pt}
    项号 & 项 & 这一项与 0 的差的绝对值\\
    1 &  $1$  & $|0-1|=1$ \\
    2 &  $\dfrac{1}{2}$  & $\left|0-\dfrac{1}{2}\right|=\dfrac{1}{2}$ \\
    3 &  $\dfrac{1}{3}$  & $\left|0-\dfrac{1}{3}\right|=\dfrac{1}{3}$ \\
    4 &  $\dfrac{1}{4}$  & $\left|0-\dfrac{1}{4}\right|=\dfrac{1}{4}$ \\
    5 &  $\dfrac{1}{5}$  & $\left|0-\dfrac{1}{5}\right|=\dfrac{1}{5}$ \\
    6 &  $\dfrac{1}{6}$  & $\left|0-\dfrac{1}{6}\right|=\dfrac{1}{6}$ \\
    7 &  $\dfrac{1}{7}$  & $\left|0-\dfrac{1}{7}\right|=\dfrac{1}{7}$ \\
    $\cdots$ &  $\cdots$  & $\cdots$ \\
  \end{tblr}
\end{table}

我们看到,无论预先指定多么小的一个正数 $\varepsilon$,总能在数列 \eqref{eq:sequence1} 中找到这样一项,使得这一项后面的所有项与 0 的差的绝对值都小于 $\varepsilon$。
例如,如果取 $\varepsilon=1/5$,那么数列 \eqref{eq:sequence1} 中第 5 项后面所有的项与 0 的差的绝对值都小于 $\varepsilon$。
如果取 $\varepsilon = 1/100$,那么数列 \eqref{eq:sequence1} 中第 100 项后面所有的项与 0 的差的绝对值都小于 $\varepsilon$。
在这种情况下,我们就说数列 \eqref{eq:sequence1} 的极限是 0。

同样,对于数列 \eqref{eq:sequence2},我们也可以列成\cref{tab:1-2}。
\begin{table}
  \caption{数列 \eqref{eq:sequence2} 各项与 0 的差的绝对值}\label{tab:1-2}
  \begin{tblr}{colspec={X[c]X[c]X[5,c]},hline{2}={0.8pt},rowsep=2pt}
    项号 & 项 & 这一项与 1 的差的绝对值\\
    1 &  $\dfrac{1}{2}    $ & $\left|\dfrac{1}{2}    -1\right|=\dfrac{1}{2}=0.5$ \\
    2 &  $\dfrac{3}{4}    $ & $\left|\dfrac{3}{4}    -1\right|=\dfrac{1}{4}=0.25$ \\
    3 &  $\dfrac{7}{8}    $ & $\left|\dfrac{7}{8}    -1\right|=\dfrac{1}{8}=0.125$ \\
    4 &  $\dfrac{15}{16}  $ & $\left|\dfrac{15}{16}  -1\right|=\dfrac{1}{16}=0.0625$ \\
    5 &  $\dfrac{31}{32}  $ & $\left|\dfrac{31}{32}  -1\right|=\dfrac{1}{32}=0.03125$ \\
    6 &  $\dfrac{63}{64}  $ & $\left|\dfrac{63}{64}  -1\right|=\dfrac{1}{64}=0.015625$ \\
    7 &  $\dfrac{127}{128}$ & $\left|\dfrac{127}{128}-1\right|=\dfrac{1}{128}=0.0078125$ \\
    $\cdots$ &  $\cdots$  & $\cdots$ \\
  \end{tblr}
\end{table}

可以看出,如果取 $\varepsilon =0.1$,那么数列 \eqref{eq:sequence2} 中第 3 项后面所有的项与 1 的差的绝对值都小于 $\varepsilon$;如果取 $\varepsilon =0.01$,那么第 6 项后面所有的项与 1 的差的绝对值都小于 $\varepsilon$。
就是说,无论预先指定多么小的一个正数 $\varepsilon$,总能在数列 \eqref{eq:sequence2} 中找到这样一项,使得这一项后面的所有项与 1 的差的绝对值都小于 $\varepsilon$。
这时,我们说数列 \eqref{eq:sequence2} 的极限是 1。

一般地,对于一个无穷数列 $\{a_n\}$,如果存在一个常数 $A$,无论预先指定多么小的正数 $\varepsilon$,都能在数列中找到一项 $a_N$,使得这一项后面所有的项与 $A$ 的差的绝对值都小于 $\varepsilon$(即当 $n>N$ 时,$|a_n-A|<\varepsilon$ 恒成立),就把常数 $A$ 叫做\Concept{数列 $\{a_n\}$ 的极限},记作
\footnotetext[1]{$\lim$ 是拉丁文 limis(极限)一词的前三个字母,一般按英文 limit (极限)一词读音。$\lim\limits_{n\to\infty }a_n=A$ 也可读作 “limit $a_n$ 当 $n$ 趋于无穷大时等于 $A$”。}
\[ \lim_{n\to\infty}a_n=A.\footnotemark[1] \]

这个式子读作 “当 $n$ 趋向于无穷大时,$a_n$ 的极限等于 $A$”。
“$\to$” 表示 “趋向于”,“$\infty$” 表示 “无穷大”,“$n\to\infty$” 表示 “$n$ 趋向于无穷大”,也就是 $n$ 无限增大的意思。

$\lim\limits_{n\to\infty}a_n=A$ 有时也可记作
\[ \text{当}\ n\to\infty\text{ 时,} a_n\to A.\]

从数列极限的定义可以看出,数列 $\{a_n\}$ 以 $A$ 为极限,是指当 $n$ 无限增大时,数列 $\{a_n\}$ 中的项 $a_n$ 无限趋近于常数 $A$。
\begin{example}
  已知数列
  \[1,-\frac{1}{2},\frac{1}{3},-\frac{1}{4},\cdots,(-1)^{n+1}\frac{1}{n},\cdots\]
  \begin{enumerate}
    \item 写出这个数列的各项与 0 的差的绝对值。
    \item 第几项后面所有的项与 0 的差的绝对值都小于 0.1?都小于 \num{0.001}?都小于 \num{0.0003}?
    \item 第几项后面所有的项与 0 的差的绝对值都小于任何预先指定的正数 $\varepsilon$?
    \item 0 是不是这个数列的极限?
  \end{enumerate}
\end{example}
\begin{solution}
  这个数列的项在数轴上的表示如\cref{fig:1-2}:
  \begin{figure}
    \includegraphics{1-2.pdf}
    \caption{}\label{fig:1-2}
  \end{figure}
  \begin{enumerate}
    \item 这个数列的各项与 0 的差的绝对值依次是
    \[ 1,\frac{1}{2},\frac{1}{3},\cdots,\frac{1}{n},\cdots\]
    \item 要使 $\frac{1}{n}<0.1$,只要 $n>10$ 就行了。这就是说,第 10 项后面所有的项与 0 的差的绝对值都小于 0.1。
    
    要使 $\frac{1}{n}<0.001$,只要 $n>1000$ 就行了。这就是说,第 1000 项后面所有的项与 0 的差的绝对值都小于 0.001。

    要使 $\frac{1}{n}<0.0003$,只要 $n>3333\frac{1}{3}$ 就行了。这就是说,第 3333 项后面所有的项与 0 的差的绝对值都小于 0.0003。
    \item \label{itm:error} 要使 $\frac{1}{n} < \varepsilon$,只要 $n > \frac{1}{\varepsilon}$ 就行了,因为所求的项数必须是正整数,因此设 $\frac{1}{\varepsilon}$ 的整数部分是 $N$,那么第 $N$ 项后面所有的项与 0 的差的绝对值都小于 $\varepsilon$。
    \item 从 \ref{itm:error} 可以知道,0 是这个数列的极限,记作:
    \[ \lim\limits_{n\to\infty} (-1)^{n+1}\frac{1}{n}=0.\]
  \end{enumerate}
\end{solution}

\begin{example}
  已知数列
  \[ \frac{1}{2},\frac{2}{3},\frac{3}{4},\cdots,\frac{n}{n+1},\cdots \]
  \begin{enumerate}
    \item 计算 $|a_n-1|$。
    \item 第几项后面所有的项与 1 的差的绝对值都小于 $\frac{1}{100}$?
    \item 第几项后面所有的项与 1 的差都小于任意指定的正数 $\varepsilon$?
    \item 1 是不是这个数列的极限?
  \end{enumerate}
\end{example}
\begin{solution}
  \begin{enumerate}
    \item $a_n-1=\left|\dfrac{n}{n+1}-1\right|=\left|\dfrac{-1}{n+1}\right|=\dfrac{1}{n+1}$。
    \item 要使 $\dfrac{1}{n+1}<\dfrac{1}{100}$,就是要使 $n+1>100$,即 $n>99$,这就是说,第 99 项后面所有的项与 1 的差的绝对值都小于 $\dfrac{1}{100}$。
    \item\label{itm:error1.2} 要使 $\dfrac{1}{n+1} < \varepsilon$,就是要使 $n+1>\dfrac{1}{\varepsilon}$,即 $n>\dfrac{1}{\varepsilon }-1$,设 $\dfrac{1}{\varepsilon}-1$ 的整数部分是 $N$,那么第 $N$ 项后面所有的项与 1 的差的绝对值都小于正数 $\varepsilon$。
    \item 从 \ref{itm:error1.2} 可以知道,这个数列的极限是 1 ,记作:
    \[ \lim\limits_{n\to\infty}\frac{n}{n+1}=1.\]
  \end{enumerate}
\end{solution}

\begin{example}
  已知数列
  \[\frac{1}{2},\frac{1}{4},\frac{1}{8},\dots,\frac{1}{2^n},\dots\]
  \begin{enumerate}
    \item 计算 $|a_n-0|$。
    \item 第几项后面所有的项与 0 的差的绝对值小于正数 $\varepsilon$?
    \item 0 是不是这个数列的极限?
  \end{enumerate}
\end{example}
\begin{solution}
  \begin{enumerate}
    \item $|a_n-0|=\left|\dfrac{1}{2^n}-0\right|=\dfrac{1}{2^n}$。
    \item\label{itm:error1.3} 要使 $\dfrac{1}{{2}^{n}}<\varepsilon$,就是要使 $n>\dfrac{\ln \dfrac{1}{\varepsilon}}{\ln2}$。设 $\dfrac{\ln\dfrac{1}{\varepsilon }}{\ln2}$ 的整数部分是 $N$,那么第 $N$ 项后面所有的项与 0 的差的绝对值都小于正数 $\varepsilon$。
    \item 从 \ref{itm:error1.3} 可以知道,这个数列的极限是 0,记作
    \[ \lim\limits_{n\to\infty} \frac{1}{2^n}=0.\]
  \end{enumerate}
\end{solution}

\begin{example}
  求常数数列 $-7,-7,-7,\cdots$ 的极限。 
\end{example}
\begin{solution}
  这个数列的各项与 $-7$ 的差的绝对值都等于 0,所以从第 1 项起,这个绝对值就能够小于任意指定的正数 $\varepsilon$,因此这个数列的极限是 $-7$。
\end{solution}

一般地,任何一个常数数列的极限都是这个常数本身,即
\[ \lim_{n\to\infty} C =C \quad\text{(}C\text{ 是常数)。}\]

应该指出,并不是每一个无穷数列都有极限。例如,数列
\[1,2,3,\dots,n,\dots\]
就没有极限。

数列
\[-1,1,-1,1,\dots,(-1)^n,\dots\]
也没有极限。

\begin{Practice}
  \begin{question}
    \item 已知数列
    \[\frac{1}{1^2},\frac{1}{2^2},\frac{1}{3^2},\frac{1}{4^2},\dots,\frac{1}{n^2},\dots\]
    \begin{tasks}
      \task 把这个数列的前 5 项在数轴上表示出来。
      \task 写出这个数列的各项与 0 的差的绝对值。
      \task 第几项后面的所有项与 0 的差的绝对值都小于 0.1?都小于 0.01 ?都小于 \num{0.0001}?都小于任何预先指定的正数 $\varepsilon$?
      \task 0 是不是这个无穷数列的极限?
    \end{tasks}
    \item 已知数列
    \[4-\frac{1}{10},4-\frac{1}{20},4-\frac{1}{30},\dots,4-\frac{1}{10n},\dots\]
    \begin{tasks}
      \task 计算 $\left|a_n-4\right|$。
      \task 第几项后面的所有项与 4 的差的绝对值都小于 0.01?都小于任意指定的正数 $\varepsilon$?
      \task 确定这个数列的极限。
    \end{tasks}
  \end{question}
\end{Practice}

\subsection{数列极限的四则运算}
前面我们看到,一些简单的数列可以从变化趋势找出它们的极限。例如,
\[\lim_{n\to\infty}\frac{1}{n}=0,\qquad\lim_{n\to\infty}\frac{1}{2^n}=0,\qquad \lim_{n\to\infty}C=C.\]
如果求极限的数列比较复杂,就要分析已知数列是由哪些简单的数列经过怎样的运算结合而成的,这样就能把复杂的数列的极限的计算问题转化为简单的数列的极限的计算问题。
因此,下面引入数列极限的四则运算法则(证明从略):
\begin{Theorem}[数列极限的四则运算法则]{法则}
  如果 $\lim\limits_{n\to\infty}a_n=A$,$\lim\limits_{n\to\infty}a_n=B$,那么,
  \begin{align*}
    \lim\limits_{n\to\infty}(a_n\pm b_n)&=A\pm B \\
    \lim\limits_{n\to\infty}(a_n\cdot b_n)&=A\cdot B \\
    \lim\limits_{n\to\infty}\dfrac{a_n}{b_n}&=\dfrac{A}{B} \quad(B\neq0)
  \end{align*}
\end{Theorem}

特别地,如果 $C$ 是常数,那么,
\[\lim\limits_{n\to\infty}(C\cdot a_n)=\lim\limits_{n\to\infty}C\cdot \lim\limits_{n\to\infty}a_n=CA.\]

上面的数列极限的四则运算法则表明:\emph{如果两个数列都有极限,那么,这两个数列的各对应项的和、差、积、商组成的数列的极限,分别等于这两个数列的极限的和、差、积、商(各项作为除数的数列的极限不能为零)}。

例如,数列
\[ \frac{1}{2},\frac{2}{3},\frac{3}{4},\dots,\frac{n}{n+1},\dots \]
与
\[ 2,2,2,\dots,2,\dots\]
的极限分别是 1 与 2 ,那么根据上面的运算法则,这两个数列的各对应项的和组成的数列的极限是 3。

\begin{example}
  已知 $\lim\limits_{n\to\infty}a_n=5$,$\lim\limits_{n\to\infty}b_n=3$,求 $\lim\limits_{n\to\infty}(3a_n-4b_n)$。
\end{example}
\begin{solution}
  $\lim\limits_{n\to\infty}(3a_n-4b_n)=\lim\limits_{n\to\infty}3a_n-\lim\limits_{n\to\infty}4b_n=3\lim\limits_{n\to\infty}a_n-4\lim\limits_{n\to\infty}b_n=3\times5-4\times3=3.$
\end{solution}

\begin{example}
  求:
  \begin{tasks}[label=(\arabic*),label-width=16pt,before-skip=5pt,after-skip=5pt,after-item-skip=7pt](2)
    \task $\lim\limits_{n\to\infty}\left(5+\dfrac{1}{n}\right)$;
    \task $\lim\limits_{n\to\infty}\dfrac{3n-2}{n}$;
    \task $\lim\limits_{n\to\infty}\dfrac{2n+1}{3n+2}$;
    \task $\lim\limits_{n\to\infty}\dfrac{3n^2-2n+8}{4-n^2}$。
  \end{tasks}
\end{example}
\begin{solution}
  \begin{enumerate}
    \item $\lim\limits_{n\to\infty}\left(5+\dfrac{1}{n}\right)=\lim\limits_{n\to\infty}5+\lim\limits_{n\to\infty}\dfrac{1}{n}=5+0=5.$
    \item 
    \begin{align*}
    \lim\limits_{n\to\infty}\dfrac{3n-2}{n} & =\lim\limits_{n\to\infty}\left(\dfrac{3n}{n}-\dfrac{2}{n}\right)=\lim\limits_{n\to\infty}3-\lim\limits_{n\to\infty}\left(2\cdot\dfrac{1}{n}\right)=3-2\lim\limits_{n\to\infty}\dfrac{1}{n}\\
    &=3-2\times0=3.
    \end{align*}
    \item 当 $n$ 无限增大时,分式 $\dfrac{2n+1}{3n+2}$ 中的分子、分母同时无限增大,上面的极限运算法则不能直接运用。为此,我们将分式中的分子、分母同时除以 $n$ 后求它的极限,得
    \[ \lim\limits_{n\to\infty}\frac{2n+1}{3n+2}=\lim\limits_{n\to\infty}\frac{2+\dfrac{1}{n}}{3+\dfrac{2}{n}}=\dfrac{\lim\limits_{n\to\infty}\left(2+\dfrac{1}{n}\right)}{\lim\limits_{n\to\infty}\left(3+\dfrac{2}{n}\right)}=\dfrac{\lim\limits_{n\to\infty}2+\lim\limits_{n\to\infty}\dfrac{1}{n}}{\lim\limits_{n\to\infty}3+\lim\limits_{n\to\infty}\dfrac{2}{n}}=\frac{2+0}{3+0}=\frac{2}{3}.\]
    \item 
    \begin{align*}
      \lim\limits_{n\to\infty}\dfrac{3n^2-2n+8}{4-n^2}&=\lim\limits_{n\to\infty}\dfrac{3-\dfrac{2}{n}+\dfrac{8}{n^2}}{\dfrac{4}{n^2}-1} =\dfrac{\lim\limits_{n\to\infty}\left(3-\dfrac{2}{n}+\dfrac{8}{n^2}\right)}{\lim\limits_{n\to\infty}\left(\dfrac{4}{n^2}-1\right)}\\ 
      &=\dfrac{\lim\limits_{n\to\infty}3-\lim\limits_{n\to\infty}\dfrac{2}{n}+\lim\limits_{n\to\infty}\dfrac{8}{n^2}}{\lim\limits_{n\to\infty}\dfrac{4}{n^2}-\lim\limits_{n\to\infty}1}=\frac{3-0+0}{0-1}=-3.
    \end{align*}
  \end{enumerate}
\end{solution}

\begin{example}\label{exp:square}
  已知等比数列
  \[\frac{1}{2},\frac{1}{4},\frac{1}{8},\dots,\frac{1}{2^n},\dots\]
  求这个数列前 $n$ 项的和当 $n\to\infty$ 时的极限。
\end{example}
\begin{solution}
  这个等比数列的公比是
  \[ q=\dfrac{\dfrac{1}{4}}{\dfrac{1}{2}}=\frac{1}{2}.\]
  根据等比数列前 n 项和的公式,得
  \[S_n=\frac{a_1(1-q^n)}{1-q}=\frac{\dfrac{1}{2}\left[1-\left(\dfrac{1}{2}\right)^n\right]}{1-\dfrac{1}{2}}=1-\frac{1}{2^n}.\]
  因此,
  \[\lim_{n\to\infty}S_n=\lim_{n\to\infty}\left(1-\frac{1}{2^n}\right)=1-\lim_{n\to\infty}\frac{1}{2^n}=1-0=1.\]
\end{solution}

\medskip\noindent
\begin{minipage}{0.65\linewidth}\parindent2em
  上述结果可从\cref{fig:1-3} 中看出,\cref{fig:1-3} 中各小矩形与小正方形面积的和 (阴影部分)的极限等于大正方形的面积。

  \cref{exp:square} 中的无穷等比数列有这样的特点: 它的公比的绝对值小于 1。一般地,设无穷等比数列
  \[ a_1,a_1q,a_1q^2,\dots,a_1q^{n-1},\dots \]
  的公比 $q$ 的绝对值小于 1,我们来求它的前 $n$ 项的和当 $n$ 无限增大时的极限。
\end{minipage}\hfill
\begin{minipage}{0.3\linewidth}\centering
  \begin{figurehere}
    \includegraphics{1-3.pdf}
    \caption{}\label{fig:1-3}
  \end{figurehere}
\end{minipage}

\medskip
无穷等比数列前 $n$ 项的和是
\[ S_n=a_1+a_1q+a_1q^2+\dots+a_1q^{n-1}=\frac{a_1(1-q^n)}{1-q},\]
因此,
\begin{align*}
  \lim\limits_{n\to\infty}S_n&= \lim\limits_{n\to\infty}\dfrac{a_1(1-q^n)}{1-q}=\lim\limits_{n\to\infty}\dfrac{a_1}{1-q}\cdot\lim\limits_{n\to\infty}(1-q^n)\\ 
  &= \dfrac{a_1}{1-q}\left(\lim\limits_{n\to\infty}1-\lim\limits_{n\to\infty}q^n\right).
\end{align*}
因为当 $|q|<1$ 时,$\lim\limits_{n\to\infty}q^n=0$(证明较繁,本书从略),所以,
\[\lim\limits_{n\to\infty}S_n=\frac{a_1}{1-q}\cdot(1-0)=\frac{a_1}{1-q}.\]

公比的绝对值小于 1 的无穷等比数列前 $n$ 项的和当 $n$ 无限增大时的极限,叫做这个无穷等比数列各项的和(注意:这与有限个数的和从意义上说是不一样的),并且用符号 $S$ 表示。
从上面知道,
\[S=a_1+a_1q+a_1q^2+\dots+a_1q^{n-1}+\dots=\frac{a_1}{1-q}\]

\begin{example}
  求无穷等比数列 $0.3,0.03,0.003,\dots$ 各项的和。
\end{example}
\begin{solution}
  $\because\quad a_1=0.3$,$q=0.1$,
  \par\medskip
  $\therefore \quad S=\dfrac{0.3}{1-0.1}=\dfrac{1}{3}$。
\end{solution}

\begin{example}
  将下列循环小数化成分数:
  \begin{tasks}[label=(\arabic*),label-width=16pt](2)
    \task $0.\dot{7}$;
    \task $0.2\dot{3}\dot{1}$。
  \end{tasks}
\end{example}
\begin{solution}
  \begin{enumerate}
    \item 纯循环小数 $0.\dot{7}=0.777\cdots$ 可以写成
    \[\frac{7}{10}+\frac{7}{100}+\frac{7}{1000}+\cdots\]
    这里各项组成公比等于 $\dfrac{1}{10}$ 的无穷等比数列,因此,
    \[0.\dot{7}=\frac{\dfrac{7}{10}}{1-\dfrac{1}{10}}=\frac{\dfrac{7}{10}}{\dfrac{9}{10}}=\frac{7}{9},\]
    也就是说,$0.\dot{7}=\dfrac{7}{9}$。
    \item 混循环小数 $0.2\dot{3}\dot{1}=0.2313131\cdots$ 可以写成
    \[ \frac{2}{10}+\frac{31}{1000}++\frac{31}{100000}++\frac{31}{10000000}+\dots \]
    这里从第 2 项起各项组成公比等于 $\dfrac{1}{100}$ 的无穷等比数列,因此,
    \[0.2\dot{3}\dot{1}=\frac{2}{10}+\frac{\dfrac{31}{1000}}{1-\dfrac{1}{100}}=\frac{2}{10}+\frac{31}{990}=\frac{2\times99+31}{990}=\frac{229}{990}.\]
  \end{enumerate}
\end{solution}

\begin{Practice}
  \begin{question}
    \item 已知 $\lim\limits_{n\to\infty}a_n= 2$,$\lim\limits_{n\to\infty}b_n=-\dfrac13$,求下列极限:
    \begin{tasks}[before-skip=5pt,after-skip=5pt](2)
      \task $\lim\limits_{n\to\infty}(2a_n+3b_n)$;
      \task $\lim\limits_{n\to\infty}\dfrac{a_n-b_n}{a_n}$。
    \end{tasks}
    \item 求下列极限:
    \begin{tasks}[before-skip=10pt,after-skip=10pt,after-item-skip=7pt](2)
      \task $\lim\limits_{n\to\infty}\left(3-\dfrac{1}{n}\right)$;
      \task $\lim\limits_{n\to\infty}\dfrac{2}{5+\dfrac{3}{n}}$;
      \task $\lim\limits_{n\to\infty}\dfrac{n+1}{n}$;
      \task $\lim\limits_{n\to\infty}\dfrac{n-1}{n^2+1}$。
    \end{tasks}
    \item 求下列无穷等比数列各项的和:
    \begin{tasks}[before-skip=5pt,after-skip=5pt](2)
      \task $3,1,\dfrac{1}{3},\dfrac{1}{9},\dots$;
      \task $1,-\dfrac{1}{2},\dfrac{1}{4},-\dfrac{1}{8},\dots$。
    \end{tasks}
  \end{question}
\end{Practice}

\begin{Exercise}
  \begin{question}
    \item 已知无穷数列 $5+1,5-\dfrac{1}{2},5+\dfrac{1}{3},5-\dfrac{1}{4},\cdots$。
    \begin{tasks}%[before-skip=5pt,after-skip=5pt]
      \task 把这个数列的前 5 项在数轴上表示出来。
      \task 计算这个数列的第 $n$ 项与 5 的差的绝对值 $|a_n-5|$。
      \task 对于任何预先指定正数 $\varepsilon$,找一个自然数 $N$,使 $n>N$ 时,$|a_n-5|<\varepsilon$。
      \task 确定这个数列的极限。
    \end{tasks}
    \item 一个无穷数列的通项公式是 $a_n=\dfrac{n+1}{n+2}$。
    \begin{tasks}
      \task 把这个数列的前 5 项在数轴上表示出来。
      \task 计算 $|a_n-1|$。
      \task 对于下表中的 $\varepsilon$,各找出一个对应自然数 $N$,使 $n>N$ 时,$|a_n-1|<\varepsilon$。
      \begin{tablehere}
        \begin{tblr}{colspec={c*4{X[r]}c},hline{2}=0.8pt}
          $\varepsilon$ & 0.2 & 0.1 & 0.05 & 0.01 & 任何给定的正数 \\
          $N$ &&&&& \\
        \end{tblr}
      \end{tablehere}
      \task 确定这个数列的极限。
    \end{tasks}
    \item 一个无穷数列的通项公式是 $a_n=\dfrac{8n+1}{2n}$。
    \begin{tasks}
      \task 把这个数列的前 5 项在数轴上表示出来。
      \task 计算 $|a_n-4|$。
      \task 确定这个无穷数列的极限。
    \end{tasks}
    \item 一个无穷数列的通项公式是 $a_n=\dfrac{n}{2n+1}$,求证这个数列的极限是 $\dfrac{1}{2}$ 。
    \item 举一个极限是 5 的无穷数列的例子。
    \item 无穷数列 $-2,0,-2,0,\cdots,(-1)^{n}-1,\cdots$ 有极限吗?
    \item 已知无穷数列
    \begin{gather*}
      \frac{5}{3},\frac{10}{4},\frac{15}{5},\cdots,\frac{5n}{n+2},\cdots;\\ 
      \frac{1}{3},\frac{2}{4},\frac{3}{5},\cdots,\frac{n}{n+2},\cdots;
    \end{gather*}
    \begin{tasks}
      \task 求证这两个数列的极限分别是 5 与 1。
      \task 另作一个每一项都等于这两个数列的对应项的和的无穷数列。验证所得数列的极限等于这两个数列的极限的和。
    \end{tasks}
    \item 求下列极限:
    \begin{tasks}[before-skip=10pt,after-skip=10pt,after-item-skip=7pt](2)
      \task $\lim\limits_{n\to\infty}\left(\dfrac{3}{n^2}+\dfrac{1}{n}+5\right)$;
      \task $\lim\limits_{n\to\infty}\dfrac{2n+1}{2n-1}$;
      \task $\lim\limits_{n\to\infty}\left(\dfrac{2}{n}+\dfrac{4n-1}{4n}\right)$;
      \task $\lim\limits_{n\to\infty}\left(1-\dfrac{2n}{n+2}\right)$;
      \task $\lim\limits_{n\to\infty}\dfrac{5+7n}{6n-11}$;
      \task $\lim\limits_{n\to\infty}\dfrac{2n^2+n-1}{3n^2-1}$;
      \task $\lim\limits_{n\to\infty}\dfrac{n+1}{n^2-9}$;
      \task $\lim\limits_{n\to\infty}\dfrac{n^2+n+1}{(n-1)^2}$。
    \end{tasks}
    \item 求下列极限:
    \begin{tasks}[before-skip=10pt,after-skip=10pt,after-item-skip=7pt]
      \task $\lim\limits_{n\to\infty}\dfrac{1+2+\dots+n}{n^2}$;
      \task $\lim\limits_{n\to\infty}\dfrac{1^2+2^2+\dots+n^2}{n^3}$。
      
      (提示:${1^2+2^2+\dots+n^2}=\dfrac{n(n+1)(2n+1)}{6}$)
    \end{tasks}
    \item\label{ex:1-10} 解答:
    \begin{tasks}
      \task 如图,在圆的内接正多边形中,$r_n$ 是边心距,$p_n$ 是周长,$S_n$ 是面积,求证 $S_n=\dfrac{1}{2}p_nr_n$。
      \task 当圆的内接正多边形的边数无限增加时,$r_n$ 的极限是圆的半径 $R$,$p_n$ 的极限是圆周长 $2\uppi R$,$S_n$ 的极限是圆面积,求证圆面积等于 $\uppi R^2$。
    \end{tasks}
    \begin{figurehere}
      \begin{minipage}[b]{0.48\linewidth}\centering
        \includegraphics{ex1-10.pdf}
        \caption*{(第 \ref{ex:1-10} 题图)}
      \end{minipage}\hfill
      \begin{minipage}[b]{0.48\linewidth}\centering
        \includegraphics{ex1-11.pdf}
        \caption*{(第 \ref{ex:1-11} 题图)}
      \end{minipage}
    \end{figurehere}
    \item\label{ex:1-11} 如图,三角形的一条底边是 $a$,这条边上的高是 $h$。
    \begin{tasks}
      \task 过高的 5 等分点分别作底边的平行线,并作出相应的 4 个矩形,求这些矩形面积的和。
      \task 把高 $n$ 等分,同样作出 $n-1$ 个矩形,求这些矩形面积的和。
      \task 求证:当 $n$ 无限增大时,这些矩形面积的和的极限等于三角形的面积 $\dfrac{ah}{2}$。
    \end{tasks}
    \item 求下列无穷等比数列各项的和:
    \begin{tasks}[before-skip=7pt,after-skip=7pt](2)
      \task $\dfrac{8}{9},-\dfrac{2}{3},\dfrac{1}{2},-\dfrac{3}{8},\dots$;
      \task $6\dfrac{2}{3},1\dfrac{1}{3},\dfrac{4}{15},\dfrac{4}{75},\dots$;
      \task $\dfrac{\sqrt{3}+1}{\sqrt{3}-1},1,\dfrac{\sqrt{3}-1}{\sqrt{3}+1},\dots$;
      \task $1,-x,x^2,-x^3,\dots$,($|x|<1$)。
    \end{tasks}
    \item\label{ex:1-13} 如图,等边三角形 $ABC$ 的面积等于 1,连结这个三角形各边的中点得到一个小三角形,又连结这个小三角形各边的中点得到一个更小的三角形,如此无限继续下去,求所有这些三角形面积的和。
    \begin{figurehere}
      \begin{minipage}[b]{0.48\linewidth}\centering
        \includegraphics{ex1-13.pdf}
        \caption*{(第 \ref{ex:1-13} 题图)}
      \end{minipage}\hfill
      \begin{minipage}[b]{0.48\linewidth}\centering
        \includegraphics{ex1-14.pdf}
        \caption*{(第 \ref{ex:1-14} 题图)}
      \end{minipage}
    \end{figurehere}
    \item\label{ex:1-14} 如图,第 1 个半圆的直径是 \qty{3}{cm},第 2 个半圆的直径是 \qty{2}{cm},以后每个半圆的直径都是前一个的 $\dfrac{2}{3}$,这样无限继续下去,求整条曲线的长。
    \item 将下列循环小数化成分数:
    \begin{tasks}(4)
      \task $0.\dot{4}$;
      \task $0.\dot{1}3\dot{5}$;
      \task $0.4\dot{3}\dot{6}$;
      \task $2.13\dot{8}$。
    \end{tasks}
  \end{question}
\end{Exercise}

\subsection{函数的极限}\label{subsec:function_limits}
前面我们研究了数列的极限的概念和极限的运算法则。从本节起,我们将讨论函数的极限的概念和运算法则。
\subsubsection{当 $x\to\infty$ 时函数的极限}
我们考察函数 $y=\dfrac{1}{x}$ 当 $x$ 无限增大时的变化趋势。为此,我们列出\cref{tab:1-3},并作出函数 $y=\dfrac{1}{x}$ 的图象(\cref{fig:1-4})。
\par\medskip\noindent
\begin{minipage}{0.6\linewidth}
\begin{tablehere}
  \caption{$y=\frac{1}{x}$ 当 $x$ 无限增大时的变化趋势}\label{tab:1-3}
  \begin{tblr}{colspec={*8{c}},vline{2}=0.8pt}
    $x$ & 1 & 10  & 100  & 1000  & \num{10000}  & \num{100000}  & $\cdots$\\
    $y$ & 1 & 0.1 & 0.01 & 0.001 & \num{0.0001} & \num{0.00001} & $\cdots$\\
  \end{tblr}
\end{tablehere}

\parindent2em
从函数 $y=\dfrac{1}{x}$ 的图象可以看出,当自变量 $x$ 取正值并无限增大时,函数 $y=\dfrac{1}{x}$ 的值无限趋近于零。
这里的“无限趋近于零”,就是表示函数值 $y$ 与 0 之差的绝对值 $|y-0|$ 可以变得任意小。
\end{minipage}\hfill
\begin{minipage}{0.35\linewidth}\centering
  \begin{figurehere}
    \includegraphics{1-4.pdf}
    \caption{}\label{fig:1-4}
  \end{figurehere}
\end{minipage}

\medskip
例如,当 $x>1000$ 时,
\[|y-0|<\num{0.001}\]

当 $x>\num{100000}$ 时,
\[|y-0|<\num{0.00001}\]

一般地,对于预先指定的任意小的正数 $\varepsilon$,当 $x>\dfrac{1}{\varepsilon}$ 时,
\[|y-0|=\frac{1}{x}<\varepsilon.\]

总之,当 $x$ 取正值并无限增大时,函数 $y=\dfrac{1}{x}$ 的值无限趋近于 0。于是我们说,当 $x$ 趋向于正无穷大时,函数 $y=\dfrac{1}{x}$ 的极限是 0 ,记作
\[\lim\limits_{x\to+\infty}\frac{1}{x}=0.\]

同样地,当 $x$ 取负值并且它的绝对值无限增大时,函数 $y=\dfrac{1}{x}$ 的值也无限趋近于 0。于是我们说,当 $x$ 趋向于负无穷大时,函数 $y=\dfrac{1}{x}$ 的极限是 0,记作。
\[\lim\limits_{x\to-\infty}\frac{1}{x}=0.\]

一般地,当自变量 $x$ 的绝对值无限增大时,如果函数 $f(x)$ 无限趋近于一个常数 $A$,就说\emph{当 $x$ 趋向于无穷大时,函数 $f(x)$ 的极限是 $A$},记作
\[\lim\limits_{x\to\infty}f(x)=A,\]
也可记作
\[ \text{当}\ x\to\infty\text{ 时,} f(x)\to A.\]

\subsubsection{当 $x\to x_0$ 时函数的极限}
我们来研究函数 $y=x^2$ 当 $x$ 无限趋近于 2 时的变化趋势。

先列出\cref{tab:1-4},并作出函数 $y=x^2$ 的图象(\cref{fig:1-5})。
\begin{table}
  \caption{$y=x^2$ 当 $x$ 无限趋近于 2 时的变化趋势}\label{tab:1-4}
  \begin{tblr}{colspec={*7{X[c]}},vline{2}=0.8pt,hline{4}=1.5pt}
    $x$     & \num{1.5} & \num{1.9} & \num{1.99} & \num{1.999} & \num{1.9999} & $\cdots$ \\
    $y$     & \num{2.25} & \num{3.61} & \num{3.96} & \num{3.996} & \num{3.9996} & $\cdots$ \\
    $|y-4|$ & \num{1.75} & \num{0.39} & \num{0.04} & \num{0.004} & \num{0.0004} & $\cdots$ \\
    $x$     & \num{2.5} & \num{2.1} & \num{2.01} & \num{2.001} & \num{2.0001} & $\cdots$ \\
    $y$     & \num{6.25} & \num{4.41} & \num{4.04} & \num{4.004} & \num{4.0004} & $\cdots$ \\
    $|y-4|$ & \num{2.25} & \num{0.41} & \num{0.04} & \num{0.004} & \num{0.0004} & $\cdots$ \\
  \end{tblr}
\end{table}

\noindent
\begin{minipage}{0.5\linewidth}\parindent2em
我们看到,当自变量 $x$ 越接近 2 时,函数 $y=x^2$ 的值越接近 4;当 $x$ 无限趋近于 2(但不等于 2)时,$y$ 的值无限趋近于 4。
于是我们说,当 $x$ 无限趋近于 2 时,函数 $y=x^2$ 的极限是 4,记作
\[\lim\limits_{x\to2}x^2=4.\]
\end{minipage}\hfill
\begin{minipage}{0.45\linewidth}\centering
  \begin{figurehere}
    \includegraphics{1-5.pdf}
    \caption{}\label{fig:1-5}
  \end{figurehere}
\end{minipage}

\medskip
我们再来研究函数 $y=\dfrac{x^2-1}{x-1}$ 当 $x$ 无限趋近于 1(但不等 1)时的变化趋势。

\medskip\noindent
\begin{minipage}{0.5\linewidth}\parindent2em
如\cref{fig:1-6},函数的图象是直线 $y=x+1$ 上除去点 $(1,2)$ 以外的部分。
从图象上看到,当 $x$ 接近于 1 时,函数 $y=\dfrac{x^2-1}{x-1}$ 的值趋近于 2。

这时,我们说,当 $x$ 无限趋近于 1(但不等于 1)时,函数
\[y=\frac{x^2-1}{x-1}\]
的极限是 2 ,记作
\end{minipage}\hfill
\begin{minipage}{0.45\linewidth}\centering
  \begin{figurehere}
    \includegraphics{1-6.pdf}
    \caption{}\label{fig:1-6}
  \end{figurehere}
\end{minipage}
\[\lim\limits_{x\to1}\frac{x^2-1}{x-1}=2.\]

一般地,当自变量 $x$ 无限趋近于常数 $x_0$(但 $x$ 不等于 $x_0$) 时,如果函数 $y=f(x)$ 无限趋近于一个常数 $A$,就说当 $x$ 趋近于 $x_0$ 时,函数 $f(x)$ 的极限是 $A$,记作
\[ \lim\limits_{x\to x_0}f(x)=A,\]
或者
\[\text{当}\ x\to x_0\text{ 时,} f(x)\to A.\]

从这个定义可以得出:

\emph{当 $x$ 趋于任何数 $x_0$ 时,常数函数的极限就是这个常数},即
\[ \lim\limits_{x\to x_0}C=C.\]

\emph{当 $x\to x_0$ 时,$f(x)=x$ 的极限是 $x_0$},即
\[ \lim\limits_{x\to x_0}x=x_0.\]

\subsubsection{函数的左极限和右极限}
首先,我们介绍在微积分中常常用到的函数 $y=[x]$,符号 $[x]$ 表示不超过数 $x$ 的整数部分,例如
\[[0]=0,\quad\left[\frac{10}{3}\right]=[3.33\cdots]=3,\quad [-2.5]=-3.\]

函数 $y=[x]$ 的图象如\cref{fig:1-7}。

\medskip\noindent
\begin{minipage}{0.55\linewidth}\parindent2em
现在,我们再来研究函数 $y=[x]$ 在点 $x=1$ 处的极限。

如\cref{fig:1-7},当 $x$ 从点 $x = 1$ 的左侧趋近于 1 时,函数 $y$ 趋近于 0;当 $x$ 从点 $x = 1$ 的右侧趋近于 1 时,函数 $y$ 趋近于 1。因此,当 $x$ 从点 $x = 1$ 的左侧和右侧分别趋近于 1 时,函数 $y$ 所趋近的值不同。根据函数在一点处的极限的定义,函数 $y$ 的极限不存在。
\end{minipage}\hfill
\begin{minipage}{0.4\linewidth}\centering
  \begin{figurehere}
    \includegraphics{1-7.pdf}
    \caption{}\label{fig:1-7}
  \end{figurehere}
\end{minipage}

\medskip
从这个例子我们看到,虽然函数 $y=[x]$ 在点 $x=1$ 处没有极限,但是当 $x$ 从点 $x=1$ 的一侧趋近于 1 时,函数 $y$ 还是趋近于确定的常数。由此我们引出单侧极限的定义。

如果当 $x$ 从点 $x=x_0$ 左侧(即 $x<x_0$)无限趋近于 $x_0$ 时,函数 $f(x)$ 无限趋近于常数 $A$,就说 $A$ 是函数 $f(x)$ 在点 $x_0$ 处的\Concept{左极限},记作
\[\lim\limits_{x\to x_0^-}f(x)=A.\]

同样,如果当 $x$ 从点 $x=x_0$ 右侧(即 $x>x_0$)无限趋近于 $x_0$ 时,函数 $f(x)$ 无限趋近于常数 $A$,就说 $A$ 是函数 $f(x)$ 在点 $x_0$ 处的\Concept{右极限},记作
\[\lim\limits_{x\to x_0^+}f(x)=A.\]

根据极限、左极限和右极限的定义,可以得出表示它们之间的关系的一条定理(证明从略):
\begin{Theorem}{定理}
  $\lim\limits_{x\to x_0}f(x)=A$ 的充要条件是
  \[ \lim\limits_{x\to x_0^+}f(x)=\lim\limits_{x\to x_0^-}f(x)=A. \]
\end{Theorem}
\begin{Practice}
  \begin{question}
    \item 给定函数 $y=\dfrac{1}{x^2+1}$,填写下表并画出函数的图像,观察函数 $y$ 当 $x\to\infty$ 时的变化趋势:
    \begin{tablehere}
      \begin{minipage}{\linewidth}
        \begin{tblr}{colspec={*7{X[c]}},vline{2}=0.8pt}
          $x$ & 0 & $\pm2$ & $\pm10$ & $\pm10^2$ & $\pm10^3$ & $\cdots$ \\
          $y$ &&&&&&\\
          $|y-0|$ &&&&&&\\
        \end{tblr}
      \end{minipage}
    \end{tablehere}
    \item 根据函数极限的定义和函数的图象,说出下列极限:
    \begin{tasks}[before-skip=10pt,after-skip=10pt](3)
      \task $\lim\limits_{x\to+\infty}\dfrac{1}{x^3}$;
      \task $\lim\limits_{x\to-\infty}\dfrac{1}{x^3}$;
      \task $\lim\limits_{x\to \infty}\dfrac{1}{x^3}$。
    \end{tasks}
    \item 对于函数 $y=2x+1$ 填写下表,并作出函数的图象,观察当 $x\to1$ 时函数 $y=2x+1$ 的变化趋势:
    \begin{tablehere}
      \begin{minipage}{\linewidth}
        \begin{tblr}{colspec={*7{X[c]}},vline{2}=0.8pt,hline{4}=1.5pt}
          $x$ & 0.5 & 0.9 & 0.99 & \num{0.999} & \num{0.9999} & \num{0.99999} \\
          $y$ &&&&&&\\
          $|y-3|$ &&&&&&\\
          $x$ & 1.5 & 1.1 & 1.01 & \num{1.001} & \num{1.0001} & \num{1.00001} \\
          $y$ &&&&&&\\
          $|y-3|$ &&&&&&\\
        \end{tblr}
      \end{minipage}
    \end{tablehere}
    \begin{tasks}
      \task 当 $|x-1|<0.01$ 时,$|y-3|$ 小于什么数?当 $|x-1|<\num{0.00001}$ 时,$|y-3|$ 小于什么数?
      \task 说出当 $x\to1$ 时函数 $y$ 的极限。
    \end{tasks}
    \item 根据函数极限的定义和函数的图象,说出下列函数的极限(其中 $C$ 是常数):
    \begin{tasks}[before-skip=10pt,after-skip=10pt,after-item-skip=7pt](2)
      \task $\lim\limits_{x\to\infty}C$;
      \task $\lim\limits_{x\to3}C$;
      \task $\lim\limits_{x\to\infty}\dfrac{1}{x^2}$;
      \task $\lim\limits_{x\to2}\dfrac{1}{x^2}$。
    \end{tasks}
    \item 说出下列各图中表示的函数在点 $x=a$ 的左极限、右极限和极限(如果存在的话):
    \begin{figurehere}
      \begin{minipage}{0.24\linewidth}\centering
        \includegraphics{pr1-3-5a.pdf}
        \subcaption{}
      \end{minipage}
      \begin{minipage}{0.24\linewidth}\centering
        \includegraphics{pr1-3-5b.pdf}
        \subcaption{}
      \end{minipage}
      \begin{minipage}{0.24\linewidth}\centering
        \includegraphics{pr1-3-5c.pdf}
        \subcaption{}
      \end{minipage}
      \begin{minipage}{0.25\linewidth}\centering
        \includegraphics{pr1-3-5d.pdf}
        \subcaption{}
      \end{minipage}
    \end{figurehere}
  \end{question}
\end{Practice}
\subsection{函数极限的四则运算法则}
函数极限的四则运算与数列极限的四则运算有类似的定理。
\begin{Theorem}{定理}
  如果 $\lim\limits_{x\to x_0}f(x)=A$,$\lim\limits_{x\to x_0}g(x)=B$,那么,
  \begin{align*}
    \lim\limits_{x\to x_0}[f(x)\pm g(x)]&=A\pm B;\\
    \lim\limits_{x\to x_0}[f(x)\cdot g(x)]&=A\cdot B;\\
    \lim\limits_{x\to x_0}\frac{f(x)}{g(x)}&=\frac{A}{B}\ (B\neq 0).
  \end{align*}
\end{Theorem}
这个定理对于 $x\to\infty$ 的情况仍然成立。

由上面第二个式子可以推出: 当 $C$ 是常数、$n$ 是正整数时,
\begin{align*}
  \lim\limits_{x\to x_0}[Cf(x)]&=C\lim\limits_{x\to x_0}f(x); \\
  \lim\limits_{x\to x_0}[f(x)]^n&=[\lim\limits_{x\to x_0}f(x)]^n. 
\end{align*}

利用函数极限的运算法则,我们可以根据已知的几个函数的极限,求出较复杂的函数的极限。

\begin{example}
  求 $\lim\limits_{x\to2}(x^2+3x)$。
\end{example}
\begin{solution}
  \begin{align*}
    \lim\limits_{x\to2}(x^2+3x)&=\lim\limits_{x\to2}x^2+\lim\limits_{x\to2}3x \\
    &= (\lim\limits_{x\to2}x)^2+3\lim\limits_{x\to2}x\\
    &=2^2+3\times2=10.
  \end{align*}
\end{solution}

\begin{example}
  求 $\lim\limits_{x\to 1}\dfrac{2x^3-x^2+1}{x+1}$。
\end{example}
\begin{solution}
  \begin{align*}
    \lim\limits_{x\to 1}\frac{2x^3-x^2+1}{x+1}&=\frac{\lim\limits_{x\to 1}(2x^3-x^2+1)}{\lim\limits_{x\to 1}(x+1)} \\
    &= \frac{\lim\limits_{x\to 1}2x^3-\lim\limits_{x\to 1}x^2 +\lim\limits_{x\to 1}1}{\lim\limits_{x\to 1}x+\lim\limits_{x\to 1}1}\\
    &= \frac{2\left(\lim\limits_{x\to 1}x\right)^3-\left(\lim\limits_{x\to 1}x\right)^2+\lim\limits_{x\to 1}1}{\lim\limits_{x\to 1} x+\lim\limits_{x\to 1} 1}\\
    &=\frac{2\times1^3-1^2+1}{1+1}\\
    &=1.
  \end{align*}
\end{solution}

\begin{example}
  求 $\lim\limits_{x\to4}\dfrac{x^2-16}{x-4}$。
\end{example}
\begin{analyze}
  当 $x\to4$ 时,分母的极限是 0,不能直接运用上面的极限运算法则。因为当 $x\to4$ 时函数的极限,只与 $x$ 无限趋近于 4 时的函数值有关,与 $x=4$ 时的函数值无关,因此可以先将分子和分母约去公因式 $x-4$ 以后再求函数的极限。
\end{analyze}

\begin{solution}
  \begin{align*}
    \lim\limits_{x\to4}\frac{x^2-16}{x-4}&=\lim\limits_{x\to4}\frac{(x+4)(x-4)}{x-4} \\
    &=\lim\limits_{x\to4}(x+4)=\lim\limits_{x\to4}x+\lim\limits_{x\to4}4 \\
    &=4+4=8.
  \end{align*}
\end{solution}

\begin{example}
  求 $\lim\limits_{x\to\infty}\dfrac{3x^2-x+2}{x^2+1}$。
\end{example}
\begin{analyze}
  当 $x\to\infty$ 时,分子、分母没有极限,不能直接运用上面的商的极限运算法则。为此,先将分子、分母同时除以 $x^2$ 后,再求它的极限。
\end{analyze}

\begin{solution}
  \begin{align*}
    \lim\limits_{x\to\infty}\frac{3x^2-x+2}{x^2+1}&=\lim\limits_{x\to\infty}\frac{\dfrac{3x^2}{x^2}-\dfrac{x}{x^2}+\dfrac{2}{x^2}}{\dfrac{x^2}{x^2}+\dfrac{1}{x^2}}\\
    &=\frac{\lim\limits_{x\to\infty}\left(3-\dfrac{1}{x}+\dfrac{2}{x^2}\right)}{\lim\limits_{x\to\infty}\left(1+\dfrac{1}{x^2}\right)}\\
    &=\frac{\lim\limits_{x\to\infty}3-\lim\limits_{x\to\infty}\dfrac{1}{x}+\lim\limits_{x\to\infty}\dfrac{2}{x^2}}{\lim\limits_{x\to\infty}1+\lim\limits_{x\to\infty}\dfrac{1}{x^2}}\\
    &=\frac{\lim\limits_{x\to\infty}3-\lim\limits_{x\to\infty}\dfrac{1}{x}+2\left(\lim\limits_{x\to\infty}\dfrac{1}{x}\right)^2}{\lim\limits_{x\to\infty}1+\left(\lim\limits_{x\to\infty}\dfrac{1}{x}\right)^2}\\
    &=\frac{3-0+0}{1+0}=3.
  \end{align*}
\end{solution}

从以上各例求极限的过程可以看出,在求有理函数的极限时,最后总是归结为求下列极限:
\begin{align*}
  \lim\limits_{x\to x_0}C&=C,   & \lim\limits_{x\to x_0}x^k&=x_0^k \quad \text{( } k \text{ 是正整数)。}\\
  \lim\limits_{x\to\infty}C&=C, & \lim\limits_{x\to\infty}\frac{1}{x^k}&=0 \quad \text{( } k \text{ 是正整数)。}
\end{align*}
这些极限可以由极限的定义和运算法则推出,以后可以直接运用这些结果。

\begin{example}
  求 $\lim\limits_{x\to\infty}\dfrac{2x^2+x-4}{3x^3-x^2+1}$。
\end{example}
\begin{solution}
  \begin{align*}
    \lim\limits_{x\to\infty}\frac{2x^2+x-4}{3x^3-x^2+1}&=\lim\limits_{x\to\infty}\frac{\dfrac{2x^2}{x^3}+\dfrac{x}{x^3}-\dfrac{4}{x^3}}{\dfrac{3x^3}{x^3}-\dfrac{x^2}{x^3}+\dfrac{1}{x^3}}\\
    &=\frac{\lim\limits_{x\to\infty}\left(\dfrac{2}{x}+\dfrac{1}{x^2}-\dfrac{4}{x^3}\right)}{\lim\limits_{x\to\infty}\left(3-\dfrac{1}{x}+\dfrac{1}{x^3}\right)}\\
    &=\frac{0+0-0}{3-0+0}=0.
  \end{align*}
\end{solution}

\begin{Practice}
  \begin{question}
    \item 利用函数极限的运算法则求下列极限:
    \begin{tasks}[before-skip=5pt,after-skip=5pt](2)
      \task $\lim\limits_{x\to\frac12}(2x-3)$;
      \task $\lim\limits_{x\to2}(2x^2-3x+1)$;
      \task $\lim\limits_{x\to4}(2x-1)(x+3)$;
      \task $\lim\limits_{x\to1}\dfrac{2x^2+1}{3x^2+4x-1}$。
    \end{tasks}
    \item 求下列极限:
    \begin{tasks}[before-skip=5pt,after-skip=5pt,after-item-skip=7pt](2)
      \task $\lim\limits_{x\to3}\dfrac{x^2-5x+6}{x^2-9}$;
      \task $\lim\limits_{x\to-1}\dfrac{x^3+1}{x+1}$;
      \task $\lim\limits_{x\to\infty}\dfrac{2x^2+x-2}{3x^2-3x+1}$;
      \task $\lim\limits_{y\to\infty}\dfrac{2y^2-y}{y^3-5}$。
    \end{tasks}
  \end{question}
\end{Practice}

\begin{Exercise}
  \begin{question}
    \item 根据函数极限的定义,说出下列极限:
    \begin{tasks}[before-skip=5pt,after-skip=5pt](2)
      \task $\lim\limits_{x\to+\infty}\left(\dfrac12\right)^x$;
      \task $\lim\limits_{x\to\infty}\dfrac{2}{x^2+1}$。
    \end{tasks}
    \item 根据函数极限的定义,说出下列极限:
    \begin{tasks}[before-skip=5pt,after-skip=5pt,after-item-skip=5pt](2)
      \task $\lim\limits_{x\to2}3x$;
      \task $\lim\limits_{x\to a}(2x-1)$;
      \task $\lim\limits_{x\to-1}\dfrac{1}{x^2}$;
      \task $\lim\limits_{x\to1}\dfrac{x^3-x}{x-1}$。
    \end{tasks}
    \item 求下列极限:
    \begin{tasks}[before-skip=5pt,after-skip=5pt,after-item-skip=5pt](2)
      \task $\lim\limits_{x\to1}(2x^3+3x+4)$;
      \task $\lim\limits_{x\to2}\dfrac{x^2+5}{x^2-3}$;
      \task $\lim\limits_{x\to0}\left(\dfrac{x^2-3x+1}{x-4}+1\right)$;
      \task $\lim\limits_{x\to\sqrt{3}}\dfrac{x^2-3}{x^4+x^2+1}$;
      \task $\lim\limits_{x\to2}\dfrac{x-2}{x^3-8}$;
      \task $\lim\limits_{x\to1}\dfrac{2x}{1+x+x^2}$;
      \task $\lim\limits_{x\to1}\dfrac{x^2-2x+1}{x^3-1}$;
      \task $\lim\limits_{x\to0}\dfrac{3x^3+x^2}{x^5+3x^4-2x^2}$;
      \task $\lim\limits_{x\to-2}\dfrac{x^3+3x^2+2x}{x^2-x-6}$;
      \task $\lim\limits_{x\to0}\dfrac{(x+m)^3-m^3}{x}$;
      \task $\lim\limits_{x\to\infty}\left(2-\dfrac{1}{x}+\dfrac{1}{x^2}\right)$;
      \task $\lim\limits_{x\to\infty}\dfrac{x^2+1}{2x^2+2x-1}$;
      \task $\lim\limits_{x\to\infty}\dfrac{x^3+x}{x^4+3x^2+1}$;
      \task $\lim\limits_{x\to\infty}\left(\dfrac{2x^3+1}{3x^3-2}\right)^2$。
    \end{tasks}
    \item 求下列极限:
    \begin{tasks}[before-skip=5pt,after-skip=5pt,after-item-skip=5pt](2)
      \task $\lim\limits_{x\to1}\dfrac{3x^2-11x+6}{2x^2-5x-3}$;
      \task $\lim\limits_{x\to\infty}\dfrac{3x^2-11x+6}{2x^2-5x-3}$;
      \task $\lim\limits_{x\to0}\dfrac{x-x^2-6x^3}{2x-5x^2-3x^3}$;
      \task $\lim\limits_{x\to\infty}\dfrac{x-x^2-6x^3}{2x-5x^2-3x^3}$。
    \end{tasks}
  \end{question}
\end{Exercise}

\subsection{函数的连续性}
我们以前学过的许多函数,例如 $y=x^2$,它的图象是连续(不间断)的曲线。
对于连续曲线 $y=f(x)$ 上的每一点 $x_0$ 来说,当 $x\to x_0$ 时,都有 $f(x)\to f(x_0)$。

如果函数 $y=f(x)$ 在点 $x_0$ 的附近有定义,而且
\[ \lim\limits_{x\to x_0}f(x)=f(x_0), \]
就说\Concept{函数 $f(x)$ 在点 $x_0$ 处连续}。

\begin{example}
  研究函数 $f(x)=x^2$ 在点 $x=2$ 处的连续性。
\end{example}
\begin{solution}
  函数 $f(x)=x^2$ 在点 $x=2$ 的附近有定义,而且
  \begin{gather*}
    \lim\limits_{x\to2}x^2=4,\\
    f(2)=2^2=4,\\
    \therefore \quad \lim\limits_{x\to2}x^2=f(2).
  \end{gather*}
  因此,函数 $f(x)=x^2$ 在点 $x=2$ 处连续。
\end{solution}

从上面的定义可以看出,函数 $f(x)$ 在点 $x = x_0$ 处连续必须具备以下三个条件:
\begin{enumerate}[1.]
  \item 函数 $f(x)$ 在点 $x_0$ 的附近有定义;
  \item $\lim\limits_{x\to x_0}f(x)$ 存在;
  \item $\lim\limits_{x\to x_0}f(x)=f(x_0)$,即函数 $f(x)$ 在点 $x_0$ 处的极限等于 $f(x)$ 在点 $x_0$ 的函数值。
\end{enumerate}

如果函数 $f(x)$ 在点 $x=x_0$ 处对上述三个条件中有一个条件不具备,那么函数在点 $x=x_0$ 处不连续,点 $x=x_0$ 称为该函数的\Concept{间断点}。

例如,函数 $y=\tan x$ 在点 $x=\dfrac{\uppi}{2}$ 处没有定义,所以 $x=\dfrac{\uppi}{2}$ 是函数 $y=\tan x$ 的间断点。

下面我们给出函数 $f(x)$ 在点 $x_0$ 处右连续和左连续的定义。
\begin{Definition}{定义}
  如果函数 $f(x)$ 在点 $x_0$ 附近右侧(或左侧)有定义,而且
  \[ \lim\limits_{x\to x_0+}f(x)=f(x_0)\quad\text{(或者} \lim\limits_{x\to x_0-}f(x)=f(x_0)\text{),}\]
  那么就说\Concept{函数 $f(x)$ 在点 $x$ 处右连续}(或者\Concept{左连续})。
\end{Definition}

\begin{example}
  研究函数 $y=[x]$ 在点 $x=2$ 处的连续性。
\end{example}
\begin{solution}
  函数 $y=[x]$ 在 $x=2$ 的附近有定义(参看\cref{fig:1-7})。

  由于
  \[ \lim\limits_{x\to 2+}[x]=2=[2],\quad \lim\limits_{x\to 2-}[x]=1\neq[2],\]
  所以函数 $y=[x]$ 在点 $x=2$ 处是右连续,但不是左连续。

  由于当 $x\to2$ 时函数 $[x]$ 没有极限,所以函数 $y=[x]$ 在点 $x=2$ 处不连续。
\end{solution}

如果函数 $f(x)$ 在某一区间 $(a,b)$ 内每一点处都连续,就说 $f(x)$ 在\Concept{区间 $(a,b)$ 内连续},或者说 $f(x)$ 是\Concept{区间 $(a,b)$ 内的连续函数}。

如果函数 $f(x)$ 在开区间 $(a,b)$ 内连续,在左端点 $x=a$ 处右连续,在右端点 $x=b$ 处左连续,就说\Concept{函数 $f(x)$ 在闭区间 $[a,b]$ 上连续}。

例如,函数 $y=1-x^2$ 在闭区间 $[-1,1]$ 上连续。而函数 $y=\dfrac{1}{x}$ 在开区间 $(0,1)$ 内连续,在闭区间 $[0,1]$ 上不连续,因为它在点 $x=0$ 不是右连续。

在区间上连续的函数的图象,是一条连续的曲线。利用闭区间上连续函数的图象可以说明它具有如下的性质。

\begin{Theorem}[最大值和最小值定理]{性质 1}
  如果 $f(x)$ 是闭区间 $[a,b]$ 上的连续函数,那么 $f(x)$ 在 $[a.b]$ 上有最大值和最小值。
\end{Theorem}

\noindent
\begin{minipage}{0.5\linewidth}\parindent2em
如\cref{fig:1-8},
\begin{gather*}
  f(x_1)\geqslant f(x),\quad x\in[a.b];\\
  f(x_0)\leqslant f(x),\quad x\in[a.b].\\
\end{gather*}

利用函数极限的运算法则,还可以证明连续函数的和、差、积、商仍然是连续函数。
\end{minipage}\hfill
\begin{minipage}{0.45\linewidth}
\begin{figurehere}\nextfloat
  \includegraphics{1-8.pdf}
  \caption{}\label{fig:1-8}
\end{figurehere}
\end{minipage}

\begin{Theorem}{性质 2}
  如果函数 $f(x)$、$g(x)$ 在某一点 $x=x_0$ 处连续,那么
  \begin{enumerate}
    \item $f(x)\pm g(x)$,
    \item\label{itm:theo2} $f(x)\cdot g(x)$,
    \item\label{itm:theo3} $\dfrac{f(x)}{g(x)}\quad(g(x)\neq 0)$
  \end{enumerate}
  在点 $x=x_0$ 处都连续。
\end{Theorem}
\begin{proof}
  因为函数 $f(x)$、$g(x)$ 在点 $x=x_0$ 处连续,所以,
  \[ \lim\limits_{x\to x_0}f(x)=f(x_0),\quad \lim\limits_{x\to x_0}g(x)=g(x_0).\]

  由极限的运算法则,得出
  \[
    \lim\limits_{x\to x_0}[f(x)\pm g(x)]=\lim\limits_{x\to x_0}f(x)\pm \lim\limits_{x\to x_0}g(x)=f(x_0)\pm g(x_0).
  \]

  因此,函数 $f(x)\pm g(x)$ 在点 $x=x_0$ 处连续。

  同样可以证明 \ref{itm:theo2} 和 \ref{itm:theo3}。
\end{proof}

\begin{Practice}
  \begin{question}
    \item\label{prac:1-5-1} 连续函数的图像有什么特点?观察下列各函数图像,说出函数在 $x=a$ 处是否连续:
    \begin{figurehere}
      \begin{minipage}{\linewidth}
      \begin{minipage}{0.33\linewidth}\centering
        \includegraphics{pr1-5-1a.pdf}
        \subcaption{}
      \end{minipage}\hfill
      \begin{minipage}{0.33\linewidth}\centering
        \includegraphics{pr1-5-1b.pdf}
        \subcaption{}
      \end{minipage}\hfill
      \begin{minipage}{0.33\linewidth}\centering
        \includegraphics{pr1-5-1c.pdf}
        \subcaption{}
      \end{minipage}
      \begin{minipage}{0.33\linewidth}\centering
        \includegraphics{pr1-5-1d.pdf}
        \subcaption{}
      \end{minipage}\hfill
      \begin{minipage}{0.33\linewidth}\centering
        \includegraphics{pr1-5-1e.pdf}
        \subcaption{}
      \end{minipage}\hfill
      \begin{minipage}{0.33\linewidth}\centering
        \includegraphics{pr1-5-1f.pdf}
        \subcaption{}
      \end{minipage}
      \begin{minipage}{0.33\linewidth}\centering
        \includegraphics{pr1-5-1g.pdf}
        \subcaption{}
      \end{minipage}\hfill
      \begin{minipage}{0.33\linewidth}\centering
        \includegraphics{pr1-5-1h.pdf}
        \subcaption{}
      \end{minipage}\hfill
      \begin{minipage}{0.33\linewidth}\centering
        \includegraphics{pr1-5-1i.pdf}
        \subcaption{}
      \end{minipage}
      \caption*{(第 \ref{prac:1-5-1} 题图)}
    \end{minipage}
    \end{figurehere}
    \item 结合下列函数的图像,说明函数在给定点或区间上是否连续:
    \begin{tasks}[before-skip=5pt,after-skip=5pt](2)
      \task $f(x)=\dfrac{1}{x^2},\ x=0$;
      \task $f(x)=|x|,\ z=0$;
      \task $f(x)=\dfrac{x^2-1}{x-1},\ (0,1)$;
      \task $f(x)=ax^2+bx+c,\ (-\infty,\infty)$。
    \end{tasks}
  \end{question}
\end{Practice}

我们学过的函数可以分为以下五类:
\begin{description}
  \item [幂函数]     $y=x^\alpha$($\alpha$ 是实数);
  \item [指数函数]   $y=a^x$($a>0$ 且 $a\neq 1$);
  \item [对数函数]   $y=\log_a x$($a>0$ 且 $a\neq 1$);
  \item [三角函数]   $y=\sin x$,$y=\cos x$,$y=\tan x$,$y=\cot x$,等等;
  \item [反三角函数] $y=\arcsin x$,$y=\arccos x$,$y=\arctan x$,$y=\arccot x$,等等。
\end{description}
这五种函数统称\Concept{基本初等函数}。

关于基本初等函数的连续性有如下结论:

\emph{基本初等函数在其定义区间上是连续函数}。

例如,$y=x^{\frac{1}{2}}$ 在 $[0,+\infty)$ 上连续,$y=\sin x$ 在 $(-\infty,+\infty)$ 上连续,$y=\tan x$ 在 $\left(n\uppi- \dfrac{\uppi}{2},n\uppi+\dfrac{\uppi}{2}\right)$ 上连续。

\bigskip
我们以前学过的许多函数是由基本初等函数和常数经过有限次四则运算得出的。例如,二次函数
\[ y=ax^2+bx+c\]
可以看作是由常数 $a$ 乘以幂函数 $x^2$ 的积,加上常数 $b$ 乘以幂函数 $x$ 的积,再加上常数 $c$ 而得到的一个函数。

另外还有一些函数,例如 $y=\sin2x$,它和 $y=\sin x$ 不同,不是基本初等函数。下面我们将会看到,它是由三角函数 $y=\sin u$ 和一次函数 $u=2x$ 经过 “复合”而成的。

同样,函数
\[y=\sqrt{1+x^2}\]
是由幂函数 $y=u^{\frac{1}{2}}$ 和二次函数 $u=1+x^2$ 经过 “复合”而成的。

一般地说,如果 $y$ 是 $u$ 的函数,而 $u$ 又是 $x$ 的函数,即 $y=f(u)$,$u=g(x)$,那么 $y$ 关于 $x$ 的函数
\[y=f[g(x)]\]
叫做函数 $f$ 和 $g$ 的\Concept{复合函数},$u$ 叫做\Concept{中间变量}。

\begin{example}\mbox{}
  \begin{enumerate}
    \item 函数 $y=f(u)=\lg u$ 与 $u=g(x)=\sin x$ 经过 “复合”以后得到什么函数?
    \item 函数 $y=f(u)=\sin u$ 与 $u=g(x)=\lg x$ 呢?
  \end{enumerate}
\end{example}
\begin{solution}
  \begin{enumerate}
    \item $\lg\sin x$;
    \item $\sin(\lg x)$。
  \end{enumerate}
\end{solution}

复合函数也可以是由三个或三个以上的函数复合而成的。

\begin{example}
说出下列函数是由哪几个简单的函数复合而成的:
\begin{enumerate}
  \item $y=a\sin(bt+c)$;
  \item $y=\log_a(1+x)^2$。
\end{enumerate}
\end{example}
\begin{solution}
\begin{enumerate}
  \item $y=a\sin(bt+c)$ 可以看成是 $y=a\sin u$ 和 $u=bt+c$ 两个函数复合而成的。
  \item $y=\log_a(1+x)^2$ 可以看成是由 $y=\log_a u$,$u=v^2$,$v=1+x$ 三个函数复合而成的。
\end{enumerate}
\end{solution}

一般地,关于复合函数连续性有如下结论:

\emph{如果函数 $u=g(x)$ 在点 $x=x_0$ 处连续,$g(x_0)=u_0$,且函数 $y=f(u)$ 在点 $u=u_0$ 处连续,那么复合函数 $y=f[g(x)]$ 在点 $x_0$ 处连续}。

由基本初等函数和常数经过有限次四则运算和有限次函数的复合而得出的函数,统称\Concept{初等函数}。

由基本初等函数的连续性、连续函数的性质 2 和复合函数的连续性可以得出如下结论:

\emph{一切初等函数在它们的定义区间上是连续函数}。

根据这条结论,函数
\begin{align*}
  y&=\frac{ax+b}{cx^2+d},& y&=a\sin(\omega t+\varphi), && \\
  y&=\log_a(1+x)^2,&y&=\sqrt{a^2-b^2\cos^2x},&y&=\frac{\upe^x-\upe^{-x}}{2}
\end{align*}
都是在其定义区间上的连续函数。

这个结论不仅给我们提供了判断一个函数是不是连续函数的根据,而且为我们提供了计算初等函数的极限问题的一种方法。
这种方法是:\emph{如果函数 $f(x)$ 是初等函数,而且点 $x_0$ 是函数定义区间内的一点,那么求 $x\to x_0$ 时函数 $f(x)$ 的极限,只要求出 $f(x)$ 在点 $x_0$ 处的函数值 $f(x_0)$ 就可以了}。

\begin{example}
  利用初等函数的连续性,求下列极限:
  \begin{tasks}[label=(\arabic*),label-width=16pt,before-skip=7pt,after-skip=7pt,after-item-skip=7pt](2)
    \task $\lim\limits_{x\to1}\sqrt{3+x^2}$;
    \task $\lim\limits_{x\to0}\dfrac{1-\upe^x}{1+\upe^x}$;
    \task $\lim\limits_{x\to\frac\uppi2}\ln\sin x$;
    \task $\lim\limits_{x\to a}\sqrt{1+\arctan^2\dfrac{x}{a}}$($a\neq 0$)。
  \end{tasks}
\end{example}
\begin{solution}
  \begin{enumerate}[itemsep=7pt]
    \item $\lim\limits_{x\to1}\sqrt{3+x^2}=\sqrt{3+1^2}=2$;
    \item $\lim\limits_{x\to0}\dfrac{1-\upe^x}{1+\upe^x}=\dfrac{1-\upe^0}{1+\upe^0}=\dfrac{1-1}{1+1}=0$;
    \item $\lim\limits_{x\to\frac\uppi2}\ln\sin x=\ln\sin\dfrac\uppi2=\ln1=0$;
    \item $\lim\limits_{x\to a}\sqrt{1+\arctan^2\dfrac{x}{a}}=\sqrt{1+\arctan^2\dfrac{a}{a}}=\sqrt{1+\left(\dfrac{\uppi}{4}\right)^2}=\dfrac{1}{4}\sqrt{16+\uppi^2}.$
  \end{enumerate}
\end{solution}

\begin{example}
  求下列极限:
  \begin{tasks}[label=(\arabic*),label-width=16pt,before-skip=7pt,after-skip=7pt](2)
    \task $\lim\limits_{x\to1}\dfrac{(x-1)\sqrt{2-x}}{x^4-1}$;
    \task $\lim\limits_{x\to0}\dfrac{\sqrt{1+x}-1}{x}$。
  \end{tasks}
\end{example}
\begin{solution}
  \begin{enumerate}[itemsep=7pt]
    \item $\lim\limits_{x\to1}\dfrac{(x-1)\sqrt{2-x}}{x^4-1}=\lim\limits_{x\to1}\dfrac{\sqrt{2-x}}{x^3+x^2+x+1}=\dfrac{\sqrt{2-1}}{1^3+1^2+1+1}=\dfrac{1}{4}$。
    \item $\lim\limits_{x\to0}\dfrac{\sqrt{1+x}-1}{x}=\lim\limits_{x\to0}\dfrac{(\sqrt{1+x}-1)(\sqrt{1+x}+1)}{x(\sqrt{1+x}+1)}=\lim\limits_{x\to0}\dfrac{x}{x(\sqrt{1+x}+1)}$
    \[{}=\lim\limits_{x\to0}\frac{1}{\sqrt{1+x}+1}=\frac12.\]
  \end{enumerate}
\end{solution}

\begin{Practice}
  \begin{question}
    \item 下列函数是由基本初等函数经过那些运算得出的函数?
    \begin{tasks}[before-skip=5pt,after-skip=5pt](2)
      \task $y=ax+b$;
      \task $y=\dfrac{1+x^2}{ax}$;
      \task $y=5a^x\sin x$;
      \task $y=(1+\ln x)\sin x$。
    \end{tasks}
    \item 写出下列函数经过复合而成的函数的表示式:
    \begin{tasks}[before-skip=5pt,after-skip=5pt](2)
      \task $y=\upe^u,\ u=x^2$;
      \task $y=u^2,\ u=\upe^x$;
      \task $y=\sqrt{u},\ u=1+\sin^2x$;
      \task $y=au^2,\ u=\sin v,\ v=bx+c$。
    \end{tasks}
    \item 说出下列函数是由哪几个简单的函数复合而成的:
    \begin{tasks}[before-skip=5pt,after-skip=5pt](2)
      \task $y=\sqrt{1+a^x}$;
      \task $y=\ln ax$;
      \task $y=\dfrac{1}{\cos(1+x^2)}$;
      \task $y=\sqrt{a^2-b^2\cos^2x}$。
    \end{tasks}
    \item 求下列极限:
    \begin{tasks}[before-skip=5pt,after-skip=7pt,after-item-skip=7pt](2)
      \task $\lim\limits_{x\to0}\lg\cos3x$;
      \task $\lim\limits_{x\to\frac\uppi2}\dfrac{\sin2x}{\cos x}$;
      \task $\lim\limits_{x\to0}\dfrac{\sqrt{3+x^2}-\sqrt{x^2-1}}{x+1}$;
      \task $\lim\limits_{x\to0}\dfrac{(1+a^x)(1-\cos x)}{\sin^2\dfrac{x}{2}}$;
    \end{tasks}
  \end{question}
\end{Practice}

\subsection{两个重要的极限}
现在我们来讨论微积分中常常用到的两个重要的极限。
\subsubsection{$\lim\limits_{x\to 0}\dfrac{\sin x}{x}=1$}
为了证明这个极限,我们先介绍一个有关极限的定理(证明略)。
\begin{Theorem}[夹逼定理]{定理}
  如果函数 $f(x)$,$g(x)$,$h(x)$ 在点 $x_0$ 的附近满足:
  \begin{enumerate}
    \item $g(x)\leqslant f(x)\leqslant h(x)$,
    \item $\lim\limits_{x\to x_0}g(x)=A$,$\lim\limits_{x\to x_0}h(x)=A$($A$ 是常数),那么有
  \end{enumerate}
  \[ \lim\limits_{x\to x_0}f(x)=A.\]
\end{Theorem}

现在我们来讨论 $\lim\limits_{x\to 0}\dfrac{\sin x}{x}$。
为此,我们先列出当 $x$ 接近于 0 时函数 $\dfrac{\sin x}{x}$ 的值如\cref{tab:1-5},并作出函数的图象(\cref{fig:1-9})。
\begin{table}
  \caption{当 $x$ 接近于 0 时函数 $\dfrac{\sin x}{x}$ 的值}\label{tab:1-5}
  \begin{tblr}{colspec={cX[r]X[r]},hline{2}=0.8pt,row{1}={m,c}}
    $x$(弧度)& $\sin x$ & $\dfrac{\sin x}{x}$ \\
    \num{1.000} & \num{0.84147098} & \num{0.84147098} \\
    \num{0.100} & \num{0.099833417} & \num{0.99833417} \\
    \num{0.010} & \num{0.0099998334} & \num{0.99998334} \\
    \num{0.001} & \num{0.00099999984} & \num{0.99999984} \\
  \end{tblr}
\end{table}
\begin{figure}
  \begin{minipage}{0.48\linewidth}\centering
    \includegraphics{1-9.pdf}
    \caption{}\label{fig:1-9}
  \end{minipage}
  \begin{minipage}{0.48\linewidth}\centering
    \includegraphics{1-10.pdf}
    \caption{}\label{fig:1-10}
  \end{minipage}
\end{figure}

从\cref{tab:1-5,fig:1-9} 可以看出,当 $x$ 无限趋近于 0 时,函数 $\dfrac{\sin x}{x}$ 的值无限趋近于 1,即 $\lim\limits_{x\to0}\dfrac{\sin x}{x}= 1$。下面我们利用上面的定理来证明这个结论。

如\cref{fig:1-10},在单位圆中,以 $OA$ 为始边的圆心角 $x$(不大的正角)是用弧度来度量的,角的终边与圆相交于 $P$。$MP\perp OA$,$AN$ 是过圆 $O$ 上 $A$ 点的切线,它和 $OP$ 相交于 $N$。从图中知道,
\[ MP=\sin x,\ \overparen{PA}=x,\ AN=\tan x,\]
那么面积
\[ S_{\triangle OAP}=\frac12\sin x,\ S_{\text{扇形} OAP}=\frac12x,\ S_{\triangle OAN} =\frac12\tan x.\]
由
\[ S_{\triangle OAP}<S_{\text{扇形} OAP}< S_{\triangle OAN},\]
得
\[\sin x< x < \tan x.\]

当自变量 $x$ 取正值趋近于 0 时,$\sin x>0$,因此,
\[ 1<\frac{x}{\sin x}<\frac{\tan x}{\sin x},\]
即
\[ 1<\frac{x}{\sin x} < \frac{1}{\cos x},\]
于是,
\[ \cos x < \frac{\sin x}{x} <1.\]

因为余弦函数是连续函数,所以 $\lim\limits_{x\to0+}\cos x = 1$。根据上面判定函数极限存在的定理,得到当 $x$ 取正值趋近于 0 时,
\[ \lim\limits_{x\to0+}\frac{\sin x}{x}=1.\]

又当 $x$ 取负值趋近于 0 时,$-x\to0$,$-x>0$,$\sin (-x)> 0$,于是
\[ \lim\limits_{x\to0-}\frac{\sin x}{x}=\lim\limits_{-x\to0+}\frac{\sin (-x)}{-x}=1.\]

根据\cref{subsec:function_limits}最后的定理,由于
\[ \lim\limits_{x\to0+}\frac{\sin x}{x}=\lim\limits_{x\to0-}\frac{\sin x}{x}=1.\]
所以有
\[ \lim\limits_{x\to0}\frac{\sin x}{x} =1.\]

在推导这个重要极限时,用到 $S_{\text{扇形 }{OAP}} = \dfrac{1}{2}x$,$x$ 以弧度为单位。一般地,在微积分中三角函数的自变量都是实数,它对应于以弧度为单位的角或弧。

\begin{example}
  求 $\lim\limits_{x\to0}\dfrac{\tan x}{x}$。
\end{example}
\begin{solution}
  $\lim\limits_{x\to0}\dfrac{\tan x}{x}=\lim\limits_{x\to0}\left(\dfrac{\sin x}{x}\cdot\dfrac{1}{\cos x}\right)=\lim\limits_{x\to0}\dfrac{\sin x}{x}\cdot\lim\limits_{x\to0}\dfrac{1}{\cos x}=1\times1=1$。
\end{solution}

\begin{example}
  求 $\lim\limits_{x\to0}\dfrac{1-\cos x}{x^2}$。
\end{example}
\begin{solution}
  \begin{align*}
    \lim\limits_{x\to0}\frac{1-\cos x}{x^2}&=\lim\limits_{x\to0}\frac{2\sin^2\dfrac{x}{2}}{x^2}=\frac12\lim\limits_{x\to0}\frac{\sin^2\dfrac{x}{2}}{\left(\dfrac{x}{2}\right)^2}\\
    &= \frac12\lim\limits_{\frac{x}{2}\to0}\left(\frac{\sin\dfrac{x}{2}}{\dfrac{x}{2}}\right)^2\quad \left(\text{当 }\ x\to0\ \text{时,}\ \frac{x}{2}\to 0\right)\\
    &=\frac12\lim\limits_{\frac{x}{2}\to0}\left(\frac{\sin\dfrac{x}{2}}{\dfrac{x}{2}}\right)\cdot\lim\limits_{\frac{x}{2}\to0}\left(\frac{\sin\dfrac{x}{2}}{\dfrac{x}{2}}\right)=\frac12\times1\times1=\frac12.
  \end{align*}
\end{solution}

\subsubsection{$\lim\limits_{x\to 0}\left(1+\dfrac{1}{x}\right)^x=\upe$}
我们从\cref{tab:1-6} 可以看出当 $x\to+\infty$ 时函数 $\left(1+\dfrac{1}{x}\right)^x$ 的变化趋势:
\begin{table}
  \caption{$x\to+\infty$ 时函数 $\left(1+\dfrac{1}{x}\right)^x$ 的变化趋势}\label{tab:1-6}
  \begin{tblr}{colspec={X[c]X[2,c]X[l]},hline{2}=0.8pt,row{1}={m,c},row{Z}={m,c}}
    $x$ & $\left(1+\dfrac1x\right)^x$ & 近似值 \\
    \num{1} & $\left(1+\dfrac1{1}\right)^1$ & \num{2} \\
    \num{10} & $\left(1+\dfrac1{10}\right)^{10}$ & \num{2.59374} \\
    \num{100} & $\left(1+\dfrac1{100}\right)^{100}$ & \num{2.70481} \\
    \num{1000} & $\left(1+\dfrac1{1000}\right)^{1000}$ & \num{2.71692} \\
    \num{10000} & $\left(1+\dfrac1{10000}\right)^{10000}$ & \num{2.71815} \\
    \num{100000} & $\left(1+\dfrac1{100000}\right)^{100000}$ & \num{2.71827} \\
    …… & …… & …… \\
  \end{tblr}
  \par \medskip
  \caption{$x\to+\infty$ 时函数 $\left(1+\dfrac{1}{x}\right)^x$ 的变化趋势}\label{tab:1-7}
  \begin{tblr}{colspec={X[c]X[2,c]X[l]},hline{2}=0.8pt,row{1}={m,c},row{Z}={m,c}}
    $x$ & $\left(1+\dfrac1x\right)^x$ & 近似值 \\
    \num{-2} & $\left(1+\dfrac1{-2}\right)^{-2}$ & \num{4} \\
    \num{-10} & $\left(1+\dfrac1{-10}\right)^{-10}$ & \num{2.86797} \\
    \num{-100} & $\left(1+\dfrac1{-100}\right)^{-100}$ & \num{2.73199} \\
    \num{-1000} & $\left(1+\dfrac1{-1000}\right)^{-1000}$ & \num{2.71964} \\
    \num{-10000} & $\left(1+\dfrac1{-10000}\right)^{-10000}$ & \num{2.71842} \\
    \num{-100000} & $\left(1+\dfrac1{-100000}\right)^{-100000}$ & \num{2.71830} \\
    …… & …… & …… \\
  \end{tblr}
\end{table}

同样,当 $x\to-\infty$ 时,函数 $\left(1+\dfrac{1}{x}\right)^x$ 有相同的变化趋势(见\cref{tab:1-7})。

可以证明,当 $x$ 趋向无穷时,$\left(1+\dfrac{1}{x}\right)^x$ 趋近于无理数 $\num{2.71828182845}\cdots$,记作 $\upe$。即
\[ \lim\limits_{x\to\infty}\left(1+\frac1x\right)^x=\upe.\]

如果作一个变换 $y=\dfrac{1}{x}$,那么当 $x\to\infty$ 时,$y\to0$,于是又得到
\[ \lim\limits_{y\to0}(1+y)^{\frac{1}{y}}=\upe.\]

无理数 $\upe$ 是自然对数的底,它在微积分和其他科学技术中经常用到。

\begin{example}
  求 $\lim\limits_{x\to\infty}\left(1+\dfrac1x\right)^{-x}$。
\end{example}
\begin{solution}
  $\lim\limits_{x\to\infty}\left(1+\dfrac1x\right)^{-x}=\lim\limits_{x\to\infty}\dfrac{1}{\left(1+\dfrac1x\right)^x}=\dfrac{1}{\lim\limits_{x\to\infty}\left(1+\dfrac1x\right)^x}=\dfrac{1}{\upe}$。
\end{solution}

\begin{Practice}
  \begin{question}
    \item 求下列极限:
    \begin{tasks}[before-skip=5pt,after-skip=5pt,after-item-skip=7pt](2)
      \task $\lim\limits_{x\to0}\dfrac{x}{\sin x}$;
      \task $\lim\limits_{x\to0}\dfrac{\sin4x}{3x}$。
    \end{tasks}
    \item 求下列极限:
    \begin{tasks}[before-skip=5pt,after-skip=5pt,after-item-skip=7pt](2)
      \task $\lim\limits_{x\to\infty}\left(1+\dfrac{1}{x}\right)^{2x}$;
      \task $\lim\limits_{x\to0}(1+x)^{-\frac{1}{x}}$。
    \end{tasks}
  \end{question}
\end{Practice}

\begin{Exercise}
  \begin{question}
    \item 根据函数连续性的定义,说明下列函数在给定点处连续:
    \begin{tasks}[before-skip=7pt,after-skip=7pt](2)
      \task $f(x)=3x+1,\ x=\dfrac{1}{2}$;
      \task $f(x)=ax^2+b,\ x=1$;
      \task $f(x)=\dfrac{4x^2-1}{2x-1},\ x=2$;
      \task $f(x)=ax^3+bx^2+c+d,\ x=0$。
    \end{tasks}
    \item 说出下列函数在实数轴上哪些点处不连续:
    \begin{tasks}[before-skip=7pt,after-skip=7pt](2)
      \task $f(x)=\dfrac{1}{x^2+3x+2}$;
      \task $f(x)=\dfrac{1}{\sin x}$。
    \end{tasks}
    \item 写出由下列各组函数复合而成的复合函数:
    \begin{tasks}[before-skip=5pt,after-skip=5pt,after-item-skip=3pt](2)
      \task $y=u^2,\ u=\sin x$;
      \task $y=\sin u,\ u=x^2$;
      \task $y=u^3,\ u=x^2+1$;
      \task $y=\ln u,\ u=v^2+1,\ v=\sin x$;
      \task $y=\upe^u,\ u=v^2,\ v=\cot x$;
      \task $y=\arcsin u,\ u=\sqrt{v},\ v=\dfrac{x-a}{b-a}$。
    \end{tasks}
    \item 下列函数是由哪几个简单函数复合而成的?
    \begin{tasks}[before-skip=5pt,after-skip=5pt,after-item-skip=7pt](2)
      \task $y=(1+x)^5$;
      \task $y=\dfrac{1}{(1-x^2)^3}$;
      \task $y=\ln\sin^23x$;
      \task $y=\upe^{2\cos^2x}$;
      \task $y=\dfrac{1}{\sqrt{1-\tan 3x}}$;
      \task $y=(1+\arctan x^2)^3$。
    \end{tasks}
    \item 求下列极限:
    \begin{tasks}[before-skip=10pt,after-skip=10pt,after-item-skip=7pt](2)
      \task $\lim\limits_{x\to1}\sqrt{x^2+3x-2}$;
      \task $\lim\limits_{u\to1}\dfrac{u^2+u+2}{3-u}$;
      \task $\lim\limits_{x\to16}\dfrac{\sqrt[4]{x}-2}{\sqrt{x}-4}$;
      \task $\lim\limits_{x\to2}\dfrac{\sqrt{2x-2}-\sqrt{x}}{x^2-4}$;
      \task $\lim\limits_{\alpha\to\frac{\uppi}{4}}(1+\sin2\alpha)^2$;
      \task $\lim\limits_{x\to2}\log_3(x^3+1)$。
    \end{tasks}
    \item 求下列极限:
    \begin{tasks}[before-skip=10pt,after-skip=10pt,after-item-skip=7pt](2)
      \task $\lim\limits_{x\to0}\dfrac{x}{\tan 3x}$;
      \task $\lim\limits_{x\to0}\dfrac{\sin3x}{\sin5x}$;
      \task $\lim\limits_{x\to0}\dfrac{\sin x\tan x}{x^2}$;
      \task $\lim\limits_{x\to0}\dfrac{1-\cos2x}{x\sin x}$。
    \end{tasks}
    \item 求下列极限:
    \begin{tasks}[before-skip=10pt,after-skip=10pt,after-item-skip=7pt](2)
      \task $\lim\limits_{x\to\infty}\left(1+\dfrac{1}{x}\right)^{x+2}$;
      \task $\lim\limits_{x\to\infty}\left(1+\dfrac{2}{x}\right)^{\frac{x}{2}}$;
      \task $\lim\limits_{x\to\infty}\left(1+2x\right)^{\frac{1}{x}}$;
      \task $\lim\limits_{x\to\infty}\left(1-\dfrac{1}{x}\right)^{2x}$;
    \end{tasks}
  \end{question}
\end{Exercise}

\section*{小结}
\begin{enumerate}[C、,itemindent=4.5em]
  \item 本章的主要内容是数列的极限的概念及其运算法则;函数的极限的概念及其运算法则;函数连续的概念和初等函数的连续性;两个重要的极限,即
  \[ \lim\limits_{x\to0}\frac{\sin x}{x}=1,\qquad \lim\limits_{x\to\infty}\left(1+\frac{1}{x}\right)^x=\upe. \]
  \item 极限是描述数列和函数在无限过程中的变化趋势的重要概念。极限方法是人们从有限中认识无限,从近似中认识精确,从量变中认识质变的一种数学方法,它是微积分的基本思想和方法。
  
  数列的极限与函数的极限的运算法则是类似的: 两个数列(或函数)的和、差、积、商的极限分别等于这两个数列(或函数)的极限的和、差、积、商(作为除数的数列或函数的极限不能为零)。运用这些运算法则,可以简化极限的计算过程。

  \item 连续的概念是用极限的概念定义的,但是连续和极限是有区别的: 极限所讨论的是函数在某一点附近的变化趋势,而不管函数在这一点上是否有定义或取什么值;函数在一点处连续不仅要求在这一点有极限,而且要求极限同这一点的函数值相等。
  \item 幂函数、指数函数、对数函数、三角函数和反三角函数,统称基本初等函数。
  
  由基本初等函数和常数经过有限次四则运算和有限次函数的复合而得出的函数,统称初等函数。

  基本初等函数和一切初等函数在它们的定义区间上是连续函数。
\end{enumerate}
\chapter*{复习参考题\chinese{chapter}}
\section*{A 组}
\begin{question}
  \item 求无穷等比数列 $\{q^n\}$ 当 $q=\dfrac{1}{2}$ 时前 10 项的和与前 100 项的和。
  \item 作图表示下列无穷数列,并说出数列是否趋近于某一常数:
  \begin{tasks}[before-skip=10pt,after-skip=10pt,after-item-skip=7pt](2)
    \task $\left\{1+ (-1)^n\dfrac{2}{n} \right\}$;
    \task $\left\{\dfrac{(n-1)^2}{2n}\right\}$;
    \task $\left\{\left(-\dfrac{1}{n}\right)^3\right\}$;
    \task $\left\{\sqrt{n}\right\}$。
  \end{tasks}
  \item 已知数列 $\left\{ \dfrac{1}{3^n}\right\}$,根据下表中给出的 $\varepsilon$ 的数值,求出相应的正整数 $N$,使得当 $n>N$ 时,$\left| {\dfrac{1}{3^n} - 0}\right| < \varepsilon$ 恒成立。
  \begin{tablehere}
    \begin{tblr}{colspec={X[c]*4{X[r]}},hline{2}=0.8pt}
      $\upe$ & 0.1 & 0.02 & 0.003 & 0.0001 \\
      $N$    & & & & \\
    \end{tblr}
  \end{tablehere}
  \item 举出两个极限是 7 的无穷数列。
  \item 举出两个没有极限的无穷数列。
  \item 求下列数列的极限:
  \begin{tasks}[before-skip=10pt,after-skip=10pt,after-item-skip=7pt](2)
    \task $\lim\limits_{n\to\infty} \dfrac{2n+1}{n}$;
    \task $\lim\limits_{n\to\infty} \dfrac{3n}{n+1}$;
    \task $\lim\limits_{n\to\infty} \dfrac{n^2}{n^2-n}$;
    \task $\lim\limits_{n\to\infty} \dfrac{1}{n^2+1}$。
  \end{tasks}
  \item 解答:
  \begin{tasks}[before-skip=10pt,after-skip=10pt,after-item-skip=7pt]
    \task 求 $\lim\limits_{n\to\infty}\dfrac{1+2+3+\cdots+n}{1+3+5+\cdots+(2n-1)}$;
    \task 求 $\lim\limits_{n\to\infty}\dfrac{1+\dfrac{1}{2}+\dfrac{1}{4}+\cdots+\dfrac{1}{2^n}}{1+\dfrac{1}{3}+\dfrac{1}{9}+\cdots+\dfrac{1}{3^n}}$。
  \end{tasks}
  \item 求下列极限:
  \begin{tasks}[before-skip=10pt,after-skip=10pt,after-item-skip=7pt](2)
    \task $\displaystyle\lim_{n\to\infty} \left(\frac{1}{n}+\frac{2n-1}{3n}\right)$;
    \task $\displaystyle\lim_{n\to\infty}\frac{5n^2+7}{3n^2+n-1} $;
    \task $\displaystyle\lim_{n\to\infty}\frac{(n-1)(n+1)(n+2)}{2n^3}$;
    \task $\displaystyle\lim_{n\to\infty}\frac{(n-1)^3}{n^3+1}$;
    \task $\displaystyle\lim_{x\to1}\left(\frac{1}{1-x}-\frac{3}{1-x^3}\right)$;
    \task $\displaystyle\lim_{x\to\infty}\left(\frac{x^3}{2x^2-1}-\frac{x^2}{2x+1}\right)$。
  \end{tasks}
  \item 已知 $f(x)=\dfrac{a_0x^m+a_1x^{m-1}+\cdots+a_m}{b_0x^n+b_1x^{n-1}+\cdots+b_n}$,而且 $x_0$ 不是 $f(x)$ 的分母的根。
  \begin{tasks}[before-skip=5pt,after-skip=5pt,after-item-skip=5pt]
    \task 求 $\lim\limits_{x\to x_0}f(x)$;
    \task 当 $m\leqslant n$ 时,求 $\lim\limits_{x\to\infty}f(x)$。
  \end{tasks}
  \item 说出下列函数是怎样复合而成的:
  \begin{tasks}[before-skip=5pt,after-skip=5pt,after-item-skip=5pt](2)
    \task $y=(a+bx)^5$;
    \task $y=(\arccos\sqrt{1-x^2})^3$;
    \task $y=\arctan\sqrt[5]{x^3-1}$;
    \task $y=\upe^{\frac{1}{2}\lg(ax+b)}$。
  \end{tasks}
  \item 求下列极限:
  \begin{tasks}[before-skip=10pt,after-skip=10pt,after-item-skip=7pt](2)
    \task $\lim\limits_{x\to\sqrt{2}}\dfrac{2x^2-1}{x^4+2x^2-1}$;
    \task $\lim\limits_{x\to\frac{\uppi}{3}}(\sin2\theta+\cos2\theta)$;
    \task $\lim\limits_{x\to a}\dfrac{\sin x-\sin a}{x-a}$;
    \task $\lim\limits_{x\to0}\dfrac{\sin^23x}{x\sin2x}$;
    \task $\lim\limits_{x\to2}\dfrac{x-2}{\sqrt{x-1}-1}$;
    \task $\lim\limits_{x\to3}\dfrac{\sqrt{1+x}-2}{x-3}$。
  \end{tasks}
  \item 求下列极限:
  \begin{tasks}[before-skip=10pt,after-skip=10pt,after-item-skip=7pt](2)
    \task $\lim\limits_{x\to0}\dfrac{x^2}{\sin^2\left(\dfrac{x}{3}\right)}$;
    \task $\lim\limits_{x\to0}(x\cdot\cot x)$;
    \task $\lim\limits_{x\to\infty}\left(\dfrac{x}{1+x}\right)^x$;
    \task $\lim\limits_{x\to\infty}\left(1+\dfrac{2}{x}\right)^x$。
  \end{tasks}
\end{question}
\section*{B 组}
\begin{question}[resume]
  \item 下面的数列中,哪些有极限?如果数列有极限,说出它的极限。
  \begin{tasks}(2)
    \task $1,\ 0.1,\ 0.01,\ 0.001,\ \dots $;
    \task $+2,\ -2,\ +2,\ -2,\ \dots $;
    \task $\left\{ \dfrac{\sqrt{n}+1}{n}\right\}$;
    \task! $\{q^n\}$(提示:分 $q=1$,$q=-1$,$|q|<1$,$|q|>1$ 四种情况讨论)。
  \end{tasks}
  \item 已知 $a>0$,求下列极限:
  \begin{tasks}[before-skip=10pt,after-skip=10pt,after-item-skip=7pt](2)
    \task $\lim\limits_{n\to\infty}\dfrac{1}{1+a^n}$;
    \task $\lim\limits_{n\to\infty}\dfrac{a^n}{1+a^n}$。
  \end{tasks}
  \item\label{exec:1t-15} 如图,从 $\angle BAC$ 的边上一点 $B$ 作 $BC \perp AC$,从 $C$ 作 $CD \perp AB$,从 $D$ 再作 $DE\perp AC$,这样无限进行下去。假定 $BC=\qty{7}{cm}$,$CD=\qty{6}{cm}$,求这些垂线长的和。
  \begin{figurehere}
    \begin{minipage}{\linewidth}\centering
      \includegraphics{1t-15.pdf}
      \caption*{(第 \ref{exec:1t-15} 题图)}
    \end{minipage}
  \end{figurehere}
  \item 将下列循环小数化成分数:
  \begin{tasks}(2)
    \task $0.\dot{2}\dot{7}$;
    \task $2.5\dot{1}4285\dot{7}$。
  \end{tasks}
  \item 求下列极限:
  \begin{tasks}[before-skip=10pt,after-skip=10pt,after-item-skip=7pt](2)
    \task $\lim\limits_{x\to\infty}\dfrac{x^2+2}{x^2+x+1}$
    \task $\lim\limits_{x\to\infty}\dfrac{x^2+1}{4x^3-1}$
    \task $\lim\limits_{x\to-\infty}\dfrac{3x^2-1}{x^2+2x}$
    \task $\lim\limits_{x\to+\infty}\dfrac{5x^2+x-3}{4x^2-2x+1}$
  \end{tasks}
  \item 求下列极限:
  \begin{tasks}[before-skip=10pt,after-skip=10pt,after-item-skip=7pt](2)
    \task $\lim\limits_{x\to3}\dfrac{x^2-9}{x^2-4x+3}$;
    \task $\lim\limits_{u\to1}\dfrac{u^3-1}{u^2-1}$;
    \task $\lim\limits_{x\to1}\dfrac{x^m-1}{x^n-1}$($m$、$n$ 是正整数)。
  \end{tasks}
  \item 求下列无穷数列各项的和 $S$:
  \begin{tasks}[before-skip=10pt,after-skip=10pt,after-item-skip=7pt]
    \task $\dfrac{1}{1\cdot2},\ \dfrac{1}{2\cdot3},\ \cdots,\ \dfrac{1}{n(n+1)},\ \cdots$(提示:$\dfrac{1}{n(n+1)}=\dfrac{1}{n}-\dfrac{1}{n+1}$);
    \task $\dfrac{2}{2^2-1},\ \dfrac{2}{3^2-1},\ \cdots,\ \dfrac{2}{(n+1)^2-1},\ \cdots$
    
    (提示:$\dfrac{1}{(n+1)^2-1}=\dfrac{1}{n(n+2)}$)。
  \end{tasks}
  \item 求下列极限:
  \begin{tasks}[before-skip=10pt,after-skip=10pt,after-item-skip=7pt](2)
    \task $\lim\limits_{x\to1}\dfrac{\sqrt[3]{x}-1}{\sqrt{x}-1}$;
    \task $\lim\limits_{x\to\infty}(\sqrt{x^2+1}-\sqrt{x^2-1})$。
  \end{tasks}
  \item 求下列极限:
  \begin{tasks}[before-skip=10pt,after-skip=10pt,after-item-skip=7pt](2)
    \task $\lim\limits_{x\to\infty}\left(1+\dfrac{3}{x}\right)^x$;
    \task $\lim\limits_{x\to0}(1+x)^{\frac{1}{x}+2}$;
    \task $\lim\limits_{x\to0}(1+\tan x)^{\cot x}$;
    \task $\lim\limits_{x\to1}\dfrac{\sin(1-x)}{1-x^2}$。
  \end{tasks}
\end{question} % ok
\chapter{导数和微分}
\section{导数概念}
\subsection{瞬时速度}\label{subsec:velocity}
我们知道,物体作匀速直线运动时,物体的位移 $s$ 与所经过的时间 $t$ 的比,就是物体运动的速度 $v$,即
\[v=\frac{s}{t}.\]
这个速度在匀速直线运动过程中的任何时刻都是一样的。

如果物体作非匀速直线运动,也就是说,在运动过程的各个时刻,物体运动的快慢不一样,这时,设已知物体的运动规律是 $s=s(t)$,从 $t_0$ 到 $t_0+\Delta t$ ($\Delta t$ 称为时间改变量) 这段时间内,物体的位移(即位置改变量)是
\[ \Delta s=s(t_0+\Delta t)-s(t_0),\]
那么,位置改变量 $\Delta s$ 与时间改变量 $\Delta t$ 的比,就是这段时间内物体的平均速度 $\bar{v}$,即
\[ \bar{v}=\frac{\Delta s}{\Delta t}=\frac{s(t_0+\Delta t)-s(t_0)}{\Delta t}.\]
平均速度的大小反映在这段时间内物体运动快慢的平均程度。

为了更精确地刻划非匀速运动,还需知道物体在某一时刻的“速度”。那么,作非匀速直线运动的物体在某一时刻的 “速度”怎样求呢?

现在我们以自由落体运动为例,来说明作非匀速直线运动的物体在某一时刻的“速度”的求法。

我们知道,自由落体运动的方程是
\[ s=s(t)=\frac{1}{2}gt^2,\]
这里 $g$ 是重力加速度,通常取 $g=\qty{9.8}{m/s^2}$。现在来求 $t= \qty{3}{s}$ 这一时刻落体的 “速度”。

当 $\Delta t$ 很小时,从 \qty{3}{s} 到 $(3+\Delta t)$\,\unit{s} 这段时间内,落体运动的快慢变化也不大,因此,可以用这段时间内的平均速度近似地反映落体在 \qty{3}{s} 时的 “速度”。当 $\Delta t$ 越小时,一般来讲,这种近似就越精确。现在我们来计算一下 $t$ 从 \qty{3}{s} 分别到 \qty{3.1}{s}、 \qty{3.01}{s}、 \qty{3.001}{s}、 \qty{3.0001}{s}、……,各段时间内的平均速度,把所得数据列表如\cref{tab:2-1}:
\begin{table}
  \caption{各段时间内的平均速度}\label{tab:2-1}
  \begin{tblr}{colspec={*{4}{X[l]}X[2,l]},hline{2}=0.8pt,row{1}={m,c}}
    $t$(\unit{s}) & $s$(\unit{m}) & $\Delta t$(\unit{s}) & $\Delta s$(\unit{m}) & $\bar{v}=\dfrac{\Delta s}{\Delta t}$(\unit{m/s})\\
    3      & $4.5g$         &        & & \\
    3.1    & $4.805g$       & 0.1    & $0.305g$       & $3.05g$ \\
    3.01   & $4.53005g$     & 0.01   & $0.03005g$     & $3.005g$ \\
    3.001  & $4.5030005g$   & 0.001  & $0.0030005g$   & $3.0005g$ \\
    3.0001 & $4.500300005g$ & 0.0001 & $0.000300005g$ & $3.00005g$ \\
    \dots  & \dots & \dots & \dots & \dots \\
  \end{tblr}
\end{table}

从\cref{tab:2-1} 可以看出,平均速度 $\dfrac{\Delta s}{\Delta t}$ 随着 $\Delta t$ 变化而变化,当 $\Delta t$ 越小时,$\dfrac{\Delta s}{\Delta t}$ 越接近于一个定值——$3g$。这个值就是 $\Delta t \to 0$ 时 $\dfrac{\Delta s}{\Delta t}$ 的极限。我们规定这个极限为落体在 $t =\qty{3}{s}$ 时的速度,也叫\Concept{瞬时速度},用 $v$ 表示。根据\cref{chp:limits}求极限的法则,得
\begin{align*}
  v &=\lim\limits_{\Delta t\to 0}\frac{s(3+\Delta t)-s(3)}{\Delta t}= \lim\limits_{\Delta t\to 0}\frac{\dfrac{1}{2}g(3+\Delta t)^2-\dfrac{1}{2}g\cdot3^2}{\Delta t}\\ 
    &= \frac{g}{2}\lim\limits_{\Delta t\to 0} (6+\Delta t)= 3g =\qty{29.4}{m/s}
\end{align*}

一般地,我们规定,非匀速直线运动在某一时刻 $t_0$ 的瞬时速度 $v$,就是运动物体在 $t_0$ 到 $t_0+\Delta t$ 一段时间内的平均速度当 $\Delta t\to 0$ 时的极限,即
\[ v=\lim\limits_{\Delta t\to 0}\frac{\Delta s}{\Delta t} =\lim\limits_{\Delta t\to 0}\frac{s(t_0+\Delta t)-s(t_0)}{\Delta t}.\]
平均速度 $\dfrac{\Delta s}{\Delta t}$ 在 $\Delta t\to 0$ 时转化为瞬时速度,瞬时速度的大小刻划了物体在某一时刻运动的快慢。

\begin{Practice}
  \begin{question}
    \item 一球沿某一斜面自由滚下,测得滚下的垂直距离 $y$ 与时间 $x$ 之间的函数关系为 $y=x^2$。
    \begin{tasks}
      \task 求时间 $x$ 从 \qty{5}{s} 分别到 \qty{6}{s}、\qty{5.1}{s}、\qty{5.01}{s}、\qty{5.001}{s}、 $(5+h)$\,\unit{s} 的时间改变量 $\Delta x$,对应的垂直距离改变量 $\Delta y$ 以及这段时间内垂直方向的平均速度 $\dfrac{\Delta y}{\Delta x}$,并填下表:
      \begin{tablehere}
        \begin{minipage}{\linewidth}
          \begin{tblr}{colspec={lX[c]X[c]X[c]},hline{2}=0.8pt}
            & $\Delta x$(\unit{s}) & $\Delta y$(\unit{m}) & $\dfrac{\Delta y}{\Delta x}$(\unit{m/s})\\
            从 \qty{5}{s} 到 \qty{6}{s}&&&\\
            从 \qty{5}{s} 到 \qty{5.1}{s}&&&\\
            从 \qty{5}{s} 到 \qty{5.01}{s}&&&\\
            从 \qty{5}{s} 到 \qty{5.001}{s}&&&\\
            从 \qty{5}{s} 到 $(5+h)\,\unit{s}$&&&\\
          \end{tblr}
        \end{minipage}
      \end{tablehere}
      \task 求在 \qty{5}{s} 时垂直方向的瞬时速度。
    \end{tasks}
    \item 质点 $M$ 按规律 $s=2t^2+3t$ 作直线运动($s$ 的单位为 \unit{cm},$t$ 的单位为 \unit{s})。
    \begin{tasks}
      \task 设 $t_0,\Delta t$ 已给定,求相应的 $\Delta s,\dfrac{\Delta s}{\Delta t}$ 和 $\lim\limits_{\Delta t\to0}\frac{\Delta s}{\Delta t}$,并说明它们的物理意义;
      \task 求出质点 $M$ 从 \qty{2}{s} 分别到 \qty{2.1}{s}、\qty{2.01}{s}、\qty{2.001}{s}、$(2+\Delta t)\,\unit{s}$ 各段时间内的平均速度;
      \task 求质点 $M$ 在 $t=\qty{2}{s}$ 时的瞬时速度。
    \end{tasks}
  \end{question}
\end{Practice}

\subsection{导数}
从\cref{subsec:velocity}的讨论可以看出,当物体作匀速直线运动时,我们可以直接根据位移和时间的比值来求速度;当物体的位移随着时间的改变可能是非均匀变化时,要研究在某一时刻物体运动的快慢,就需要引进瞬时速度,也就是位移函数对时间的瞬时变化率(简称变化率)的概念。
在自然科学和工程技术中,经常遇到非均匀变化的问题,例如化学反应速度、物体温度变化速度、放射物质的蜕变速度、电流强度等等。
因此,撇开具体实际意义,一般地从数量关系上来研究函数的变化率,将对很多实际问题的解决具有普遍意义。
为此,我们引进一个新的数学概念——导数。

设函数 $y=f(x)$ 在点 $x=x_0$ 及其附近有定义。当自变量 $x$ 在 $x_0$ 处有改变量 $\Delta x$($\Delta x$ 可正可负),则函数 $y$ 相应地有改变量
\[ \Delta y=f(x_0+\Delta x)-f(x_0).\]
这两个改变量的比
\[ \frac{\Delta y}{\Delta x}=\frac{f(x_0+\Delta x)-f(x_0)}{\Delta x},\]
叫做函数 $y=f(x)$ 在 $x_0$ 到 $x_0+\Delta x$ 之间的\Concept{平均变化率}。

如果 当 $\Delta x\to 0$ 时,$\dfrac{\Delta y}{\Delta x}$ 有极限,我们就说函数 $y=f(x)$ 在点 $x_0$ 处可导,并把这一极限叫做 $f(x)$ 在点 $x_0$ 处的\Concept{导数}(或变化率),记作 $f'(x_0)$ 或 $y'\vert_{x=x_0}$,即
\begin{equation}
  f'(x_0)=\lim\limits_{\Delta x\to 0}\frac{\Delta y}{\Delta x}=\lim\limits_{\Delta x\to 0}\frac{f(x_0+\Delta x)-f(x_0)}{\Delta x}.
\end{equation}

因此,函数 $f(x)$ 在点 $x_0$ 处的导数就是函数的平均变化率当自变量的改变量趋向于零时的极限 $f'(x_0)$,它反映了函数 $f(x)$ 在点 $x_0$ 处变化的 “速度”。

如果上述极限不存在,我们就说函数 $f(x)$ 在点 $x_0$ 处不可导或导数不存在。

根据导数的定义,瞬时速度就是位移函数 $s(t)$ 对时间 $t$ 的导数,自由落体在 $t=\qty{3}{s}$ 时的速度就是 $s=\dfrac{1}{2}gt^2$ 在点 $t =3$ 处的导数。即
\[
  v=s'\vert_{t=3}=\lim\limits_{\Delta t\to 0}\frac{\dfrac{1}{2}g(3+\Delta t)^2-\dfrac{1}{2}g\cdot3^2}{\Delta t}= 3g =\qty{29.4}{m/s^2}.
\]

由导数的定义,直接得出求函数 $f(x)$ 在点 $x_0$ 处的导数的方法:
\begin{enumerate}[1.]
  \item 求改变量 $\Delta y= f(x_0+\Delta x)-f(x)$;
  \item 求比 $\dfrac{\Delta y}{\Delta x}=\dfrac{f(x_0+\Delta x)-f(x)}{\Delta x}$;
  \item 求极限 $\lim\limits_{\Delta x\to 0}\dfrac{\Delta y}{\Delta x}$。
\end{enumerate}

\begin{example}
  求 $y=x^2$ 在点 $x=1$ 处的导数。
\end{example}
\begin{solution}
  \begin{align*}
    \Delta y& = (1+\Delta x)^2-1^2 =2\Delta x+(\Delta x)^2,\\
    \frac{\Delta y}{\Delta x}&=\frac{2\Delta x+(\Delta x)^2}{\Delta x}=2+\Delta x, \\
    \lim\limits_{\Delta x\to 0} \frac{\Delta y}{\Delta x}&=\lim\limits_{\Delta x\to 0}(2+\Delta x)=2, \\
    \therefore \quad y'\vert_{x=1} &= 2
  \end{align*}
\end{solution}

如果函数 $f(x)$ 在开区间 $(a,b)$ 内每一点处都可导,就说 $f(x)$ 在开区间 $(a,b)$ 内可导。这时,对于开区间 $(a,b)$ 内每一个确定的值 $x_0$,都对应着一个确定的导数 $f'(x_0)$,这样就在开区间 $(a,b)$ 内,构成一个新的函数,我们把这一新函数叫做 $f(x)$ 的\Concept{导函数},记为 $f'(x)$ 或 $y'$(有必要指明自变量 $x$ 时记作 $y'_x$)。根据导数的定义,就可得出导函数
\begin{equation}
  y'=f'(x)=\lim\limits_{\Delta x\to 0}\frac{\Delta y}{\Delta x}=\lim\limits_{\Delta x\to 0}\frac{f(x+\Delta x)-f(x)}{\Delta x}.
\end{equation}

导函数也简称为导数。
今后,如不特别指明求某一点处的导数,求导数就是指求导函数。
但要注意,函数 $y=f(x)$ 的导函数 $f'(x)$ 与函数 $y=f(x)$ 在点 $x_0$ 处的导数是有区别的,$f'(x)$ 是 $x$ 的函数,而 $f'(x_0)$ 是一个数值;但它们又是有联系的,$f(x)$ 在点 $x_0$ 处的导数 $f'(x_0)$ 就是导函数 $f'(x)$ 在点 $x_0$ 处的函数值。

这样,如果知道了导函数 $f'(x)$,要求 $f(x)$ 在点 $x_0$ 处的导数,只要把 $x=x_0$ 代入 $f'(x)$ 中去求函数值就可以了。

\begin{example}
  已知 $y=x^3-2x+1$,求 $y'$,并求在点 $x=2$ 处的导数。
\end{example}
\begin{solution}
\begin{align*}
  \Delta y&=(x+\Delta x)^3-2(x+\Delta x)+1-(x^3-2x+1)\\
  &=(3x^2-2)\Delta x+3x(\Delta x)^2+(\Delta x)^3,\\
  \frac{\Delta y}{\Delta x}&=3x^2-2+3x\Delta x+(\Delta x)^2,\\
  \therefore \quad y'&=\lim\limits_{\Delta x\to 0}\frac{\Delta y}{\Delta x}=3x^2-2\\
  y'\vert_{x=2}&=3\times2^2-2=10.
\end{align*}
\end{solution}

\begin{example}
  已知 $y=\sqrt{x}$,求 $y'$。
\end{example}
\begin{solution}
  \begin{align*}
    \Delta y&=\sqrt{x+\Delta x}-\sqrt{x},\\
    \frac{\Delta y}{\Delta x}&=\frac{\sqrt{x+\Delta x}-\sqrt{x}}{\Delta x},\\
    \therefore \quad y'&=\lim\limits_{\Delta x\to 0}\frac{\Delta y}{\Delta x}=\lim\limits_{\Delta\to 0}\frac{\sqrt{x+\Delta x}-\sqrt{x}}{\Delta x}=\lim\limits_{\Delta\to 0}\frac{1}{\sqrt{x+\Delta x}+\sqrt{x}}=\frac{1}{2\sqrt{x}}.
  \end{align*}
\end{solution}

\begin{Practice}
  \begin{question}
    \item 求下列函数在指定点处的导数:
    \begin{tasks}[before-skip=5pt,after-skip=5pt](2)
      \task $y=x^3$,点 $x=0$;
      \task $y=\dfrac{2}{x}$,点 $x=1$。
    \end{tasks}
    \item 已知一物体作直线运动,运动方程为 $s=4t^2$(\unit{m}),求:
    \begin{tasks}
      \task $t$ 秒时的瞬时速度;
      \task $t=\qty{3}{s}$、$t=\qty{5}{s}$、$t=\qty{8}{s}$ 时的瞬时速度。
    \end{tasks}
    \item 已知 $y=\sqrt{x+4}$,求 $y'$,并求在点 $x=5$ 处的导数。
  \end{question}
\end{Practice}
\subsection{导数的几何意义\texorpdfstring{\quad}{ }切线方程和法线方程}
\noindent
\begin{minipage}{0.5\linewidth}\parindent2em
如\cref{fig:2-1},设曲线 $C$ 是函数 $y = f(x)$ 的图象。在曲线 $C$ 上取一点 $P\,(x_0,y_0)$ 及点 $P$ 邻近的任一点 $Q\,(x_0+\Delta x,y_0+\Delta y)$,过 $P$,$Q$ 作割线,并作 $MP\perp Ox$,$NQ\perp Ox$,$PR\perp NQ$。又设割线 $PQ$ 的倾斜角为 $\beta$,那么
\[ \frac{\Delta y}{\Delta x}=\tan\angle QPR=\tan\beta.\]
这就是说,$\dfrac{\Delta y}{\Delta x}$ 就是割线 $PQ$ 的斜率。
\end{minipage}\hfill
\begin{minipage}{0.45\linewidth}
  \begin{figurehere}
    \includegraphics{2-1.pdf}
    \caption{}\label{fig:2-1}
  \end{figurehere}
\end{minipage}

\par\medskip
当点 $Q$ 沿着曲线 $O$ 无限地趋近于点 $P$,即 $\Delta x\to 0$ 时,割线 $PQ$ 绕着点 $P$ 转动,它的极限位置 $PT$ 叫做曲线 $C$ 在点 $P$ 处的切线。这时如果函数 $y=f(x)$ 在点 $x_0$ 处可导,那么当 $\Delta x\to 0$ 时,$\dfrac{\Delta y}{\Delta x}\to f'(x_0)$,而 $\tan\beta$ 以 $PT$ 的斜率 $\tan\alpha$($\alpha$ 是 $PT$ 的倾斜角)为极限,所以
\[f'(x_0)=\tan\alpha.\]

因此,\emph{函数 $y=f(x)$ 在点 ${x}_{0}$ 处的导数 $f'(x_0)$ 的几何意义,就是曲线 $y=f(x)$ 在点 $(x_0,f(x_0))$ 处的切线的斜率}。

这样,求曲线 $y=f(x)$ 在点 $(x_0,f(x_0))$ 处的切线,只要先求出函数 $y=f(x)$ 在点 $x_0$ 处的导数 $f'(x_0)$,然后根据直线方程的点斜式,就得到切线的方程
\[ y-y_0=f'(x_0)(x-x_0),\]
这里 $y_0= f(x_0)$,下同。

当 $\alpha=\dfrac{\uppi}{2}$ 时,导数不存在,这时切线 $PT$ 平行于 $y$ 轴,切线的方程为
\[x=x_0.\]

经过点 $P$ 且和切线 $PT$ 垂直的直线叫做曲线 $C$ 在点 $P$ 处的\Concept{法线}。
由解析几何得知,如果两条有斜率的直线互相垂直,那么,它们的斜率互为负倒数。
已知曲线 $y=f(x)$ 在点 $(x_0,f(x_0))$ 处的切线的斜率为 $f'(x_0)$,那么,当 $f'(x_0) \neq 0$ 时,曲线在该点处的法线的斜率为 $- \dfrac{1}{f'(x_0)}$,法线的方程为
\[ y-y_0=-\frac{1}{f'(x_0)}(x-x_0).\]
当 $f'(x_0)=0$ 时,法线平行于 $y$ 轴,法线的方程为
\[x=x_0.\]
当切线 $PT$ 平行于 $y$ 轴时,法线平行于 $x$ 轴,这时法线的方程为
\[y=y_0.\]

\begin{example}
  已知曲线 $y=\dfrac{1}{3}x^3$ 上一点 $P\,\left(2,\dfrac{8}{3}\right)$。求:
  \begin{enumerate*}
    \item 过 $P$ 点的切线的斜率;
    \item 过 $P$ 点的切线的方程;
    \item 过 $P$ 点的法线的方程。
  \end{enumerate*}
\end{example}
\begin{solution}\par\noindent
  \begin{minipage}{0.65\linewidth}
  \begin{enumerate}
    \item $y=\dfrac{1}{3}x^3$,
    \begin{align*}
      \therefore\quad y'&=\lim\limits_{\Delta x\to 0}\frac{\Delta y}{\Delta x} = \lim\limits_{\Delta x\to 0}\frac{\dfrac{1}{3}(x+\Delta x)^3-\dfrac{1}{3}x^3}{\Delta x}\\ 
      &= \frac{1}{3}\lim\limits_{\Delta x\to 0}\frac{3x^2\Delta x+3x(\Delta x)^2+(\Delta x)^3}{\Delta x}\\ 
      &= \frac{1}{3}\lim\limits_{\Delta x\to 0}\left[3x^2+3x\Delta x+(\Delta x)^2\right]= x^2,\\ 
      y'\vert_{x=2}&= 2^2=4. 
    \end{align*}
    $\therefore\quad$ 在点 $P$ 处的切线的斜率等于 4。
    \item 在点 $P$ 处的切线的方程为
    \[ y-\frac{8}{3}=4(x-2)\]
    即
    \[ 12x-3y-16=0.\]
  \end{enumerate}
\end{minipage}\hfill
\begin{minipage}{0.33\linewidth}
  \begin{figurehere}
    \includegraphics{2-2.pdf}
    \caption{}\label{fig:2-2}
  \end{figurehere}
\end{minipage}
\par\medskip
\begin{enumerate}[start=3]
  \item 在点 $P$ 处的法线的方程为
  \[y-\frac{8}{3}=-\frac{1}{4}(x-2),\]
  即
  \[3x+12y-38=0\]
  图形如\cref{fig:2-2} 所示。
\end{enumerate}
\end{solution}

\begin{Practice}
  \begin{question}
    \item 求抛物线 $y=4x-x^2$ 在点 $A\,(4,0)$ 和点 $B\,(2,4)$ 处的 
    \begin{enumerate*}
      \item 切线的斜率;\item 切线的方程。
    \end{enumerate*}
    \item 求等边双曲线 $y=\dfrac{9}{x}$ 在点 $M\,(3,3)$ 处的切线的斜率与倾斜角。
    \item 求抛物线 $y=x^2+2$ 在点 $M\,(2,6)$ 处的切线方程和法线方程。
  \end{question}
\end{Practice}
\subsection*{变化率举例}
我们知道,函数 $y= f(x)$ 的导数 $f'(x)$ 就是函数对自变量 $x$ 的变化率,因此,很多非均匀变化的变化率问题都可以应用导数来研究。
为了更好地理解和应用导数概念,我们再举几个非均匀变化的变化率的例子。
\begin{example}
  \textbf{瞬时功率}\par
  已知物体所作的功 $W$ 是时间 $t$ 的函数:$W=W(t)$。求在时刻 $t= t_0$ 的功率。
\end{example}
\begin{analyze}
  功率表示作功的效率,物体在某段时间内所作的功 $\Delta W$ 和这段时间 $\Delta t$ 的比 $\dfrac{\Delta W}{\Delta t}$,就是这段时间内的(平均)功率。
  如果物体所作的功随时间增加而均匀改变,这个比是个常数,这个常数就是任一时刻的功率。
  但是,如果物体作功随时间的变化是非均匀的,那么,$\dfrac{\Delta W}{\Delta t}$ 只表示某段时间内的平均功率。
因此,要求任一时刻 $t_0$ 的功率,就是求 $[ t_0,t_0+\Delta t]$ 内的平均功率 $\dfrac{\Delta W}{\Delta t}$ 当 $\Delta t\to 0$ 时的极限——瞬时功率。
\end{analyze}

\begin{solution}
  已知物体从 0 到 $t$ 这段时间内所作的功是 $W=W(t)$,那么,从时刻 $t_0$ 到 $t_0+\Delta t$ 这段时间内物体所作的功(即功的改变量)为
  \[ \Delta W=W(t_0+\Delta t)-W(t_0)\]
  它和完成这些功所用时间(即时间改变量)$\Delta t$ 的比
  \[ \frac{\Delta W}{\Delta t}=\frac{W(t_0+\Delta t)-W(t_0)}{\Delta t}\]
  就是 $t_0$ 到 $t_0+\Delta t$ 这段时间内的平均功率。

  当 $\Delta t \to 0$ 时,平均功率的极限
  \[ \lim\limits_{\Delta t\to 0}=\frac{W(t_0+\Delta t)-W(t_0)}{\Delta t}=P_0\]
  就是在时刻 $t_0$ 的\Concept{瞬时功率}。

  因此,功率 $P$ 是功 $W$ 对时间 $t$ 的导数,即
  \[ P=W'(t).\]
\end{solution}
\begin{example}
  \textbf{瞬时电流强度}\par
  设有一随时间而变化的电流,从 0 到 $t$ 这段时间内通过导线横截面的电量为 $q=q(t)$。求在时刻 $t_0$ 时导线中的电流强度。
\end{example}
\begin{analyze}
电流强度表示电流的强弱,在某段时间内通过导线横截面的电量 $\Delta q$ 和这段时间 $\Delta t$ 的比 $\dfrac{\Delta q}{\Delta t}$,就是这段时间内的电流强度。
  如果给出的电流是稳恒电流,比 $\dfrac{\Delta q}{\Delta t}$ 是一个常数,这个常数就是任一时刻的电流强度。
  但是,如果电路中电流的强弱是变化的,那么,$\dfrac{\Delta q}{\Delta t}$ 只表示某段时间内的平均电流强度。
  因此,要求在时刻 $t_0$ 的电流强度,就是求 $[t_0,t_0+\Delta t]$ 内的平均电流强度 $\dfrac{\Delta q}{\Delta t}$ 当 $\Delta t\to 0$ 时的极限——瞬时电流强度。
\end{analyze}

\begin{solution}
  已知从 0 到 $t$ 这段时间内通过导线横截面的电量为 $q(t)$,那么,从时刻 $t_0$ 到 $t_0+\Delta t$ 这段时间内通过导线横截面的电量为
  \[ \Delta q=q(t_0+\Delta t)-q(t_0),\]
  这段时间内的平均电流强度为
  \[ \frac{\Delta q}{\Delta t}=\frac{q(t_0+\Delta t)-q(t_0)}{\Delta t}.\]
  当 $\Delta t\to 0$ 时,平均电流强度的极限
  \[ \lim\limits_{\Delta t\to 0}=\frac{q(t_0+\Delta t)-q(t_0)}{\Delta t}=I_0\]
  就是在时刻 $t_0$ 的\Concept{电流强度}。

  因此,电流强度 $I$ 是电量 $q$ 对时间 $t$ 的导数,即
  \[I=q'(t).\]
\end{solution}

\begin{Practice}
  \begin{question}
    \item 在匀速圆周运动中,连结运动质点和圆心的半径转过的角度和所用时间的比值,叫做匀速圆周运动的角速度。现有一质点作变速圆周运动,已知在时刻 $t$ 连结运动质点和圆心的半径转过的角度为 $\varphi(t)$,求它在时 刻 $t_0$ 的角速度 $\omega_0$。
    \item 某物质在一个化学反应中的浓度 $C$ 与反应开始后的时间 $t$ 之间的函数关系为 $C=C(t)$,写出 $t=a$ 时浓度的变化率。
    \item 求导数的方法适用不适用于求均匀变化的量的变化率?这时,平均变化率和瞬时变化率有什么关系?试加说明。
  \end{question}
\end{Practice}

\subsection{函数的可导性与连续性的关系}
由导数的定义,可以推出函数在一点处可导与函数在该点处连续的关系:

\emph{如果函数 $y=f(x)$ 在点 $x_0$ 处可导,那么 $y=f(x)$ 在点 $x_0$ 处连续}。

\begin{proof}
  我们是要根据
  \[\lim\limits_{\Delta x\to 0}\frac{f(x_0+\Delta x)-f(x_0)}{\Delta x}=f'(x_0)\]
  来证明
  \[ \lim\limits_{x\to x_0}f(x)=f(x_0).\]

  考虑 $\lim\limits_{x\to x_0}f(x)$,令 $x=x_0+\Delta x$,$x\to x_0$ 相当于 $\Delta x\to 0$,于是
  \begin{align*}
    \lim\limits_{x\to x_0}f(x_0)&= \lim\limits_{\Delta x\to 0}f(x_0+\Delta x) = \lim\limits_{\Delta x\to 0}[f(x_0+\Delta x)-f(x_0)+f(x_0)]\\ 
    &= \lim\limits_{\Delta x\to 0}\left[\frac{f(x_0+\Delta x)-f(x_0)}{\Delta x}\cdot\Delta x +f(x_0)\right]\\ 
    &= \lim\limits_{\Delta x\to 0}\frac{f(x_0+\Delta x)-f(x_0)}{\Delta x}\cdot\Delta x +f(x_0)\\ 
    &= \lim\limits_{\Delta x\to 0}\frac{f(x_0+\Delta x)-f(x_0)}{\Delta x}\cdot\lim\limits_{\Delta x\to 0}\Delta x +f(x_0)\\ 
    &= f'(x_0)\cdot 0+f(x_0)= f(x_0).
  \end{align*}
\end{proof}

\medskip\noindent
\begin{minipage}{0.5\linewidth}\parindent2em
但是,如果函数 $f(x)$ 在点 $x_0$ 连续,$f(x)$ 在该点不一定可导。
例如 $y=|x|$ 在点 $x=0$ 连续,但在点 $x=0$ 处不可导。
从图形上看,就是曲线 $y=f(x)$ 在点 $O\,(0,0)$ 处没有切线(\cref{fig:2-3})。
\end{minipage}\hfill
\begin{minipage}{0.45\linewidth}\centering
  \begin{figurehere}
    \includegraphics{2-3.pdf}
    \caption{}\label{fig:2-3}
  \end{figurehere}
\end{minipage}

\medskip
下面我们根据导数的定义证明 $y=|x|$ 在 $x=0$ 处不可导。并从而导出左、右导数的概念。
\[\because \Delta y=|0+\Delta x|-|0|=|\Delta x|=\begin{cases}
  \phantom{-}\Delta x,\quad\text{当\ }\Delta x>0,\\
  -\Delta x,\quad\text{当\ }\Delta x<0,\\
\end{cases}
\]
\begin{align*}
  \therefore\quad \lim\limits_{\Delta x\to 0+}\frac{\Delta y}{\Delta x}&=\lim\limits_{\Delta x\to 0+}\frac{\Delta x}{\Delta x}=1,\\
  \lim\limits_{\Delta x\to 0-}\frac{\Delta y}{\Delta x}&=\lim\limits_{\Delta x\to 0-}\frac{-\Delta x}{\Delta x}=-1.
\end{align*}
也就是说,当 $\Delta x\to 0$ 时,$\dfrac{\Delta y}{\Delta x}$ 的左、右极限不相等,所以 $\dfrac{\Delta y}{\Delta x}$ 当 $\Delta x \to 0$ 时极限不存在。
因此,函数 $y=|x|$ 在点 $x=0$ 处不可导。

一般地,设已知函数 $y=f(x)$,$\Delta y=f(x_0+\Delta x)-f(x_0)$,如果 $\dfrac{\Delta y}{\Delta x}$ 的左极限存在,就把左极限 $\lim\limits_{\Delta x\to 0-}\dfrac{\Delta y}{\Delta x}$ 叫做 $f( x)$ 在点 $x_0$ 处的\Concept{左导数};如果 $\dfrac{\Delta y}{\Delta x}$ 的右极限存在,就把右极限 $\lim\limits_{\Delta x\to 0+}\dfrac{\Delta y}{\Delta x}$ 叫做 $f(x)$ 在点 $x_0$ 处的\Concept{右导数}。

根据左、右极限存在且相等是极限存在的充要条件,可得\emph{左、右导数存在且相等是导数存在的充要条件}。

如果函数 $y =f(x)$ 在开区间 $(a,b)$ 内可导,在左端点 $x=a$ 处存在右导数,在右端点 $x=b$ 处存在左导数,我们就说函数 $f(x)$ 在闭区间 $[a,b]$ 上可导。

\begin{Practice}
先从函数的图像观察,然后根据定义判断函数 $y=\sqrt[3]{x^2}$ 在点 $x=0$ 处是否连续,在点 $x=0$ 处是否可导。
\end{Practice}

\begin{Exercise}
  \begin{question}
    \item 已知作直线运动的某一物体的运动方程为 $s=\dfrac{5}{2}t^2$ (\unit{m}),当 $t=\qty{2}{s}$,$\Delta t$ 分别为 \qty{0.1}{s}、 \qty{0.01}{s}、\qty{0.001}{s}、 \qty{0.0001}{s}、\qty{0.00001}{s}时,求从 $t_0$ 到 $t_0+\Delta t$ 这段时间内的平均速度及 $t=\qty{2}{s}$ 时的瞬时速度。
    \item 已知质点按规律 $s=2t^2+4t$(\unit{m})作直线运动,求:
    \begin{tasks}
      \task 质点在运动开始后前 \qty{3}{s} 内的平均速度;
      \task 质点在 \qty{2}{s} 到 \qty{3}{s} 内的平均速度;
      \task 质点在 \qty{3}{s} 时的瞬时速度。
    \end{tasks}
    \item 求下列函数在指定点处的导数:
    \begin{tasks}[before-skip=5pt,after-skip=5pt](2)
      \task $y=(x-2)^2$,点 $x=2$;
      \task $y=\dfrac{1}{x-1}$,点 $x=0$。
    \end{tasks}
    \item 说明函数 $y=f(x)$ 在点 $x_0$ 处的导数也可定义为
    \[ f'(x_0)=\lim\limits_{x\to x_0}\frac{f(x)-f(x_0)}{x-x_0}.\]
    \item 求下列函数的导数:
    \begin{tasks}(4)
      \task $y=ax+b$;
      \task $y=\dfrac{1}{x}$;
      \task $y=\dfrac{1}{x^2}$;
      \task $y=\dfrac{1}{\sqrt{x}}$。
    \end{tasks}
    \item 已知 $f(x)=\dfrac{1}{1-x}$,求 $f'(x)$,$f'(0)$,$f'(2)$。
    \item 已知 $y=\sqrt{a^2-x^2}$,求证 $y'=-\dfrac{x}{\sqrt{a^2-x^2}}$。
    \item 设质点 $M$ 沿 $x$ 轴作变速直线运动,在时刻 $t$(\unit{s}),质点 $M$ 所在位置为 $x=t^2-5t+6$(\unit{m})。求从 \qty{1}{s} 到 \qty{3}{s} 这段时间内质点 $M$ 的平均速度。质点 $M$ 在什么时刻的速度等于这段时间内的平均速度?
    \item 求曲线 $y=2x-x^3$ 在点 $(-1,-1)$ 处的切线的倾斜角。
    \item 求抛物线 $y=\dfrac{1}{4}x^2$ 在点 $(-2,1)$ 及点 $(2,1)$ 处的切线方程和法线方程。
    \item 从时刻 $t=0$ 开始的 $t$\,\unit{s} 内,通过某导体的电量(单位:\unit{C})可由公式 $q=2t^2+3t$ 表示。求第 \qty{5}{s} 时的电流强度及第 \qty{7}{s} 时的电流强度(即通过的电量 $q$ 对时间 $t$ 的导数 $q'_t$),什么时刻电流强度达到 \qty{43}{A}(即 \unit{C/s})。
  \end{question}
\end{Exercise}

\section{求导方法}
\subsection{几种常见函数的导数}
为了能够较快地求出某个函数的导数,在下几节中我们将研究求导数的一般运算法则以及基本初等函数的导数公式。
这一节,我们根据导数的定义先来证明几个常见函数的导数公式。
\subsubsection{设 $y=C$($C$ 为常数),则 $y'=0$}
\begin{proof}
  \begin{align*}
    y&=f(x)=C\\
    \Delta y&=f(x+\Delta x)-f(x)=C-C=0\\
    \frac{\Delta y}{\Delta x}&=0,\\
    \therefore\quad f'(x)&=C'=\lim\limits_{\Delta x\to 0}\frac{\Delta y}{\Delta x}=0.
  \end{align*}
\end{proof}
\subsubsection{$(x^n)'=nx^n-1$($n$ 为正整数)}
\begin{proof}
  \begin{align*}
    y&=f(x)=x^n \\
    \Delta y&=f(x+\Delta x)-f(x)=(x+\Delta x)^n-x^n\\
    &= [x^n+C_n^1x^{n-1}\Delta x+C_n^2x^{n-2}(\Delta x)^2+\cdots+C_n^n(\Delta x)^n]-x^n\\
    % &= \\
    &= C_n^1x^{n-1}\Delta x +C_n^2x^{n-2}(\Delta x)^2+\cdots+C_n^n(\Delta x)^n,\\
    \frac{\Delta y}{\Delta x}&= C_n^1x^{n-1}+C_n^2x^{n-2}\Delta x+\cdots+C_n^n(\Delta x)^{n-1},\\
    \therefore y'&=(x^n)'=\lim\limits_{\Delta x\to 0}\frac{\Delta y}{\Delta x}=\lim\limits_{\Delta x\to 0}[C_n^1x^{n-1}+C_n^2x^{n-2}\Delta x+\cdots+C_n^n(\Delta x)^{n-1}]\\
    &=nx^{n-1}.
  \end{align*}
  例如,$(x^3)'=3x^2$,$(x^7)'=7x^6$,$(x)'=1x^0=1$。
\end{proof}
\subsubsection{$(\sin x)'=\cos x$}
\begin{proof}
  $y=\sin x$,
  \begin{align*}
    \Delta y&=\sin(x+\Delta x)-\sin x=2\cos\left(x+\frac{\Delta x}{2}\right)\sin\frac{\Delta x}{2},\\
    \frac{\Delta y}{\Delta x} & =\cos\left(x+\frac{\Delta x}{2}\right)\frac{\sin\dfrac{\Delta x}{2}}{\dfrac{\Delta x}{2}},
  \end{align*}
  \begin{align*}
    \because\quad \lim\limits_{\Delta x\to 0}\frac{\sin \dfrac{\Delta x}{2}}{\dfrac{\Delta x}{2}}& =1,\\
    \therefore\quad y'&=(\sin x)'=\lim\limits_{\Delta x \to 0}\frac{\Delta y}{\Delta x}\\
    &=\lim\limits_{\Delta x\to 0}\cos\left(x+\frac{\Delta x}{2}\right)\cdot \lim\limits_{\Delta x\to 0}\frac{\sin\dfrac{\Delta x}{2}}{\dfrac{\Delta x}{2}}=\cos x.
  \end{align*}
\end{proof}
\subsubsection{$(\cos x)'=-\sin x$}
请同学们自己证明。

\begin{Practice}
  (口答)求下列函数的导数:
  \begin{tasks}(4)
    \task $y=x^5$;
    \task $y=x^6$;
    \task $x=\sin t$;
    \task $u=\cos \varphi$。
  \end{tasks}
\end{Practice}

\subsection{函数的和、差、积、商的导数}
相应于函数极限的四则运算法则,我们根据导数的定义来导出求导数的四则运算法则,以简化求导数的计算。在下面的公式中,$u$ 及 $v$ 都是 $x$ 的函数,而且都是可导的。

\subsubsection{和(或差)的导数}
\begin{Theorem}{法则 1}
  两个函数的和(或差)的导数,等于这两个函数的导数的和(或差)。即
  \[\tcbhighmath{(u\pm v)'=u'\pm v'.}\]
\end{Theorem}
\begin{proof}
$y=f(x)=u(x)\pm v(x)$,
\begin{align*}
  \Delta y &=[u(x+\Delta x)\pm v(x+\Delta x)]-[u(x)\pm v(x)] \\
           &=[u(x+\Delta x)-u(x)]\pm[v(x+\Delta x)-v(x)] \\
           &=\Delta u\pm \Delta v,\\
  \frac{\Delta y}{\Delta x} &=\frac{\Delta u}{\Delta x}\pm\frac{\Delta v}{\Delta x},\\
  \lim\limits_{\Delta x\to 0}\frac{\Delta y}{\Delta x} &= \lim\limits_{\Delta x\to 0}\left(\frac{\Delta u}{\Delta x}\pm\frac{\Delta v}{\Delta x}\right)=\lim\limits_{\Delta x\to 0}\frac{\Delta u}{\Delta x}\pm\lim\limits_{\Delta x\to 0}\frac{\Delta v}{\Delta x},
\end{align*}
即
\[y'=(u\pm v)'=u'\pm v'.\]
\end{proof}

这个法则可以推广到任意有限个函数,即
\[(u_1\pm u_2\pm\cdots u_n)'=u'_1\pm u'_2\pm\cdots u'_n.\]

\begin{example}
  求 $y=x^3+\sin x$ 的导数。
\end{example}
\begin{solution}
  $y'=(x^3)'+(\sin x)'=3x^2+\cos x$。
\end{solution}

\begin{example}
  求 $y=x^4-x^2-x+3$ 的导数。
\end{example}
\begin{solution}
  $y'=4x^3-2x-1$。
\end{solution}

\subsubsection{积的导数}
\begin{Theorem}{法则 2}
  两个函数的积的导数,等于第一个函数的导数乘以第二个函数,加上第一个函数乘以第二个函数的导数。即
  \[\tcbhighmath{(uv)'=u'v+v'u.}\]
\end{Theorem}
\begin{proof}
  $y=f(x)=u(x)v(x)$,
  \begin{align*}
    \Delta y &= u(x+\Delta x)v(x+\Delta x)-u(x)v(x)\\
             &= u(x+\Delta x)v(x+\Delta x)-u(x)v(x+\Delta x)+u(x)v(x+\Delta x)-u(x)v(x)\\
    \frac{\Delta y}{\Delta x} &=\frac{u(x+\Delta x)-u(x)}{\Delta x}v(x+\Delta x)+u(x)\frac{v(x+\Delta x)-v(x)}{\Delta x}.
  \end{align*}

  因为 $v(x)$ 在点 $x$ 处可导,所以它在点 $x$ 处连续,于是当 $\Delta x\to 0$ 时,$v(x+\Delta x) \to v(x)$。从而
  \begin{align*}
    \lim\limits_{\Delta x\to 0}\frac{\Delta y}{\Delta x}&=\lim\limits_{\Delta x\to 0}\frac{u(x+\Delta x)-u(x)}{\Delta x}v(x+\Delta x)+u(x)\lim\limits_{\Delta x\to 0}\frac{v(x+\Delta x)-v(x)}{\Delta x}\\
    &=u'v+uv'
  \end{align*}
  即
  \[ y'=(uv)'=u'v+uv'.\]
\end{proof}

从法则 2 立即可以得出
\[(Cu)'=C'u+Cu'=0+Cu'=Cu',\]
也就是,常数与函数的积的导数,等于常数乘以函数的导数。即
\[\tcbhighmath{(Cu)'=Cu'.}\]

\begin{example}
  求 $y=2x^3-3x^2+5x-4$ 的导数。
\end{example}
\begin{solution}
  $y'=6x^2-6x+5$。
\end{solution}

\begin{example}
  求 $y =(2x^2+3)(3x-2)$ 的导数。
\end{example}
\begin{solution}
  $y'=(2x^2+3)'(3x-2)+(2x^2+3)(3x-2)'=4x\cdot(3x-2)+(2x^2+3)\cdot3=18x^2-8x+9$。
\end{solution}

\subsubsection{商的导数}
\begin{Theorem}{法则 3}
  两个函数的商的导数,等于分字的导数与分母的积,减去分母的导数与分子的积,再除以分母的平方。即
  \[\tcbhighmath{\left(\frac{u}{v}\right)'=\frac{u'v-uv'}{v^2}\quad (v\neq0).}\]
\end{Theorem}

\begin{proof}
  $y=f(x)=\dfrac{u(x)}{v(x)}$,
  \begin{align*}
    \Delta y &=\frac{u(x+\Delta x)}{v(x+\Delta x)}-\frac{u(x)}{v(x)}\\
      &=\frac{\left[u(x+\Delta x)v(x)-u(x)v(x)\right]-\left[u(x)v(x+\Delta x)-u(x)v(x)\right]}{v(x+\Delta x)v(x)}\\
      &=\frac{\left[u(x+\Delta x)-u(x)\right]v(x)-u(x)\left[v(x+\Delta x)-v(x)\right]}{v(x+\Delta x)v(x)}.\\
    \frac{\Delta y}{\Delta x}&=\frac{\dfrac{u(x+\Delta x)-u(x)}{\Delta x}v(x)-u(x)\dfrac{v(x+\Delta x)-v(x)}{\Delta x}}{u(x+\Delta x)v(x)}.
  \end{align*}

  因为 $v(x)$ 在点 $x$ 处可导,所以它在点 $x$ 处连续,于是当 $\Delta x\to 0$ 时,$v(x+\Delta x)\to v(x)$。从而
  \[\lim\limits_{\Delta x\to 0}\frac{\Delta y}{\Delta x}=\frac{u'(x)v(x)-u(x)v'(x)}{\left[v(x)\right]^2}.\]
  即
  \[y'=\left(\frac{u}{v}\right)'=\frac{u'v-uv'}{v^2}.\]
\end{proof}

\begin{example}
  求 $y = \dfrac{x^2}{\sin x}$ 的导数。
\end{example}
\begin{solution}
  $y'=\dfrac{(x^2)'\cdot\sin x-x^2\cdot(\sin x)'}{\sin^2x}=\dfrac{2x\sin x-x^2\cos x}{\sin^2x}$。
\end{solution}

\begin{example}
  求 $y=\dfrac{x+3}{x^2+3}$ 在点 $x=3$ 处的导数。
\end{example}
\begin{solution}
  $y'=\dfrac{1\cdot(x^2+3)-(x+3)\cdot 2x}{(x^2+3)^2}=\dfrac{-x^2-60+3}{(x^2+3)^2}$,
  \[ \therefore \quad y'\biggm|_{x=3}=\dfrac{-9-18+3}{(9+3)^2}=\frac{-24}{144}=-\frac16.\]
\end{solution}

\begin{example}
  求证当 $n$ 是负整数时,公式
  \[ (x^n)'=n(x^{n-1})\]
  仍然成立。
\end{example}
\begin{proof}
  设 $n=-m$,则 $m$ 为正整数。
  
  $\therefore \quad (x^n)'=(x^{-m})'=\left(\dfrac{1}{x^m}\right)=\dfrac{0\cdot x^m-mx^{m-1}}{x^{2m}}=-mx^{-m-1}=nx^{n-1}$。
\end{proof}

\begin{example}
  求 $y=2x^2-3x+4-\dfrac{3}{x}+\dfrac{2}{x^2}$ 的导数。
\end{example}
\begin{solution}
  $y=2x^2-3x+4-3x^{-1}+2x^{-2}$,
  
  $\therefore \quad y'=4x-3+3x^{-2}-4x^{-3}=4x-3+\dfrac{3}{x^2}-\dfrac{4}{x^3}$。
\end{solution}

\begin{Practice}
  \begin{question}
    \item 求下列函数的导数:
    \begin{tasks}[before-skip=5pt,after-skip=5pt,after-item-skip=5pt](2)
      \task $y=3x^4-23x^3+40x-10$;
      \task $y=ax^3-bx+c$;
      \task $y=\sin x-x+1$;
      \task $y=x^2+2\cos x$。
    \end{tasks}
    \item 填空:
    \begin{tasks}[before-skip=5pt,after-skip=5pt,after-item-skip=5pt]
      \task $[(3x^2+1)(4x^2-3)]'=(\qquad)(4x^2-3)+(3x^2+1)(\qquad)$;
      \task $(x^3\sin x)'=(\qquad)x^2\sin x+x^3(\qquad)$。
    \end{tasks}
    \item 求下列函数的导数:
    \begin{tasks}[before-skip=5pt,after-skip=5pt,after-item-skip=5pt](2)
      \task $y=(3x^2+1)(2-x)$;
      \task $y=(1-2x^3)(-3x^2)$;
      \task $y=(1+x^2)\cos x$;
      \task $y=(1+\sin x)(1-2x)$。
    \end{tasks}
    \item 填空:
    \begin{tasks}[before-skip=10pt,after-skip=10pt,after-item-skip=7pt]
      \task $\left(\dfrac{x}{x^2+1}\right)'=\dfrac{(\qquad)(x^2+1)-x(\qquad)}{(x^2+1)^2}$;
      \task $\left(\dfrac{1-x^2}{\sin x}\right)'=\dfrac{(\qquad)\sin x-(1-x^2)(\qquad)}{\sin^2x}$。
    \end{tasks}
    \item 求下列函数的导数:
    \begin{tasks}[before-skip=10pt,after-skip=10pt,after-item-skip=7pt](2)
      \task $y=\dfrac{a-x}{a+x}$;
      \task $y=\dfrac{1+x}{3-x^2}$;
      \task $y=\dfrac{\cos x}{1-x^2}$;
      \task $y=\dfrac{1}{1+\sin x}$;
      \task $y=1+\dfrac{2}{x}+\dfrac{3}{x^2}-\dfrac{4}{x^3}$;
      \task $y=\dfrac{-3x^4+3x^2-5}{x^2}$。
    \end{tasks}
    \item 下列做法是否正确?如果不正确,加以改正:
    \begin{tasks}[before-skip=5pt,after-skip=5pt,after-item-skip=7pt]
      \task $[(3+x^2)(2-x^3)]'=2x(2-x^3)+3x^2(3+x^2)$;
      \task $\left(\dfrac{1+\cos x}{x^2}\right)=\dfrac{2x(1+\cos x)+x^2\sin x}{x^2}$。
    \end{tasks}
  \end{question}
\end{Practice}

\subsection{复合函数的导数}
我们先看一个例子。设已知
\[ y=(3x-2)^2\]
那么,
\[ y'=[(3x-2)^2]'=(9x^2-12x+4)'=18x-12.\]

函数 $y=(3x-2)^2$ 又可以看成由
\[ y=u^2,\quad u=3x-2\]
复合而成的。由于
\[ y'_u=2u,\quad u'_x=3\]
因而
\[ y'_u\cdot u'_x=2u\cdot 3=2(3x-2)\cdot 4=18x-12.\]
于是在本例中,我们有等式
\[ y'_x=y'_u\cdot u'_x.\]

一般地,\emph{设函数 $u =\varphi(x)$ 在点 $x$ 处有导数 $u'_x=\varphi' (x)$,函数 $y=f(u)$ 在点 $x$ 的对应点 $u$ 处有导数 $y'_u= f'( u)$,则复合函数 $y=f\left[\varphi (x)\right]$ 在点 $x$ 处也有导数,且}
\[ \tcbhighmath{y'_x=y'_u\cdot u'_x.}\]
或写作 
\[ f'_x\left[\varphi(x)\right]=f'(u)\varphi'(x).\]

\begin{proof}
  设 $x$ 有一改变量 $\Delta x$,则对应的 $u$、$y$ 分别有改变量 $\Delta u$、$\Delta y$。
  因为 $u=\varphi(x)$ 在点 $x$ 处可导,所以 $u=\varphi(x)$ 在点 $x$ 处连续。
  因此当 $\Delta x\to 0$ 时,$\Delta u\to 0$。
  设 $\Delta u\neq0$\footnote{$\Delta u=0$ 时公式也成立,证明从略。},由
  \[ \frac{\Delta y}{\Delta x}=\frac{\Delta y}{\Delta u} \cdot \frac{\Delta u}{\Delta x},\]
  且
  \[\lim\limits_{\Delta x\to 0}\frac{\Delta y}{\Delta u}=\lim\limits_{\Delta u\to 0}\frac{\Delta y}{\Delta u},\]
  得
  \[\lim\limits_{\Delta x\to 0}\frac{\Delta y}{\Delta x}=\lim\limits_{\Delta x\to 0}\frac{\Delta y}{\Delta u}\cdot\lim\limits_{\Delta x\to 0}\frac{\Delta u}{\Delta x}=\lim\limits_{\Delta u\to 0}\frac{\Delta y}{\Delta u}\cdot\lim\limits_{\Delta x\to 0}\frac{\Delta u}{\Delta x}\]
  即
  \[y'_x=y'_u\cdot u'_x.\]
\end{proof}

这就是复合函数得求导法则,即:\emph{复合函数对自变量的导数,等于已知函数对中间变量的导数,乘以中间变量对自变量的导数}。

这个法则可以推广到两个以上的中间变量。例如,如果
\[y=y(u),\quad u=u(v),\quad v=v(x),\]
那么有
\[y'_x=y'_u\cdot u'_v\cdot v'_x.\]

\begin{example}
  求 $y=(2x+1)^5$ 的导数。
\end{example}
\begin{solution}
  设 $y=u^5$,$u=2x+1$。根据复合函数求导法则,有
  \[
    y'_x= y'_u\cdot u'_x=(u^5)'_u\cdot(2x+1)'_x= 5u^4\cdot 2=5(2x+1)^4\cdot 2=10(2x+1)^4. 
  \]
\end{solution}

\alertwarning{在利用复合函数的求导法则求导数后,要把中间变量换成自变量的函数。}

\begin{example}
  求 $y=\dfrac{1}{(1-3x)^4}$ 的导数。
\end{example}
\begin{solution}
  $y=\dfrac{1}{(1-3x)^4}=(1-3x)^{-4}$。

  设 $y=u^{-4}$,$u=(1-3x)$,则
  \begin{align*} 
    y'_x&=y'_u\cdot u'_x=(u^{-4})'_u\cdot(1-3x)'_x=-4u^{-5}\cdot(-3)\\
    &=12u^{-5}=12(1-3x)^{-5}=\frac{12}{(1-3x)^5}.
  \end{align*}
\end{solution}

\begin{example}
  求 $y=\sin^2\left(2x+\dfrac{\uppi}{3}\right)$ 的导数。
\end{example}
\begin{solution}
  设 $y=u^2$,$u=\sin v$,$v=2x+\dfrac{\uppi}{3}$,
  \begin{align*}
    y'_x&=y'_u\cdot u'_v\cdot v'_x=(u^2)'_u\cdot(\sin v)'_v\cdot\left(2x+\frac{\uppi}{3}\right)'_x\\
    &=2u\cdot\cos v\cdot 2=2\sin\left(2x+\frac{\uppi}{3}\right)\cdot\cos\left(2x+\frac{\uppi}{3}\right)\cdot 2\\
    &=2\sin\left(4x+\frac{2\uppi}{3}\right).
  \end{align*}
\end{solution}

求复合函数的导数,关键在于分析清楚函数的复合关系,适当选定中间变量,明确每次是哪个变量对哪个变量求导数。在熟练以后,就不必再写出中间步骤。如以上三例可分别直接写成
\begin{align*}
  y'&=[(2x+1)^5]'=5(2x+1)^4\cdot 2=10(2x+1)^4. \\
  y'&=[(1-3x)^{-4}]'=-4(1-3x)^{-5}\cdot(-3)=12(1-3x)^{-5} \\
  y'&=\left[\sin^2\left(2x+\frac{\uppi}{3}\right)\right]'=2\sin\left(2x+\frac{\uppi}{3}\right)\cos\left(2x+\frac{\uppi}{3}\right)\cdot 2=2\sin\left(4x+\frac{2\uppi}{3}\right).\\
\end{align*}

对经过多次复合及四则运算而成的复合函数,也可利用复合函数的求导法则,由外向里,逐层求导。

\begin{example}
  求 $y=(at- b\sin^2\omega t)^3$ 对 $t$ 的导数。
\end{example}
\begin{solution}
  \begin{align*}
    y'&=3(at-b\sin^2\omega t)^2\cdot(at-b\sin^2\omega t)'\\
      &=3(at-b\sin^2\omega t)^2[a-2b\sin\omega t\cdot(\sin\omega t)']\\
      &=3(at-b\sin^2\omega t)^2[a-2b\sin\omega t\cdot\cos\omega t\cdot(\omega t)']\\
      &=3(at-b\sin^2\omega t)^2(a-2b\sin\omega t\cdot\cos\omega t\cdot\omega)\\
      &=3(at-b\sin^2\omega t)^2(a-b\omega\sin2\omega t).
  \end{align*}
\end{solution}

熟练以后,也可省去中间步骤,直接写成
\begin{align*}
  y'&=3(at-b\sin^2\omega t)^2(a-2b\sin\omega t\cdot\cos\omega t\cdot\omega)\\
  &=3(at-b\sin^2\omega t)^2(a-b\omega\sin2\omega t).
\end{align*}

在\cref{subsec:power_dif} 我们将要证明,公式 $(x^\alpha)'=\alpha x^{\alpha-1}$ 对一切实数 $\alpha$ 都成立。
现在先运用这个公式和复合函数的求导法则来求一些无理函数的导数(中间步骤省略不写)。

\begin{example}
  求 $y=\sqrt[3]{ax^2+bx+c}$ 的导数。
\end{example}
\begin{solution}
  $y=\sqrt[3]{ax^2+bx+c}=(ax^2+bx+c)^{\frac13}$,
  \begin{align*}
    \therefore\quad y'&=\frac13(ax^2+bx+c)^{-\frac23}\cdot(2ax+b)\\
    &=\frac{2ax+b}{3\sqrt[3]{(ax^2+bx+c)^2}}.
  \end{align*}
\end{solution}

\begin{example}
  求 $y=(2x^2-3)\sqrt{1+x^2}$ 的导数。
\end{example}
\begin{solution}
  $y=(2x^2-3)\sqrt{1+x^2}=(2x^2-3)(1+x^2)^{\frac12}$,
  \begin{align*}
    \therefore\quad y'&=4x\cdot(1+x^2)^{\frac12}+(2x^2-3)\cdot\frac12(1+x^2)^{-\frac12}\cdot 2x \\
    &= 4x\sqrt{1+x^2}+\frac{x(2x^2-3)}{\sqrt{1+x^2}}\\
    &= \frac{4x(1+x^2)+x(2x^2-3)}{\sqrt{1+x^2}}\\
    &=\frac{6x^3+x}{\sqrt{1+x^2}}. 
  \end{align*}
\end{solution}

\begin{Practice}
  \begin{question}
    \item 把下列函数看成由一些比较简单的函数复合而成的,写出它们的复合过程:
    \begin{tasks}[before-skip=10pt,after-skip=10pt,after-item-skip=7pt](2)
      \task $y=(x^2-1)^3$;
      \task $y=\tan\left(\dfrac{\uppi}{4}-x\right)$;
      \task $y=\upe^{1+x^2}$;
      \task $y=\sin\dfrac{1}{\sqrt{1+x^2}}$。
    \end{tasks}
    \item 按例 1 中的步骤,对下列函数,先设中间变量,然后求导:
    \begin{tasks}[before-skip=10pt,after-skip=10pt,after-item-skip=7pt](2)
      \task $y=(5x-3)^4$;
      \task $y=(2-x^2)^3$;
      \task $y=\sin\left(3x-\dfrac{\uppi}{6}\right)$;
      \task $y=\cos(1+x^2)$。
    \end{tasks}
    \item 填空:
    \begin{tasks}[before-skip=10pt,after-skip=10pt,after-item-skip=7pt]
      \task $y=[(2x^3+x)^2]'=2(2x^3+x)(\qquad)$;
      \task $y=[(1+x^2)^2\sin(ax+b)]'=2(1+x^2)(\qquad)\sin(ax+b)+(1+x^2)^2\cos(ax+b)(\qquad)$;
      \task $y=[(1+\cos^2x)^3]'=3(1+\cos^2x)^2(\qquad)(\qquad)$;
      \task $y=\left[\frac{1}{(2+3x)^5}\right]'=[(2+3x)^{-5}]'=(\qquad)(2+3x)^{-6}(\qquad)$。
    \end{tasks}
    \item 求下列函数的导数:
    \begin{tasks}[before-skip=10pt,after-skip=10pt,after-item-skip=7pt](2)
      \task $y=\sin x^2-\sin3x$;
      \task $y=\dfrac{x^2}{(2x+1)^3}$;
      \task $y=x^2\sqrt{x}-\dfrac{1}{\sqrt{x}}$;
      \task $y=\sqrt{x^2-a^2}$;
      \task $y=\dfrac{1}{\sqrt[3]{x^2-1}}$;
      \task $y=\sqrt{4x+3}\cos2x$。
    \end{tasks}
  \end{question}
\end{Practice}

\begin{Exercise}
  \begin{question}
    \item 求下列函数的导数:
    \begin{tasks}[before-skip=10pt,after-skip=10pt,after-item-skip=7pt](2)
      \task $y=x^2\sin x+x^3$;
      \task $y=\dfrac{a^2-x^2}{a^2+x^2}$;
      \task $y=(2+3x)(1-x+x^2)$;
      \task $y=\dfrac{\cos x}{1-\sin x}$;
      \task $y=x^3\left(\sin x+\sin\dfrac{\uppi}{4}\right)$;
      \task $y=\dfrac{x-1}{x^2-3x+6}$。
    \end{tasks}
    \item 已知 $u$,$v$,$w$ 是 $x$ 的可导函数,求证
    \[(uvw)'=u'vw+uv'w+uvw'.\]
    \item 求下列函数在指定点处的导数:
    \begin{tasks}[before-skip=5pt,after-skip=5pt,after-item-skip=7pt]
      \task $y=x\sin x$ 在点 $x=\dfrac{\uppi}{4}$ 处;
      \task $y=\dfrac{2-3x^2}{1+2x}$ 在点 $x=1$ 处。
    \end{tasks}
    \item 求正弦函数 $y=\sin x$ 在点 $\left(\dfrac{\uppi}{6},\dfrac{1}{2}\right)$ 处的切线方程和法线方程。
    \item 已知曲线 $y=x^3+3x$,求这条曲线平行于直线 $y=15x+2$ 的切线的方程。
    \item 已知曲线 $y=2x^3+3x^2-12x+1$,求这条曲线的与 $x$ 轴平行的切线的方程。
    \item 已知曲线 $y=x^3+x^2-1$,在曲线上哪一点处作切线,它的倾斜角等于 \ang{45}?求在这点处的切线和法线的方程。
    \item 已知函数 $f(x)=x^2(x-1)$ 当 $x=x_0$ 时有 $f'(x_0)=f(x_0)$,求 $x_0$ 的值。
    \item 已知两个作直线运动的物体的运动方程 $s_1(t)=\dfrac13t^3$ 及 $s_2(t)=20t-4t^2$($t\geqslant 0$),在什么时刻它们运动的速度相等?
    \item 在直线轨道上运行的一列火车,从刹车到停车这段时间内,测得刹车后 $t$\,\unit{s} 内列车前进的距离 $s=27t-0.45t^2$(\unit{m})。这列车在刹车后几秒钟才停车?刹车后又运行了多少米?
    \item 求下列函数的导数:
    \begin{tasks}[before-skip=10pt,after-skip=10pt,after-item-skip=7pt](2)
      \task $y=x^2\sqrt{x}-3\sqrt{x}+\frac{1}{x\sqrt{x}}$;
      \task $y=\dfrac{3x^3-x^2+5x-2}{\sqrt[3]{x}}$。
    \end{tasks}
    \item 把下列函数看成由一些比较简单的函数复合而成的,写出它们的复合过程:
    \begin{tasks}[before-skip=10pt,after-skip=10pt,after-item-skip=7pt](2)
      \task $y=\dfrac{1}{\sqrt[5]{1+3x}}$;
      \task $y=\arcsin\dfrac{1-x}{1+x}$;
      \task $y=\lg\sin\sqrt{x}$;
      \task $y=\sqrt{3+\cos2x}$。
    \end{tasks}
    \item 对下列函数,先设中间变量,然后求导:
    \begin{tasks}[before-skip=10pt,after-skip=10pt,after-item-skip=7pt](3)
      \task $y=(ax+b)^n$;
      \task $y=\sqrt{2-x^2}$;
      \task $y=\sin^3(4x+3)$。
    \end{tasks}
    \item 求下列函数的导数:
    \begin{tasks}[before-skip=10pt,after-skip=10pt,after-item-skip=7pt](2)
      \task $y=(2x-1)^2(2-3x)^3$;
      \task $y=\dfrac{1}{(a+bx^2)^3}$;
      \task $y=\dfrac{x}{\sqrt{1+x}}$;
      \task $y=2\sin^2\dfrac{x}{2}-\sqrt{x}$;
      \task $y=\sqrt[3]{1+x^2}\sin5x$;
      \task $y=\left(\dfrac{x}{1+x}\right)^5$;
      \task $y=\dfrac{x}{2}\sqrt{a^2-x^2}$;
      \task $y=\dfrac{x}{\sqrt{x^2-a^2}}$。
    \end{tasks}
    \item 求下列曲线在指定点 $M$ 处的切线和法线的方程:
    \begin{tasks}[before-skip=5pt,after-skip=5pt,after-item-skip=7pt]
      \task $y=\frac35\sqrt{25-x^2}$,点 $M\,\left(4,\dfrac95\right)$;
      \task $y=x^2-4\sqrt{x}$,点 $M\,(1,-3)$。
    \end{tasks}
    \item 把 $y=\dfrac{u}{v}$(其中 $u$,$v$ 都是 $x$ 的函数,$v \neq 0$)改写成 $y=u{v}^{-1}$,利用积的求导法则和复合函数的求导法则,导出商的求导法则 $\left(\dfrac{u}{v}\right)'=\dfrac{u'v-uv'}{v^2}$。
  \end{question}
\end{Exercise}

\subsection{三角函数的导数}
\subsubsection{$(\sin x)'=\cos x$}
这一公式根据定义已证明。

现在,利用复合函数求导法则以及和、差、积、商的求导法则,可以简便地推导出余弦函数、正切函数及余切函数的导数公式。
\subsubsection{$(\cos x)'=-\sin x$}
\begin{proof}
  $(\cos x)'=\left[\sin\left(\dfrac{\uppi}{2}-x\right)\right]'=-\cos\left(\dfrac{\uppi}{2}-x\right)=-\sin x$。
\end{proof}
\subsubsection{$(\tan x)'=\sec^2 x$}
\begin{proof}
  $(\tan x)'=\left(\dfrac{\sin x}{\cos x}\right)'=\dfrac{\cos x\cos x-\sin x(-\sin x)}{\cos^2x}=\dfrac{1}{\cos^2x}=\sec^2x.$
\end{proof}
\subsubsection{$(\cot x)'=-\csc^2 x$}
这个公式由同学自己证明。

\begin{example}
  求证:
  \begin{align*}
    (\sec x)'&=\sec x\tan x;\\
    (\csc x)'&=-\csc x\cot x.
  \end{align*}
\end{example}
\begin{proof}
  \begin{align*}
    (\sec x)'&=[(\cos x)^{-1}]'=-(\cos x)^{-2}(-\sin x)\\ 
    &=\frac{1}{\cos x}\cdot\frac{\sin x}{\cos x}=\sec x\tan x.
  \end{align*}
  \begin{align*}
    (\csc x)'&=[(\sin x)^{-1}]'=-(\cos x)^{-2}(\cos x)\\ 
    &=-\frac{1}{\sin x}\cdot\frac{\cos x}{\sin x}=-\csc x\cot x.
  \end{align*}
\end{proof}

\begin{example}
  求 $y=\sin nx\sin^nx$ 的导数。
\end{example}
\begin{solution}
  \begin{align*}
    y'&=n\cos nx\cdot\sin^nx+\sin nx\cdot n\sin^{n-1}x\cos x\\
    &=n\sin^{n-1}x(\cos nx\sin x+\sin nx\cos x) \\
    &=n\sin^{n-1}x\sin(n+1)x.
  \end{align*}
\end{solution}

\begin{example}
  求 $y=\sin(x+\alpha)\sin(x-\alpha)$ 的导数。
\end{example}
\begin{solution}
  $y=-\dfrac12(\cos2x-\cos2\alpha)$。

  $y'=-\dfrac12(-\sin2x\cdot2)=\sin2x.$
\end{solution}

\begin{example}
  求 $y=\tan\sqrt{1-x}$ 的导数。
\end{example}
\begin{solution}
  \[
    y'=\sec^2\sqrt{1-x}\cdot(\sqrt{1-x})'=\sec^2\sqrt{1-x}\cdot\dfrac{-1}{2\sqrt{1-x}}=-\dfrac{\sec^2\sqrt{1-x}}{2\sqrt{1-x}}.
  \]
\end{solution}

\begin{Practice}
  求下列函数的导数:
  \begin{tasks}[before-skip=10pt,after-skip=10pt,after-item-skip=7pt](2)
    \task $f(\theta)=\dfrac{1+\cos\theta}{1-\cos\theta}$;
    \task $y=\cos x^2-\sin\sqrt{x}$;
    \task $f(\theta)=\tan\theta-\theta$;
    \task $y=\tan\dfrac{x}{2}-\cot\dfrac{x}{2}$。
  \end{tasks}
\end{Practice}

\subsection{反三角函数的导数}
反正弦函数与正弦函数互为反函数。
已知正弦函数的导数,能否从而求出反正弦函数的导数呢?

下面,我们先来研究互为反函数的两个函数的导数之间的关系。
\subsubsection{反函数的导数}
我们来看一个例子。设
\[y=2x-3,\]
则
\[y'_x=(2x-3)'=2\]
$y=2x-3$ 的反函数是 $x=\dfrac{y+3}{2}$(这里 $y$ 是自变量,$x$ 是 $y$ 的函数),
\[x'_y=\left(\frac{y+3}{2}\right)'_y=\frac12.\]
因此,在本例中我们有
\[y'_x=\frac{1}{x'_y}.\]

一般地,如果函数 $y=f(x)$ 与 $x=\varphi(y)$ 互为反函数,它们的导数是不是也有上述关系呢?

设已知函数 $y=f(x)$ 是 $x=\varphi(y)$ 的反函数,$y=f(x)$ 在点 $x$ 处连续,$x=\varphi(y)$ 在对应点 $y$ 处的导数不等于零。
给 $x$ 以改变量 $\Delta x$,相应地 $y=f(x)$ 就有改变量
\[\Delta y=f(x+\Delta x)-f(x).\]
当 $\Delta x\neq 0$ 时,一定有 $\Delta y\neq 0$,否则不等的两个值 $x$ 与 $x+\Delta x$ 将对应同一函数值 $y$,这和 “$y=f(x)$ 与 $x=\varphi(y)$ 互为反函数” 矛盾。因此
\[\frac{\Delta y}{\Delta x}=\dfrac{1}{\dfrac{\Delta x}{\Delta y}}.\]

由于 $y=f(x)$ 在点 $x$ 处连续,即当 $\Delta x\to 0$ 时,$\Delta y\to 0$,又由于 $x=\varphi(y)$ 在对应点 $y$ 处有不等于零的导数,所以
\[\lim\limits_{\Delta x\to 0}\frac{\Delta y}{\Delta x}=\lim\limits_{\Delta y\to 0}\frac{1}{\dfrac{\Delta x}{\Delta y}}=\frac{1}{\lim\limits_{\Delta y\to 0}\dfrac{\Delta x}{\Delta y}}=\frac{1}{x'_y},\]
即有
\[y'_x=\frac{1}{x'_y}.\]

于是,我们得到反函数的求导法则如下:
\begin{Theorem}{法则}
  已知函数 $y=f(x)$ 是函数 $x=\varphi(y)$ 的反函数,$y=f(x)$ 在点 $x$ 处连续,$x=\varphi(y)$ 在对应点 $y$ 处的导数不等于零,那么,$y=f(x)$ 在点 $x$ 处有导数,且
  \[\tcbhighmath{y'_x=\frac{1}{x'_y}.}\]
\end{Theorem}
或记作
\[f'(x)=\frac{1}{\varphi'(y)}.\]

\subsubsection{反三角函数的导数}
根据反函数的求导法则,我们可以得出反三角函数的导数公式如下:
\paragraph{$(\arcsin x)'=\dfrac{1}{\sqrt{1-x^2}}$}\mbox{}

\begin{proof}
  设 $y=\arcsin x$,($-1\leqslant x\leqslant 1$,则
  \[ x=\sin y\quad \left(-\frac{\uppi}{2}\leqslant y \leqslant \frac{\uppi}{2}\right).\]
  $y=\arcsin x$ 在 $(-1<x<1)$ 上连续,且当 $-\dfrac{\uppi}{2}<y<\dfrac{\uppi}{2}$ 时,$x'_y=\cos y>0$,这时,由反函数的求导法则,有
  \[y'_x=\frac{1}{x'_y}=\frac{1}{(\sin y)'}=\frac{1}{\cos y}=\frac{1}{\sqrt{1-\sin^2y}}=\frac{1}{1-x^2},\]
  即
  \[(\arcsin x)'=\frac{1}{\sqrt{1-x^2}}.\]
\end{proof}
\alertwarning{公式只在 $-1<x<1$ 时成立。当 $x=\pm1$ 时,对应的 $y$ 值是 $\pm\dfrac{\uppi}{2}$,这时 $x'_y=\cos y= 0$,不满足反函数求导法则要求的条件。}

\paragraph{$(\arccos x)'=-\dfrac{1}{\sqrt{1-x^2}}$}\mbox{}\par
这个公式由同学自己证明。
\paragraph{$(\arctan x)'=\dfrac{1}{1+x^2}$}\mbox{}\par
\begin{proof}
  设 $y=\arctan x$($-\infty<x<\infty$),则
  \[ x=\tan y \quad\left(-\frac{\uppi}{2}<y<\frac{\uppi}{2}\right).\]

  容易验证它满足反函数求导法则要求的条件,于是有
  \[y'_x=\frac{1}{x'_y}=\frac{1}{(\tan y)'}=\frac{1}{\sec^2 y}=\frac{1}{1+\tan^2y}=\frac{1}{1+x^2}\]
  即
  \[ (\arctan)'=\frac{1}{1+x^2}.\]
\end{proof}

\paragraph{$(\arccot x)'=-\dfrac{1}{1+x^2}$}\mbox{}\par
这个公式由同学自己证明。

\begin{example}
  求 $y=\arcsin\dfrac{x}{3}$ 的导数。
\end{example}
\begin{solution}
  $y'=\dfrac{1}{\sqrt{1-\left(\dfrac{x}{3}\right)^2}}\cdot\left(\dfrac{x}{3}\right)'=\dfrac{1}{\sqrt{9-x^2}}$。
\end{solution}

\begin{example}
  求 $y=\arcsin\dfrac{1-x^2}{1+x^2}$($x>0$)的导数。
\end{example}
\begin{solution}
    $y'=\dfrac{1}{\sqrt{1-\left(\dfrac{1-x^2}{1+x^2}\right)^2}}\cdot\left(\dfrac{1-x^2}{1+x^2}\right)'
    =\dfrac{1+x^2}{2x}\cdot\dfrac{-4x}{(1+x^2)^2}
    =-\dfrac{2}{1+x^2}.$
\end{solution}

\begin{example}
  求 $y=\dfrac{\arccos x}{\sqrt{1-x^2}}$ 的导数。
\end{example}
\begin{solution}
  $y'=\dfrac{-\dfrac{1}{\sqrt{1-x^2}}\cdot\sqrt{1-x^2}-\arccos x\cdot\dfrac{-x}{\sqrt{1-x^2}}}{1-x^2}=\dfrac{x\arccos x-\sqrt{1-x^2}}{\sqrt{(1-x^2)^3}}$
\end{solution}

\begin{example}
  求 $y=\arctan2x$ 的导数。
\end{example}
\begin{solution}
  $y'=\dfrac{1}{1+(2x)^2}\cdot(2x)'=\dfrac{2}{1+4x^2}$。
\end{solution}

\begin{example}
  求 $y=\arctan\dfrac{x}{1-x^2}$ 的导数。
\end{example}
\begin{solution}
  $y'=\dfrac{1}{1+\left(\dfrac{x}{1-x^2}\right)^2}\cdot\left(\dfrac{x}{1-x^2}\right)'=\dfrac{(1-x^2)^2}{1-x^2+x^4}\cdot\dfrac{1+x^2}{(1-x^2)^2}=\dfrac{1+x^2}{1-x^2+x^4}$。
\end{solution}

\begin{Practice}
  求下列函数的导数:
  \begin{tasks}[before-skip=10pt,after-skip=10pt,after-item-skip=7pt](2)
    \task $y=\arcsin\dfrac{x}{a}$;
    \task $y=x\arcsin x$;
    \task $y=2\arcsin x^2$;
    \task $y=\arccos\dfrac{x}{2}$;
    \task $y=\arctan\dfrac{x}{a}$;
    \task $y=(\arccot x)^2$。
  \end{tasks}
\end{Practice}

\subsection{对数函数的导数}
\subsubsection{$(\ln x)'=\dfrac{1}{x}$}
\begin{proof}
  $y=f(x)=\ln x$。
  \begin{align*}
    \Delta y&=\ln(x+\Delta x)-\ln x=ln\dfrac{x+\Delta}{x}=\ln\left(1+\dfrac{\Delta x}{x}\right), \\
    \frac{\Delta y}{\Delta x}&=\frac{1}{\Delta x}\ln\left(1+\frac{\Delta x}{x}\right)=\frac{1}{x}\cdot\frac{x}{\Delta x}\ln\left(1+\frac{\Delta x}{x}\right)=\frac{1}{x}\ln\left(1+\frac{\Delta x}{x}\right)^{\frac{x}{\Delta x}},\\
    \lim\limits_{\Delta x\to 0}\frac{\Delta y}{\Delta x}&=\frac{1}{x}\lim\limits_{\Delta x\to 0}\ln\left(1+\frac{\Delta x}{x}\right)^{\frac{x}{\Delta x}}.
  \end{align*}

  令 $\alpha=\dfrac{\Delta x}{x}$,则当 $\Delta x\to 0$ 时,$\alpha \to 0$,从而
  \[ \lim\limits_{\Delta x\to 0}\left(1+\frac{\Delta x}{x}\right)^{\frac{x}{\Delta x}}=\lim\limits_{\alpha\to 0}(1+\alpha)^{\frac{1}{\alpha}}=\upe.\]
  令 $\left(1+\dfrac{\Delta x}{x}\right)^{\frac{x}{\Delta x}}= u$,根据上式,当 $\Delta x\to 0$ 时,$u\to\upe$。由于对数函数是连续函数,$\ln u$ 在点 $u=\upe$ 处连续,于是有
  \[ \lim\limits_{\Delta x\to 0}\ln\left(1+\frac{\Delta x}{x}\right)^{\frac{x}{\Delta x}}=\lim\limits_{u\to\upe}\ln u=\ln\upe.\]
  所以
  \[y'=\lim\limits_{\Delta x\to 0}\frac{\Delta y}{\Delta x}=\frac{1}{x}\cdot\lim\limits_{\Delta x\to 0}\ln\left(1+\frac{\Delta x}{x}\right)^{\frac{x}{\Delta x}}=\frac{1}{x}\ln\upe=\frac{1}{x}.\]
\end{proof}
\subsubsection{$(\log_a x)'=\dfrac{1}{x\ln a}=\dfrac{\log_a \upe}{x}$}
\begin{proof}
  $(\log_a x)'=\left(\dfrac{\ln x}{\ln a}\right)'=\dfrac{1}{\ln a}\cdot\dfrac{1}{x}=\dfrac{1}{x\ln a}.$
  \begin{align*}
    \because\quad \ln a&=\frac{\log_a a}{\log_a \upe}=\frac{1}{\log_a\upe},\\
    \therefore\quad (\log_a x)'&=\frac{\log_a\upe}{x}.
  \end{align*}
\end{proof}

\begin{example}
  求 $y=\ln(2x^2+3x+1)$ 的导数。
\end{example}
\begin{solution}
  $y'=\dfrac{1}{2x^2+3x+1}\cdot(2x^2+3x+1)'=\dfrac{4x+3}{2x^2+3x+1}$。
\end{solution}

\begin{example}
  求证 $(\ln|x|)'=\dfrac{1}{x}$。
\end{example}
\begin{proof}
  当 $x>0$ 时,$y=\ln x$,
  \[y'=\frac1x;\]

  当 $x<0$ 时,$y=\ln(-x)$,
  \[y'=\frac{1}{-x}\cdot(-x)'=\dfrac1x.\]

  所以,不论 $x>0$ 或 $x<0$,都有
  \[(\ln|x|)'=\frac1x.\]
\end{proof}

\begin{example}\label{exp:lgsqrt}
  求 $y=\lg\sqrt{1-x^2}$ 的导数。
\end{example}
\begin{solution}
  \begin{enumerate}[label={解法 \arabic*:},leftmargin=2em]
    \item $y'=\dfrac{\lg\upe}{\sqrt{1-x^2}}(\sqrt{1-x^2})'=\dfrac{\lg\upe}{\sqrt{1-x^2}}\cdot\dfrac{-x}{\sqrt{1-x^2}}=\dfrac{x\lg\upe}{x^2-1}$。
    \item $y=\lg\sqrt{1-x^2}=\dfrac12\lg(1-x^2)$。\par$y'=\dfrac12\cdot\dfrac{\lg\upe}{1-x^2}\cdot(1-x^2)'=\dfrac{x\lg\upe}{x^2-1}$。
  \end{enumerate}
\end{solution}

\medskip
从\cref{exp:lgsqrt} 可以看出,求对数函数的导数,有时先把所给对数函数变形,然后再求导数,做起来要简便一些。

\begin{example}
  求 $y=\ln\sqrt{\dfrac{1+x^2}{1-x^2}}$ 的导数。
\end{example}
\begin{solution}
  $y=\ln\sqrt{\dfrac{1+x^2}{1-x^2}}=\dfrac12[\ln(1+x^2)-\ln(1-x^2)]$。

  $y'=\dfrac{1}{2}\left(\dfrac{2x}{1+x^2}-\dfrac{-2x}{1-x^2}\right)=\dfrac{2x}{1-x^4}$
\end{solution}

\begin{example}
  求 $y=\ln\arctan\left(\dfrac{t}{2}+\dfrac{\uppi}{4}\right)$ 的导数。
\end{example}
\begin{solution}
  \begin{align*}
    y'&=\dfrac{1}{\tan\left(\dfrac{t}{2}+\dfrac{\uppi}{4}\right)}\left[\tan\left(\dfrac{t}{2}+\dfrac{\uppi}{4}\right)\right]'=\dfrac{1}{\tan\left(\dfrac{t}{2}+\dfrac{\uppi}{4}\right)}\cdot\dfrac{1}{\cos^2\left(\dfrac{t}{2}+\dfrac{\uppi}{4}\right)}\cdot\dfrac{1}{2}\\
    &=\dfrac{1}{\sin\left(t+\dfrac{\uppi}{2}\right)}=\dfrac{1}{\cos t}=\sec t
  \end{align*}
\end{solution}

\begin{Practice}
  求下列函数的导数:
  \begin{tasks}[before-skip=10pt,after-skip=10pt,after-item-skip=7pt](2)
    \task $y=x\ln x$;
    \task $y=\ln\dfrac{1+3x^2}{2-x^2}$;
    \task $y=\log_a(2x^3+3x^2)$;
    \task $y=\ln\sqrt{\dfrac{1+x}{1-x}}$;
    \task $y=\lg(1+\cos x)$;
    \task $y=\ln(\ln x)$。
  \end{tasks}
\end{Practice}

\subsection{指数函数的导数}
\subsubsection{$(\upe^x)'=\upe^x$}
\begin{proof}
  指数函数 $y=\upe^x$ 与对数函数 $x=\ln y$ 互为反函数。

  根据对数函数的导数公式,有
  \[x'_y=(\ln y)'_y=\frac{1}{y},\]
  因此,根据反函数的求导法则,有
  \[y'_x=\frac{1}{x'_y}=y=\upe^x,\]
  即
  \[(\upe^x)'=\upe^x.\]
\end{proof}
\subsubsection{$(a^x)'=a^x\ln a$}
\begin{proof}
  $\because \quad a^x=(\upe^{\ln a})^x=\upe^{x\ln a}$,

  $\therefore \quad (a^x)'=(\upe^{x\ln a})'=\upe^{x\ln a}\cdot(x\ln a)'=a^x\ln a$。
\end{proof}

\begin{example}
  求 $y=x^3\upe^x$ 的导数。
\end{example}
\begin{solution}
  $y'=3x^2\upe^x+x^3\upe^x=(3+x)x^2\upe^x$。
\end{solution}

\begin{example}
  求 $y=\upe^{3x}$ 的导数。
\end{example}
\begin{solution}
  $y'=\upe^{3x}\cdot 3=3\upe^{3x}$
\end{solution}

\begin{example}
  求 $y=\upe^{ax}\cos bx$ 的导数。
\end{example}
\begin{solution}
  $y'=\upe^{ax}\cdot a\cdot\cos bx+\upe^{ax}(-\sin bx\cdot b)=\upe^{ax}(a\cos bx-b\sin bx)$。
\end{solution}

\begin{example}
  求 $y=a^{5x}$ 的导数。
\end{example}
\begin{solution}
  $y'=a^{5x}\ln a\cdot(5x)'=5a^{5x}\ln a$。
\end{solution}

\begin{Practice}
  求下列函数的导数:
  \begin{tasks}[before-skip=10pt,after-skip=10pt,after-item-skip=7pt](2)
    \task $y=e^x\sin x$;
    \task $y=\dfrac{e^x-1}{e^x+1}$;
    \task $y=x^ne^{-x}$;
    \task $y=\dfrac{a}{2}(e^{\frac{x}{a}}-e^{-\frac{x}{a}})$;
    \task $y=x^3+3^x$;
    \task $y=2^xe^x$;
    \task $y=e^{2x}\ln x$;
    \task $y=e^{x^2+1}$。
  \end{tasks}
\end{Practice}

\subsection{幂函数的导数}\label{subsec:power_dif}
当 $\alpha$ 为任意实数时,有公式
\[(x^\alpha)'=\alpha x^{\alpha-1}.\]
\begin{proof}
  当 $\alpha$ 为任意实数时,我们只考虑 $x>0$。这时,
  \begin{align*} 
    x^\alpha&=(\upe^{\ln x})^\alpha\\
    (x^\alpha)'&=(\upe^{\alpha\ln x})'=\upe^{\alpha\ln x}(\alpha\ln x)'=\upe^{\alpha\ln x}\cdot\alpha\cdot\frac{1}{x}
    =x^\alpha\cdot\alpha\cdot\frac{1}{x}=\alpha x^{\alpha-1}.
  \end{align*}
\end{proof}

这就是一般幂函数的导数公式。

\begin{example}
  求 $y={x}^{-\frac13}\left(1-x^{\frac83}\right)$ 的导数。
\end{example}
\begin{solution}
  $y=x^{-\frac13}-x^{\frac73}$。\par\medskip
  $y'=-\dfrac13x^{-\frac43}-\dfrac73x^{\frac43}$。
\end{solution}

\begin{example}
  求 $y=\sqrt{(x^2-a^2)^3}$ 的导数。
\end{example}
\begin{solution}
  $y=(x^2-a^2)^{\frac32}$。\par\medskip
  $y'=\dfrac32(x^2-a^2)^{\frac12}\cdot 2x=3x\sqrt{x^2-a^2}$。
\end{solution}

\begin{example}
  求 $y=\sqrt[\uproot{10}\leftroot{-3}5]{\dfrac{x}{1-x}}$ 的导数。
\end{example}
\begin{solution}
  $y=\left(\dfrac{x}{1-x}\right)^{\frac15}$。
  
  $y'=\dfrac15\left(\dfrac{x}{1-x}\right)^{-\frac45}\cdot\left(\dfrac{x}{1-x}\right)'=\dfrac15\left(\dfrac{x}{1-x}\right)^{-\frac45}\cdot\dfrac{1}{(1-x)^2}=\dfrac15x^{-\frac45}(1-x)^{-\frac65}$。
\end{solution}

\subsection*{导数公式表}
到目前为止,我们已经学习了基本初等函数的导数公式、求导数的四则运算法则以及复合函数的求导法则,这样,我们也就学会了求初等函数的导数的一般方法。

为便于查阅和记忆,我们把学过的导数公式列表如下:

\begin{center}\bfseries 导数公式表\end{center}
\begin{tabbing}
  \hspace*{2em}\=(12)\hspace{0.5em}\= $y=x^\alpha$($\alpha$ 是实数),\quad \=  $y'=\dfrac{1}{x\ln a}=\dfrac{\log_a\upe}{x}$;\kill
  \>(1)\>$y=C$, \>$y'=0$;\\
  \>(2)\>$y=x^\alpha$($\alpha$ 是实数),\>$y'=\alpha x^{\alpha-1}$;\\
  \>(3)\>$y=\log_a x$,\>$y'=\dfrac{1}{x\ln a}=\dfrac{\log_a\upe}{x}$;\\[7pt]
  \>\>$y=\ln x$,\>$y'=\dfrac{1}{x}$;\\
  \>(4)\>$y=a^x$,\>$y'=a^x\ln a$;\\
  \>\>$y=\upe^x$,\>$y'=\upe^x$;\\
  \>(5)\>$y=\sin x$,\>$y'=\cos x$;\\
  \>(6)\>$y=\cos x$,\>$y'=-\sin x$;\\
  \>(7)\>$y=\tan x$,\>$y'=\sec^2 x$;\\
  \>(8)\>$y=\cot x$,\>$y'=-\csc^2 x$;\\
  \>(9)\>$y=\arcsin x$,\>$y'=\dfrac{1}{\sqrt{1-x^2}}$;\\[7pt]
  \>(10)\>$y=\arccos x$,\>$y'=-\dfrac{1}{\sqrt{1-x^2}}$;\\[7pt]
  \>(11)\>$y=\arctan x$,\>$y'=\dfrac{1}{1+x^2}$;\\[7pt]
  \>(12)\>$y=\arccot x$,\>$y'=-\dfrac{1}{1+x^2}$。
\end{tabbing}

\begin{Practice}
  求下列函数的导数:
  \begin{tasks}[before-skip=10pt,after-skip=10pt,after-item-skip=7pt](2)
    \task $y=x^{\frac{2}{3}}-2x^{-\frac{1}{2}}+5x^{\frac{7}{6}}$;
    \task $y=\left(\dfrac{1}{\sqrt[\uproot{2}3]{x^2}}+\dfrac{1}{\sqrt{x}}\right)^2$;
    \task $y=\sqrt[\uproot{2}3]{(4-3x^2)^2}$;
    \task $y=\sqrt[\uproot{10}\leftroot{-2}3]{\dfrac{x-a}{x+a}}$。
  \end{tasks}
\end{Practice}

\begin{Exercise}
  \begin{question}
    \item 求下列函数的导数:
    \begin{tasks}[before-skip=10pt,after-skip=10pt,after-item-skip=7pt](2)
    \task $y=\dfrac{\sin x}{1+\cos x}$;
    \task $y=\sin3x\cos2x$;
    \task $y=\dfrac{\sin\left(2x-\dfrac{\uppi}{4}\right)}{\sin\left(2x+\dfrac{\uppi}{4}\right)}$;
    \task $y=\dfrac{1}{3}\tan^3x-\tan x+x$;
    \task $y=\tan x-\sec x$;
    \task $y=\cot x+\csc x$;
    \task $y=\tan\left(\dfrac{\uppi}{4}-\dfrac{x}{2}\right)$;
    \task $y=\sin^nx\cos nx$;
    \task $y=\sqrt{\tan\dfrac{x}{2}}$;
    \task $y=\sin^2\sqrt{1+x^2}$。
  \end{tasks}
    \item 先把下列函数变形成较易求导数的形式,再求导数:
    \begin{tasks}[before-skip=10pt,after-skip=10pt,after-item-skip=7pt](2)
    \task! $y=\sin mx\cos nx+\cos mx\sin nx$;
    \task $y=\dfrac{2\tan x}{1-\tan^2x}$;
    \task $y=\dfrac{2\tan x}{1+\tan^2x}$;
    \task $y=\dfrac{1-\tan^2x}{1+\tan^2x}$;
    \task $y=1-4\sin^2x\cos^2x$。
  \end{tasks}
    \item 求正切曲线 $y=\tan x$ 在点 $M\,\left(\dfrac{\uppi}{4},1\right)$ 处的切线方程和法线方程。
    \item 求下列函数的导数:
    \begin{tasks}[before-skip=10pt,after-skip=10pt,after-item-skip=7pt](2)
    \task $y=\dfrac{\arcsin x}{\sqrt{1-x^2}}$;
    \task $y=\arccos(1-x)$;
    \task $y=(4+x^2)\arctan\dfrac{x}{2}$;
    \task $y=\arccot x^2$;
    \task! $y=x\sqrt{a^2-x^2}+a^2\arcsin\dfrac{x}{a}$($a>0$);
    \task $y=x+\dfrac{8x}{x^2+4}-4\arctan\dfrac{x}{2}$。
  \end{tasks}
    \item 求下列函数的导数:
    \begin{tasks}[before-skip=10pt,after-skip=10pt,after-item-skip=7pt](2)
    \task $y=x\ln^2x$;
    \task $y=\dfrac{1-\ln x}{1+\ln x}$;
    \task $y=\ln\dfrac{1-\sin x}{1+\sin x}$;
    \task $y=x\arctan x-\dfrac12\ln(1+x^2)$;
    \task $y=x\log_3 x$;
    \task $y=\lg(x+\sqrt{1+x^2})$。
  \end{tasks}
    \item 已知物体的运动方程是 $s=10\ln\dfrac{4}{t+4}$,求 $t=1$ 及 $t=10$ 时的瞬时速度($s$ 的单位是 \unit{m},$t$ 的单位是 \unit{s})。
    \item $a$ 等于什么数时,曲线 $y=\ln(x-7a)+\arctan ax$ 在点 $x=1$ 处的切线平行于 $x$ 轴?
    \item 求下列函数的导数:
    \begin{tasks}[before-skip=10pt,after-skip=10pt,after-item-skip=7pt](2)
    \task $y=\dfrac{\upe^x-\upe^{-x}}{\upe^x+\upe^{-x}}$;
    \task $y=x^na^x$;
    \task $y=\upe^{-3x}\sin2x$;
    \task $y=\dfrac{1+x}{2^x}$;
    \task $y=\upe^{-\frac1x}$;
    \task $y=a^{2x}$。
  \end{tasks}
    \item 求下列函数的导数:
    \begin{tasks}[before-skip=10pt,after-skip=10pt,after-item-skip=7pt](2)
    \task $y=x\sqrt[5]{6x-1}$;
    \task $y=\dfrac{x}{\sqrt{x^2+x+1}}$;
    \task $y=x^a+a^x$;
    \task $y=\sqrt{\dfrac{a^2-x^2}{a^2+x^2}}$;
    \task $y=\dfrac{2a-x}{\sqrt[3]{a-x}}$;
    \task $y=\sqrt[3]{\dfrac{x+1}{x+4}}$。
  \end{tasks}
  \end{question}
\end{Exercise}

\subsection{隐函数的导数}
如果要求椭圆 $\dfrac{x^2}{a^2}+\dfrac{y^2}{b^2}=1$ 上一点 $(x,y)$ 处的切线的方程,就要求出切线的斜率 $y'_x$。
当然我们可以从方程中解出 $y=\pm\dfrac{b}{a}\sqrt{a^2-x^2}$,再求 $y'_x$。
但是,有时解方程很麻烦,而且有些方程,例如 $xy-\upe^x+\upe^y=0$,就不能用 $x$ 的初等函数把 $y$ 表示出来。

如果变量 $x$,$y$ 之间的函数关系是由某一方程
\[F(x,y)=0\]
所确定,这样确定的函数叫做\Concept{隐函数}。
例如,由方程 $y^2-2px= 0$,$\dfrac{x^2}{a^2}+\dfrac{y^2}{b^2}=1$,$x^{\frac23}+y^{\frac23}=a^{\frac23}$ 等所确定的 $x,y$ 之间的函数关系(有时所确定的是几个函数关系)就是隐函数。
下面举例来说明求隐函数的导数的方法。

\begin{example}\mbox{}\par
  \begin{enumerate}
    \item 已知 $y^2=2px$,求 $y'_x$。
    \item 求证抛物线 $y^2=2px$ 上点 $(x_0,y_0)$ 处的切线的方程为 $y_0y=p(x+x_0)$。
  \end{enumerate}
\end{example}
\begin{solution}
  \begin{enumerate}
    \item 把 $y$ 看成 $x$ 的函数,则 $y^2$ 是 $x$ 的复合函数,运用复合函数的求导法则,在方程两边同时对 $x$ 求导:
    \begin{align*}
      (y^2)'_x&=(2px)'_x,\\
      2y\cdot y'_x&=2p,\\
      \therefore \quad y'_x&=\frac{p}{y}.
    \end{align*}
    这里的 $y$ 仍由方程 $y^2=2px$ 确定。

    (上式在分母不等于零的条件下成立,以后不再一一注明。)    
    \item 当 $x=x_0$,$y=y_0\neq 0$ 时,$y'_x=\dfrac{p}{y_0}$,所以所求的切线的方程为
    \[y-y_0=\frac{p}{y_0}(x-x_0),\]
    即
    \begin{align*}
      y_0y-y_0^2&=px-px_0.\\
      \therefore \quad y_0^2&=2px_0.\\
    \end{align*}
    所求的切线的方程为
    \[ y_0y-2px_0=px-px_0,\]
    即
    \[y_0y=p(x+x_0).\]

    当 $y_0=0$ 时,$x_0=\dfrac{y_0^2}{2p}=0$。抛物线 $y^2=2px$ 在点 $(0,0)$ 处的切线为 $y$ 轴,它的方程 $x=0$ 是方程 $y_0y=p(x+x_0)$ 的特殊形式。
  \end{enumerate}
\end{solution}

\begin{example}
  求证椭圆 $\dfrac{x^2}{a^2}+\dfrac{y^2}{b^2}=1$ 上点 ($x_0,y_0)$ 处的切线的方程为
  \[ \frac{x_0x}{a^2}+\frac{y_0y}{b^2}=1\]
\end{example}
\begin{proof}
  \begin{align*}
    \left(\frac{x^2}{a^2}+\frac{y^2}{b^2}\right)'_x&=(1)'_x,\\
    \frac{}{}+\frac{}{}&=0,\\
    \therefore y'_x&=-\frac{b^2x}{a^2y}.
  \end{align*}
  $y_0\neq 0$ 时,在点 $(x_0,y_0)$ 处的切线的方程为
  \[ y-y_0=-\frac{b^2x_0}{a^2y_0}(x-x_0),\]
  即
  \[ b^2x_0x+a^2y_0y=b^2x_0^2+a^2y_0^2.\]

  $\because\quad$ 点 $(x_0,y_0)$ 在椭圆 $\dfrac{x^2}{a^2}+\dfrac{y^2}{b^2}=1$ 上,
  \begin{align*}
    \therefore \frac{x_0^2}{a^2}+\frac{y_0^2}{b^2}&=1,\\
    b^2x_0^2+a^2y_9^2&=a^2b^2.
  \end{align*}
  所求的切线的方程为
  \[ b^2x_0x+a^2y_0y=a^2b^2,\]
  即
  \[ \frac{x_0x}{a^2}+\frac{y_0y}{b^2}=1.\]
\end{proof}

在点 $(-a,0)$ 和点 $(a,0)$ 处,切线的方程分别为 $x=-a$,$x=a$,它们是方程 $\dfrac{x_0x}{a^2}+\dfrac{y_0y}{b^2}=1$ 的特殊形式。

请同学们证明,双曲线 $\dfrac{x^2}{a^2}-\dfrac{y^2}{b^2}=1$ 上点 $(x_0,y_0)$ 处的切线的方程为 $\dfrac{x_0x}{a^2}-\dfrac{y_0y}{b^2}=1$.

\begin{Practice}
  \begin{question}
    \item 求曲线 $x^2+2xy-y^2=2x$ 在点 $(2,4)$ 处的切线的方程。
    \item 求曲线 $\sqrt{x}+\sqrt{y}=3$ 在点 $(1,4)$ 处的切线和法线方程。
    \item 解答:
    \begin{tasks}[before-skip=7pt,after-skip=7pt,after-item-skip=7pt]
      \task 写出椭圆 $9x^2+y^2=25$ 在点 $P\,(-1,-4)$ 处的切线和法线方程;
      \task 写出双曲线 $\dfrac{x^2}{18}-\dfrac{y^2}{4}=1$ 在点 $P\,(6,2)$ 处的切线和法线方程。
    \end{tasks}
    \item 求过点 $P\,(2,\sqrt{3})$ 且与椭圆 $\dfrac{x^2}{4}+ \dfrac{y^2}{9}=1$ 相切的切线方程时,直接利用公式 $\dfrac{x_0x}{a^2}+\dfrac{y_0y}{b^2}=1$,得出切线方程为 $\frac{2x}{4}+\frac{\sqrt{3}y}{9}=1$。这个结果对不对?为什么?
  \end{question}
\end{Practice}

\subsection{二阶导数}
我们知道,函数 $y=f(x)$ 的导数 $f'(x)$ 仍旧是 $x$ 的函数。如果 $f'(x)$ 可导,那么它的导数 $(f'(x))'$ 叫做 $f(x)$ 的二阶导数,记作 $f''(x)$ 或 $y''$。

速度 $v$ 是位移函数 $s=s(t)$ 对于时间 $t$ 的一阶导数:$v=s'(t)$。加速度 $a$ 是速度 $v=v(t)$ 对于时间 $t$ 的一阶导数:$a=v'(t)$。所以,加速度 $a$ 是位移函数 $s=s(t)$ 对于时间 $t$ 的二阶导数:
\[a=v'(t)=(s'(t))'=s''(t).\]

\begin{example}
  设 $y=2x^3-x^2+1$,求 $y''$。
\end{example}
\begin{solution}
  $y'=6x^2-2x$,$\therefore\quad y''=12x-2.$
\end{solution}

\begin{example}
  设 $y=\upe^{x}\cos x$,求 $y'|_{x=0}$,$y''|_{x=0}$。
\end{example}
\begin{solution}
  \begin{align*}
    y'&=\upe^x\cos x+\upe^x(-\sin x)=\upe^x(\cos x-\sin x),\\
    y''&=\upe^x(\cos x-\sin x)+\upe^x(-\sin x-\cos x)=-2\upe^x\sin x.\\
    \therefore \quad y'|_{x=0}&=1,\quad y''|_{x=0}=0
  \end{align*}
\end{solution}

$f(x)$ 的二阶导数的导数叫做 $f(x)$ 的三阶导数,记作 $f'''(x)$ 或 $y'''$。一般地,$f(x)$ 的 $n-1$ 阶导数的导数叫做 $f(x)$ 的 $n$ 阶导数。$y=f(x)$ 的 $n$ 阶导数记作 $f^{(n)}(x)$。

\begin{Practice}
  \begin{question}
    \item 某运动方程为 $s=2t^3-\dfrac12gt^2$,求 $t=2$ 时的加速度。
    \item 求下列函数的二阶导数:
    \begin{tasks}[before-skip=5pt,after-skip=5pt,after-item-skip=7pt](2)
      \task $y=ax^2+bx+c$;
      \task $y=x\ln x$;
      \task $y=\tan x$;
      \task $y=(1+x^2)\arctan x$。
    \end{tasks}
  \end{question}
\end{Practice}

\begin{Exercise}
  \begin{question}
    \item 写出椭圆 $4x^2+9y^2=36$ 在下列点处的切线方程:
    \begin{tasks}[before-skip=10pt,after-skip=10pt](2)
      \task $M_1\left(1,\dfrac{4}{3}\sqrt{2}\right)$;
      \task $M_2\left(\dfrac{3}{2},-\sqrt{3}\right)$。
    \end{tasks}
    \item 写出双曲线 $3x^2-y^2=1$ 在下列点处的切线方程:
    \begin{tasks}[before-skip=10pt,after-skip=10pt](2)
      \task $M_1\left(1,-\sqrt{2}\right)$;
      \task $M_2\left(\sqrt{3},2\sqrt{2}\right)$。
    \end{tasks}
    \item 求圆 $(x-1)^2+(y-2)^2=25$ 在点 $P\,(5,5)$ 处的切线和法线的方程。
    \item 求证曲线 $ax^2+2hxy+by^2=1$ 上点 $M\,(x_0,y_0)$ 处的切线的方程为 $ax_0x+h(y_0x+x_0y)+by_0y=1$。
    \item 求下列函数的二阶导数:
    \begin{tasks}[before-skip=10pt,after-skip=10pt,after-item-skip=7pt](2)
      \task $y=x\sqrt{1+x^2}$;
      \task $y=\upe^{-x^2}$;
      \task $y=x[\sin(\ln x)+\cos(\ln x)]$;
      \task $y=4(x-2)\upe^x+(x-1)\upe^{2x}$。
    \end{tasks}
    \item 如果物体的运动方程为 $s=t+\dfrac{1}{4}{t}^{3}$($s$ 的单位是 \unit{m}),求这一物体的初速度,并求出 $t=\qty{3}{s}$ 时,物体运动的速度及加速度。
    \item 一物体的运动方程为 $s=a\upe^t+b\upe^{-t}$,求这个物体运动的加速度。
    \item 质点 $M$ 作简谐运动,运动规律为
    \[x=A\sin\omega t\quad\text{(}A,\omega\text{ 是常数),}\]
    求质点 $M$ 的速度和加速度,并求质点 $M$ 到达点 $x=A$ 和点 $x=-A$ 时的速度和加速度。
  \end{question}
\end{Exercise}

\section{微分}
\subsection{微分概念}
\medskip\noindent
\begin{minipage}{0.6\linewidth}\parindent2em
在实际问题中,有时需要考虑:当自变量有较小的改变时,函数改变多少。如果函数很复杂,计算函数的改变量也就会很复杂。能不能找到一个既简便而又具有较好精确度的计算函数改变量的近似值的方法呢?下面先来分析一个实例。

设有边长为 $x$ 的正方形铁片,加热后边长增加了 $\Delta x$(\cref{fig:2-4}),求铁片的面积约增加多少。
\end{minipage}\hfill
\begin{minipage}{0.35\linewidth}
  \begin{figurehere}
    \includegraphics{2-4.pdf}
    \caption{}\label{fig:2-4}
  \end{figurehere}
\end{minipage}

\medskip
加热前铁片的面积为 $y=f(x)=x^2$,当边长增加了 $\Delta x$,铁片面积的增加量就是函数 $f(x)$ 的改变量
\begin{align}
  \Delta y & = (x+\Delta x)^2-x^2 \notag \\
           & = x^2+2x\cdot\Delta x +(\Delta x)^2-x^2 \notag \\
           & = 2x\cdot\Delta x +(\Delta x)^2.
\end{align}
$\Delta y$ 由两部分组成:一部分是 $\Delta x$ 的线性函数 $2x\cdot\Delta x$(\cref{fig:2-4});另一部分是 $(\Delta x)^2$(\cref{fig:2-4} 中双线阴影部分的面积)。

如果以 $2x\cdot\Delta x$ 作为 $\Delta y$ 的近似值,其误差为
\[ \Delta y -2x\cdot\Delta x=(\Delta x)^2.\]
这个误差 $(\Delta x)^2$ 显然随着 $|\Delta x|$(在这个实际问题中,$\Delta x>0$,可以去掉绝对值符号)的减小而减小,而且,$(\Delta x)^2$ 要比 $|\Delta x|$ 减小得更快些(例如,$|\Delta x|$ 从 0.1 减小到 0.01,$(\Delta x)^2$ 就相应地从 0.01 减小到 0.0001),当 $|\Delta x|$ 很小时,$(\Delta x)^2$ 比 $|\Delta x|$ 要小得多(例如 $|\Delta x|=10^{-5}$,则 $(\Delta x)^2=10^{-10}$)。因此式子 $\Delta y=2x\cdot\Delta x+(\Delta x)^2$ 右边的两项中,第一项 $2x \cdot\Delta x$ 是主要部分。当 $|\Delta x|$ 很小时,可认为铁片面积增加量
\[ \Delta y \approx 2x\cdot\Delta x.\]

这样我们可以用计算 $2x\Delta x$ 来代替计算 $y=x^2$ 的改变量 $\Delta y$,这比计算 $\Delta y$ 来得简便,且有一定的精确度。

由 $2x=(x^2)'=f'(x)$,于是在上例中有
\begin{equation}
  \label{eq:approx_delta_y}
  \Delta y\approx f'(x)\cdot\Delta x.
\end{equation}

一般地,设函数 $y=f(x)$ 在点 $x$ 处可导,则 $y=f(x)$ 在点 $x$ 处的导数 $f'(x)$ 与自变量的改变量 $\Delta x$ 的积叫做函数 $y=f(x)$ 在点 $x$ 处关于改变量 $\Delta x$ 的微分,简称函数 $y$ 的微分,记作 $\dif y$,即
\begin{equation}
  \label{eq:dy}
  \dif y=f'(x)\Delta x.
\end{equation}

因此,在 $y=x^2$ 时,
\begin{equation}
  \label{eq:approx_delta_y2}
  \Delta y\approx f'(x)\Delta x=\dif y.
\end{equation}
即函数的改变量 $\Delta y$ 可用它的微分近似地表示出来。
对于一般的可导函数也有同样的结果,我们就不证了。
这样,就可以把计算较为复杂的 $\Delta y$ 转化为计算 $\dif y$,即只要求出导数值 $f'(x)$ 再乘以 $\Delta x$ 就行了。

\begin{example}
  半径为 \qty{10}{cm} 的金属圆片加热后,半径伸长了 \qty{0.05}{cm},求此时刻面积的微分 $\dif A$ 与 $\Delta A-\dif A$ 的值。
\end{example}
\begin{solution}
  以 $A$ 表示圆片的面积,$r$ 表示圆片的半径,则
  \[A=\uppi r^2.\]
  根据题意,取 $r=10$,$\Delta r=0.05$。这时,
  \[\dif A=2\uppi r\cdot\Delta r=2\uppi\times 10\times 0.05=\uppi\,\unit{cm^2}.\]
  \begin{align*}
    \because   \Delta A        & = ,\\
    \therefore \Delta A-\dif A & = 2\uppi r\cdot\Delta r+\uppi(\Delta r)^2=2\uppi r\cdot \Delta r=\uppi(\Delta r)^2,\\
    & = \uppi(0.05)^2\\
    & = 0.0025\uppi\,\unit{cm^2}.
  \end{align*}
  答:面积的微分 $\dif A$ 为 $\uppi\,\unit{cm^2}$,$\Delta A-\dif A$ 为 $0.0025\uppi\,\unit{cm^2}$。
\end{solution}
在本例中,如果“加热”改为“冷却”,“伸长”改为“缩短”,这时,可取 $r=10$,$\Delta r=-0.05$,于是 $\dif A$ 为 $-\uppi\,\unit{cm^2}$,$\Delta A- \dif A$ 为 $-0.0025\uppi\,\unit{cm^2}$。

通常把自变量的改变量 $\Delta x$ 记作 $\dif x$,即 $\dif x=\Delta x$,称为自变量的微分。于是函数 $y=f(x)$ 的微分也可以写成
\begin{equation}
  \label{eq:dif_y_dif_x}
  \dif y=f'(x)\dif x.
\end{equation}

在\cref{eq:dif_y_dif_x} 两边同时除以 $\dif x$,得到 $f'(x)=\dfrac{\dif y}{\dif x}$。
这样,函数 $y=f(x)$ 的导数 $f'(x)$ 就等于函数的微分 $\dif y$ 与自变量的微分 $\dif x$ 的商,所以导数也叫做\Concept{微商}。
今后,我们也采用记号 $\dfrac{\dif y}{\dif x}$ 来表示函数 $y=f(x)$ 的导数 $f'(x)$,即
\[ \frac{\dif y}{\dif x}=f'(x)=y'_x=\lim\limits_{\Delta x\to 0}\frac{\Delta y}{\Delta x}.\]
我们还采用记号 $\dfrac{\dif^2y}{\dif x^2}$ 来表示二阶导数 $f''(x),\cdots \cdots$,采用记号 $\dfrac{\dif^n y}{\dif x^n}$ 来表示 $n$ 阶导数 $f^{(n)}(x)$ 等等。

下面我们来说明函数的微分的几何意义。

\noindent\begin{minipage}{0.6\linewidth}\parindent2em
设函数 $y=f(x)$ 在点 $x$ 处可导,如\cref{fig:2-5},在 $y=f(x)$ 所表示的曲线上取点 $P(x,y)$ 及它邻近的点 $P'(x+\Delta x,y+\Delta y)$,过点 $P$ 及 $P'$ 作 $MP$ 及 $M'P'$ 垂直于 $x$ 轴,分别交 $x$ 轴于点 $M$ 及 $M'$,过点 $P$ 作平行于 $x$ 轴的直线交 $M'P'$ 于点 $N$,又作曲线 $y=f(x)$ 在点 $P$ 处的切线,交 $M'P'$ 于点 $T$,则
\begin{align*}
  PN& = \Delta x,\quad NP'=\Delta y,\\
  NT& = f'(x)\Delta x=\dif y,\\
\end{align*}
\end{minipage}\hfill
\begin{minipage}{0.35\linewidth}
  \begin{figurehere}
    \includegraphics{2-5.pdf}
    \caption{}\label{fig:2-5}
  \end{figurehere}
\end{minipage}

所以,\emph{当自变量的改变量为 $\Delta x$ 时,$\Delta y$ 就是曲线的纵坐标的改变量,$\dif y$ 就是切线的纵坐标的改变量},这就是函数的微分的几何意义。
$\Delta y$ 与 $\dif y$ 的差的绝对值在图形上是 $|TP'|$,一般地,它是随着 $|\Delta x|$ 减小而减小,而且要比 $|\Delta x|$ 减小得更快些。
所以,当 $|\Delta x|$ 很小时,$\Delta y\approx\dif y$。
这时,可以用切线的纵坐标的改变量来代替曲线的纵坐标的改变量。
用 $\dif y$ 近似表示 $\Delta y$,相当于在点 $P\,(x,y)$ 附近用切线段 $PT$ 近似地代替曲线段 $PP'$。
这种在一定条件下以直代曲的方法是微分和积分中常用的典型方法。

\subsection{微分的运算}
由微分的表示式 $\dif y=f'(x)\dif x$ 知道,已知函数求微分时,只要求出函数的导数再乘以自变量的微分。
计算微分或导数的方法也叫做\Concept{微分法}。

\begin{example}
  求 $y=\sin x$ 的微分。
\end{example}
\begin{solution}
  $\dif y=(\sin x)'\dif x=\cos x\dif x.$
\end{solution}

\begin{example}
  求 $y=\arctan x$ 的微分。
\end{example}
\begin{solution}
  $\dif y=(\arctan x)'\dif x=\dfrac{\dif x}{1+x^2}.$
\end{solution}

根据导数的基本公式,利用 $\dif y=f'(x)\dif x$ 我们就可求出相应的微分公式。

\begin{center}\bfseries 微分公式表\end{center}
\begin{tasks}[label=(\arabic*),label-width=2em,before-skip=10pt,after-skip=10pt,after-item-skip=7pt](2)
  \task $\dif(C)=0$;
  \task $\dif(x^\alpha)=\alpha x^{\alpha-1}\dif x$;
  \task $\dif(\log_a x)=\dfrac{\dif x}{x\ln a}$;$\dif(\ln x)=\dfrac{\dif x}{x}$;
  \task $\dif(a^x)=a^x\ln a\dif x$;$\dif(\upe^x)=\upe^x\dif x$;
  \task $\dif(\sin x)=\cos x\dif x$;
  \task $\dif(\cos x)=-\sin x\dif x$;
  \task $\dif(\tan x)=\sec^2 x\dif x$;
  \task $\dif(\cot x)=-\csc^2 x\dif x$;
  \task $\dif(\arcsin x)=\dfrac{\dif x}{\sqrt{1-x^2}}$;
  \task $\dif(\arccos x)=-\dfrac{\dif x}{\sqrt{1-x^2}}$;
  \task $\dif(\arctan x)=\dfrac{\dif x}{1+x^2}$;
  \task $\dif(\arccot x)=-\dfrac{\dif x}{1+x^2}$。
\end{tasks}

同样,由求导数的四则运算法则,可以得出相应的求微分的四则运算法则:
\begin{enumerate}[itemsep=5pt]
  \item\label{itm:dif_1} $\dif(u\pm v)=\dif u\pm\dif v$。
  \item\label{itm:dif_2} $\dif(uv)=u\dif v+v\dif u$。
  
  请同学们自己证明 \ref{itm:dif_1}、\ref{itm:dif_2}。
  \item $\dif \left(\dfrac{u}{v}\right)=\dfrac{v\dif u-u\dif v}{v^2}$。
\end{enumerate}
\par\medskip
\begin{proof}
  \begin{align*}
    \dif\left(\frac{u}{v}\right)&=\left(\frac{u}{v}\right)'\dif x=\frac{vu'-uv'}{v^2}\dif x\\
    &=\frac{v(u'\dif x)-u(v'\dif x)}{v^2}=\frac{v\dif u-u\dif v}{v^2}.
  \end{align*}
\end{proof}
\begin{example}
  求 $y=\upe^{ax}\sin bx$ 的微分。
\end{example}
\begin{solution}
  \begin{align*}
    \dif y&=\upe^{ax}\dif(\sin bx)+\sin bx\dif(\upe^{ax})\\
    &=\upe^{ax}\cdot b\cos bx\dif x+\sin bx\cdot a\upe^{ax}\dif x\\
    &=\upe^{ax}(b\cos bx+a\sin bx)\dif x.
  \end{align*}
\end{solution}

\begin{Practice}
  \begin{question}
    \item 对于函数 $y=x^3-x$ 和下列的 $\Delta x$ 的值,求点 $x=2$ 处的 $\Delta y$ 和 $\dif y$:
    \begin{tasks}(4)
      \task $\Delta x=1$;
      \task $\Delta x=0.1$;
      \task $\Delta x=0.01$;
      \task $\Delta x=0.01$。
    \end{tasks}
    \item 求下列函数的微分:
    \begin{tasks}[before-skip=10pt,after-skip=10pt,after-item-skip=7pt](2)
      \task $y=3x^4-5x^2+1$;
      \task $y=(2x^2-3)(3x+4)$;
      \task $y=\dfrac{1-x^2}{1+x^2}$;
      \task $y=\dfrac13\tan^3\theta+\tan\theta$;
      \task $y=\upe^{2x}\sin5x$;
      \task $y=\arctan(\ln x)$。
    \end{tasks}
    \item 函数 $y=f(x)$ 的微分 $\dif y$ 依赖于哪两个量?
    \item 边长 \qty{4}{cm} 的正方形铁皮,在加热中边长增加了 \qty{0.001}{cm},求此时刻面积的微分 $\dif S$ 与 $\Delta S-\dif S$ 的值。
    \item 半径 $R$ 为 \qty{10}{cm} 的球,在冷却中 $R$ 缩短了 \qty{0.001}{cm},求此时刻体积的微分 $\dif V$ 与 $\Delta V-\dif V$ 的值。
    \item 求证 $\dfrac{\dif(\sin x)}{\dif(\cos x)}=-\cot x.$
  \end{question}
\end{Practice}

\subsection{近似计算}
我们知道,当 $|\Delta x|$ 很小时,可以用微分 $\dif y$ 近似代替函数 $y=f(x)$ 的改变量 $\Delta y$。由于在点 $x_0$ 处,
\begin{align*}
  \dif y&= f'(x_0)\Delta x,\\
 \Delta y &= f(x_0+\Delta x)-f(x_0),
\end{align*}
于是有
\[f(x_0+\Delta x)-f(x_0)\approx f'(x_0)\Delta x,\]
改写一下,得
\begin{equation}
  \label{eq:func_x_plus_delta_x}
f(x_0+\Delta x)\approx f(x_0)+f'(x_0)\Delta x.
\end{equation}
如果把 $x_0+\Delta x$ 记作 $x$,那么 $\Delta x=x-x_0$,于是\cref{eq:func_x_plus_delta_x} 也可写成
\begin{equation}
  \label{eq:func_x_approx}
f(x)\approx f(x_0)+f'(x_0)(x-x_0).
\end{equation}

如果我们要求函数 $y=f(x)$ 在某一点 $x$ 处的值,但这点的值不易求出,而在 $x$ 附近的 $x_0$ 处,$f(x_0)$ 和 $f'(x_0)$ 的值都容易求出,这时,我们可以利用\cref{eq:func_x_approx}(或\cref{eq:func_x_plus_delta_x})来求 $f(x)$ 的近似值。要注意的是这里要求 $|x-x_0|=|\Delta x|$ 很小,$|x-x_0|$ 越小,近似程度越好。

\begin{example}
  \label{exp:sin_value}
  不查表,求 $\sin\ang{46}$ 的近似值。
\end{example}
\begin{solution}
  令 $y=\sin x$,由\cref{eq:func_x_approx},
  \[ \sin x\approx\sin x_0+\cos x_0\cdot(x-x_0).\]
  因为 $\ang{46}=\ang{45}+\ang{1}=\left(\dfrac{\uppi}{4}+\dfrac{\uppi}{180}\right)\,\unit{rad}$,取 $x_0=\dfrac{\uppi}{4}$,$x=\dfrac{\uppi}{4}+\dfrac{\uppi}{180}$,于是 $x-x_0=\dfrac{\uppi}{180}$,代入上式得
  \begin{align*}
    \sin\ang{46}&=\sin\left(\frac{\uppi}{4}+\frac{\uppi}{180}\right) \\
    &\approx \sin\frac{\uppi}{4}+\cos\frac{\uppi}{4}\cdot\frac{\uppi}{180}=\frac{\sqrt{2}}{2}+\frac{\sqrt{2}}{2}\cdot\frac{\uppi}{180}\\
    &\approx 0.7071+0.0123=0.7194.
  \end{align*}
\end{solution}

如果把\cref{eq:func_x_approx} 中的 $x$ 看成变量,\cref{eq:func_x_approx} 右边就是 $x$ 的一次函数。也就是说,当 $|x-x_0|$ 很小时,可以用一次函数 $y=f(x_0)+f'(x_0)(x-x_0)$ 来近似表示函数 $y=f(x)$;从几何上看,也就是在点 $(x_0,y_0)$ 附近,用曲线 $y=f(x)$ 上点 $(x_0,y_0)$ 处的切线 $y-y_0=f'(x_0)(x-x_0)$ 来近似表示曲线 $y=f(x)$。

当 $x_0=0$ 时,\cref{eq:func_x_approx} 变为
\begin{equation}
  \label{eq:func_x_approx_0}
f(x)\approx f(0)+f'(0)x.
\end{equation}
利用\cref{eq:func_x_approx_0},当 $|x|$ 充分小时,可以导出常用的一些近似公式:
\begin{align}
  \label{eq:approx_fx_sqrt}\sqrt{1+x} &\approx 1+\frac{x}{2};\\
  \label{eq:approx_ln}\ln (1+x) &\approx x;\\
  \label{eq:approx_ex}\upe^x &\approx 1+x;\\
  \label{eq:approx_tan}\tan x &\approx x.
\end{align}

下面写出这些近似公式的导出过程。
\begin{itemize}
  \item[\eqref{eq:approx_fx_sqrt}] 取 $f(x)=\sqrt{1+x}$,则
  \[f'(x)=\frac{1}{2\sqrt{1+x}}, \]
  于是
  \[f(0)=1,\quad f'(0)=\frac12.\]
  由\cref{eq:func_x_approx_0},得
  \[\sqrt{1+x}=f(x)\approx f(0)+f'(0)x=1+\frac{x}{2},\]
  即
  \[\sqrt{1+x} \approx 1+\frac{x}{2}.\]
  \item[\eqref{eq:approx_ln}] 取 $f(x)=\ln(1+x)$,则
  \[f'(x)=\frac{1}{1+x}\]
  于是
  \[f(0)=0,\quad f'(0)=1.\]
  由\cref{eq:func_x_approx_0},得
  \[\ln(1+x)\approx x.\]
  \item[\eqref{eq:approx_ex}] 取 $f(x)=\upe^x$,则
  \[f'(x)=\upe^x\]
  于是
  \[f(0)=1,\quad f'(0)=1.\]
  由\cref{eq:func_x_approx_0},得
  \[\upe^x\approx x.\] 
  \item[\eqref{eq:approx_tan}] 取 $f(x)=\tan x$,则
  \[f'(x)=\sec^2x\]
  于是
  \[f(0)=0,\quad f'(0)=1.\]
  由\cref{eq:func_x_approx_0},得
  \[\tan x\approx x.\]
\end{itemize}

\begin{example}
  根据根据导出的近似公式,求下列各式的近似值:
  \begin{tasks}[label=(\arabic*)](3)
    \task $\sqrt{4.01}$,$\sqrt{8.997}$;
    \task $\ln 1.002$,$\ln0.998$;
    \task $\upe^{0.01}$,$\upe^{-0.02}$。
  \end{tasks}
\end{example}
\begin{solution}
  \begin{enumerate}
    \item $\sqrt{4.01}=\sqrt{4\left(1+\dfrac{0.01}{4}\right)}=2\sqrt{1+\dfrac{0.01}{4}}$。利用\cref{eq:approx_fx_sqrt},取 $x=\dfrac{0.01}{4}$,得
    \[\sqrt{4.01}\approx2\left(1+\frac{0.01}{2\times4}\right)=2.0025.\]

    $\sqrt{8.997}=\sqrt{9\left(1-\dfrac{0.003}{9}\right)}=3\sqrt{1-\dfrac{0.003}{9}}$,利用\cref{eq:approx_fx_sqrt},取 $x=-\dfrac{0.003}{9}$,得
    \[\sqrt{8.997}\approx3\left(1-\frac{0.003}{2\times9}\right)=2.9995.\]
    \item 利用\cref{eq:approx_ln}。取 $x=0.002$,得
    \[ \ln1.002=\ln(1+0.002)\approx0.002.\]
    取 $x=-0.002$,得
    \[ \ln0.998=\ln(1-0.002)\approx-0.002.\]
    \item 利用\cref{eq:approx_ex}。取 $x=0.01$,得
    \[ \upe^{0.01}\approx1+0.01=1.01.\]
    取 $x=-0.02$,得
    \[ \upe^{-0.02}\approx 1-0.02=0.98.\]
  \end{enumerate}
\end{solution}

\begin{example}
  如\cref{fig:2-6},加工锥形工件时,已知工件两头直径分别为 $d_1$,$d_2$,长度为 $l$,当斜角 $\alpha$ 很小时,导出近似关系式:
  \[\alpha\approx\ang{28.6}\times\frac{d_1-d_2}{l}\]
\end{example}
\noindent
\begin{minipage}{0.5\linewidth}
\begin{solution}
  由图,
  \[\tan \alpha=\frac{d_1-d_2}{2l}.\]
  当 $\alpha$ 很小时,根据\cref{eq:approx_tan},
  \begin{gather*}
    \tan\alpha\approx\alpha,\\
    \therefore\quad \alpha\approx\frac{d_1-d_2}{2l}.
  \end{gather*}
\end{solution}
\end{minipage}\hfill
\begin{minipage}{0.45\linewidth}
  \begin{figurehere}
    \includegraphics{2-6.pdf}
    \caption{}\label{fig:2-6}
  \end{figurehere}
\end{minipage}

\medskip\noindent
又
\begin{gather*} 
  \qty{1}{rad}=\left(\frac{180}{\uppi}\right)\unit{\degree}\approx\ang{57.3},\\
  \therefore \quad \alpha\approx\ang{57.3}\times\frac{d_1-d_2}{2l}\approx\ang{28.6}\times\frac{d_1-d_2}{2l}.
\end{gather*}

\begin{Practice}
  \begin{question}
    \item 在\cref{exp:sin_value} 中,$x-x_0=\dfrac{\uppi}{180}$,是用弧度为单位,为什么?如果用度为单位,取 $x-x_0=1$,得出的结果对不对?
    \item 已知 $\ln10=2.3026$,利用 $f(x)\approx f(x_0)+f'(x_0)( x-x_0)$ 求 $\ln 10.012$ 的近似值。
    \item $|x|$ 很小时,导出下列近似公式:
    \begin{tasks}[before-skip=5pt,after-skip=5pt](2)
      \task $\sqrt[n]{1+x}\approx1+\dfrac{x}{n}$;
      \task $\sin x\approx x$。
    \end{tasks}
    \item 计算下列各式的近似值:
    \begin{tasks}(3)
      \task $\sqrt[5]{1.02}$;
      \task $\sqrt[3]{0.998}$;
      \task $\upe^{-0.1}$;
      \task $\tan0.01$;
      \task $\sin\ang{0.1}$;
      \task $\ln1.0021$。
    \end{tasks}
  \end{question}
\end{Practice}

\begin{Exercise}
  \begin{question}
    \item\label{ex:8-1} 解答:
    \begin{tasks}
      \task 在下列图形中,标出相应的 $\Delta y$ 和 $\dif y$。
      \begin{figurehere}
        \begin{minipage}{\linewidth}\centering
          \includegraphics{ex8-1a.pdf}\hfill\includegraphics{ex8-1b.pdf}
          \caption*{(第 \ref{ex:8-1} 题图)}
        \end{minipage}
      \end{figurehere}
      \task 自变量 $x$ 的微分 $\dif x$ 是否一定为正?当 $\dif x$ 为正时,函数 $y=f(x)$ 的微分 $\dif y$ 是否一定为正?
    \end{tasks}
    \item 求下列函数的微分:
    \begin{tasks}[before-skip=7pt,after-skip=7pt,after-item-skip=7pt](2)
      \task $y=ax^3+bx^2+cx+d$;
      \task $y=(a^2-x^2)^5$;
      \task $y=\dfrac{(x-1)(x-2)}{(x+1)(x+2)}$;
      \task $r=(1+\theta^2)\tan\theta$;
      \task $y=\upe^{-x}\ln x$;
      \task $y=\arcsin3x+3x\sqrt{1-9x^2}$;
      \task $y=\dfrac{x^2n}{(1+x^2)^n}$;
      \task $yt-\dfrac{\upe^t-\upe^{-t}}{\upe^t+\upe^{-t}}$。
    \end{tasks}
    \item 求函数 $y=x^3$ 在自变量的值由 $x= $ 变为 $x+\Delta x=1.01$ 时的改变量 $\Delta y$ 及 $\dif y$。用 $\dif y$ 来近似地代替 $\Delta y$ 的误差是多少?
    \item 某运动方程为
    \[s=4t^2,\]
    其中 $t$ 的单位是 \unit{s},$s$ 的单位是 \unit{m},求 $t=\qty{2}{s}$,$\Delta t= \qty{0.001}{s}$ 时路程的改变量 $\Delta s$ 及路程的微分 $\dif s$,并加以比较。
    \item 一金属圆管的内半径为 \qty{10}{cm},厚度为 \qty{0.5}{cm},求圆管的横截面积的近似值。
    \item 球壳的外直径是 \qty{30}{cm},厚度是 \qty{0.1}{cm},求球壳体积的近似值。
    \item 要在直径为 \qty{50}{cm} 的半球面形锅的内侧镀锌(锌的比重为 \qty{7}{g/cm^3}),镀层厚 \qty{0.005}{cm},约需用锌多少克?
    \item 计算下列各式的近似值:
    \begin{tasks}(3)
      \task $\sqrt{9.01}$;
      \task $\sqrt{15.98}$;
      \task $\sqrt[3]{1.004}$;
      \task $\sqrt[3]{0.982}$;
      \task $\upe^{-0.03}$;
      \task $(1.002)^5$;
      \task $\ln 1.01$;
      \task $\ln 0.97$;
      \task $\sin 0.016$;
      \task $\tan 0.025$;
    \end{tasks}
    \item 当 $|x|$ 很小时,导出近似公式:
    \[ (1+x)^\alpha\approx 1+\alpha x\quad \text{( }\alpha \text{为实数),}\]
    并求 ${1.001}^{0.13}$ 的近似值。
  \end{question}
\end{Exercise}

\section*{小结}
\begin{enumerate}[C、,itemindent=4.5em]
  \item 本章主要内容是导数和微分的概念、求导数和求微分的方法以及微分在近似计算上的某些应用。
  \item 导数概念是微积分学的基本概念之一。函数 $y=f(x)$ 的导数 $f'(x)$,就是函数的改变量 $\Delta y$ 与自变量的改变量 $\Delta x$ 的比 $\dfrac{\Delta y}{\Delta x}$ 当 $\Delta x\to 0$ 时的极限,即
  \[f'(x)=\lim\limits_{\Delta x\to 0}\frac{\Delta y}{\Delta x} =\lim\limits_{\Delta x\to 0}\frac{f(x+\Delta x)-f(x)}{\Delta x},  \]
  它表示在点 $x$ 处函数 $y$ 对自变量的变化率,它的几何意义是曲线 $y=f(x)$ 在点 $(x,f(x))$ 处的切线的斜率。

  如果 $f'(x)$ 存在,曲线 $y=f(x)$ 上点 $(x_0,y_0)$ 处的切线的方程为
  \[ y-y_0=f'(x_0)(x-x_0);\]
  法线的方程(当 $f'(x_0) \neq 0$ 时)为
  \[ y-y_0=-\frac{1}{f'(x_0)}(x-x_0).\]
  \item 函数 $y=f(x)$ 的微分 $\dif y$ 就是函数的导数 $f'(x)$ 与自变量的微分 $\dif x$($\dif x=\Delta x\neq 0$)的积,即
  \[ \dif y=f'(x)\dif x.\]
  求函数 $y=f(x)$ 的导数 $f'(x)$ 与求函数的微分 $f'(x)\dif x$ 是互通的,即
  \[ \frac{\dif y}{\dif x}=f'(x) \Longleftrightarrow  \dif y=f'(x)\dif x\]
  所以导数也叫做微商。
  \item 求函数的导数或微分的方法叫做微分法。根据定义求导数是最基本的方法。对初等函数来说,只要根据定义先求得一些基本初等函数的导数,再利用求导数(或微分)的四则运算法则以及复合函数、反函数的求导(或微分)法则,就可以求出任一初等函数的导数(或微分)。这里,复合函数的求导法则特别重要,应切实掌握。
  \item 根据微分的定义及其几何意义,有下列近似公式:
  \begin{gather*}
    \Delta y\approx f'(x_0)\Delta x\quad\text{(}|\Delta x|\text{很小),}\\
    f(x)\approx f(x_0)+f'(x_0)(x-x_0)\quad\text{(}|x-x_0|\text{很小)。}
  \end{gather*}
  利用这些公式,可以求出函数的改变量或在某一点处的函数值的近似值。

  特别当 $x_0=0$ 时,有
  \[ f(x)\approx f(0)+f'(0)x \quad\text{(}|x|\text{很小)。}\]
  由此可以导出一些常用的近似公式,从而使实际问题中的一些计算大为简化。
\end{enumerate}

\chapter*{复习参考题\chinese{chapter}}
\section*{A 组}
\begin{question}
  \item 解答:
  \begin{tasks}
    \task 在导数的定义中,$\lim\limits_{\Delta x\to0}\dfrac{\Delta y}{\Delta x}$ 的 ${\Delta x}$ 是正的还是负的?还是可正可负?$\lim\limits_{\Delta x\to0+}\dfrac{\Delta y}{\Delta x}$ 的 ${\Delta x}$ 呢?$\lim\limits_{\Delta x\to0-}\dfrac{\Delta y}{\Delta x}$ 的 ${\Delta x}$ 呢?$\lim\limits_{\Delta x\to0}\dfrac{\Delta y}{\Delta x}$ 与 $\lim\limits_{\Delta x\to0+}\dfrac{\Delta y}{\Delta x}$ 及 $\lim\limits_{\Delta x\to0-}\dfrac{\Delta y}{\Delta x}$ 之间有什么关系?
    \task $f'(x)$ 和 $f'(x_0)$ 有什么不同?$f'(x_0)$ 与 $[f(x_0)]'$ 有什么不同?
  \end{tasks}
  \item 一质点沿 $OA$ 轴运动,在时刻 $t$(\unit{s})时,质点的位置 $s=6t-t^2$(\unit{m})。
  \begin{tasks}
    \task 求 $t=0,2,3,6,7$ 时质点的位置。当 $s$ 取负值时意味着什么?
    \task 求速度 $v$(\unit{m/s})的表示式,并求 $t=0,2,3,6,7$ 时的 $v$ 值。$v$ 值为负时意味着什么?
    \task 什么时刻以后,质点改变运动的方向?
  \end{tasks}
  \item 在受到制动后的 $t$\,\unit{s} 内飞轮转过的角度(弧度)由函数 $\varphi(t)=4t-0.3t^2$ 给出,求:
  \begin{tasks}
    \task $t=\qty{2}{s}$ 时,飞轮转过的角度;
    \task 飞轮停止旋转的时刻。
  \end{tasks}
  \item 假设 \qty{1}{kg} 的铁从 \qty{0}{\celsius} 加热到 $t\unit{\celsius}$($0\leqslant t\leqslant 200$)时,所吸收的热量 $Q$(\unit{kCal})由公式 $Q(t)=0.1053t+0.000071t^2$ 确定,求 $t=\qty{50}{\celsius}$ 时铁的比热 $C$。(提示:比热 $C=Q'(t)\,\unit{Cal.g^{-1}.\celsius^{-1}}$)。
  \item 已知抛物线 $y=x^2-4$ 及直线 $y=x+2$,求直线与抛物线在交点处的切线的交角。
  \item 已知两曲线 $y=x^2-1$ 与 $y=1-x^3$。
  \begin{tasks}
    \task 这两曲线在横坐标为 $x_0$ 的点处的切线互相平行,求 $x_0$ 的值;
    \task 这两曲线在横坐标为 $x_1$ 的点处的切线互相垂直,求 $x_1$ 的值。
  \end{tasks}
  \item 按定义求 $y=\tan x$ 的导数。
  \item 求下列函数的导数:
  \begin{tasks}[before-skip=10pt,after-skip=10pt,after-item-skip=7pt](2)
    \task $y=(x^2-1)(x^2-3)(x^2-5)$;
    \task $y=\dfrac{x^2}{(x^2-1)^3}$;
    \task $y=\dfrac{\sin2x}{1+x}$;
    \task $y=\dfrac{\sec x}{1+\tan x}$;
    \task $y=\sin^43x\cos^34x$;
    \task $y=2\left(\upe^{\frac{x}{2}}+\upe^{-\frac{x}{2}}\right)$;
    \task $y=x\arcsin\dfrac{x}{2}+\sqrt{4-x^2}$;
    \task $y=a^{2x+1}$;
    \task $y=\arccos\dfrac{x-3}{3}-2\sqrt{\dfrac{6-x}{x}}$;
    \task $y=\dfrac{x\ln x}{x+1}-\ln(x+1)$;
    \task $y=\lg(x^2+x+1)$;
    \task $y=\ln(\cos^2x)+2x\tan x-x^2$。
  \end{tasks}
  \item 求下列函数的导数:
  \begin{tasks}[before-skip=10pt,after-skip=10pt,after-item-skip=7pt](2)
    \task $y=\dfrac{x-a}{\sqrt{x^2-2ax}}$;
    \task $y=\arctan\frac{x-a}{x+a}$;
    \task! $y=\dfrac{x}{2}\sqrt{x^2+a^2}+\dfrac{a^2}{2}\ln\dfrac{x+\sqrt{x^2+a^2}}{a}$;
    \task $y=\arctan(\sec x+\tan x)$;
    \task $y=\ln(\ln x)-\dfrac{1}{\ln x}$;
    \task $y=\ln\left(\tan\left(\dfrac{\uppi}{4}-\dfrac{x}{2}\right)\right)$;
    \task $y=\upe^{\arcsin\sqrt{x}}$;
    \task $y=\lg(a^x+b^x)$;
    \task $y=a^{\tan x}$;
    \task $y=\arcsin\dfrac{2}{\upe^x+\upe^{-x}}$($x>0$);
    \task $y=\sqrt{(x-a)(x-b)(x-c)}$;
    \task $y=\ln(\csc x-\cot x)$。
  \end{tasks}
  \item 一金属圆盘受热膨胀,它的半径以 \qty{0.01}{cm/s} 的速度均匀增大,当它的半径等于 \qty{2}{cm} 时,它的面积的增大速度是多少?(提示:圆面积 $S=\uppi r^2$,把 $S$ 和 $r$ 都看成 $t$ 的函数。)
  \item 求证抛物线 $\sqrt{x}+\sqrt{y}=\sqrt{a}$ 上任意一点处的切线在两坐标轴上截距的和等于 $a$。
  \item 求垂直于直线 $2x+4y-3=0$ 并与双曲线 $\dfrac{x^2}{2}-\dfrac{y^2}{7}=1$ 相切的直线的方程。
  \item 某运动物体由点 $O$ 开始作直线运动,经过 $t$ 秒后它和点 $O$ 的距离为
  \[ s=\frac14t^4-4t^3+16t^2.\]
  \begin{tasks}
    \task 此物体什么时刻在 $O$ 点(即 $s=0$)?
    \task 什么时刻它的速度为 0?
    \task 什么时刻它的加速度值为 11?
  \end{tasks}
  \item 求下列函数的二阶导数:
  \begin{tasks}[before-skip=7pt,after-skip=7pt,after-item-skip=7pt](2)
    \task $y=\sin ax+\cos bx$;
    \task $y=\upe^{\sqrt{x}}+\upe^{-\sqrt{x}}$;
    \task $y=\dfrac{x^2+1}{(x+1)^3}$;
    \task $y=\arctan\dfrac{\upe^x-\upe^{-x}}{2}$。
  \end{tasks}
  \item 求下列函数 $y$ 在指定点处的一阶导数及二阶导数:
  \begin{tasks}[before-skip=7pt,after-skip=7pt,after-item-skip=7pt](2)
    \task $y=\sqrt{3x}+\dfrac{13}{\sqrt{3x}}$,点 $x=3$;
    \task $y=x\sqrt{x^2-16}$,点 $x=5$。
  \end{tasks}
  \item 一物体作阻尼运动,运动规律为
  \[ x=\upe^{-2t}\sin\left(3t+\dfrac{\uppi}{6}\right),\]
  求运动物体的速度和加速度。
  \item 求下列函数的微分:
  \begin{tasks}[before-skip=7pt,after-skip=7pt,after-item-skip=7pt](2)
    \task $y=(2x^3-3x^2+6x)^2$;
    \task $y=(\upe^x+\upe^{-x})^2$;
    \task $y=\dfrac{\ln x}{\sqrt{x}}$;
    \task $y=\arctan\dfrac{1-x^2}{1+x^2}$。
  \end{tasks}
  \item 设 $y=\cos^2\varphi$,当 $\varphi$ 从 \ang{60} 变到 \ang{60;30;} 时,求函数 $y$ 的微分。
  \item 单摆的周期 $T$(\unit{s})与单摆的长度 $l$(\unit{cm})之间,有关系 $T=2\uppi\sqrt{\dfrac{l}{980}}$。长度为 \qty{20}{cm}的单摆加长 \qty{1}{cm} 后,它的周期大约增加多少?
  \item 利用近似公式 $f(x)\approx f(x_0)+f'(x_0)(x-x_0)$ 求下列各式的近似值:
  \begin{tasks}(2)
    \task $\sin\ang{29}$;
    \task $\arctan 0.97$。
  \end{tasks}
  \item 当 $|x|$ 很小时,导出下列近似公式:
  \begin{tasks}(2)
    \task $\dfrac{1}{1+x}\approx 1-x$;
    \task $\arctan x\approx x$。
  \end{tasks}
  \item 计算下列各式的近似值:
  \begin{tasks}(3)
    \task $\sqrt[3]{1.02}$;
    \task $\dfrac{1}{\sqrt{99.5}}$;
    \task $\log_{1.01}0.997$。
  \end{tasks}
  \item 已知 $f(x)=\dfrac{x}{\sqrt{x^2+9}}$,求 $f(0.03)$ 的近似值。
  \item\label{ex:2t-24} 如图,一透镜的凸面半径是 $R$,口径是 $2h$($h$ 比 $R$ 小得多),厚度是 $\delta$。导出近似关系式:
  \[ \delta\approx \frac{h^2}{2R}.\]
  \begin{figurehere}
    \begin{minipage}{\linewidth}
      \centering
      \includegraphics{2t-24.pdf}
      \caption*{(第 \ref{ex:2t-24} 题图)}
    \end{minipage}
  \end{figurehere}
\end{question}
\section*{B 组}
\begin{question}[resume]
  \item 证明双曲线 $xy=a$($a$ 为不等于零的常数)上任意一点处的切线和坐标轴所构成的三角形的面积等于 $2|a|$。
  \item 设函数
  \[f(x)=\begin{cases}x^2, & x\leqslant 1;\\ ax+b,& x>1.\end{cases}\]
  为了使 $f(x)$ 在点 $x=1$ 处连续而且可导,应该怎样选取系数 $a$,$b$?
  \item 把 $y=\arcsin x$ 变形为 $x=\sin y$,利用隐函数的求导法则,证明:
  \[ (\arcsin x)'=\frac{1}{\sqrt{1-x^2}}\]
  \item\label{ex:2t-28} 一质量为 \qty{3}{kg} 的物体挂于弹簧的下端,在 $O$ 点上下振动,振动规律为 $x=10\sin t$。求振动过程中的最大动能。(位移单位为 \unit{cm},时间单位为 \unit{s}。)
  \begin{figurehere}
    \begin{minipage}{\linewidth}\centering
      \includegraphics{2t-28.pdf}
      \caption*{(第 \ref{ex:2t-28} 题图)}
    \end{minipage}
  \end{figurehere}
  \item 一倒置圆锥形的容器,它的轴截面是一等边三角形。以 \qty{100}{m^3/s} 的速度往容器内加水,求当水面高度为 \qty{20}{cm} 时水面上升的速度。(提示: 先写出容器内水的体积与水面高度的函数关系。)
  \item 从上口直径为 \qty{12}{cm},深为 \qty{18}{cm} 的锥形漏斗中流出溶液,当液面高度从 \qty{10}{cm} 下降到 \qty{9.8}{cm}时,求流出溶液的容积的近似值。
\end{question} % ok
\chapter{导数的应用}
\section{一阶导数的应用}

\subsection{预备知识}
\begin{Practice}
  \begin{question}
    \item 
    \item 
    \item 
  \end{question}
\end{Practice}

\begin{Exercise}
  \begin{question}
    \item 
    \item 
    \item 
    \item 
    \item 
  \end{question}
\end{Exercise}

\subsection{函数的单调性}
\begin{Practice}
  \begin{question}
    \item 
    \item 
    \item 
    \item 
    \item 
  \end{question}
\end{Practice}

\subsection{函数的极大值与极小值}
\begin{Practice}
  \begin{question}
    \item 
    \item 
    \item 
    \item 
  \end{question}
\end{Practice}

\begin{Exercise}
  \begin{question}
    \item 
    \item 
    \item 
    \item 
    \item 
    \item 
    \item 
    \item 
    \item 
    \item 
    \item 
  \end{question}
\end{Exercise}

\subsection{函数的最大值与最小值}
\begin{Practice}
  \begin{question}
    \item 
    \item 
    \item 
    \item 
  \end{question}
\end{Practice}

\begin{Exercise}
  \begin{question}
    \item 
    \item 
    \item 
    \item 
    \item 
    \item 
    \item 
    \item 
    \item 
    \item 
    \item 
    \item 
    \item 
    \item 
    \item 
    \item 
    \item 
    \item 
    \item 
    \item 
    \item 
    \item 
    \item 
    \item 
    \item 
  \end{question}
\end{Exercise}

\section{二阶导数的应用}


\subsection{预备知识}
\subsection{函数极值的判定}
\begin{Practice}
  应用二阶导数求下列函数的极值:
  \begin{tasks}
    \task $f(x)=x^3+3x^2-9x+6$;
    \task $f(x)=x-2\sin x$($0\leqslant x \leqslant 2\uppi $);
    \task $g(x)=ax^2+bx+c$($a\neq 0$)。
  \end{tasks}
\end{Practice}

\subsection{曲线的凸向和拐点}
\begin{Practice}
  \begin{question}
    \item 
    \item 
  \end{question}
\end{Practice}

\subsection{函数的图像}

\begin{Exercise}
  \begin{question}
    \item 
    \item 
    \item 
    \item 
    \item 
    \item 
  \end{question}
\end{Exercise}

\section*{小结}
\begin{enumerate}[C、,itemindent=4.5em]
  \item 
  \item 
  \item 
  \item 
  \item 
\end{enumerate}

\chapter*{复习参考题\chinese{chapter}}
\section*{A 组}
\begin{question}
  \item 
  \item 
  \item 
  \item 
  \item 
  \item 
  \item 
  \item 
  \item 
  \item 
  \item 
  \item 
  \item 
\end{question}
\section*{B 组}
\begin{question}
  \item 
  \item 
  \item 
  \item 
  \item 
  \item 
  \item 
  \item 
  \item 
  \item 
  \item 
  \item 
  \item 
  \item 
\end{question} % ok 
\chapter{不定积分}
\phantomsection\pdfbookmark[1]{不定积分}{phantomsection2}
\subsection{原函数}
假设已知物体运动的路程函数
\[s=s(t),\]
把路程函数对时间求导数,就得到速度函数 $v(t)$,即
\[s'(t)=v(t).\]

在实践中,也需要解决相反的问题:已知物体运动的速度函数 $v(t)$,如何求路程函数 $s(t)$。
一般地说,已知某个函数的导数,如何求这个函数。
下面我们就来研究这类问题。

设 $f(x)$ 是定义在区间 $I$ 上的一个函数,如果存在函数 $F(x)$,在区间 $I$ 上任何一点 $x$ 处都有
\[F'(x)=f(x)\]
那么 $F(x)$ 叫做函数 $f(x)$ 在区间 $I$ 上的一个\Concept{原函数}。

根据定义,求函数 $f(x)$ 的原函数,就是要求一个函数 $F(x)$,使它的导数 $F'(x)$ 等于 $f(x)$。
\begin{example}
  求出下列函数的一个原函数:
  \begin{tasks}[label=(\arabic*),label-width=16pt](2)
    \task $f(x)=3x^2$;
    \task $f(x)=\cos x$;
  \end{tasks}
\end{example}
\begin{solution}
  \begin{enumerate}
    \item $\because\quad (x^3)'=3x^2$,
    \par $\therefore\quad x^3$ 是函数 $3x^2$ 的一个原函数;
    \item $\because\quad (\sin x)'=\cos x$,
    \par $\therefore\quad \sin x$ 是函数 $\cos x$ 的一个原函数。
  \end{enumerate}
\end{solution}

已知函数 $f(x)$ 有一个原函数 $F(x)$,函数 $f(x)$ 是否还有其他原函数?我们看下面的例子。因为
\begin{align*}
  (x^2)'  &=2x,\\
  (x^2+1)'&=2x,\\
  (x^2-1)'&=2x,
\end{align*}
所以 $x^2$、$x^2+1$、$x^2-1$ 都是函数 $2x$ 的原函数。设 $C$ 为任意常数,由于
\[(x^2+C)' =2x\]
便知 $x^2+C$ 也是函数 $2x$ 的原函数。

一般地,有下面的定理:
\begin{Theorem}{定理}
  设 $F(x)$ 是函数 $f(x)$ 在区间 $I$ 上的一个原函数,对于任意常数 $C$,则
  \begin{enumerate}
    \item $F(x)+C$ 也是 $f(x)$ 的原函数;
    \item $f(x)$ 在区间 $I$ 上的任何一个原函数都可以表示成 $F(x)+C$ 的形式。
  \end{enumerate}
\end{Theorem}
\begin{proof}
  \begin{enumerate}
    \item 因为
    \[ [F(x)+C]'=F'(x)=f(x),\]
    所以 $F(x)+C$ 也是函数 $f(x)$ 的原函数。
    \item 设 $\Phi(x)$ 是 $f(x)$ 在区间 $I$ 上的任一原函数,则
    \begin{align*}
      \Phi'(x)&=f(x).\\
      \because\quad F'(x)&=f(x),\\
      \therefore\quad [\Phi(x)-F(x)]' &=\Phi'(x)-F'(x)\\
       &=f(x)-f(x)=0.
    \end{align*}
    已知在某区间 $I$ 上导数恒等于零的函数必为常数,由此得到
    \[ \Phi(x)-F(x) = C,\]
    即
    \[ \Phi(x)=F(x)+C\quad\text{(}C\text{ 为常数)。}\]
    也就是,函数 $f(x)$ 的任一原函数 $\Phi(x)$ 都可表示成 $F(x)+C$ 的形式。
  \end{enumerate}
\end{proof}

\subsection{不定积分}
上节定理告诉我们,如果 $F(x)$ 是函数 $f(x)$ 的一个原函数,则 $F(x)+C$ 也是函数 $f(x)$ 的原函数,而且所有的原函数都可表示成 $F(x)+C$ 的形式,其中 $C$ 为任意常数。

设 $F(x)$ 是函数 $f(x)$ 的一个原函数,我们把函数 $f(x)$ 的所有的原函数 $F(x)+C$ ($C$ 为任意常数)叫做函数 $f(x)$ 的\Concept{不定积分},记作 $\int f(x)\dif x$,即
\[\int f(x) \dif x = F(x) + C,\]
其中 $\int$ 叫做\Concept{积分号},$f(x)$ 叫做\Concept{被积函数},$x$ 叫做\Concept{积分变量},$f(x)\dif x$ 叫做\Concept{被积式},$C$ 叫做\Concept{积分常数}。
求已知函数的不定积分的过程叫做对这函数进行\Concept{积分}。

求函数 $f(x)$ 的不定积分,就是要求出 $f(x)$ 的所有的原函数。
由上节定理可知,只要求出函数 $f(x)$ 的任一个原函数 $F(x)$,再加上任意常数 $C$,就得到 $f(x)$ 的不定积分。

\begin{example}
  求下列不定积分:
  \begin{tasks}[label=(\arabic*),label-width=16pt](2)
    \task $\displaystyle\int x\dif x$;
    \task $\displaystyle\int \cos x\dif x$。
  \end{tasks}
\end{example}
\begin{solution}
  \begin{enumerate}
    \item $\because\quad \dfrac{1}{2}x^2$ 是 $x$ 的一个原函数,
    \par $\therefore\quad \displaystyle \int x\dif x=\frac{1}{2}x^2+C$。
    \item $\because\quad \sin x$ 是 $\cos x$ 的一个原函数,
    \par $\therefore\quad \displaystyle \int \cos x\dif x=\sin x+C$。
  \end{enumerate}
\end{solution}

根据不定积分的定义,可以推出下面两个性质:
\begin{Theorem}{性质}
  \begin{enumerate}
    \item $\displaystyle \left( \int f(x) \dif x \right)'=f(x)$;
    \item $\displaystyle \int F'(x) \dif x =F(x)+C$。
  \end{enumerate}
\end{Theorem}

上面的性质表明: 如果对函数 $f(x)$ 先求不定积分后求导数,那么两者的作用相互抵消,结果仍为 $f(x)$。
如果对函数 $F(x)$ 先求导数后求不定积分,那么作用相互抵消后结果与 $F(x)$ 只差一个任意常数。
从这里可以看出,求导数与求不定积分互为逆运算。

\begin{Practice}
  \begin{question}
    \item 将下面表中空格处填上适当的函数:\par\noindent
    \begin{tablehere}
      \begin{tblr}{colspec={X[c]X[c]X[c]},hline{2}=0.8pt}
        函数 $f(x)$        & 理由 & $f(x)$ 的一个原函数 \\
        $k$(常数)        & $(kx)'=k$ & $kx$ \\
        $4x^3$             &           &      \\
        $\cos x$           &           &      \\
        $\upe^x$     &           &      \\
        $\dfrac{1}{1+x^2}$ &           &      \\
      \end{tblr}
    \end{tablehere}
    \item 写出下列函数的一个原函数:
    \begin{tasks}[before-skip=10pt,after-skip=10pt,after-item-skip=7pt](4)
      \task $6x^5$
      \task $-\sin x$
      \task $\dfrac{1}{2\sqrt{x}}$
      \task $\dfrac{1}{\sqrt{1-x^2}}$
    \end{tasks}
    \item 在括号内填入一个适当的函数,并求出相应的不定积分:
    \begin{tasks}[before-skip=10pt,after-skip=10pt,after-item-skip=7pt]
      \task $\displaystyle (\qquad)'=3,\quad\quad\quad\ \ \ \ \int 3\dif x = \qquad$;
      \task $\displaystyle (\qquad)'=3x^2,\quad\quad\quad\ \int 3x^2\dif x = \qquad$;
      \task $\displaystyle (\qquad)'=\frac{1}{\cos^2x},\quad\quad \int \frac{1}{\cos^2x}\dif x = \qquad$;
      \task $\displaystyle (\qquad)'=\frac{1}{1+x^2},\quad\quad \int \frac{1}{1+x^2}\dif x = \qquad$。
    \end{tasks}
    \item 根据不定积分的定义,验证下列等式:
    \begin{tasks}[before-skip=10pt,after-skip=10pt,after-item-skip=7pt](2)
      \task $\displaystyle\int x^4\dif x=\frac{1}{5}x^5+C$
      \task $\displaystyle\int \frac{1}{x^3}\dif x=-\frac{1}{2}x^{-2}+C$
      \task! $\displaystyle\int (\sin x+\cos x)\dif x=-\cos x+\sin x +C$
    \end{tasks}
  \end{question}
\end{Practice}

\subsection{基本积分公式}
我们已经知道,求不定积分是求导数的逆运算。
因此,我们可以从导数公式得到相应的不定积分公式。
例如,因为
\[ \left(\frac{x^{n+1}}{n+1}\right)'=x^n\quad(n\neq -1),\]
所以有不定积分公式
\[ \int x^n \dif x=\frac{1}{n+1} x^{n+1}+C\quad(n\neq -1).\]

用同样方法可以得到其他的不定积分公式。下面是基本积分公式表。
\subsubsection*{基本积分公式表}
\begin{enumerate}[itemsep=5pt]
  \item $\displaystyle \int 0\dif x= C$;
  \item $\displaystyle \int 1\dif x= x+C$;
  \item $\displaystyle \int x^n\dif x= \frac{1}{n+1}x^{n+1}+C$($n\neq-1$);
  \item $\displaystyle \int \frac{1}{x}\dif x=\ln|x| +C$;
  \item $\displaystyle \int a^x\dif x= \frac{a^x}{\ln a}+C$(其中 $a>0$,且 $a\neq-1$);
  \item $\displaystyle \int \upe^x\dif x=\upe^x +C$;
  \item $\displaystyle \int \sin x\dif x=-\cos x +C$;
  \item $\displaystyle \int \cos x\dif x=\sin x +C$;
  \item $\displaystyle \int \frac{1}{\cos^2x}\dif x= \int\sec^2x \dif x = \tan x+C$;
  \item $\displaystyle \int \frac{1}{\sin^2x}\dif x= \int\csc^2x \dif x =-\cot x+C$;
  \item $\displaystyle \int \frac{1}{\sqrt{1-x^2}}\dif x=\arcsin x +C$;
  \item $\displaystyle \int \frac{1}{1+x^2}\dif x=\arctan x +C$。
\end{enumerate}

\medskip
表中 $C$ 为积分常数。

\begin{example}
  求 $\displaystyle \int \frac{1}{x^4}\dif x$。
\end{example}
\begin{solution}
  $\displaystyle\int \frac{1}{x^4}\dif x=\int x^{-4}\dif x=\frac{1}{-4+1}x^{-4+1}+C=-\frac{1}{3}x^{-3}+C$。
\end{solution}
\begin{example}
  求 $\displaystyle \int \frac{1}{x^3\sqrt{x}}\dif x$。
\end{example}
\begin{solution}
  $\displaystyle\int \frac{1}{x^3\sqrt{x}}\dif x = \int x^{-\frac{7}{2}}\dif x= \frac{1}{-\dfrac{7}{2}+1}x^{-\frac{7}{2}+1}+C= -\frac{2}{5}x^{-\frac{5}{2}}+C$。
\end{solution}
\begin{example}
  求 $\displaystyle \int 10^x\dif x$。
\end{example}
\begin{solution}
  $\displaystyle \int 10^x\dif x=\frac{10^x}{\ln10}+C$。
\end{solution}

\begin{Practice}
  \begin{question}
    \item (口答)求不定积分:
    \begin{tasks}[before-skip=10pt,after-skip=10pt,after-item-skip=7pt](4)
      \task $\displaystyle\int \sin x\dif x$;
      \task*(2) $\displaystyle\int m\dif x$($m$ 为常数);
      \task $\displaystyle\int a^x\dif x$;
      \task $\displaystyle\int \upe^x\dif x$;
      \task $\displaystyle\int \cos x\dif x$;
      \task $\displaystyle\int \frac{1}{x^2}\dif x$;
      \task $\displaystyle\int \frac{1}{x}\dif x$;
      \task $\displaystyle\int x\dif x$;
      \task*(2) $\displaystyle\int \frac{1}{\sqrt{1-x^2}}\dif x$;
      \task $\displaystyle\int \frac{1}{\sin^2x}\dif x$。
    \end{tasks}
    \item 求不定积分:
    \begin{tasks}[before-skip=10pt,after-skip=10pt,after-item-skip=7pt](4)
      \task $\displaystyle\int x^{\frac{1}{2}}\dif x$;
      \task $\displaystyle\int x^{-\frac{1}{2}}\dif x$;
      \task $\displaystyle\int x^7\dif x$;
      \task $\displaystyle\int x^3\sqrt[3]{x^2}\dif x$;
      \task $\displaystyle\int \frac{1}{x\sqrt{x}}\dif x$;
      \task $\displaystyle\int x^{-9}\dif x$;
      \task $\displaystyle\int 5^x\dif x$;
      \task $\displaystyle\int (3\upe)^x\dif x$;
      \task $\displaystyle\int \sin\theta\dif \theta$;
      \task $\displaystyle\int \sec^2\alpha\dif \alpha$;
      \task $\displaystyle\int \theta^3\dif \theta$;
      \task $\displaystyle\int t^5\dif t$。
    \end{tasks}
  \end{question}
\end{Practice}

\subsection{不定积分的运算法则}
根据导数的运算法则,可以推出下面不定积分的两个运算法则。
\begin{enumerate}[1.]
  \item \emph{被积式的常数因子可以提到积分号前面},即如果 $k$ 为不等于零的常数,那么
  \[ \tcbhighmath{\int kf(x)\dif x= k\int f(x)\dif x}.\]
  \begin{proof}
    由于
    \[\left[k\!\int f(x)\dif x \right]'=k\left[ \int f(x) \dif x\right]'=kf(x)\]
    即 $\displaystyle k\int f(x) \dif x$ 是 $kf(x)$ 的原函数,由此得出
    \[ \int kf(x)\dif x= k\!\int f(x)\dif x. \]
  \end{proof}
  \item \emph{两个函数的和(或差)的不定积分等于这两个函数的不定积分的和(或差)},即
  \[\tcbhighmath{\int[f(x)\pm g(x)] \dif x =\int f(x)\dif x \pm \int g(x)\dif x}.\]
  \begin{proof}
    由于
    \[\left[\int f(x)\dif x \pm \int g(x) \dif x\right]'=\left[ \int f(x)\dif x \right]'\pm\left[ \int g(x)\dif x\right]' =f(x)\pm g(x).\]
    即 $\displaystyle \int f(x)\dif x \pm \int g(x)\dif x $ 是 $f(x)\pm g(x)$ 的原函数,由此得出:
    \[ \int[f(x)\pm g(x)] \dif x =\int f(x)\dif x \pm \int g(x)\dif x.\]
  \end{proof}
\end{enumerate}

这个法则可以推广: 有限个函数的和(或差)的不定积分等于各个函数的不定积分的和(或差),即
\begin{multline*}
  \qquad \int [ f_1(x)\pm f_2(x)\pm\cdots\pm f_n(x)]\dif x\\
  =\int f_1(x)\dif x\pm \int f_2(x)\dif x\pm\cdots\pm \int f_n(x)\dif x.\qquad
\end{multline*}

\begin{example}
  求 $\displaystyle \int (2x^2+5x+3) \dif x$。
\end{example}
\begin{solution}
  \begin{align*}
    \int (2x^2+5x+3) \dif x &= 2\!\int x^2\dif x + 5\!\int x\dif x + 3\!\int \dif x = 2\cdot\frac{1}{3}x^3+5\cdot\frac{1}{2}x^2+3x+C\\
    &= \frac{2}{3}x^3+\frac{5}{2}x^2+3x+C.
  \end{align*}
\end{solution}

\alertinfo{在各项积分后,每个不定积分的结果都含有任意常数。但因任意常数的和仍然是任意常数,所以只要写一个任意常数就可以了。}

\begin{example}
  求 $\displaystyle \int \left(\frac{6}{\sqrt[3]{x}}-\frac{5}{x^2}\right) \dif x$。
\end{example}
\begin{solution}
  \begin{align*}
    \int \left(\frac{6}{\sqrt[3]{x}}-\frac{5}{x^2}\right) \dif x &= 6\!\int x^{-\frac{1}{3}}\dif x - 5\!\int x^{-2}\dif x = \frac{6}{-\dfrac{1}{3}+1}\cdot x^{-\frac{1}{3}+1}-\frac{5}{-2+1}\cdot x^{-2+1}\\
    &= 9x^{\frac{2}{3}}+\frac{5}{x}+C.
  \end{align*}
\end{solution}

\begin{example}
  求 $\displaystyle \int\left(\frac{1}{x}-\cos x\right) \dif x$。
\end{example}
\begin{solution}
  $\displaystyle \int\left(\frac{1}{x}-\cos x\right) \dif x=\int\frac{1}{x}\dif x-\int \cos x\dif x = \ln|x|-\sin x+C.$
\end{solution}

\begin{Practice}
  求不定积分:
  \begin{tasks}[before-skip=10pt,after-skip=10pt,after-item-skip=7pt](2)
    \task $\displaystyle\int (x^3-3x+1)\dif x$
    \task $\displaystyle\int \left(5x^4+2\sqrt{x}\right)\dif x$
    \task $\displaystyle\int \left(\frac{x^2}{2}-\frac{2}{x^2}\right)\dif x$
    \task $\displaystyle\int (2^x+x^2)\dif x$
    \task $\displaystyle\int \left(\frac{5}{x}+2\upe^x-\frac{1}{x\sqrt{x}}\right)\dif x$
    \task $\displaystyle\int (\sin x-\cos x)\dif x$
    \task $\displaystyle\int (\sec^2x-\csc^2x)\dif x$
    \task $\displaystyle\int \left(\frac{3}{1+x^2}-\frac{2}{\sqrt{1-x^2}}\right)\dif x$
  \end{tasks}
\end{Practice}

\subsection{直接积分法}
在求某些函数的不定积分时,只需经过简单的恒等变形,直接运用不定积分的两个运算法则与基本积分公式来求出结果,这种积分方法叫做\Concept{直接积分法}。
\begin{example}
  求 $\displaystyle \int \frac{x^3-3x^2+2x}{x}\dif x$。
\end{example}
\begin{solution}
  \begin{align*}
    \int \frac{x^3-3x^2+2x}{x}\dif x & = \int\left(x-3+\frac{2}{x}\right)\dif x=\int x\dif x-3\!\int \dif x+2\!\int\frac{1}{x}\dif x \\
    &= \frac{x^2}{2}-3x+2\ln|x|+C. 
  \end{align*}
\end{solution}

\begin{example}
  求 $\displaystyle \int (x-\sqrt{x})^2\dif x$。
\end{example}
\begin{solution}
  \begin{align*}
    \int (x-\sqrt{x})^2\dif x & = \int(x^2-2x\sqrt{x}+x)\dif x=\int x^2\dif x-2\!\int x^{\frac{3}{2}}\dif x+\int x\dif x \\
    &= \frac{1}{3}x^3-\frac{4}{5}x^{\frac{5}{2}}+\frac{1}{2}x^2+C. 
  \end{align*}
\end{solution}

\begin{example}
  求 $\displaystyle \int \frac{x^2}{1+x^2}\dif x$。
\end{example}
\begin{solution}
  $\displaystyle \int \frac{x^2}{1+x^2}\dif x = \int\frac{x^2+1-1}{1+x^2}\dif x=\int\left(1-\frac{1}{1+x^2}\right) \dif x=x-\arctan x +C $。
\end{solution}

\begin{example}
  求 $\displaystyle \int \frac{\cos 2x}{\cos x-\sin x}\dif x$。
\end{example}
\begin{solution}
  \begin{align*}
    \int \frac{\cos 2x}{\cos x-\sin x}\dif x & = \int\frac{\cos^2x-\sin^2x}{\cos x-\sin x}\dif x=\int(\cos x+\sin x)\dif x \\
    &=\int\cos x\dif x+\int\sin x\dif x = \sin x-\cos x+C. 
  \end{align*}
\end{solution}

\begin{example}
  求 $\displaystyle \int \frac{1}{\sin^2 x\cos^2 x}\dif x$。
\end{example}
\begin{solution}
  \begin{align*}
    \int \frac{1}{\sin^2 x\cos^2 x}\dif x & = \int\frac{\sin^2 x+\cos^2 x}{\sin^2 x\cos^2 x}\dif x=\int\left(\frac{1}{\cos^2 x}+\frac{1}{\sin^2 x}\right)\dif x \\
    &=\int\frac{1}{\cos^2x}\dif x+\int\frac{1}{\sin^2x}\dif x = \tan x-\cot x+C. 
  \end{align*}
\end{solution}

\begin{Practice}
  求不定积分:
  \begin{tasks}[before-skip=10pt,after-skip=10pt,after-item-skip=7pt](2)
    \task $\displaystyle\int (x^2+2)\sqrt{x}\dif x$;
    \task $\displaystyle\int (x-5)^2\dif x$;
    \task $\displaystyle\int \frac{x^4-x-5}{x^2}\dif x$;
    \task $\displaystyle\int \frac{x^2-9}{x+3}\dif x$;
    \task $\displaystyle\int (a^x+\csc^2x)\dif x$;
    \task $\displaystyle\int \left(\sec^2x+\frac{2}{1+x^2}\right)\dif x$;
    \task $\displaystyle\int \frac{\sqrt{x}-x^3\upe^x}{x^3}\dif x$;
    \task $\displaystyle\int \left(\sin\frac{x}{2}-\cos\frac{x}{2}\right)^2\dif x$;
    \task $\displaystyle\int \tan^2 x\dif x$;
    \task $\displaystyle\int \frac{5}{1-x^2}\dif x$。
  \end{tasks}
\end{Practice}

\begin{Exercise}
  \begin{question}
    \item 求不定积分:
    \begin{tasks}[before-skip=10pt,after-skip=10pt,after-item-skip=7pt](2)
      \task! $\displaystyle\int \sqrt[n]{x^m}\dif x$($m$、$n$ 为正整数);
      \task  $\displaystyle\int (x^3+px+q)\dif x$;
      \task  $\displaystyle\int \frac{\upe^x+x^{\upe}}{2}\dif x$。
    \end{tasks}
    \item 根据不定积分的定义,验证下列各等式:
    \begin{tasks}[before-skip=10pt,after-skip=10pt,after-item-skip=7pt]
      \task $\displaystyle\int \sin x\cos x \dif x=\frac{1}{2}\sin^2x+C$;
      \task $\displaystyle\int \sin x\cos x \dif x=-\frac{1}{2}\cos^2x+C$;
      \task $\displaystyle\int \sin x\cos x \dif x=-\frac{1}{4}\cos2x+C$。
    \end{tasks}
    \item 求不定积分:
    \begin{tasks}[before-skip=10pt,after-skip=10pt,after-item-skip=7pt](2)
      \task $\displaystyle\int (a+bx)^2\dif x$;
      \task $\displaystyle\int (x+5)(x-5)\dif x$;
      \task $\displaystyle\int (3x+1)(2x-1)\dif x$;
      \task $\displaystyle\int x(4x^2-4x-1)\dif x$;
      \task $\displaystyle\int \left(x^{\frac{1}{2}}-x^{-\frac{1}{2}}\right)^3\dif x$;
      \task $\displaystyle\int \frac{x^2+x+1}{x^2}\dif x$;
      \task $\displaystyle\int \frac{x+5}{\sqrt{x}}\dif x$;
      \task $\displaystyle\int \left(\frac{2}{x}+\frac{x}{3}\right)^3\dif x$;
      \task $\displaystyle\int \left(\frac{t-1}{t}\right)^2\dif t$;
      \task $\displaystyle\int \frac{\sqrt[3]{x^2}-\sqrt[4]{x}}{\sqrt{x}}\dif x$;
      \task $\displaystyle\int \frac{x-4}{\sqrt{x}-2}\dif x$;
      \task $\displaystyle\int \frac{x^3-27}{x-3}\dif x$;
      \task $\displaystyle\int \frac{x^4}{1+x^2}\dif x$;
      \task $\displaystyle\int \frac{1+x+x^2}{x(1+x^2)}\dif x$。
    \end{tasks}
    \item 求不定积分:
    \begin{tasks}[before-skip=10pt,after-skip=10pt,after-item-skip=7pt](2)
      \task $\displaystyle\int\frac{\sqrt{1+x^2}}{\sqrt{1-x^4}} \dif x$;
      \task $\displaystyle\int \upe^x\left(a^x-\frac{\upe^{-x}}{\sqrt{1-x^2}}\right)\dif x$;
      \task $\displaystyle\int \left(\frac{1}{x}-\frac{3}{\sqrt{1-x^2}}\right)\dif x$;
      \task $\displaystyle\int \frac{2\cdot3^x-5\cdot2^3}{3^x}\dif x$;
      \task $\displaystyle\int \frac{\cos 2x}{\sin x+\cos x}\dif x$;
      \task $\displaystyle\int \frac{\cos 2x}{\sin^2 x}\dif x$;
      \task $\displaystyle\int \frac{\sin 2x}{\cos x}\dif x$;
      \task $\displaystyle\int \frac{\sin 2x}{\sin x}\dif x$;
      \task $\displaystyle\int \cos^2\frac{x}{2}\dif x$;
      \task $\displaystyle\int \cot^2x\dif x$。
    \end{tasks}
  \end{question}
\end{Exercise}

\subsection{换元积分法}\label{subsec:Substitution_Integration}
换元积分法就是通过适当的变量替换,使被积式化为基本积分公式表中的某一被积式,然后求出结果。例如,在不定积分 $\displaystyle \int 2\sin 2x \dif x$ 中,如果令 $u=2x$,那么 $\dif u=2\dif x$,于是
\[ \int 2\sin2x\dif x=\int \sin 2x\cdot 2\dif x =\int \sin u \dif u\]
这样,原来的被积式就化为基本积分公式表中的一个被积式,可得
\[ \int 2\sin2x\dif x=-\cos u +C.\]
最后,换回原来的变量,就有
\[ \int 2\sin2x\dif x=-\cos 2x +C.\]

\begin{example}
  求 $\displaystyle \int (2x+1)^5\dif x$。
\end{example}
\begin{solution}
  设 $u=2x+1$,$\dif u=2\dif x$,于是
  \begin{align*}
    \int(2x+1)^5\dif x &= \frac{1}{2}\int(2x+1)^5\cdot2\dif x =\frac{1}{2}\int u^5\dif u=\frac{1}{12}u^6+C \\
    &= \frac{1}{12}(2x+1)^6+C.
  \end{align*}
\end{solution}

\begin{example}
  求 $\displaystyle \int \frac{2x}{(x^2+1)^3}\dif x$。
\end{example}
\begin{solution}
  设 $x^2+1=u$,$\dif u=2x \dif x$,于是
  \begin{align*}
    \int\frac{2x}{(x^2+1)^3}\dif x &= \int(x^2+1)^{-3}\cdot 2x\dif x =\int u^{-3}\dif u=\frac{1}{-3+1}u^{-2}+C \\
    &= -\frac{1}{2(x^2+1)^2}+C.
  \end{align*}
\end{solution}

从上面的例子我们看到,求某些不定积分的关键是设法将被积式 $f(x)\dif x$ 凑成微分 $g(\varphi(x))\varphi'(x)\dif x$ 的形式,即
\[ \int f(x)\dif x=\int g(\varphi(x))\varphi'(x)\dif x,\]
再进行变量替换 $u=\varphi(x)$,于是
\[ \int g(\varphi(x))\varphi'(x)\dif x=\int g(u)\dif u.\]
对 $g\left( u\right)$ 进行积分,得
\[ \int g(u)\dif u =F(u)+C,\]
用 $u=\varphi(x)$ 将变量 $u$ 换回到原来的变量 $x$,那么所求的不定积分就是
\[ \int f(x)\dif x=F[\varphi(x)]+C.\]
这种求不定积分的方法叫做\Concept{第一换元积分法},这一方法实质上是把被积式凑成某个函数的微分,因此也叫做\Concept{凑微分法}。

\begin{example}
  求 $\displaystyle \int x\sqrt{1-x^2} \dif x$。
\end{example}
\begin{solution}
  设 $u=1-x^2$,则 $\dif u=-2x\dif x$,于是 
  \begin{align*}
    \int x\sqrt{1-x^2} \dif x & =-\frac{1}{2}\int \sqrt{1-x^2}(-2x\dif x) = -\frac{1}{2}\int u^{\frac{1}{2}}\dif u=-\frac{1}{3}u^{\frac{3}{2}}+C\\ 
    & =-\frac{1}{3}(1-x^2)^{\frac{3}{2}}+C.
  \end{align*}
\end{solution}

\begin{example}
  求 $\displaystyle \int \cot x \dif x$。
\end{example}
\begin{solution}
  设 $\sin x=u$,则 $\dif u=\cos x\dif x$,于是
  \[\int \cot x\dif x= \int \frac{1}{\sin x}\cos x\dif x=\int\frac{1}{u}\dif u=\ln|u|+C=\ln|\sin x|+C.\]
\end{solution}

\medskip
在运算熟练之后,可以省去其中写出变量替换的步骤。

\begin{example}
  求 $\displaystyle \int \sin5x\cos x \dif x$。
\end{example}
\begin{solution}
  由三角函数积化和差公式,得
  \[ \sin5x\cos x=\frac{1}{2}(\sin6x+\sin4x)\]
  于是
  \begin{align*}
    \int \sin5x\cos x \dif x & =\frac{1}{2}\int (\sin6x+\sin4x)\dif x = \frac{1}{2}\left(\int\sin6x\dif x +\int\sin4x\dif x \right)\\
    &=\frac{1}{2}\left[\frac{1}{6}\int\sin6x\dif(6x)+\frac{1}{4}\int\sin4x\dif(4x)\right]\\ 
    & =-\frac{1}{12}\cos6x-\frac{1}{8}\cos4x+C.
  \end{align*}
\end{solution}

\begin{example}\label{exp:arctan}
  求 $\displaystyle \int \frac{1}{a^2+x^2} \dif x$。
\end{example}
\begin{solution}
\[ \int \dif x= \frac{1}{a^2}\!\int \frac{1}{1+\left(\dfrac{x}{a}\right)^2}\dif x =\frac{1}{a}\!\int\frac{1}{1+\left(\dfrac{x}{a}\right)^2}\dif\left(\frac{x}{a}\right) =\frac{1}{a}\arctan\frac{x}{a}+C.\]
\end{solution}

求不定积分有时将所求的不定积分 $\int f(x)\dif x$ 作另一种形式的变量替换。
设 $x=\varphi(t)$,则 $\dif x=\varphi'(t)\dif t$,于是
\[ \int f(x) \dif x=\int f[\varphi(t)]\varphi'(t)\dif t.\]
如果已知右端的积分为
\[ \int f[\varphi(t)]\varphi'(t)\dif t = F(t)+C.\]
那么以 $x=\varphi(x)$ 的反函数 $t=\varphi^{-1}(x)$ 代入上式,将变量 $t$ 还原为原来变量 $x$,即得到所求的不定积分
\[ \int f(x) \dif x =F[\varphi^{-1}(x)]+C.\]
这种求不定积分的方法叫做\Concept{第二换元积分法}。

\begin{example}
  求 $\displaystyle \int\frac{1}{1+\sqrt{x}}\dif x$。
\end{example}
\begin{solution}
  设 $x=t^2$($t>0$),则 $\dif x=2t\dif t$,于是
  \[ \int\frac{1}{1+\sqrt{x}}\dif x =\int\frac{1}{1+t}\cdot2t\dif  t =2\!\int\left(1-\frac{1}{1+t}\right)\dif t=2t-2\ln|1+t|+C.\]
  将变量 $t$ 还原为变量 $x$。由 $x = t^2$,得 $t = \sqrt{x}$。代入上式,得
  \[ \int\frac{1}{1+\sqrt{x}}\dif x =2\sqrt{x}-2\ln|1+\sqrt{x}|+C.\]
\end{solution}

\begin{example}
  求 $\displaystyle \int \sqrt{a^2-x^2} \dif x$($a>0$)。
\end{example}
\begin{solution}
  由于被积式含有根式 $\sqrt{a^2-x^2}$,为此考虑利用三角公式进行变量替换,消去根号。根据公式
  \[1-\sin^2t=\cos^2 t,\]
  可设 $x=a\sin t\;\left(-\dfrac{\uppi}{2}\leqslant t \leqslant \dfrac{\uppi }{2}\right)$,那么
  \begin{gather*} 
    \sqrt{a^2-x^2}=\sqrt{a^2-a^2\sin^2t}=a\cos t,\\
    \dif x= a\cos t\dif t, 
  \end{gather*}
  于是
  \begin{align*}
    \int \sqrt{a^2-x^2}\dif x &=\int a\cos t\cdot a \cos t\dif t=a^2\int \cos^2t\dif t\\ 
    &= a^2\int \frac{1+\cos2t}{2}\dif t =a^2\left(\frac{1}{2}t+\frac{1}{4}\sin2t\right)+C. 
  \end{align*}
  由 $x=a\sin t$ 可得 $t=\arcsin\dfrac{x}{a}$,
  \begin{align*}
    \cos t &= \sqrt{1-\left(\frac{x}{a}\right)^2},\\
    \sin2t &= 2\sin t\cos t=2\cdot\frac{x}{a}\cdot\sqrt{1-\left(\frac{x}{a}\right)^2}=\frac{2}{a^2}x\sqrt{a^2-x^2}.
  \end{align*}
  因此,
  \begin{align*}
    \int \sqrt{a^2-x^2} \dif x  &= a^2\left(\frac{1}{2}\arcsin\frac{x}{a}+\frac{1}{4}\cdot\frac{2}{a^2}\cdot x\cdot\sqrt{a^2-x^2}\right)+C\\
    &= \frac{a^2}{2}\arcsin\frac{x}{a}+\frac{x}{2}\sqrt{a^2-x^2}+C.
  \end{align*}
\end{solution}

\begin{Practice}
  \begin{question}
    \item 在下列等式的空白处填入适当的系数,使等式成立,并求出相应的不定积分。
    \begin{tasks}[before-skip=10pt,after-skip=10pt,after-item-skip=7pt]
      \task $\displaystyle \dif x = \qquad \dif (3x+1),\quad\quad\quad\ \  \int (3x+1)^4\dif x = \qquad$;
      \task $\displaystyle x\dif x = \qquad \dif (x^2+1),\ \ \, \quad\quad \int \frac{x}{(x^2+1)^2}\dif x = \qquad$;
      \task $\displaystyle \sin 3x\dif x = \qquad \dif (\cos 3x),\quad \int \cos^2 30\sin3x\dif x = \qquad$;
      \task $\displaystyle \frac{1}{x}\dif x = \qquad \dif (\ln x),\quad\ \ \quad\quad \int \frac{\ln^3 x}{x}\dif x = \qquad$。
    \end{tasks}
    \item 用换元积分法求不定积分:
    \begin{tasks}[before-skip=10pt,after-skip=10pt,after-item-skip=7pt](2)
      \task $\displaystyle\int \sqrt{1+2x}\dif x$;
      \task $\displaystyle\int \frac{1}{1-x}\dif x$;
      \task $\displaystyle\int x\sqrt{1+x^2}\dif x$;
      \task $\displaystyle\int \frac{2x}{1+x^2}\dif x$;
      \task $\displaystyle\int \frac{\ln x}{x}\dif x$;
      \task $\displaystyle\int \upe^{\sin x}\cos x\dif x$;
      \task $\displaystyle\int \sin(3x+5)\dif x$;
      \task $\displaystyle\int \tan x\dif x$。
    \end{tasks}
    \item 用换元积分法求不定积分:
    \begin{tasks}[before-skip=10pt,after-skip=10pt,after-item-skip=7pt](2)
      \task $\displaystyle\int \frac{1}{1+\sqrt{x+1}}\dif x$;
      \task $\displaystyle\int x\sqrt{x-6}\dif x$;
      \task $\displaystyle\int \frac{1}{x\sqrt{x-1}}\dif x$;
      \task $\displaystyle\int \sqrt{1-x^2}\dif x$。
    \end{tasks}
  \end{question}
\end{Practice}

\subsection{分部积分法}
我们知道,两个函数乘积的导数法则是
\[(uv)'=vu'+uv'.\]
移项,得
\[ uv'=(uv)'-vu'.\]
两边积分,得
\[ \int uv'\dif x= uv -\int vu'\dif x.\]
或者简写成如下形式:
\[ \tcbhighmath{\int u\dif v= uv- \int v \dif u}.\]
这个公式叫做\Concept{分部积分公式}。
如果求 $\int v\dif u$ 比较容易时,就可以利用分部积分公式,将求 $\int u\dif v$ 形式的不定积分转化为求 $\int v\dif u$ 形式的不定积分。
用这个公式求不定积分的方法叫做\Concept{分部积分法}。

\begin{example}
  求 $\displaystyle \int x\cos x\dif x$。
\end{example}
\begin{solution}
  设 $u=x$,$\dif v= \cos x\dif x$,则
  \[ \dif u=\dif x,\quad v=\sin x.\]
  由分部积分公式,得
  \begin{align*} 
  \int x\cos x \dif x & = x\sin x - \int\sin x\dif x \\
   &= x\sin x +\cos x +C.
  \end{align*}
\end{solution}

\begin{example}\label{exp:int-uv}
  求 $\displaystyle \int x\upe^{x} \dif x$。
\end{example}
\begin{solution}
  设 $u=x$,$\dif v= \upe^{x}\dif x$,则
  \[ \dif u=\dif x,\quad v=\upe^x.\]
  由分部积分公式,得
  \begin{align*} 
    \int x\upe^{x} \dif x & = x\upe^x - \int\upe^x\dif x \\
     &= x\upe^x - \upe^x +C.
    \end{align*}
\end{solution}

\alertwarning{用分部积分公式的关键是 $u$ 与 $\dif v$ 的选择要得当,否则可能会使问题愈来愈繁。
在\cref{exp:int-uv} 中,如果改设 $u=\upe^x$,$\dif v=x\dif x$,则 $\dif u= \upe^x\dif x$,$v=\frac{1}{2}x^2$,按分部积分公式得
\[\int x\upe^x\dif x= \frac{1}{2}x^2\upe^x-\dfrac{1}{2}\int x^2\upe^x\dif x\]
这时,上式右端第二项的积分比原来的积分更复杂了。}

\begin{example}
  求 $\displaystyle\int \arctan x\dif x$。
\end{example}
\begin{solution}
  将 $\arctan x$ 看作 $u$,$\dif x$ 看作 $\dif v$,即

  设 $u=\arctan x$,$\dif v= \dif x$,则
  \[ \dif u=\frac{1}{1+x^2}\dif x,\quad v=x.\]
  由分部积分公式,得
  \begin{align*} 
    \int \arctan x \dif x & = x\arctan x - \int\frac{x}{1+x^2}\dif x \\
     &= x\arctan x -\frac{1}{2}\int\frac{1}{1+x^2} \dif(1+x^2)\\
     &= x\arctan x -\frac{1}{2}\ln(1+x^2) +C.
    \end{align*}
\end{solution}

\begin{example}
  求 $\displaystyle\int x\ln x\dif x$。
\end{example}
\begin{solution}
  设 $u=\ln x$,$\dif v= x\dif x$,则
  \[ \dif u=\frac{1}{x}\dif x,\quad v=\frac{1}{2}x^2.\]
  由分部积分公式,得
  \begin{align*} 
    \int x\ln x \dif x & = \frac{1}{2}x^2\ln x - \int\frac{1}{x}\cdot\frac{1}{2}x^2\dif x \\
     &= \frac{1}{2}x^2\ln x -\frac{1}{2}\int x\dif x\\
     &= \frac{1}{2}x^2\ln x -\frac{1}{4}x^2 +C.
    \end{align*}
\end{solution}

\begin{Practice}
  用分部积分法求不定积分:
  \begin{tasks}[before-skip=10pt,after-skip=10pt,after-item-skip=7pt](2)
    \task $\displaystyle\int x\sin x\dif x$;
    \task $\displaystyle\int (x+1)\cos x\dif x$;
    \task $\displaystyle\int x\upe^{-x}\dif x$;
    \task $\displaystyle\int x\upe^{2x}\dif x$;
    \task $\displaystyle\int \arccot x\dif x$;
    \task $\displaystyle\int x\arctan x\dif x$;
    \task $\displaystyle\int \ln x\dif x$;
    \task $\displaystyle\int (x+1)\ln x\dif x$。
  \end{tasks}
\end{Practice}

\subsection{积分表的用法}
求一个函数的不定积分,比求它的导数或微分往往困难得多,因此我们把常见的被积函数的积分结果列成积分表。
这样,在实际计算积分时,就可以利用积分表来求得函数的积分。
本书后面附有简易积分表。

积分表是按被积函数的类型编排的。
查表求积分时,根据被积函数的类型,或经过适当的变换,化成表中所列函数类型,查出相应的公式,便可求得结果。

\begin{example}
  求 $\displaystyle\int\frac{1}{x^2(2+3x)} \dif x$
\end{example}
\begin{solution}
  这个被积函数是有理函数,可在\cref{sec:rational_int_eq}中查出公式 \ref{eqn:int-24} 是
  \[ \int \frac{1}{x^2(a+bx)}\dif x=-\frac{1}{ax}+\frac{b}{a^2}\ln\left|\frac{a+bx}{x}\right|+C \]
  因此,以 $a=2$,$b=3$ 代入这个公式,便得所求不定积分
  \[ \int \frac{1}{x^2(2+3x)}\dif x=-\frac{1}{2x}+\frac{3}{4}\ln\left|\frac{2+3x}{x}\right|+C \]
\end{solution}

\begin{example}
  求 $\displaystyle\int\frac{1}{x\sqrt{4x^2+9}} \dif x$
\end{example}
\begin{solution}
  这个被积式不能在表中直接查到,需要先进行变量替换。

  设 $2x=u$,那么 $\sqrt{4x^2+9}=\sqrt{u^2+3^2}$,$x = \dfrac{u}{2}$,$\dif x= \dfrac{1}{2}\dif u$,于是
  \begin{align*} 
    \int\frac{1}{x\sqrt{4x^2+9}} \dif x & = \int\frac{1}{\dfrac{u}{2}\sqrt{u^2+3^2}}\cdot\frac{1}{2}\dif u \\
    & = \int\frac{1}{u\sqrt{u^2+3^2}}\cdot\dif u.
  \end{align*}
  上式中的被积函数是无理函数,可在\cref{sec:irrational_int_eq}中查出公式 \ref{eqn:int-50} 是
  \[ \int \frac{1}{u\sqrt{u^2+a^2}}\dif u =-\frac{1}{a}\ln\left|\frac{a+\sqrt{u^2+a^2}}{u}\right| +C. \]
  因此,以 $a=3$ 代入这个公式,便得所求不定积分
  \[ \int \frac{1}{x\sqrt{u^2+3^2}}\dif u =-\frac{1}{3}\ln\left|\frac{3+\sqrt{u^2+3^2}}{u}\right| +C. \]
  最后把 $u=2x$ 代入,得到所求不定积分
  \[ \int \frac{1}{x\sqrt{4x^2+9}}\dif x =-\frac{1}{3}\ln\left|\frac{3+\sqrt{4x^2+9}}{2x}\right| +C. \]
\end{solution}

\begin{Practice}
  利用积分表求不定积分:
  \begin{tasks}[before-skip=10pt,after-skip=10pt,after-item-skip=7pt](2)
    \task $\displaystyle\int \frac{x}{(3x+5)^2}\dif x$;
    \task $\displaystyle\int \frac{x}{(1+x^2)^2}\dif x$;
    \task $\displaystyle\int \sin3x\sin5x\dif x$;
    \task $\displaystyle\int \sqrt{2x^2+1}\dif x$;
    \task $\displaystyle\int \frac{1}{3+5x^2}\dif x$;
    \task $\displaystyle\int \frac{1}{\sin x}\dif x$;
    \task $\displaystyle\int \sin^4x\dif x$;
    \task $\displaystyle\int x^4\ln x\dif x$。
  \end{tasks}
\end{Practice}

\begin{Exercise}
  \begin{question}
    \item 用换元积分法求不定积分:
    \begin{tasks}[before-skip=10pt,after-skip=10pt,after-item-skip=7pt](2)
      \task $\displaystyle\int \sqrt{2+3x}\dif x$;
      \task $\displaystyle\int \frac{1}{2x-1}\dif x$;
      \task $\displaystyle\int \frac{x^2}{1+x^3}\dif x$;
      \task $\displaystyle\int \frac{2x-3}{x^2-3x+8}\dif x$;
      \task $\displaystyle\int \sin(3x+1)\dif x$;
      \task $\displaystyle\int \cos(2-5x)\dif x$;
      \task $\displaystyle\int x\cos x^2\dif x$;
      \task $\displaystyle\int \frac{1}{x^2}\sin\frac{1}{x}\dif x$;
      \task $\displaystyle\int \sin^5x\cos x\dif x$;
      \task $\displaystyle\int \frac{\sin x}{(1+\cos x)^3}\dif x$;
      \task $\displaystyle\int \frac{\upe^x}{1+\upe^x}\dif x$;
      \task $\displaystyle\int \upe^{-3x}\dif x$;
      \task $\displaystyle\int 2x\upe^{-x^2}\dif x$;
      \task $\displaystyle\int \frac{\upe^x}{1+\upe^{2x}}\dif x$;
      \task $\displaystyle\int \frac{1}{9+4x^2}\dif x$;
      \task $\displaystyle\int (1+\tan x)\sec^2x\dif x$;
      \task $\displaystyle\int \sin6x\sin4x\dif x$;
      \task $\displaystyle\int \cos7x\cos3x\dif x$;
      \task $\displaystyle\int \frac{1}{x(1+2\ln x)}\dif x$;
      \task $\displaystyle\int \frac{x^2}{\sqrt{2-x^3}}\dif x$。
    \end{tasks}
    \item 用换元积分法求不定积分:
    \begin{tasks}[before-skip=10pt,after-skip=10pt,after-item-skip=7pt](2)
      \task $\displaystyle\int \frac{x+1}{\sqrt{3x+1}}\dif x$;
      \task $\displaystyle\int \frac{1}{1+\sqrt{2x}}\dif x$;
      \task $\displaystyle\int \frac{1}{x+\sqrt{x}}\dif x$;
      \task $\displaystyle\int \frac{\sqrt[4]{x}}{1+\sqrt{x}}\dif x$;
      \task $\displaystyle\int \frac{\upe^{\sqrt{x}}}{\sqrt{x}}\dif x$;
      \task $\displaystyle\int \frac{\sin\sqrt{x}}{\sqrt{x}}\dif x$;
      \task $\displaystyle\int \frac{\cos\sqrt{x}}{\sqrt{x}}\dif x$;
      \task $\displaystyle\int \frac{x^2}{\sqrt{1-x^2}}\dif x$。
    \end{tasks}
    \item 用分部积分法求不定积分:
    \begin{tasks}[before-skip=10pt,after-skip=10pt,after-item-skip=7pt](3)
      \task $\displaystyle\int x\sin3x\dif x$;
      \task $\displaystyle\int x\cos2x\dif x$;
      \task $\displaystyle\int (2x+1)\sin x\dif x$;
      \task $\displaystyle\int x\cos(\omega x+\varphi)\dif x$;
      \task $\displaystyle\int t\sin(\omega t+\varphi)\dif t$;
      \task $\displaystyle\int (x+1)\upe^{x}\dif x$;
      \task $\displaystyle\int x\upe^{4x}\dif x$;
      \task $\displaystyle\int x\upe^{-3x}\dif x$;
      \task $\displaystyle\int x\arccot x\dif x$;
      \task $\displaystyle\int \arcsin x\dif x$;
      \task $\displaystyle\int x\ln(3x-2)\dif x$;
      \task $\displaystyle\int \ln(1+x^2)\dif x$。
    \end{tasks}
    \item 用积分表求不定积分:
    \begin{tasks}[before-skip=10pt,after-skip=10pt,after-item-skip=7pt](3)
      \task $\displaystyle\int \frac{x}{\left(2x+\dfrac{1}{2}\right)^2}\dif x$;
      \task $\displaystyle\int \frac{1}{x^2+9}\dif x$;
      \task $\displaystyle\int \frac{1}{(x^2+2)^2}\dif x$;
      \task $\displaystyle\int \frac{1}{\sqrt{4x^2-9}}\dif x$;
      \task $\displaystyle\int \sqrt{2x^2+9}\dif x$;
      \task $\displaystyle\int \sqrt{3x^2-2}\dif x$;
      \task $\displaystyle\int \frac{1}{x^2+5x+6}\dif x$;
      \task $\displaystyle\int x^2\sin4x\dif x$;
      \task $\displaystyle\int x^5\ln x\dif x$;
      \task $\displaystyle\int \frac{1}{x^2(1-x)}\dif x$。
    \end{tasks}
  \end{question}
\end{Exercise}

\section*{小结}
\begin{enumerate}[C、,itemindent=4.5em]
  \item 本章主要内容是不定积分的概念及其求法。
  \item 如果 $F'(x)=f(x)$,那么就称 $F(x)$ 为函数 $f(x)$ 的一个原函数。

  如果函数 $f(x)$ 有一个原函数 $F(x)$,那么它的原函数就。有无穷多个,而且所有原函数可以用 $F(x)+C$($C$ 是任意常数)的形式表示出来。
  
  我们把函数 $f(x)$ 的所有的原函数 $F(x)+C$ 叫做函数 $f(x)$ 的不定积分,记作
  \[\int f(x)\dif x=F(x)+C.\]
  \item 与求已知函数的导数的运算相反,求不定积分是从导数求其原函数的运算,因此,求不定积分与求导数互为逆运算。于是,根据导数公式可以得到相应的不定积分公式;根据导数的某些运算法则可以导出相应的不定积分运算法则:
  \begin{gather*}
    \int kf(x) \dif x=k\int f(x)\dif x \quad(k\neq 0),\\
    \int [f(x)\pm g(x)]\dif x=\int f(x)\dif x\pm \int g(x)\dif x.
  \end{gather*}
  \item 求不定积分的基本方法有:
  \begin{description}
    \item[直接积分法]利用基本积分公式和不定积分的运算法则直接求得结果;
    \item[换元积分法]通过适当的变量替换,将被积式化为基本积分公式表中的某一被积式,求出结果;
    \item[分部积分法]利用分部积分公式将求 $\int u\dif v$ 形式的 不定积分化为求 $\int v\dif u$ 形式的不定积分,当后者较易求出结果时,即可用分部积分公式。
  \end{description}
\end{enumerate}

\chapter*{复习参考题\chinese{chapter}}
\section*{A 组}
\begin{question}
  \item 求不定积分:
  \begin{tasks}[before-skip=10pt,after-skip=10pt,after-item-skip=7pt](3)
    \task $\displaystyle \int x(3x^2-4)\dif x$
    \task $\displaystyle \int \left(\frac{2x^2+1}{x}\right)^3\dif x$
    \task $\displaystyle \int x(3x^2-4)\dif x$
    \task $\displaystyle \int x(3x^2-4)\dif x$
    \task $\displaystyle \int \frac{5\theta^2}{1+\theta^2}\dif \theta$
    \task $\displaystyle \int \frac{2x^2+1}{1+x^2}\dif x$
    \task $\displaystyle \int \frac{2+\cos^2x}{\cos^2x}\dif x$
    \task $\displaystyle \int \frac{1}{1+\cos2\theta}\dif \theta$
    \task $\displaystyle \int \left(\upe^{\frac{x}{2}}+\upe^{-\frac{x}{2}}\right)^2\dif x$
    \task $\displaystyle \int \frac{1+\cos^2x}{1+\cos2x}\dif x$
  \end{tasks}
  \item 用换元积分法求不定积分:
  \begin{tasks}[before-skip=10pt,after-skip=10pt,after-item-skip=7pt](2)
    \task $\displaystyle\int (3x+2)^3\dif x$;
    \task $\displaystyle\int x\sqrt[3]{8+9x^2}\dif x$;
    \task $\displaystyle\int \frac{2+\ln x}{x}\dif x$;
    \task $\displaystyle\int \upe^x(\upe^x+2)^2\dif x$;
    \task $\displaystyle\int \frac{3}{(1+x^2)\arctan x}\dif x$;
    \task $\displaystyle\int \frac{1}{\upe^x+\upe^{-x}}\dif x$;
    \task $\displaystyle\int (\sin x-\cos x)^2\dif x$;
    \task $\displaystyle\int \frac{1}{\cos^2(1-x)}\dif x$;
    \task $\displaystyle\int \tan nx \dif x$;
    \task $\displaystyle\int \frac{\tan m\theta}{\cos m\theta}\dif \theta$;
    \task $\displaystyle\int (\tan \theta-\sec \theta)^2\dif \theta$;
    \task $\displaystyle\int \upe^{\cos2\theta}\sin2\theta\dif \theta$;
    \task $\displaystyle\int \sqrt{\upe^x}\dif x$;
    \task $\displaystyle\int 2\upe^{\tan \theta}\sec^2\theta\dif \theta$;
    \task $\displaystyle\int \frac{1}{x^2\sqrt{1+x^2}}\dif x$;
    \task $\displaystyle\int \frac{\sqrt{x^2-1}}{x}\dif x$。
  \end{tasks}
  \item 用分部积分法求不定积分:
  \begin{tasks}[before-skip=10pt,after-skip=10pt,after-item-skip=7pt](3)
    \task $\displaystyle\int x\cos\frac{x}{2} \dif x$;
    \task $\displaystyle\int (x^2+1)\sin x\dif x$;
    \task $\displaystyle\int x\upe^{-2x}\dif x$;
    \task $\displaystyle\int (3x+4)\upe^{-3x}\dif x$;
    \task $\displaystyle\int \arctan\frac{x}{3}\dif x$;
    \task $\displaystyle\int x\arctan5x \dif x$;
    \task $\displaystyle\int x\sec^2x\dif x$;
    \task $\displaystyle\int x\csc^2x\dif x$;
    \task $\displaystyle\int \ln9x\dif x$;
    \task $\displaystyle\int \frac{\ln3x}{x^3}\dif x$。
  \end{tasks}
\end{question}
\section*{B 组}
\begin{question}[resume]
  \item 假设 
  \[\int f(x)\dif x=F(x)+C,\]
  求证
  \[\int f(ax+b)\dif x= \frac{1}{a}F(ax+b)+C,\]
  其中 $a$,$b$ 为常数,而且 $a\neq 0$。
  \item 证明下式:
  \[ \int f(x) \dif x = xf(x)-\int xf'(x)\dif x.\]
  \item 求不定积分:
  \begin{tasks}[before-skip=10pt,after-skip=10pt,after-item-skip=7pt](2)
    \task!$\displaystyle\int(a_nx^n+a_{n-1}x^{n-1}+\cdots+a_1x+a_0)\dif x$;
    \task $\displaystyle\int\frac{1}{(x-a)(x-b)} \dif x$;
    \task $\displaystyle\int\frac{1}{x^2-a^2}\dif x$;
    \task $\displaystyle\int\frac{x^2+7x+12}{x+4}\dif x$;
    \task $\displaystyle\int\cot^2x\dif x$;
    \task $\displaystyle\int(2^x+3^x)^2\dif x$。
  \end{tasks}
  \item 用换元积分法求不定积分:
  \begin{tasks}[before-skip=10pt,after-skip=10pt,after-item-skip=7pt](3)
    \task $\displaystyle\int\frac{1}{\sqrt{x^2+3x}} \dif x$;
    \task $\displaystyle\int\frac{1}{\sqrt{x-x^2}} \dif x$;
    \task $\displaystyle\int\frac{\sin x}{3+4\cos x} \dif x$;
    \task $\displaystyle\int\frac{\arcsin x}{\sqrt{1-x^2}} \dif x$;
    \task $\displaystyle\int\frac{\sqrt{\arctan2x}}{1+4x^2} \dif x$;
    \task $\displaystyle\int\frac{\upe^x-\upe^{-x}}{\upe^x+\upe^{-x}} \dif x$;
    \task $\displaystyle\int\frac{\sqrt{x^2-a^2}}{x} \dif x$;
    \task $\displaystyle\int\frac{x^3}{\sqrt{1-x^2}} \dif x$。
  \end{tasks}
  \item 用分部积分法求不定积分:
  \begin{tasks}(3)
    \task $\displaystyle\int x^2\cos x\dif x$;
    \task $\displaystyle\int \upe^x\sin x\dif x$;
    \task $\displaystyle\int \upe^x\cos x\dif x$;
    \task $\displaystyle\int x\arcsin x\dif x$;
    \task $\displaystyle\int x^2\upe^x\dif x$;
    \task $\displaystyle\int \ln^2x\dif x$。
  \end{tasks}
\end{question} % ok
\chapter{定积分及其应用}
\section{定积分的概念和计算}
\subsection{定积分的概念}\label{subsec:calculus_concept}
在生产和科学技术中,许多实际问题,如求面积、路程、体积等,最后都可归结为求一种和的极限。
现在我们以求面积和求路程为例,说明解决这类问题的方法,从而引出定积分的概念。

\subsubsection*{问题 1\quad 求曲边梯形的面积}

\medskip\noindent
\begin{minipage}{0.6\linewidth}\parindent2em
曲边梯形是指由三条直线 $x=a,\ x=b,\ y=0$ 和一条曲线 $y=f(x)$ 围成的图形(\cref{fig:5-1}),其中 $f(x)$ 是连续函数(这里假设 $f(x) \geqslant 0$)。

我们知道,矩形的面积公式是
\[\text{面积} = \text{长} \times \text{宽}.\]
\end{minipage}\hfill
\begin{minipage}{0.35\linewidth}\centering
  \begin{figurehere}
    \includegraphics{5-1.pdf}
    \caption{}\label{fig:5-1}
  \end{figurehere}
\end{minipage}

\medskip\noindent
现在研究的是曲边梯形的面积,就不能直接用这个公式来计算。
如\cref{fig:5-2} 所表示,为了计算曲边梯形的面积,可以将它分割成许多小曲边梯形,每个小曲边梯形用相应的小矩形近似代替,把这些小矩形的面积累加起来,就得到曲边梯形面积的一个近似值,当分割无限变细时,这个近似值就无限趋近于所求的曲边梯形面积。
现在用下例来说明具体的做法:
\begin{figure}
  \includegraphics{5-2.pdf}
  \caption{}\label{fig:5-2}
\end{figure}

\begin{example}\label{exp:x2_area}
  求由直线 $x=0,\ x=1,\ y=0$ 和曲线 $y=x^2$ 围成的图形面积。
\end{example}
\begin{solution}
  \begin{enumerate}
    \item 将曲边梯形分割成 $n$ 个小曲边梯形。
    
    用分点
    \[ 0=\frac{0}{n}<\frac{1}{n}<\cdots<\frac{i-1}{n}<\frac{i}{n}<\cdots<\frac{n-1}{n}<\frac{n}{n}=1\]
    \medskip\par\noindent
    \begin{minipage}{0.6\linewidth}
      把区间 $[0,1]$ 等分成 $n$ 个小区间(\cref{fig:5-3}):
    \[\left[0,\frac{1}{n}\right],\left[\frac{1}{n},\frac{2}{n}\right],\dots,\left[\frac{i-1}{n},\frac{i}{n}\right],\dots,\left[\frac{n-1}{n},\frac{n}{n}\right],\]
    每个小区间的长度为
    \[ \Delta x=\frac{i}{n}-\frac{i-1}{n}=\frac{1}{n}.\]
    过各分点作 $x$ 轴的垂线,把曲边梯形分成 $n$ 个小曲边梯形,它们的面积分别记作
    \end{minipage}\hfill
    \begin{minipage}{0.35\linewidth}
      \begin{figurehere}
        \includegraphics{5-3.pdf}
        \caption{}\label{fig:5-3}
      \end{figurehere}
    \end{minipage}
    \[\Delta S_1,\Delta S_2,\dots,\Delta S_i,\dots,\Delta S_n.\]
    \item 用小矩形面积近似代替小曲边梯形面积。
    
    在小区间 $\left[\dfrac{i-1}{n},\dfrac{i}{n}\right]$ 上任取一点 $\xi_i(i=1,2,\cdots,n)$,为了计算方便,取 $\xi_i$ 为小区间的左端点,用以点 $\xi_i$ 的纵坐标 $f(\xi_i)=\left(\dfrac{i-1}{n}\right)^{2}$ 为长,以小区间长度 $\Delta x=\dfrac{1}{n}$ 为宽的小矩形面积近似代替第 $i$ 个小曲边梯形面积,可以近似地表示为
    \[\Delta S_i\approx f(\xi_i)\Delta x=\left(\frac{i-1}{n}\right)^2\cdot\frac{1}{n}.\quad(i=1,2,\cdots\cdots,n)\]
    \item 取和。
    
    因为每一个小矩形的面积都可以作为相应的小曲边梯形面积的近似值,所以 $n$ 个小矩形面积的和就是曲边梯形面积 $S$ 的近似值,即
    \begin{equation}
      \label{eq:sum_of_area}
      S=\sum\limits_{i=1}^{n}\Delta S_i\approx\sum\limits_{i=1}^{n}f(\xi_i)\Delta x=\sum\limits_{i=1}^{n}\left(\frac{i-1}{n}\right)^2\cdot\frac{1}{n}.
    \end{equation}
    \item 求\cref{eq:sum_of_area} 的极限。
    
    当分点数目愈多,即 ${\Delta x}$ 愈小时,从\cref{fig:5-4} 可以看出,和\cref{eq:sum_of_area} 的值就愈接近曲边梯形的面积 $S$。因此,当 $n\to\infty$,即 $\Delta x\to0$ 时,和\cref{eq:sum_of_area} 的极限,就是所求的曲边梯形的面积。
    \begin{figure}
      \includegraphics{5-4.pdf}
      \caption{}\label{fig:5-4}
    \end{figure}
  因为
  \begin{align*}
    \sum\limits_{i=1}^{n}\left(\frac{i-1}{n}\right)^2\cdot\frac{1}{n}&=\frac{1}{n^3}\sum\limits_{i=1}^{n}(i-1)^2\\
    &=\frac{1}{n^3}[0^2+1^2+\cdots+(n-1)^2]\\
    &=\frac{1}{n^3}\cdot\frac{n(n-1)(2n-1)}{6}\\
    &=\frac16\left(1-\frac{1}{n}\right)\left(2-\frac{1}{n}\right).
  \end{align*}
  由此得到
  \[ S=\lim\limits_{n\to\infty}\sum\limits_{i=1}^{n}f(\xi_i)\Delta x=\lim\limits_{n\to\infty}\frac16\left(1-\frac{1}{n}\right)\left(2-\frac{1}{n}\right)=\frac{1}{3}.\]
  \end{enumerate}
\end{solution}

\subsubsection*{问题 2\quad 求变速直线运动的路程}
设一物体沿直线运动,它的速度 $v$ 是时间 $t$ 的函数 $v(t)$,求物体从时刻 $t=a$ 到 $t=b$ 这段时间所经过的路程 $s$。

我们知道,匀速直线运动的路程公式是
\[\text{路程}=\text{速度}\times\text{时间}.\]
现在研究的是变速直线运动,即速度是随时间的变化而变化的,因此不能直接用这公式计算路程。
为了计算变速直线运动的路程,将时间区间(即时间间隔)分割成许多小区间,当时间区间很短时,速度变化很小,可以认为在同一个时间区间内速度是不变的,这样在这很短的一段时间内可以用匀速运动公式计算小区间内路程的近似值,把这些小区间内路程的近似值加起来,得到所求路程的近似值,当分割无限变细时,这个近似值的极限就是所求的路程。

我们以自由落体运动为例,采用上述的方法和步骤,求物体在给定的时间内下落的距离。

\begin{example}\label{exp:falling}
  已知自由落体的运动速度 $v=gt$($g$ 是常数),求在时间区间 $[0,t]$ 内,物体由点 0 下落的距离 $\varepsilon$。
\end{example}
\begin{solution}
  \begin{enumerate}
    \item 将时间区间 $[0,t]$ 分成 $n$ 等份。
    
    用分点
    \[0=t_0<t_1<\cdots<t_{i-1}<t_i<\cdots<t_n=t\]
    把时间 $[0,t]$ 等分成 $n$ 个小区间
    \[\left[0,\frac{t}{n}\right],\left[\frac{t}{n},\frac{2t}{n}\right],\cdots,\left[\frac{(i-1)t}{n},\frac{it}{n}\right],\cdots,\left[\frac{(n-1)t}{n},\frac{nt}{n}\right],\]
    每个小区间所表示的时间为 $\Delta t=\dfrac{it}{n}-\dfrac{(i-1)t}{n}=\dfrac{t}{n}$。

    在各小区间物体下落的距离记作
    \[ \Delta s_1,\Delta s_2,\cdots,\Delta s_n.\]
    \item 在每个小区间上以匀速运动的路程近似代替变速运动的路程。
    
    在小区间 $[\dfrac{i-1}{n}t,\dfrac{i}{n}t]$ 上任取一时刻 $\xi_i\,(i=1,2,\cdots,n)$;为了计算方便,取 $\xi_i$ 为小区间的左端点,用时刻 $\xi_i$ 的速度 $v(\xi_i)=g\cdot\dfrac{(i-1)t}{n}$ 近似代替第 $i$ 个小区间上的速度,这样利用匀速运动的路程公式,每个小区间上自由落体在 $\Delta t=\frac{t}{n}$ 内所经过的距离,可以近似地表示为
    \[\Delta s_i\approx v(\xi_i)\Delta t=g\cdot\left(\frac{i-1}{n}\cdot t\right)\cdot\frac{t}{n}.\quad (i=1,2,\cdots,n)\]
    \item 取和。
    
    因为每个小区间上自由落体运动的距离可用该区间匀速直线运动的路程近似代替,所以在 $[0,t]$ 内落体运动的距离 $s$ ,就可以用这同一物体分别在 $n$ 个小区间上作 $n$ 个匀速直线运动的路程的和近似代替,即
    \begin{equation}
      \label{eq:sum_distance}
      s=\sum\limits_{i=1}^{n}\Delta s_i\approx\sum\limits_{i=1}^{n}v(\xi_i)\Delta t=\sum\limits_{i=1}^{n}g\cdot\frac{(i-1)t}{n}\cdot\frac{t}{n}
    \end{equation}
    \item 求和\cref{eq:sum_distance} 的极限。
    
    当所分时间区间愈短,即 $\Delta t=\dfrac{t}{n}$ 愈小时,和\cref{eq:sum_distance} 的值就愈接近 $s$。
    因此,当 $n\to\infty$,即 $\Delta t=\dfrac{t}{n}\to0$ 时,和\cref{eq:sum_distance} 的极限,就是所求的自由落体运动在时间 $[0,t]$ 内所经过的距离,因为
    \begin{align*}
      \sum\limits_{i=1}^{n}g\cdot\frac{(i-1)t}{n}\cdot\frac{t}{n}&=\frac{gt^2}{n^2}\sum\limits_{i=1}^{n}(i-1)=\frac{gt^2}{n^2}[0+1+2+\cdots+(n-1)]\\
      &=\frac{gt^2}{n^2}\left[\frac{n(n-1)}{2}\right]=\frac{gt^2}{2}\left(1-\frac{1}{n}\right).
    \end{align*}
    由此得到
    \[ s=\lim\limits_{n\to\infty}\sum\limits_{i=1}^{n}v(\xi_i)\Delta t=\lim\limits_{n\to\infty}\frac{gt^2}{2}\left(1-\frac{1}{n}\right)=\frac{1}{2}gt^2.
    \]
  \end{enumerate}
\end{solution}

上面两个实际问题,一个是求曲边梯形面积,一个是求变速直线运动的路程,虽然实际意义不同,但是解决问题的方法和计算步骤是完全相同的,最后都归结为求一个连续函数在某一闭区间上的和式的极限问题:
\begin{align*}
  &\text{曲边梯形的面积}&S&=\lim\limits_{n\to\infty}\sum\limits_{i=1}^{n} f(\xi_i)\Delta x;\\
  &\text{变速直线运动的路程}&s&=\lim\limits_{n\to\infty}\sum\limits_{i=1}^{n} v(\xi_i)\Delta t.
\end{align*}

类似的实际问题很多,都可以归结为求这种和式的极限。
因此,我们抛开这些问题的具体意义,抽象出解决这类问题的一般思想,给出定积分的概念。

设函数 $f(x)$ 在区间 $[a,b]$ 上连续,用分点
\[a=x_0<x_1<\cdots<x_{i-1}<x_i<\cdots<x_n=b\]
把区间 $[a,b]$ 等分成 $n$ 个小区间,在每个小区间 $[x_{i-1},x_i]$ 上取任一点 $\xi_i$($i=1,2,\cdots,n$),作和式
\[ I_n=\sum\limits_{i=1}^{n}f(\xi_i)\Delta x\ \text{(其中}\ \Delta x\text{ 为小区间长度),}\]
我们把 $n\to\infty$ 即 $\Delta x\to 0$ 时,和式 $I_n$ 的极限叫做函数 $f(x)$ 在区间 $[a,b]$ 上的\Concept{定积分},记作
\[\int_a^b f(x)\dif x,\]
即
\[\int_a^b f(x)\dif x=\lim\limits_{n\to\infty}\sum\limits_{i=1}^{n} f(\xi_i)\Delta x.\]
这里,$a$ 与 $b$ 分别叫做\Concept{积分下限}与\Concept{积分上限},区间 $[a,b]$ 叫做\Concept{积分区间},函数 $f(x)$ 叫做\Concept{被积函数},$x$ 叫做\Concept{积分变量},$f(x)\dif x$ 叫做\Concept{被积式}。

根据定积分的定义,就可以说:

曲边梯形的面积 $S$ 等于其曲边所对应的函数 $y=f(x)$ 在底边所在的区间 $[a,b]$ 上的定积分,即
\[ S=\int_a^b f(x)\dif x.\]
于是\cref{exp:x2_area} 的结果可写为
\[s=\int_0^1 f(x)\dif x=\int_0^1 x^2\dif x=\frac{1}{3}.\]

物体作变速直线运动所经过的路程 $s$ 等于其速度函数 $v=v(t)$ 在时间区间 $[a,b]$ 上的定积分,即
\[s=\int_a^b v(t)\dif t.\]
于是例 2 的结果可写为
\[s=\int_0^t v(t)\dif t=\int_0^t gt\dif t=\frac12gt^2.\]

定积分有下列三个主要性质:
\begin{Theorem}{性质 1}
  被积函数的常数因子可以提到积分号前面,即
  \[ \int_a^b kf(x)\dif x=k\int_a^b f(x)\dif x\quad\text{(}k\text{为常数)。}\]
\end{Theorem}
\begin{proof}
  由定义可得
  \begin{align*}
    \int_a^b kf(x)\dif x&=\lim\limits_{n\to\infty}\sum\limits_{i=1}^{n}kf(\xi_i)\Delta x=\lim\limits_{n\to\infty}k\sum\limits_{i=1}^{n}f(\xi_i)\Delta x\\
    &=k\lim\limits_{n\to\infty}\sum\limits_{i=1}^{n}f(\xi_i)\Delta x=k\int_a^b f(x)\dif x,
  \end{align*}
  即
  \[\int_a^b kf(x)\dif x=k\int_a^b f(x)\dif x.\]
\end{proof}
\begin{Theorem}{性质 2}
  两个函数的和(或差)在 $[a,b]$ 上的定积分,等于这两个函数在 $[a,b]$ 上的定积分的和(或差),即
  \[ \int_a^b [f(x)\pm g(x)]\dif x=\int_a^b f(x)\dif x\pm \int_a^b f(x)\dif x.\]
\end{Theorem}
\begin{proof}
  由定义可得
  \begin{align*}
    \int_a^b[f(x)\pm g(x)]\dif x&=\lim\limits_{n\to\infty}\sum\limits_{i=1}^{n}[f(\xi_i)\pm g(\xi_i)]\Delta x\\
    &=\lim\limits_{n\to\infty}\sum\limits_{i=1}^{n}f(\xi_i)\Delta x\pm \lim\limits_{n\to\infty}\sum\limits_{i=1}^{n}g(\xi_i)\Delta x\\
    &=\int_a^b f(x)\dif x\pm \int_a^b g(x)\dif x,
  \end{align*}
  即
  \[\int_a^b[f(x)\pm g(x)]\dif x=\int_a^b f(x)\dif x\pm \int_a^b g(x)\dif x.\]
\end{proof}
\begin{Theorem}{性质 3}
  如果将区间 $[a,b]$ 分成两个区间 $[a,c]$ 及 $[c,b]$(其中 $a<c<b$),那么
  \[ \int_a^b f(x)\dif x=\int_a^c f(x)\dif x+\int_c^b f(x)\dif x.\]
\end{Theorem}
\par\medskip\noindent
\begin{minipage}{0.55\linewidth}\parindent2em
定积分的性质 3,可以用\cref{fig:5-5} 直观地表示出来,即 
\[ S_{\text{曲边梯形} AabB}=S_{\text{曲边梯形} AacC}+S_{\text{曲边梯形} CcbB}.\]
这个性质的证明从略。
\end{minipage}\hfill
\begin{minipage}{0.4\linewidth}
\begin{figurehere}
  \includegraphics{5-5.pdf}
  \caption{}\label{fig:5-5}
\end{figurehere}
\end{minipage}

\begin{Practice}
  \begin{question}
    \item 用定积分定义求由 $y=x,\ x=1,\ x=2,\ y=0$ 围成的图形的面积。
    \item 将由 $y=\sin x,\ x=0,\ x=\dfrac{\uppi}{2},\ y=0$ 围成的图形的面积写成定积分的形式。
  \end{question}
\end{Practice}

\subsection{微积分基本公式}
从\cref{subsec:calculus_concept}我们看到,根据定义求定积分,要计算一个和式的极限,这往往是比较困难的。
下面,我们用求变速直线运动的路程的例子,来研究计算定积分的新方法。

设物体沿直线运动,速度为 $v(t)$,求从时刻 $t=a$ 到 $t=b$ 这段时间所经过的路程 $s$。

由定积分定义,物体所经过的路程为
\[s=\int_a^b v(t)\dif t\]
\begin{figure}
  \includegraphics{5-6.pdf}
  \caption{}\label{fig:5-6}
\end{figure}

另一方面,如\cref{fig:5-6},$t=a$ 时,路程为 $s(a)$;$t=b$ 时,路程为 $s(b)$。
由图中可以看出,从时刻 $t=a$ 到 $t=b$ 这段时间物体所经过的路程
\[s=s(b)-s(a).\]

由此得到
\[ \int_a^b v(t)\dif t=s(b)-s(a).\]

我们又知道,路程函数 $s(t)$ 与速度函数 $v(t)$ 有下面的关系:
\[ s'(t)=v(t),\]
即 $s(t)$ 是 $v(t)$ 的一个原函数。
因此,由上式可知,函数 $v(t)$ 在区间 $[a,b]$ 上的定积分,等于它的一个原函数 $s(t)$ 在积分区间 $[a,b]$ 上的改变量 $s(b)-s(a)$。

一般地,有下面的定理:
\begin{Theorem}{定理}
  设 $f(x)$ 是区间 $[a,b]$ 上的连续函数,$F(x)$ 是函数 $f(x)$ 在区间 $[a,b]$ 上的任一原函数,即 $F'(x)=f(x)$,则
  \[\tcbhighmath{\int_a^b f(x)\dif x=F(b)-F(a).}\]
\end{Theorem}
\begin{proof}
  用分点
  \[a=x_0<x_1<\cdots<x_{i-1}<x_i<\cdots<x_n=b\]
  将区间 $[a,b]$ 分为 $n$ 等份,每个小区间长度为 $\Delta x= \dfrac{b-a}{n}$。相应的函数 $F(x)$ 的总改变量 $F(b)- F(a)$ 可分为 $n$ 个部分改变量的和,即
  \begin{align}
    F(b)-F(a)=&F(x_n)-F(x_0)\notag\\
    =&[F(x_n)-F(x_{n-1})]+[F(x_{n-1})-F(x_{n-2})]+\cdots+\notag\\
    &[F(x_{i})-F(x_{i-1})]+\cdots+[F(x_1)-F(x_0)]\notag\\
    \label{eq:sum_function_1}=&\sum\limits_{i=1}^{n}[F(x_{i})-F(x_{i-1})].
  \end{align}
  根据拉格朗日中值定理,在每个小区间 $[x_{i-1},x_i]$ 内一定存在一点 $\xi_i$,使得
  \begin{equation}
    \label{eq:differ_function}
    F(x_i)-F(x_{i-1})=f(\xi_i)\Delta x.
  \end{equation}
  将\cref{eq:differ_function} 代入\cref{eq:sum_function_1},从而
  \[ F(b)-F(a)=\sum\limits_{i=1}^n f(\xi_i)\Delta x.\]
  当 $n\to\infty$ 时,根据定积分的定义,得
  \[ F(b)-F(a)=\int_a^b f(x)\dif x.\]
\end{proof}

这个公式叫做\Concept{微积分基本公式},又叫做\Concept{牛顿—莱布尼茨\footnote{牛顿(Isaac Newton,1642--1727 年),英国物理学家、数学家。莱布尼茨 (Gottfried Wilhelm Leibniz,1646--1716 年),德国数学家。}公式}。它表示了定积分与不定积分(或原函数)之间的关系,使我们可以借助求原函数来计算定积分,就是说,连续函数 $f(x)$ 在区间 $[a,b]$ 上的定积分 $\displaystyle\int_a^bf(x)\dif x$,等于函数 $f(x)$ 的任一原函数 $F(x)$ 在积分区间 $[a,b]$ 上的改变量 $F(b)- F(a)$。

通常地,原函数在区间 $[a,b]$ 的改变量 $F(b)-F(a)$ 简记作 $\left.F(x) \middle|_{a}^{b}\right.$,因此,微积分基本公式可以写成下面的形式:
\[\int_a^b f(x)\dif x=F(x)\Bigm|_a^b=F(b)-F(a).\]

\alertwarning{在计算定积分时,只写 $f(x)$ 的一个原函数 $F(x)$ ,不需要再加上任意常数 $C$,这是因为
\[\left.[F(x)+C]\middle|_a^b=[F(b)+C]-[F(a)+C]=F(b)-F(a).\right.\]
}
\begin{example}
  计算定积分:
  \begin{tasks}[before-skip=10pt,after-skip=10pt,label={(\arabic*)}](2)
    \task $\displaystyle \int_0^1 x^2\dif x$;
    \task $\displaystyle \int_1^2 \left(2x+\dfrac{1}{x}\right)\dif x$。
  \end{tasks}
\end{example}
\begin{solution}
  \begin{enumerate}
    \item 因为 $\dfrac13x^3$ 是 $x^2$ 的一个原函数,由微积分基本公式,有
    \[\int_0^1 x^2\dif x=\frac12x^3\Bigm|_0^1=\frac13\cdot 1^3-\frac13\cdot0^3=\frac13.\]
    这个结果与\cref{exp:x2_area} 按照定积分定义计算的结果相同。
    \item 
    \begin{align*}
      \int_1^2\left(2x+\frac{1}{x}\right)\dif x&=\int_1^22x\dif x+\int_1^2\frac{1}{x}\dif x=x^2\bigm|_1^2+\ln x\bigm|_1^2\\
      &=(4-1)+(\ln2-\ln1)=3+\ln2.
    \end{align*}
  \end{enumerate}
\end{solution}

\begin{example}
  计算定积分:
  \begin{tasks}[before-skip=10pt,after-skip=10pt,label={(\arabic*)}](2)
    \task $\displaystyle \int_0^{\frac\uppi2} \sin^2\frac{x}{2}\dif x$;
    \task $\displaystyle \int_0^a \frac{1}{a^2+x^2}\dif x$。
  \end{tasks}
\end{example}
\begin{solution}
  \begin{enumerate}
    \item
    \begin{align*}
      \int_0^{\frac\uppi2}\sin^2\frac x2\dif x&=\int_0^{\frac\uppi2}\frac12(1-\cos x)\dif x=\frac12\int_0^{\frac\uppi2}(1-\cos x)\dif x=\frac12(x-\sin x)\Bigm|_0^{\frac\uppi2}\\
      &=\frac{\uppi}{4}-\frac12.
    \end{align*}
    \item
    \[
      \int_0^a\frac{1}{a^2+x2}\dif x=\frac1a\arctan\frac{x}{a}\Bigm|_0^a=\frac1a\arctan1-\frac1a\arctan0=\frac1a\cdot\frac{\uppi}{4}-0=\frac{\uppi}{4a}.
    \]
  \end{enumerate}
\end{solution}

\begin{example}
  计算 $\displaystyle\int_0^1x\sqrt{1+x^2}\dif x$。
\end{example}
\begin{solution}
  \begin{align*}
    \because\quad \int x\sqrt{1+x^2}\dif x&=\frac12\int(1+x^2)^{\frac12}\dif (1+x^2)=\frac12\cdot\frac23(1+x^2)^{\frac32}+C\\
    &=\frac13(1+x^2)^{\frac32}+C.\\
    \therefore\quad \int_0^1 x\sqrt{1+x^2}\dif x&=\frac13(1+x^2)^{\frac32}\Bigm|_0^1=\frac{2\sqrt{2}-1}{3}.
  \end{align*}
\end{solution}

\begin{example}
  求 $\displaystyle \int_0^a\sqrt{a^2-x^2}\dif x$。
\end{example}
\begin{solution}
  由\cref{subsec:Substitution_Integration}\cref{exp:arctan} 有
  \[\int \sqrt{a^2-x^2}\dif x=\frac{a^2}{2}\arcsin\frac{x}{a}+\frac{x}{2}\sqrt{a^2-x^2}+C,\]
  因此
  \begin{align*}
    \int_0^a\sqrt{a^2-x^2}\dif x&=\left(\frac{a^2}{2}\arcsin\frac{x}{a}+\frac{x}{2}\sqrt{a^2-x^2}\right)\biggm|_0^a=\frac{a^2}{2}\arcsin 1-0\\&=\frac{a^2}{2}\cdot\frac{\uppi}{2}=\frac{1}{4}\uppi a^2.
  \end{align*}
\end{solution}

\begin{Practice}
  计算定积分:
  \begin{tasks}[before-skip=10pt,after-skip=10pt,after-item-skip=7pt](2)
    \task $\displaystyle \int_0^5 2x \dif x$;
    \task $\displaystyle \int_0^2 (x^2-2x)\dif x$;
    \task $\displaystyle \int_1^2 (\sqrt{x}-1)\dif x$;
    \task $\displaystyle \int_0^2 (4-2x)(4-x^2) \dif x$;
    \task $\displaystyle \int_1^2 \left(x-\frac1x\right)^2 \dif x$;
    \task $\displaystyle \int_1^2 \frac{x^2+2x-3}{x}\dif x$;
    \task $\displaystyle \int_0^{\uppi} \cos x \dif x$;
    \task $\displaystyle \int_0^{\frac\uppi2} \sin x\dif x$;
    \task $\displaystyle \int_1^2 \left(\upe^x+\frac1x\right) \dif x$;
    \task $\displaystyle \int_0^{\frac{1}{2}} \frac{1}{\sqrt{1-x^2}}\dif x$。
  \end{tasks}
\end{Practice}

\begin{Exercise}
  计算定积分:
  \begin{tasks}[before-skip=10pt,after-skip=10pt,after-item-skip=7pt](2)
    \task $\displaystyle\int_{-1}^{3} (3x^2-2x+1)\dif x$;
    \task $\displaystyle\int_{1}^{2} \frac{1}{x^2}\dif x$;
    \task $\displaystyle\int_{2}^{3} \left(\sqrt{x}+\frac{1}{\sqrt{x}}\right)^2\dif x$;
    \task $\displaystyle\int_{0}^{\frac{\uppi}{2}} (3x+\sin x)\dif x$;
    \task $\displaystyle\int_{0}^{\frac{\uppi}{2}} \cos x\dif x$;
    \task $\displaystyle\int_{\frac{\uppi}{6}}^{\frac{\uppi}{4}} \cos 2x\dif x$;
    \task $\displaystyle\int_{0}^{\frac\uppi6} \frac{1}{\cos^22x} \dif x$;
    \task $\displaystyle\int_{\frac\uppi4}^{\frac\uppi3}\cot^2x \dif x$;
    \task $\displaystyle\int_{0}^{1} \frac{1}{1+x^2}\dif x$;
    \task $\displaystyle\int_{0}^{2} \frac{1}{4+x^2}\dif x$;
    \task $\displaystyle\int_{-\frac12}^{\frac12} \frac{1}{\sqrt{1-x^2}}\dif x$;
    \task $\displaystyle\int_{-1}^{1} \frac{1}{\sqrt{5-4x}}\dif x$;
    \task $\displaystyle\int_{0}^{1} \frac{x}{(1+x^2)^3}\dif x$;
    \task $\displaystyle\int_{1}^{\upe} \frac{2+\ln x}{x}\dif x$。
  \end{tasks}
\end{Exercise}

\section{定积分的应用}
\subsection{平面图形的面积}
我们已经知道,由三条直线 $x=a$,$x=b$($a<b$),$x$ 轴及一条曲线 $y=f(x)$ 围成的曲边梯形的面积为
\[\tcbhighmath{S=\int_a^bf(x)\dif x.}\]
这里 $f(x)\geqslant 0$(\cref{fig:5-7})。
\begin{figure}
  \begin{minipage}[b]{0.48\linewidth}\centering
    \includegraphics{5-7.pdf}
    \caption{}\label{fig:5-7}
  \end{minipage}
  \begin{minipage}[b]{0.48\linewidth}\centering
    \includegraphics{5-8.pdf}
    \caption{}\label{fig:5-8}
  \end{minipage}
\end{figure}

如果图形由曲线 $y_1=f_1(x)$,$y_2=f_2(x)$(不妨设 $f_1(x)\geqslant f_2(x)\geqslant 0$),及直线 $x=a$,$x=b$($a<b$)围成(\cref{fig:5-8}),那么所求面积为
\[ S=\int_a^bf_1(x)\dif x-\int_a^bf_2(x)\dif x.\]

\begin{example}
  计算曲线 $y=x^2-2x+3$ 与直线 $y=x+3$ 所围图形的面积。
\end{example}
\begin{solution}
如\cref{fig:5-9},为了确定图形的范围,先求出已知曲线与直线的交点的横坐标。解方程组
\[\begin{cases}y=x+3,\\ y=x^2-2x+3,\end{cases}\]
得出交点的横坐标围 $x=0$ 及 $x=3$。
\end{solution}
\par\medskip\noindent
\begin{minipage}{0.6\linewidth}\parindent2em
从而所求图形的面积
\begin{align*}
  S&=\int_0^3(x+3)\dif x-\int_0^3(x^2-2x+3)\dif x\\
  &=\int_0^3[(x+3)-(x^2-2x+3)]\dif x\\
  &=\int_0^3(-x^2+30)\dif x\\
  &=\left(-\frac13x^3+\frac32x^2\right)\biggr\vert_0^3=\frac92.\\
\end{align*}
\end{minipage}\hfill
\begin{minipage}{0.35\linewidth}
  \begin{figurehere}
    \includegraphics{5-9.pdf}
    \caption{}\label{fig:5-9}
  \end{figurehere}
\end{minipage}


\begin{example}
  求椭圆 $\dfrac{x^2}{a^2}+\dfrac{y^2}{b^2}=1$ 的面积。
\end{example}
\noindent
\begin{minipage}{0.6\linewidth}\parindent2em
\begin{solution}
  如\cref{fig:5-10},这个椭圆关于坐标轴对称,所以只须求出椭圆在第一象限的面积
  \[S_1=\int_a^b y\dif x,\]
  便可得出椭圆面积:
  \[S=4S_1.\]
\end{solution}
\end{minipage}\hfill
\begin{minipage}{0.35\linewidth}
  \begin{figurehere}
    \includegraphics{5-10.pdf}
    \caption{}\label{fig:5-10}
  \end{figurehere}
\end{minipage}
\par\medskip
由椭圆方程 $\dfrac{x^2}{a^2}+\dfrac{y^2}{b^2}=1$ 得 $y=\pm\dfrac{b}{a}\sqrt{a^2-x^2}$。因为我们只考虑第一象限,所以取 $y=\dfrac{b}{a}\sqrt{a^2-x^2}$。于是
\begin{align*}
  S&=4\int_0^a\frac{b}{a}\sqrt{a^2-x^2}\dif x=4\cdot\frac{b}{a}\int_0^a\sqrt{a^2-x^2}\dif x\\
  &=4\cdot\frac{b}{a}\left(\frac{a^2}{2}\arcsin\frac{x}{a}+\frac{x}{2}\sqrt{a^2-x^2}\right)\biggr|_0^a=4\cdot\frac{b}{a}\left(\frac{a^2}{2}\arcsin1-0\right)\\
  &=4\cdot\frac{b}{a}\cdot\frac{a^2}{2}\cdot\frac{\uppi}{2}=\uppi ab.\\
\end{align*}

当 $a=b=r$ 时,上式就成为圆的面积公式
\[S=\uppi r^2,\]
其中 $r$ 为圆的半径。

\begin{example}
  计算曲线 $y^2=x$,$y=x^2$ 所围图形的面积。
\end{example}
\noindent
\begin{minipage}{0.6\linewidth}\parindent2em
\begin{solution}
  如\cref{fig:5-11},为确定图形的范围,首先求这两条曲线的交点横坐标。解方程组
  \[\begin{cases}y^2=x,\\y=x^2,\end{cases}\]
  得出交点横坐标为 $x=0$ 及 $x=1$。
\end{solution}
\end{minipage}\hfill
\begin{minipage}{0.35\linewidth}
  \begin{figurehere}
    \includegraphics{5-11.pdf}
    \caption{}\label{fig:5-11}
  \end{figurehere}
\end{minipage}
\par\medskip
因此所求图形的面积
\[ S=\int_0^1\sqrt{x}\dif x-\int_0^1x^2\dif x=\frac23x^{\frac32}\Bigr|_0^1-\frac13x^3\Bigr|_0^1=\frac23-\frac13=\frac13.\]

\alertwarning{
  \begin{minipage}{0.55\linewidth}
    \begin{enumerate}
      \item 如果在区间 $[a,b]$ 上 $f(x)\leqslant0$(\cref{fig:5-12}),那么 $\displaystyle \int_a^b f(x)\dif x\leqslant 0$,这时曲边梯形的面积
      \[S=\left|\int_a^b f(x)\dif x\right|=-\int_a^b f(x)\dif x.\]
      \item 如果在区间 $[a,c]$ 上 $f(x)\leqslant0$,在区间 $[c,b]$ 上 $f(x)\geqslant0$(\cref{fig:5-13}),那么我们看到,所求面积
      \[S=\left|\int_a^c f(x)\dif x\right|+\int_c^b f(x)\dif x.\]
    \end{enumerate}
  \end{minipage}\hfill
  \begin{minipage}{0.4\linewidth}
    \begin{figurehere}
      \includegraphics{5-12.pdf}
      \caption{}\label{fig:5-12}
    \end{figurehere}\par
    \begin{figurehere}
      \includegraphics{5-13.pdf}
      \caption{}\label{fig:5-13}
    \end{figurehere}
  \end{minipage}
}

\begin{Practice}
  求下列曲线围成的图形的面积:
  \begin{tasks}[before-skip=5pt,after-skip=5pt,after-item-skip=7pt](2)
    \task $y=x^2$,$y=2x+3$。
    \task $y=1-x^2$,$y=0$。
    \task $y=\upe^3$,$x=2$,$x=4$,$y=0$。
    \task $y=\cos x$,$x=-\dfrac\uppi2$,$x=\dfrac\uppi2$,$y=0$。
    \task $y=x^2$,$y=-x^2+8$。
  \end{tasks}
\end{Practice}

\subsection{旋转体的体积}
以前我们学过的圆柱、圆锥、圆台、球等几何体都是简单的旋转体。
旋转体就是一平面图形绕这平面内的一条直线旋转一周而成的几何体。
现在我们利用计算定积分的方法来计算它的体积。

设旋转体是曲线 $y=f(x)$,直线 $x=a$,$x=b$ 以及 $x$ 轴所
围曲边梯形 ${AabB}$ 绕 $x$ 轴旋转一周而成的(\cref{fig:5-14})。
\begin{figure}
  \includegraphics{5-14.pdf}
  \caption{}\label{fig:5-14}
\end{figure}

我们仿照计算曲边梯形面积的方法,用 $n-1$ 个垂直于 $x$ 轴的平面,把区间 $[a,b]$ 等分成 $n$ 个小区间 $[x_{i-1},x_i]$,($i=1,2,\ldots,n$,其中 $x_0=a$,$x_n=b$),旋转体被分成 $n$ 个厚度相同的薄片,每个薄片的厚度为 $\Delta x=\dfrac{b-a}{n}$。
如果薄片很薄,可以把每个薄片近似地认为是小圆柱体,它的底面半径可用区间上任一点 $\xi_i$ 的纵坐标 $f(\xi_i)$ 来近似代替,它的高为薄片的厚度 $\Delta x=\dfrac{b-a}{n}$。这样,第 $i$ 个小薄片的体积 $\Delta V_i$,就可用同这个区间对应的小圆柱体的体积来近似代替,即
\[ \Delta V_i\approx \uppi[f(\xi_i)]^2\Delta x\quad(i=1,2,\ldots,n).\]
所以,整个旋转体的体积
\[ V\approx \sum\limits_{i=1}^{n}\uppi[f(\xi_i)]^2\Delta x.\]
当 $n\to\infty$,即 $\Delta x\to0$ 时,上式右边的极限,就是旋转体的体积 $V$,即
\[ V=\lim\limits_{n\to\infty}\sum\limits_{i=1}^{n}\uppi[f(\xi_i)]^2\Delta x.\]
根据定积分定义,便得到旋转体的体积公式
\[\tcbhighmath{V=\uppi\int_a^b[f(x)]^2\dif x.}\]

\begin{example}
  用积分法推出两底面半径分别为 $R$,$r$,高为 $H$ 的圆台的体积计算公式。
\end{example}
\noindent
\begin{minipage}{0.55\linewidth}
\begin{solution}
  取坐标系如\cref{fig:5-15},圆台母线 $AB$ 过点 $A\,(0,r)$,$B\,(H,R)$,因此,母线 $AB$ 的方程为
  \[\frac{y-r}{x-0}=\frac{R-r}{H-0}\]
  即
  \[y=\frac{R-r}{H}x+r\quad(0\leqslant x\leqslant H).\]
\end{solution}
\end{minipage}\hfill
\begin{minipage}{0.4\linewidth}
  \begin{figurehere}
    \includegraphics{5-15.pdf}
    \caption{}\label{fig:5-15}
  \end{figurehere}
\end{minipage}
\par\medskip
根据旋转体体积公式得出圆台的体积
\begin{align*}
  V&=\uppi\int_0^H\left(\frac{R-r}{H}+r\right)^2\dif x=\frac{\uppi H}{3(R-r)}\left(\frac{R-r}{H}+r\right)^3\biggr|_0^H\\
   &=\frac{\uppi H}{3(R-r)}(R^3-r^3)=\frac{\uppi H}{3}(R^2+Rr+r^2).
\end{align*}

如果其中一个底面的半径 $r=0$,则圆台就变成了圆锥,其体积公式为
\[V=\frac{\uppi R^2H}{3}.\]

\begin{example}
  求椭圆 $\dfrac{x^2}{a^2}+\dfrac{y^2}{b^2}=1$ 绕 $x$ 轴旋转所成的旋转体的体积。
\end{example}
\noindent
\begin{minipage}{0.6\linewidth}
\begin{solution}
  如\cref{fig:5-16},因为 $y^2=\dfrac{b^2}{a^2}(a^2-x^2)$,并且所求的体积 $V$ 是曲边梯形 $BOA$ 绕 $x$ 轴旋转一周所成旋转体的体积 $V_1$ 的 2 倍,所以,根据旋转体的体积公式,有
\end{solution}
\end{minipage}
\begin{minipage}{0.35\linewidth}
  \begin{figurehere}
    \includegraphics{5-16.pdf}
    \caption{}\label{fig:5-16}
  \end{figurehere}
\end{minipage}

\begin{align*}
  V&=2V_1=2\uppi\int_0^a\frac{b^2}{a^2}(a^2-x^2)\dif x=2\uppi\frac{b^2}{a^2}\int_0^a(a^2-x^2)\dif x\\
  &=2\uppi\frac{b^2}{a^2}\left(a^2x-\frac13x^3\right)\bigg|_0^a=\frac43\uppi ab^2.
\end{align*}

当 $a=b=r$ 时,上式就成为球的体积公式
\[ V_\text{球}=\frac43\uppi r^3,\]
其中 $r$ 为球的半径。

\begin{example}
  求圆 $x^2+(y-b)^2=a^2\ (0<a<b)$ 绕 $x$ 轴旋转所成的旋转体的体积。
\end{example}
\begin{solution}
  如\cref{fig:5-17},圆的方程可以改写成
  \[y=b\pm\sqrt{a^2-x^2},\]
  \begin{figure}
    \includegraphics{5-17.pdf}
    \caption{}\label{fig:5-17}
  \end{figure}
  其中,上半圆 $MKN$ 的方程是
  \[y=b+\sqrt{a^2-x^2},\]
  下半圆 $MLN$ 的方程是
  \[y=b-\sqrt{a^2-x^2}.\]
  所求的体积,是这两个半圆分别与直线 $x=\pm a$ 及 $x$ 轴围成的曲边梯形绕 $x$ 轴旋转所成的两个旋转体的体积的差,因此
  \begin{align*}
    V&=\uppi\int_{-a}^{a}(b+\sqrt{a^2-x^2})\dif x-\uppi\int_{-a}^{a}(b-\sqrt{a^2-x^2})\dif x\\
    &=4\uppi b\int_{-a}^a\sqrt{a^2-x^2}\dif x=4\uppi b\left(\frac{a^2}{2}\arcsin\frac{x}{a}+\frac{x}{2}\sqrt{a^2-x^2}\right)\biggr|_{-a}^{a}\\
    &=2\uppi ba^2(\arcsin1+\arcsin1)=2\uppi^2a^2b.
  \end{align*}
\end{solution}

\begin{Practice}
  \begin{question}
    \item 将直线 $\dfrac{x}{2}+y=1$ 及两条坐标轴围成的三角形绕 $x$ 轴旋转,利用定积分计算所得的圆锥体的体积。
    \item 求曲线 $y^2=4x$,直线 $x=0$,$x=4$ 及 $x$ 轴所围图形绕 $x$ 轴旋转而成的旋转体的体积。
    \item 求曲线 $y=\cos x$ 从 $x=-\dfrac{\uppi}{2}$ 到 $x=\dfrac{\uppi}{2}$ 的一段绕 $x$ 轴旋转而成的旋转体的体积。
  \end{question}
\end{Practice}

\subsection{平面曲线的弧长}
我们已经知道,任一线段的长度,可以直接度量求得,任一已知半径和弧度或角度的圆弧的长度,可以用公式 $l=\alpha R$(或 $\dfrac{n\uppi}{180}R$)求得。
但是,我们还不会求平面上任意一条曲线的弧长。
现在,我们采用与求曲边梯形面积类似的方法来求曲线的弧长。

设曲线 $AB$ 的方程为 $y=f(x)$,($a\leqslant x\leqslant b$)。
这里函数 $f(x)$ 在区间 $[a,b]$ 上可导,而且 $f'(x)$ 连续。

现在求曲线 $AB$ 的长 $l$(见\cref{fig:5-18})。
\begin{figure}
  \includegraphics{5-18.pdf}
  \caption{}\label{fig:5-18}
\end{figure}
\begin{enumerate}
  \item 将$\overparen{AB}$ 分割为 $n$ 段小弧。
  
  用 $n-1$ 个垂直于 $x$ 轴的垂线,把区间 $[a,b]$ 等分成 $n$ 个小区间
  \[[x_{i-1},x_i]\quad\text{(}i=1,2,\ldots,n\text{,其中}\ x_0=a,x_n=b\text{ )。}\]
  每个小区间长度为
  \[ \Delta x=\frac{b-a}{n}.\]

  平面曲线 ${AB}$ 被分成 $n$ 段弧,连结每段弧的弦,得内接曲线 ${AB}$ 的折线 (如\cref{fig:5-18} 所示),第 $i$ 个小区间上所对应的第 $i$ 段弧 $A_{i-1}A_i$ 两个端点坐标为
  \[A_{i-1}\,(x_{i-1},f(x_{i-1})),\quad A_i\,(x_i,f(x_i)).\]
  根据两点间的距离公式,得折线第 $i$ 段长
  \[ |A_{i-1}A_i|=\sqrt{(x_i-x_{i-1})^2+[f(x_i)-f(x_{i-1})]^2} \quad (i=1,2,\ldots,n).\]
  \item 用折线代替曲线。
  
  当小区间很小时,第 $i$ 段弧 $A_{i-1}A_i$ 的长就可以用第 $i$ 段折线 $|A_{i-1}A_i|$ 的长近似代替,于是曲线 $AB$ 的长 $l$ 的近似值
  \begin{equation}
    \label{eq:curve_length}
    l_n=\sum\limits_{i=1}^{n}\sqrt{(x_i-x_{i-1})^2+[f(x_i)-f(x_{i-1})]^2},
  \end{equation}
  由拉格朗日中值定理,有
  \[ f(x_i)-f(x_{i-1})=f'(\xi_i)\Delta x \quad (x_{i-1}<\xi_i<x_i),\]
  其中
  \[\Delta x=x_i-x_{i-1}=\frac{b-a}{n}.\]
  这时,
  \begin{align*}
    t_n&=\sum\limits_{i=1}^{n}\sqrt{(\Delta x)^2+[f'(\xi_i)\Delta x]^2}\\
    &=\sum\limits_{i=1}^{n}\sqrt{1+[f'(\xi_i)]^2}\Delta x,
  \end{align*}
  当 $n\to\infty$,即 $\Delta x\to0$ 时,折线长 $l_n$ 的极限,就是曲线 $AB$ 的长 $l$,即
  \[ l=\lim\limits_{n\to\infty}l_n=\lim\limits_{n\to\infty}\sum\limits_{i=1}^{n}\sqrt{1+[f'(\xi_i)]^2}\Delta x.\]
  根据定积分的定义,便得到平面曲线的弧长公式:
  \[\tcbhighmath{l=\int_a^b\sqrt{1+[f'(x)]^2}\dif x.}\]
\end{enumerate}

\begin{example}
  已知圆的方程 $x^2+y^2=R^2$,求第一象限内端点横坐标 $x=0$ 到 $x=b$($b<R$)的弧 $AB$ 的长 $l$。
\end{example}
\noindent
\begin{minipage}{0.65\linewidth}
\begin{solution}
  如\cref{fig:5-19},第一象限圆弧方程为
  \[y=\sqrt{R^2-x^2},\]
  于是
  \begin{gather*}
    y'=-\frac{x}{\sqrt{R^2-x^2}},\\[5pt]
    \sqrt{1+(y')^2}=\sqrt{1+\left(-\frac{x}{\sqrt{R^2-x^2}}\right)^2}=\frac{R}{\sqrt{R^2-x^2}}.
  \end{gather*}
\end{solution}
\end{minipage}\hfill
\begin{minipage}{0.3\linewidth}
  \begin{figurehere}
    \includegraphics{5-19.pdf}
    \caption{}\label{fig:5-19}
  \end{figurehere}
\end{minipage}
\par\medskip\noindent
根据弧长公式,得弧 $AB$ 的长
\[l=\int_0^b\sqrt{1+(y')^2}\dif x=\int_0^b\frac{R}{\sqrt{R^2-x^2}}\dif x=R\arcsin\frac{x}{R}\biggr|_0^b=R\arcsin\frac{b}{R}.\]

\begin{example}
  已知悬链线方程为 $y=\dfrac{a}{2}\left(\upe^{\frac{x}{a}}+\upe^{-\frac{x}{a}}\right)$,求端点横坐标从 $x=0$ 到 $x=a$($a>0$)一段的弧长 $l$。
\end{example}
\noindent
\begin{minipage}{0.6\linewidth}
\begin{solution}
  如\cref{fig:5-20}。
  \begin{align*}
    y'&=\left[\frac{a}{2}\left(\upe^{\frac{x}{a}}+\upe^{-\frac{x}{a}}\right)\right]'\\
    &=\frac12\left(\upe^{\frac{x}{a}}-\upe^{-\frac{x}{a}}\right)\\
    \sqrt{1+(y')^2}&=\sqrt{1+\frac14\left(\upe^{\frac{x}{a}}-\upe^{-\frac{x}{a}}\right)^2}\\
    &=\frac12\left(\upe^{\frac{x}{a}}+\upe^{-\frac{x}{a}}\right),
  \end{align*}
\end{solution}
\end{minipage}\hfill
\begin{minipage}{0.35\linewidth}
  \begin{figurehere}
    \includegraphics{5-20.pdf}
    \caption{}\label{fig:5-20}
  \end{figurehere}
\end{minipage}
\begin{align*}
  l&=\int_0^a\sqrt{1+(y')^2}\dif x=\int_0^a\frac12\left(\upe^{\frac{x}{a}}+\upe^{-\frac{x}{a}}\right)\dif x\\
  &=\frac{a}{2}\left(\upe^{\frac{x}{a}}-\upe^{-\frac{x}{a}}\right)\biggr|_0^a=\frac{a}{2}\left(\upe-\frac1\upe\right).
\end{align*}

\begin{Practice}
  \begin{question}
    \item 求抛物线 $y=\dfrac12x^2$ 在端点横坐标 $x=-1$ 到 $x=1$ 之间的弧长。
    \item 求半立方抛物线 $y=\dfrac23x^{\frac32}$ 在端点横坐标 $x=0$ 到 $x=8$ 之间的弧长。
  \end{question}
\end{Practice}

\subsection{旋转体的侧面积}
设旋转体是由曲线 $y=f(x)$,直线 $x=a$,$x=b$ 以及 $x$ 轴所围曲边梯形 $AabB$ 绕 $x$ 轴旋转一周而成的(图\cref{fig:5-21a})。
\begin{figure}
  \begin{minipage}[b]{0.48\linewidth}\centering
    \includegraphics{5-21a.pdf}
    \subcaption{}\label{fig:5-21a}
  \end{minipage}
  \begin{minipage}[b]{0.48\linewidth}\centering
    \includegraphics{5-21b.pdf}
    \subcaption{}\label{fig:5-21b}
  \end{minipage}
  \caption{}\label{fig:5-21}
\end{figure}

用 $n-1$ 个垂直于 $x$ 轴的平面把区间 $[a,b]$ 等分成 $n$ 个小区间 $[x_{i-1},x_i]$,($i=1,2,\cdots,n$,其中 $x_0=a$,$x_n=b$),旋转体被分成 $n$ 个厚度相同的薄片,取第 $i$ 片,它可以看作是曲线 $MN$,直线 $x=x_{i-1}$,$x=x_i$ 及 $x$ 轴所围小曲边梯形 $Mx_{i-1}x_iN$ 绕 $x$ 轴旋转一周而成的小旋转体,它的侧面积用 $\Delta S_i$ 表示(\cref{fig:5-21})。

从\cref{fig:5-21b} 可以看出,当 $n$ 充分大,即 $\Delta x$ 充分小时,弧 $MN$ 可近似地用弦 $MN$ 代替。
因此,由弧 $MN$ 绕 $x$ 轴旋转而成的第 $i$ 个小薄片的侧面积 $\Delta S_i$,就可近似地用由弦 $MN$ 绕 $x$ 轴旋转而成的小圆台的侧面积来代替。
这第 $i$ 个小圆台上、下底面的半径分别为 $f(x_{i-1})$ 与 $f(x_i)$,母线 $MN=\sqrt{MQ^2+QN^2}$,由\cref{fig:5-21b} 可知其中的 $MQ=\Delta x$,$QN=f(x_1)-f(x_{i-1})$,根据圆台的侧面积公式,可得
\begin{align*}
  \Delta S_i&\approx \uppi[f(x_{i-1})+f(x_i)]\sqrt{(\Delta x)^2+[f(x_i)-f(x_{i-1})]^2}\\
  &=\uppi[f(x_{i-1})+f(x_i)]\sqrt{1+\left[\frac{f(x_i)-f(x_{i-1})}{\Delta x}\right]^2}\cdot\Delta x.
\end{align*}
所以,整个旋转体的侧面积
\[S\approx\sum\limits_{i=1}^{n}\uppi[f(x_{i-1})+f(x_i)]\sqrt{1+\left[\frac{f(x_i)-f(x_{i-1})}{\Delta x}\right]^2}\Delta x\]
当 $n \rightarrow \infty$ 时,上式右端的极限,就是侧面积 $S$,即
\[S=\lim\limits_{n\to\infty}\sum\limits_{i=1}^{n}\uppi[f(x_{i-1})+f(x_i)]\sqrt{1+\left[\frac{f(x_i)-f(x_{i-1})}{\Delta x}\right]^2}\Delta x\]
可以证明(本书从略)这个极限就是
\[2\uppi\int_a^b f(x)\sqrt{1+[f'(x)]^2}\dif x.\]
于是旋转体的侧面积公式为
\[\tcbhighmath{S_\text{侧}=2\uppi\int_a^b f(x)\sqrt{1+[f'(x)]^2}\dif x.}\]

\begin{example}
  圆 $x^2+y^2=r^2$ 绕 $x$ 轴旋转形成球面,求由 $x=x_1$ 到 $x=x_2$ 的球带面积。
\end{example}
\noindent
\begin{minipage}{0.6\linewidth}
\begin{solution}
  如\cref{fig:5-22},球带的表面积 $S$ 等于曲线
  \[y=\sqrt{r^2-x^2}\quad(x_1\leqslant x\leqslant x_2),\]
  绕 $x$ 轴旋转所成的曲面面积。由旋转体侧面积公式,得
  \[ S=2\uppi\int_{x_1}^{x_2}y\sqrt{1+(y')^2}\dif x,\]
\end{solution}
\end{minipage}\hfill
\begin{minipage}{0.36\linewidth}
  \begin{figurehere}
    \includegraphics{5-22.pdf}
    \caption{}\label{fig:5-22}
  \end{figurehere}
\end{minipage}
\par\noindent 其中
\begin{align*}
  y&=\sqrt{r^2-x^2}\\
  y'&=-\frac{x}{\sqrt{r^2-x^2}}=-\frac{x}{y},\\
  \sqrt{1+(y')^2}&=\sqrt{1+\left(\frac{x}{y}\right)^2}=\frac{\sqrt{x^2+y^2}}{y}=\frac{r}{y},\\
  y\sqrt{1+(y')^2}&=y\cdot\frac{r}{y}=r.
\end{align*}
于是
\[S=2\uppi\int_{x_1}^{x_2}r\dif x=2\uppi r\int_{x_1}^{x_2}\dif x=2\uppi rx\Bigr|_{x_1}^{x_2}=2\uppi r(x_2-x_1).\]

如果把球带的高 $x_2-x_1$ 记为 $h$,那么得出球带的面积
\[ S=2\uppi rh.\]

特别当 $x_1=-r$,$x_2=r$ 时,$h=x_2-x_1=2r$,上面公式变成了球面积公式:
\[ S=4\uppi r^2.\]
\begin{example}
  求在 $x=0$ 与 $x=3a$ 之间的抛物线 $y^2=4ax$ 绕 $x$ 轴旋转而成的曲面面积。
\end{example}
\begin{solution}
  如\cref{fig:5-23},由旋转体侧面积公式,得
  \[ S=2\uppi\int_0^{3a} y\sqrt{1+(y')^2}\dif x,\]
  \begin{figure}
    \includegraphics{5-23.pdf}
    \caption{}\label{fig:5-23}
  \end{figure}
  \par\noindent
  其中
  \begin{align*}
    y&=\sqrt{4ax},\\
    y'&=(\sqrt{4ax})'=\frac{\sqrt{a}}{\sqrt{x}}\\
    \sqrt{1+(y')^2} &=\sqrt{1+\frac{x}{a}}=\frac{\sqrt{x+a}}{\sqrt{x}}\\
    y\sqrt{1+(y')^2}&=\sqrt{4ax}\cdot\frac{\sqrt{x+a}}{\sqrt{x}}=2\sqrt{a}\cdot\sqrt{x+a}.
  \end{align*}
  于是
  \begin{align*}
    S&=2\uppi\int_0^{3a}2\sqrt{a}\cdot\sqrt{x+a}\dif x=4\uppi\sqrt{a}\cdot\frac23(x+a)^{\frac32}\biggr|_0^{3a}\\
    &=\frac83\uppi\sqrt{a}\Bigl[(4a)^{\frac32}-a^{\frac32}\Bigr]=\frac{56}{3}\uppi a^2.
  \end{align*}
\end{solution}

\begin{Practice}
  求曲线 $y^2=x$,直线 $x=0$,$x=6$ 所围图形绕 $x$ 轴旋转所得旋转体的侧面积。
\end{Practice}

\begin{Exercise}
  \begin{question}
    \item 求下列曲线所围图形的面积:
    \begin{tasks}[before-skip=10pt,after-skip=10pt,after-item-skip=7pt]
      \task 曲线 $y=4-x^2$ 与 $x$ 轴;
      \task 曲线 $2y=x^2$ 与直线 $x=y-4$;
      \task 半圆 $y=\sqrt{25-x^2}$,$x$ 轴,直线 $x=-3$,$x=4$;
      \task 曲线 $y=2x-x^2$,$y=2x^2-4x$;
      \task 曲线 $y=x^2+2$,$y=2x$,$x=0$,$x=2$;
      \task 曲线 $y=2x^2$,$y=x^2$,$x=1$;
      \task 曲线 $\sqrt{x}+\sqrt{y}=1$,$x=0$,$y=0$;
      \task 曲线 $y=\sin x$,$x=\dfrac{\uppi}{4}$,$x=\uppi$,$y=0$;
      \task 曲线 $y=\dfrac{1}{x}$,$x=1$,$x=\upe$,$y=0$。
    \end{tasks}
    \item 求下列曲线围成的图形绕 $x$ 轴旋转所成的旋转体的体积:
    \begin{tasks}[before-skip=10pt,after-skip=10pt,after-item-skip=7pt]
      \task $y=4-x^2$ 与 $x$ 轴;
      \task $y=\sqrt{4+x^2}$,$x=-2$,$x=2$,$x$ 轴;
      \task $y=x^2$,$y=\sqrt{x}$;
      \task $y=\sin x$,$y=\cos x$,$x$ 轴上的线段 $\left[0,\dfrac{\uppi}{2}\right]$。
    \end{tasks}
    \item 求曲线 $y=\ln(1-x^2)$ 由 $x=0$ 到 $x=\dfrac12$ 之间的弧长。
    \item 求抛物线 $y=\dfrac{x^2}{2p}$ 由顶点到点 $A\,(\sqrt{2}p,p)$ 之间的弧长。
    \item 求曲线 $y=\dfrac14x^2-\dfrac12\ln x$ 在 $1\leqslant x\leqslant \upe$ 之间的弧长。
    \item 求抛物线 $y^2=4x$,直线 $x=10$ 所围图形绕 $x$ 轴旋转所得旋转体的侧面积。
    \item 求圆 $x^2+(y-2)^2=1$ 绕 $x$ 轴旋转而成的旋转体的表面积。
    \item 用旋转体侧面积公式验证高为 $H$,底面半径为 $R$ 的圆锥侧面积公式为 $\uppi R\sqrt{R^2+H^2}$。
  \end{question}
\end{Exercise}

\section*{小结}
\begin{enumerate}[C、,itemindent=4.5em]
  \item 本章主要内容是定积分的概念、计算及其简单应用。
  \item 定积分的概念是从求曲边梯形的面积、变速直线运动的路程等实际问题引入的。解决这类问题都是通过分割,取近似,最后归结为求一种和式的极限:
  \[ \lim_{n \to \infty}\sum_{i=1}^{n} f(\xi_i)\Delta x. \]
  (其中 $f(x)$ 为区间 $[a,b]$ 上的连续函数,把区间 $[a,b]$ $n$ 等分后,$\Delta x= \dfrac{b-a}{n}$,而 $\xi_i$ 是第 $i$ 个小区间上的任意一点)。这个极限叫做函数 $f(x)$ 在区间 $[a,b]$ 上的定积分,记作
  \[\int_a^b f(x)\dif x=\lim_{n\to\infty}\sum_{i=1}^{n} f(\xi_i)\Delta x.\]
  \item 微积分基本公式是
  \[\int_a^b f(x) \dif x= F(b)-F(a)\]
  其中 $F(x)$ 是函数 $f(x)$ 的任一原函数,即 $F'(x)=f(x)$,就是说,函数 $f(x)$ 在区间 $[a,b]$ 上的定积分 $\int_a^b f(x)\dif x$,等于它的任一原函数 $F(x)$ 在区间 $[a,b]$ 上的改变量 $F(b)-F(a)$。
  这个公式是定积分与原函数之问的关系式,它使定积分的计算大为简化。
  \item 定积分的一些简单应用:
  \begin{enumerate}[1.]
    \item 求曲边梯形的面积,公式是
    \[S=\int_a^b f(x)\dif x;\]
    \item 求旋转体的体积,公式是
    \[V=\uppi\int_a^b [f(x)]^2\dif x;\]
    \item 求平面曲线弧长,公式是
    \[l=\int_a^b \sqrt{1+[f'(x)]^2}\dif x;\]
    \item 求旋转体的侧面积,公式是
    \[S=2\uppi\int_a^b f(x)\sqrt{1+[f'(x)]^2}\dif x.\]
  \end{enumerate}
\end{enumerate}

\chapter*{复习参考题\chinese{chapter}}
\section*{A 组}
\begin{question}
  \item 计算定积分:
  \begin{tasks}[before-skip=10pt,after-skip=10pt,after-item-skip=7pt](2)
    \task $\displaystyle \int_0^a (3x^2-x+1) \dif x$;
    \task $\displaystyle \int_1^2 \left(x^2+\frac{1}{x^4}\right) \dif x$;
    \task $\displaystyle \int_2^4 \frac{x^3-3x^2+5}{x^2} \dif x$;
    \task $\displaystyle \int_1^3 y^2(y-2) \dif y$;
    \task $\displaystyle \int_{-1}^{1} x(x-3) \dif x$;
    \task $\displaystyle \int_{-2}^{2} (6x^3+x+1) \dif x$;
    \task $\displaystyle \int_{-1}^{1} \left(x^2-\frac{x}{x^2+1}\right) \dif x$;
    \task $\displaystyle \int_{0}^{\frac{1}{3}} \frac{1}{4-3x} \dif x$;
    \task $\displaystyle \int_{\frac{1}{2}}^{1} \sqrt{3-2x} \dif x$.
  \end{tasks}
  \item 计算定积分:
  \begin{tasks}[before-skip=10pt,after-skip=10pt,after-item-skip=7pt](2)
    \task $\displaystyle \int_0^{\uppi} \sqrt{1-\cos2x} \dif x$;
    \task $\displaystyle \int_{\frac{\uppi}{6}}^{\frac{\uppi}{2}} \cos^2u \dif u$;
    \task $\displaystyle \int_0^{\frac{\uppi}{4}} \tan^2\theta \dif \theta$;
    \task $\displaystyle \int_{\frac{\uppi}{3}}^{\frac{2\uppi}{3}} (2\sin x+\cos x) \dif x$;
    \task $\displaystyle \int_0^{\frac{\uppi}{2}} \sin\varphi\cos^2\varphi \dif \varphi$;
    \task $\displaystyle \int_0^4 \frac{1}{1+\sqrt{x}} \dif x$;
    \task $\displaystyle \int_0^{e-1} \ln(x+1) \dif x$;
    \task $\displaystyle \int_0^1 xe^x \dif x$。
  \end{tasks}
  \item 求下列各曲线围成的图形的面积:
  \begin{tasks}
    \task 曲线 $y=x^3$,$y=x^2$,直线 $x=1$,$x=2$;
    \task 曲线 $y=\sin x$,$y=\cos x$,直线 $x=-\dfrac{\uppi}{4}$,$x=\dfrac{\uppi}{4}$;
    \task 曲线 $y=\dfrac{1}{x}$,直线 $y=x$,$x=2$,$y=0$;
    \task 曲线 $y=x^2$,直线 $y=x$,$y=2x$;
    \task 曲线 $y=x^2-4x+5$,直线 $x=3$,$x=5$,$y=0$;
    \task 曲线 $y=3-2x-x^2$,$y=0$。
  \end{tasks}
  \item 求下列曲线所围图形绕 $x$ 轴旋转而成的旋转体体积:
  \begin{tasks}
    \task $y=x^3$,$x=2$,$y=0$;
    \task $y=\cos x$,$x=-\dfrac{\uppi}{4}$,$x=\dfrac{\uppi}{4}$,$y=0$;
    \task $xy=4$,$x=1$,$x=4$,$y=0$;
    \task $x^2-y^2=a^2$,$x=a+h$,($a>0,h>0$);
    \task $y=1+\sqrt{x}$,$y=3$,$x=0$。
  \end{tasks}
  \item 求曲线 $y=\dfrac{x^2}{2}-2$ 与 $x$ 轴交点间的曲线弧长。
  \item 将立方抛物线 $a^2y=x^3$ 由 $x=0$ 到 $x=a$ 的一段弧,绕 $x$ 轴旋转一周,求旋转面的面积。
  \item 星形线 $x^{\frac{2}{3}}+y^{\frac{2}{3}}=a^{\frac{2}{3}}$ 绕 $x$ 轴旋转一周,求所得曲面面积。
\end{question}
\section*{B 组}
\begin{question}[resume]
  \item 计算定积分:
  \begin{tasks}[before-skip=10pt,after-skip=10pt,after-item-skip=7pt](2)
    \task $\displaystyle \int_0^{2a} (x-a)^3 \dif x$;
    \task $\displaystyle \int_{-2}^{0} x^3(x-a)^2 \dif x$;
    \task $\displaystyle \int_{-a}^{0} \left(\frac{x+a}{a}\right)^2 \dif x$;
    \task $\displaystyle \int_{-\uppi}^{\uppi} \sin 2x\sin 4x \dif x$。
  \end{tasks}
  \item 求抛物线 $y=-x^2+4x-3$ 及其在点 $A\,(0,-3)$ 与点 $B\,(3,0)$ 处的切线所围图形的面积。
  \item \label{exec:t-10}如图,已知曲线方程 $y^2=x^2(1-x^2)$,求图中阴影部分的面积。
  \begin{figurehere}
    \begin{minipage}{\linewidth}\centering
      \includegraphics{ext-10.pdf}
      \caption*{(第 \ref{exec:t-10} 题图)}
    \end{minipage}
  \end{figurehere}
  \item 过椭圆 $\dfrac{x^2}{5}+y^2=1$ 的两个焦点作 $x$ 轴的垂线,将椭圆的夹在这两条垂线间的部分与这两条垂线及 $x$ 轴所围曲边梯形绕 $x$ 轴旋转,求得到的旋转体的体积。
  \item 求曲线 $9ay^2=x(x-3a)^2$ 由 $x=0$ 到 $x=3a$ 的弧长。
  \item 求 $x^2+(y-b)^2=a^2$($b>a$)绕 $x$ 轴旋转所成的旋转体的表面积。
\end{question} % ok
\appendix
\ctexset{section/name={,},section/number={(\chinese{section})}}
\chapter*{\appendixname\quad 简易积分表}
\addcontentsline{toc}{chapter}{附表}
\section{基本积分公式}\label{sec:basic_int_eq}
\begin{enumerate}[1.,itemsep=5pt]
  \item $\displaystyle \int \dif x=x+C$
  \item $\displaystyle \int x^n\dif x=\frac{x^n}{n+1}+C$
  \item $\displaystyle \int \frac{1}{x}\dif x=\ln|x|+C$
  \item $\displaystyle \int e^x\dif x=e^x+C$
  \item $\displaystyle \int a^x\dif x=\frac{1}{\ln a}a^x+C$
  \item $\displaystyle \int \sin x\dif x=-\cos x+C$
  \item $\displaystyle \int \cos x\dif x=\sin x+C$
  \item $\displaystyle \int \tan x\dif x=-\ln|\cos x|+C$
  \item $\displaystyle \int \cot x\dif x=\ln|\sin x|+C$
  \item $\displaystyle \int \sec^2 x\dif x=\int \frac{1}{\cos^2 x} \dif x=\tan x+C$
  \item $\displaystyle \int \csc^2 x\dif x=\int \frac{1}{\sin^2 x} \dif x=-\cot x+C$
  \item $\displaystyle \int \sec x\dif x=\int \frac{1}{\cos x} \dif x=\ln|\sec x+\tan x|+C =\ln\left|\tan\left(\frac{x}{2}+\frac{\uppi}{4}\right)\right| +C$
  \item $\displaystyle \int \csc x\dif x=\int \frac{1}{\sin x} \dif x=\ln|\csc x-\cot x|+C =\ln\left|\tan\frac{x}{2}\right|+C$
  \item $\displaystyle \int \sec x\tan x\dif x=\sec x+C$
  \item $\displaystyle \int \csc x\cot x\dif x=-\csc x+C$
  \item $\displaystyle \int \frac{1}{\sqrt{a^2-x^2}}\dif x=\arcsin \frac{x}{a}+C = -\arccos\frac{x}{a} +C$
  \item $\displaystyle \int \frac{1}{a^2+x^2}\dif x=\frac{1}{a}\arctan \frac{x}{a}+C$
\end{enumerate}
\section{有理函数的积分}\label{sec:rational_int_eq}
\begin{enumerate}[1.,itemsep=5pt,resume]
  \item $\displaystyle \int \frac{1}{a+bx}\dif x=\frac{1}{b}\ln|a+bx|+C$
  \item $\displaystyle \int (a+bx)^n\dif x=\frac{(a+bx)^{n+1}}{b(n+1)}+C\,(n\neq -1)$
  \item $\displaystyle \int \frac{x}{(a+bx)^2}\dif x=\frac{1}{b^2}\left[ \frac{a}{a+bx}+ \ln|a+bx|\right]+C$
  \item $\displaystyle \int \frac{x^2}{(a+bx)^2}\dif x=\frac{1}{b^3}\left[ a+bx-\frac{a^2}{a+bx}-2a\ln|a+bx|\right]+C$
  \item $\displaystyle \int \frac{1}{x(a+bx)}\dif x=-\frac{1}{a}\ln\left|\frac{a+bx}{x}\right|+C$
  \item $\displaystyle \int \frac{1}{(x+a)(x+b)}\dif x=\frac{1}{b-a}\ln\left|\frac{x+a}{x+b}\right|+C$
  \item\label{eqn:int-24} $\displaystyle \int \frac{1}{x^2(a+bx)}\dif x=-\frac{1}{ax}+\frac{b}{a^2}\ln\left|\frac{a+bx}{x}\right|+C$
  \item $\displaystyle \int \frac{1}{x(a+bx)^2}\dif x=\frac{1}{a(a+bx)}-\frac{1}{a^2}\ln\left|\frac{a+bx}{x}\right|+C$
  \item $\displaystyle \int \frac{1}{x^2(a+bx)^2}\dif x=-\frac{1}{a^3}\left[\frac{a+bx}{x}-2b\ln\left|\frac{a+bx}{x}\right|-\frac{b^2x}{a+bx}\right]+C$
  \item $\displaystyle \int \frac{1}{a+bx^2}\dif x =\frac{1}{\sqrt{ab}}\arctan\frac{x\sqrt{ab}}{a} +C$($a$、$b$ 同号)
  \item $\displaystyle \int \frac{1}{a+bx^2}\dif x =\frac{1}{2\sqrt{-ab}}\ln\left|\frac{a+\sqrt{-ab}x}{a-\sqrt{-ab}x}\right|+C$($a$、$b$ 异号)
  \item $\displaystyle \int \frac{x}{a+bx^2}\dif x =\frac{1}{2b}\ln\left|a+bx^2\right|+C$
  \item $\displaystyle \int \frac{x}{a^2\pm b^2x^2}\dif x =\frac{1}{\pm 2b^2}\ln\left|a^2\pm b^2x^2\right|+C$
  \item $\displaystyle \int \frac{1}{a^2+b^2x^2}\dif x =\frac{1}{ab}\arctan\frac{bx}{a}+C$
  \item $\displaystyle \int \frac{1}{a^2-b^2x^2}\dif x =\frac{1}{2ab}\ln\left|\frac{a+bx}{a-bx}\right|+C$
  \item $\displaystyle \int \frac{1}{x(a^2\pm b^2x^2)}\dif x =\frac{1}{2a^2}\ln\left|\frac{x^2}{a^2\pm b^2x^2}\right|+C$
  \item $\displaystyle \int \frac{1}{x^2(a^2+b^2x^2)}\dif x =-\frac{x}{a^2}-\frac{b}{a^3}\arctan\frac{bx}{a}+C$
  \item $\displaystyle \int \frac{1}{(a^2+b^2x^2)^2}\dif x =\frac{x}{2a^2(a^2+b^2x^2)}+\frac{1}{2a^3b}\arctan\frac{bx}{a}+C$
  \item $\displaystyle \int \frac{1}{(a^2-b^2x^2)^2}\dif x =\frac{x}{2a^2(a^2-b^2x^2)}+\frac{1}{4a^3b}\ln\left|\frac{a+bx}{a-bx}\right|+C$
  \item $\displaystyle \int \frac{1}{a+bx+cx^2}\dif x =\frac{2}{\sqrt{4ac-b^2}}\arctan\left(\frac{2cx+b}{\sqrt{4ac-b^2}} \right) +C$($b^2<4ac$)
  \item $\displaystyle \int \frac{1}{a+bx+cx^2}\dif x =\frac{1}{\sqrt{b^2-4ac}}\ln\left|\frac{2cx+b-\sqrt{b^2-4ac}}{2cx+b+\sqrt{b^2-4ac}}\right|+C$($b^2>4ac$)
\end{enumerate}
\section{无理函数的积分}\label{sec:irrational_int_eq}
\begin{enumerate}[1.,itemsep=5pt,resume]
  \item $\displaystyle \int x\sqrt{a+bx}\dif x=-\frac{2(2a-3bx)(a+bx)^{\frac{3}{2}}}{15b^2} +C$
  \item $\displaystyle \int x^2\sqrt{a+bx}\dif x=-\frac{2(8a^2-12abx+15b^2x^2)(a+bx)^{\frac{3}{2}}}{105b^3} +C$
  \item $\displaystyle \int \frac{x}{\sqrt{a+bx}}\dif x =-\frac{2(2a-bx)\sqrt{a+bx}}{3b^2} +C$
  \item $\displaystyle \int \frac{x^2}{\sqrt{a+bx}}\dif x =\frac{2(8a^2-4abx+3b^2x^2)\sqrt{a+bx}}{15b^3} +C$
  \item $\displaystyle \int \frac{1}{x\sqrt{a+bx}}\dif x =\frac{1}{\sqrt{a}}\ln\left|\frac{\sqrt{a+bx}-\sqrt{a}}{\sqrt{a+bx}+\sqrt{a}}\right|+C$($a>0$)
  \item $\displaystyle \int \frac{1}{x\sqrt{a+bx}}\dif x =\frac{2}{\sqrt{-a}}\arctan\sqrt{\frac{a+bx}{-a}}+C$($a<0$)
  \item $\displaystyle \int \frac{1}{\sqrt{x^2\pm a^2}}\dif x = \ln\left|x+\sqrt{x^2\pm a^2}\right|+C$
  \item $\displaystyle \int \sqrt{x^2\pm a^2}\dif x =\frac{x}{2}\sqrt{x^2\pm a^2}\pm\frac{a^2}{2}\ln\left|x+\sqrt{x^2\pm a^2}\right| +C$
  \item $\displaystyle \int (x^2\pm a^2)^{\frac{3}{2}}\dif x =\frac{x}{8}(2x^2\pm5a^2)\sqrt{x^2\pm a^2}+\frac{3a^4}{8}\ln\left|x+\sqrt{x^2\pm a^2}\right| +C$
  \item $\displaystyle \int \frac{1}{(x^2\pm a^2)^{\frac{3}{2}}}\dif x =\frac{x}{\pm a^2\sqrt{x^2\pm a^2}} +C$
  \item $\displaystyle \int \frac{x^2}{\sqrt{x^2\pm a^2}}\dif x =\frac{x}{2}\sqrt{x^2+a^2}\mp\frac{a^2}{2}\ln\left|x+\sqrt{x^2\pm a^2}\right| +C$
  \item\label{eqn:int-50} $\displaystyle \int \frac{1}{x\sqrt{x^2+a^2}}\dif x =-\frac{1}{a}\ln\left|\frac{a+\sqrt{x^2+a^2}}{x}\right| +C$
  \item $\displaystyle \int \frac{1}{x\sqrt{x^2-a^2}}\dif x =\frac{1}{a}\arccos\frac{a}{x} +C$
  \item $\displaystyle \int \frac{1}{x^2\sqrt{x^2\pm a^2}}\dif x =\mp\frac{\sqrt{x^2\pm a^2}}{a^x} +C$
  \item $\displaystyle \int \frac{\sqrt{x^2+a^2}}{x}\dif x =\sqrt{x^2+a^2}-a\ln\left|\frac{a+\sqrt{x^2+a^2}}{x}\right| +C$
  \item $\displaystyle \int \frac{\sqrt{x^2-a^2}}{x}\dif x =\sqrt{x^2-a^2}-a\arccos\frac{a}{x} +C$
  \item $\displaystyle \int \frac{\sqrt{x^2\pm a^2}}{x^2}\dif x =-\frac{\sqrt{x^2\pm a^2}}{x}+\ln\left|x+\sqrt{x^2\pm a^2}\right| +C$
  \item $\displaystyle \int \sqrt{a^2-x^2}\dif x =\frac{x}{2}\sqrt{a^2-x^2}+\frac{a^2}{2}\arcsin\frac{x}{a} +C$
  \item $\displaystyle \int (a^2-x^2)^{\frac{3}{2}}\dif x =\frac{x}{8}(5a^2-2x^2)\sqrt{a^2-x^2}+\frac{3}{8}a^4\arcsin\frac{x}{a} +C$
  \item $\displaystyle \int \frac{1}{\sqrt{a^2-x^2}}\dif x =\arcsin\frac{x}{a} +C$
  \item $\displaystyle \int \frac{1}{(a^2-x^2)^{\frac{3}{2}}}\dif x =\frac{x}{a^2\sqrt{a^2-x^2}} +C$
  \item $\displaystyle \int x^2\sqrt{a^2-x^2}\dif x =\frac{x}{8}(2x^2-a^2)\sqrt{a^2-x^2}+\frac{a^4}{8}\arcsin\frac{x}{a}+C$
  \item $\displaystyle \int \frac{x^2}{\sqrt{a^2-x^2}}\dif x =-\frac{x}{2}\sqrt{a^2-x^2}+\frac{a^2}{2}\arcsin\frac{x}{a} +C$
  \item $\displaystyle \int \frac{1}{x\sqrt{a^2-x^2}}\dif x =-\frac{1}{a}\ln\left|\frac{a+\sqrt{a^2-x^2}}{x}\right| +C$
  \item $\displaystyle \int \frac{1}{x^2\sqrt{a^2-x^2}}\dif x = -\frac{\sqrt{a^2-x^2}}{a^2x}+C$
  \item $\displaystyle \int \frac{\sqrt{a^2-x^2}}{x}\dif x =\sqrt{a^2-x^2}-a\ln\left|\frac{a+\sqrt{a^2-x^2}}{x}\right| +C$
  \item $\displaystyle \int \frac{\sqrt{a^2-x^2}}{x^2}\dif x =-\frac{\sqrt{a^2-x^2}}{x}-\arcsin\frac{x}{a}+C$
  \item $\displaystyle \int \sqrt{2ax-x^2}\dif x =\frac{x-a}{2}\sqrt{2ax-x^2}+\frac{a^2}{2}\arccos\left(1-\frac{x}{a}\right)+C$
  \item $\displaystyle \int x\sqrt{2ax-x^2}\dif x =-\frac{3a^2+ax-2x^2}{6}\sqrt{2ax-x^2}+\frac{a^3}{2}\arccos\left(1-\frac{x}{a}\right)+C$
  \item $\displaystyle \int \frac{\sqrt{2ax-x^2}}{x}\dif x = \sqrt{2ax-x^2}+a\arccos\left(1-\frac{x}{a}\right)+C$
  \item $\displaystyle \int \frac{\sqrt{2ax-x^2}}{x^2}\dif x = -\frac{2\sqrt{2ax-x^2}}{x}-\arccos\left(1-\frac{x}{a}\right)+C$
  \item $\displaystyle \int \frac{1}{\sqrt{2ax-x^2}}\dif x = \arccos\left(1-\frac{x}{a}\right)+C$
  \item $\displaystyle \int \frac{x}{\sqrt{2ax-x^2}}\dif x =-\sqrt{2ax-x^2}+a\arccos\left(1-\frac{x}{a}\right) +C$
  \item $\displaystyle \int \frac{x^2}{\sqrt{2ax-x^2}}\dif x =-\frac{(x+3a)\sqrt{2ax-x^2}}{2}+\frac{3a^2}{2}\arccos\left(1-\frac{x}{a}\right) +C$
  \item $\displaystyle \int \frac{1}{x\sqrt{2ax-x^2}}\dif x = -\frac{\sqrt{2ax-x^2}}{ax} +C$
  \item $\displaystyle \int \frac{1}{\sqrt{2ax+x^2}}\dif x = \ln\left|x+a+\sqrt{2ax+x^2}\right| +C$
  \item $\displaystyle \int \sqrt{\frac{a+x}{b+x}}\dif x = \sqrt{(a+x)(b+x)}+(a-b)\ln(\sqrt{a+x}+\sqrt{b+x})+C$
  \item $\displaystyle \int \sqrt{\frac{a-x}{b+x}}\dif x =\sqrt{(a-x)(b+x)}+(a+b)\arcsin\sqrt{\frac{x+b}{a+b}} +C$
  \item $\displaystyle \int \sqrt{\frac{a+x}{b-x}}\dif x =-\sqrt{(a+x)(b-x)}-(a+b)\arcsin\sqrt{\frac{b-x}{a+b}} +C$
  \item $\displaystyle \int \frac{1}{\sqrt{(x-a)(b-x)}}\dif x =2\arcsin\sqrt{\frac{x-a}{b-a}} +C$
\end{enumerate}
\section{超越函数的积分}\label{sec:transcendental_int_eq}
\begin{enumerate}[1.,itemsep=5pt,resume]
  \item $\displaystyle \int e^{ax}\dif x=\frac{e^{ax}}{a}+C$
  \item $\displaystyle \int b^{ax}\dif x=\frac{b^{ax}}{a\ln b}+C$
  \item $\displaystyle \int \ln x\dif x=x\ln x-x+C$
  \item $\displaystyle \int x^n\ln x\dif x=x^{n+1}\left[ \frac{\ln x}{n+1}-\frac{1}{(n+1)^2}\right]+C$
  \item $\displaystyle \int \sin^2 x\dif x=\frac{1}{2}x-\frac{1}{4}\sin2x+C$
  \item $\displaystyle \int \cos^2 x\dif x=\frac{1}{2}x+\frac{1}{4}\sin2x+C$
  \item $\displaystyle \int \cos^n x\sin x\dif x=-\frac{\cos^{n+1}x}{n+1}+C$
  \item $\displaystyle \int \sin^n x\cos x\dif x=\frac{\sin^{n+1}x}{n+1}+C$
  \item $\displaystyle \int \sin mx\sin nx\dif x=-\frac{\sin(m+n)x}{2(m+n)}+\frac{\sin(m-n)x}{2(m-n)}+C$
  \item $\displaystyle \int \cos mx\cos nx\dif x=\frac{\sin(m+n)x}{2(m+n)}+\frac{\sin(m-n)x}{2(m-n)}+C$
  \item $\displaystyle \int \sin mx\cos nx\dif x=-\frac{\cos(m+n)x}{2(m+n)}-\frac{\cos(m-n)x}{2(m-n)}+C$
  \item $\displaystyle \int \frac{1}{1+\cos x}\dif x=\tan\frac{x}{2}+C$
  \item $\displaystyle \int \frac{1}{1-\cos x}\dif x=-\cot\frac{x}{2}+C$
  \item $\displaystyle \int x\sin nx\dif x=\frac{1}{n^2}\sin nx-\frac{1}{n}x\cos nx+C$
  \item $\displaystyle \int x\cos nx\dif x=\frac{1}{n^2}\cos nx+\frac{1}{n}x\sin nx+C$
  \item $\displaystyle \int x^2\sin nx\dif x=\frac{x}{n^2}(2\sin nx-nx\cos nx)+\frac{2}{n^3}\cos nx+C$
  \item $\displaystyle \int x^2\cos nx\dif x=\frac{x}{n^2}(nx\sin nx+2\cos nx)-\frac{2}{n^3}\sin nx+C$
  \item $\displaystyle \int \arcsin x\dif x=x\arcsin x+\sqrt{1-x^2}+C$
  \item $\displaystyle \int \arccos x\dif x=x\arccos x-\sqrt{1-x^2}+C$
  \item $\displaystyle \int \arctan x\dif x=x\arctan x-\ln\sqrt{1+x^2}+C$
  \item $\displaystyle \int \arccot x\dif x=x\arccot x+\ln\sqrt{1+x^2}+C$
\end{enumerate}
 % ok
\end{document}