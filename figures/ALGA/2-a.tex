\documentclass{standalone}
\usepackage{tikz}
\usepackage{ctex,siunitx,upgreek}
\setCJKmainfont{Noto Serif CJK SC}
\usepackage{tkz-euclide}
\usepackage{amsmath,amsfonts,amssymb}
\usetikzlibrary{patterns, calc,3d}
\usetikzlibrary {decorations.pathmorphing,decorations.pathreplacing,decorations.shapes}
\tikzset{
  label style/.append style={font=\small},
  func/.style={rectangle,draw}
}
\begin{document}
\small
\begin{tikzpicture}[>=latex,scale=1.0]
  \node (a) [func]{\parbox{5em}{\centering 任意负角的三角函数}};
  \node (b) [func,right=1.8cm of a] {\parbox{5em}{\centering 任意正角的三角函数}};
  \node (c) [func,right=1.8cm of b] {\parbox{6em}{\centering \ang{0}~\ang{360} 间的三角函数}};
  \node (d) [func,right=1.8cm of c] {\parbox{5em}{\centering \ang{0}~\ang{90} 间的三角函数}};
  \node (e) [func,right=1cm of d] {\parbox{1em}{\centering 求\\值}};
  \draw[->](a)--(b)node[midway,above]{\footnotesize 用公式三、一};
  \draw[->](b)--(c)node[midway,above]{\footnotesize 用公式一};
  \draw[->](c)--(d)node[midway,above]{\footnotesize 用公式}node[midway,below]{\scriptsize 二、四、五};
  \draw[->](d)--(e)node[midway,above]{\footnotesize 查表};
\end{tikzpicture}
\end{document}