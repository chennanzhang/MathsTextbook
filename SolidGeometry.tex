\documentclass[colortheme=teal,txconfig=txmaths.cfg]{textbook}
\stylesetup{ 
  fullwidth-stop = catcode,
  boldemph = false,
}
\Booksetup{
  BookSeries  = 中学经典教材丛书, 
  BookTitle   = 立体几何(甲种本),
  BookTitle*  = {Textbook for Middle School Solid Geometry},
  SubTitle    = 全一册,
  % SubTitle*   = Volume I,
  BriefIntro    = 
    { 
      本书供六年制中学高中一年级选用,每周授课 2 课时。本书内容包括直线和平面、多面体和旋转体以及多面角和正多面体等三章。本书习题共分四类:练习、习题、复习参考题以及总复习参考题。练习主要供课堂练习用;习题主要供课内外作业用;复习参考题和总复习参考题都分 A、B 两组。复习参考题 A 组供复习本章知识时使用;总复习参考题 A 组供复习全书知识时使用;两类题中的 B 组综合性与灵活性较大,仅供学有余力的学生参考使用。习题及复习参考题、总复习参考题中的 A 组题的题量较多,约为学生通常所需题量的 1.5 倍,教学时可根据情况选用。本书是在中小学通用教材编写组编写的全日制十年制高中课本(试用本)《数学》第二册第五章“空间图形”的基础上编写的。初稿编出后,曾向各省、市、自治区的教研部门、部分师范院校征求了意见,并向部分中学教师征求了意见,有的省还进行了试教。他们都提出了许多宝贵的意见。本书由人民教育出版社数学室编写。参加编写的有鲍珑、李慧君、孙福元等,全书由孙福元校订。
    },
  DedicatedTo   = 奔赴高考的莘莘学子,
  % CoverGraph    = graphics/A.pdf,
  AuthorList    = {人民教育出版社数学室},
  ReleaseDate   = 2024-12-12,
  % Url           = https://www.tjad.cn,
  % ISBN          = 978-7-302-11622-6,
  % Publisher     = 同济极客出版社,
  % Logo          = graphics/logo.pdf,
  % Editor        = {张晨南},
  WrittenStyle  = 编,
}
\graphicspath{{figures/SG/}}
\begin{document}
\frontmatter
\chapter{引言}
在初中,我们学习了平面几何,研究过一些平面图形(由同一个平面内的点、线所构成的图形)的形状、大小和位置关系,还有平面图形的画法和计算,以及它们的应用。
可是,在解决实际问题中,只知道这些几何知识还是不够用的。
例如,建造厂房、制造机器、修筑堤坝等,都需要进一步研究空间图形的问题。

空间图形是由空间的点、线、面所构成,也可看成是空间点的集合。
以前我们学过的长方体、圆柱、圆锥等,都属于空间图形。
平面图形是空间图形的一部分。

立体几何的研究对象是空间图形。
我们将在平面几何知识的基础上,来研究空间图形的性质、画法、计算,以及它们的应用。
\tableofcontents
\mainmatter
\chapter{直线和平面}
\section{平面}
\subsection{平面}
\begin{Practice}
  \begin{question}
    \item 
    \item 
  \end{question}
\end{Practice}
\subsection{平面的基本性质}
\begin{Practice}
  \begin{question}
    \item 
    \item 
    \item 
  \end{question}
\end{Practice}
\subsection{水平放置的平面图形的直观图的画法}
\begin{Practice}
  \begin{question}
    \item 
    \item 
  \end{question}
\end{Practice}
\begin{Exercise}
  \begin{question}
    \item 
    \item 
    \item 
    \item 
    \item 
    \item 
    \item 
    \item 
    \item 
    \item 
    \item 
  \end{question}
\end{Exercise}
\section{空间两条直线}
\subsection{两条直线的位置关系}
\begin{Practice}
  \begin{question}
    \item 
    \item 
    \item 
  \end{question}
\end{Practice}
\subsection{平行直线}
\begin{Practice}
  \begin{question}
    \item 
    \item 
  \end{question}
\end{Practice}
\subsection{两条异面直线所成的角}
\begin{Practice}
  \begin{question}
    \item 
    \item 
    \item 
  \end{question}
\end{Practice}
\begin{Exercise}
  \begin{question}
    \item 
    \item 
    \item 
    \item 
    \item 
    \item 
    \item 
    \item 
    \item 
    \item 
    \item 
  \end{question}
\end{Exercise}

\section{空间直线和平面}
\subsection{直线和平面的位置关系}
\begin{Practice}
  \begin{question}
    \item 
    \item 举出直线和平面三种位置关系的实例。
  \end{question}
\end{Practice}
\subsection{直线和平面平行的判定与性质}
\begin{Practice}
  \begin{question}
    \item ;
    \item ;
    \item 。
  \end{question}
\end{Practice}
\begin{Exercise}
  \begin{question}
    \item 
    \item 
    \item 
    \item 
    \item 
    \item 
    \item 
    \item 
    \item 
    \item 
  \end{question}
\end{Exercise}
\subsection{直线和平面垂直的判定与性质}
\begin{Practice}
  \begin{question}
    \item ;
    \item ;
    \item ;
    \item 。
  \end{question}
\end{Practice}
\subsection{斜线在平面上的射影、直线和平面所成的角}
\begin{Practice}
  \begin{question}
    \item ;
    \item ;
    \item 。
  \end{question}
\end{Practice}
\subsection{三垂线定理}
\begin{Practice}
  \begin{question}
    \item ;
    \item 。
  \end{question}
\end{Practice}
\begin{Exercise}
  \begin{question}
    \item 
    \item 
    \item 
    \item 
    \item 
    \item 
    \item 
    \item 
    \item 
    \item 
  \end{question}
\end{Exercise}

\section{空间两个平面}
\subsection{两个平面的位置关系}
\begin{Practice}
  \begin{question}
    \item 举出两个平面平行和相交的一些实例。
    \item 画两个平行平面和分别在这两个平面内的两条平行直线,再画一个经过这两条平行直线的平面。
  \end{question}
\end{Practice}
\subsection{两个平面平行的判定和性质}
\begin{Practice}
  \begin{question}
    \item 
    \item 
    \item 
  \end{question}
\end{Practice}
\begin{Exercise}
  \begin{question}
    \item 
    \item 
    \item 
    \item 
    \item 
    \item 
    \item 
    \item 
    \item 
    \item 
  \end{question}
\end{Exercise}

\subsection{二面角}
\begin{Practice}
  \begin{question}
    \item 
    \item 
    \item 
    \item 
  \end{question}
\end{Practice}

\subsection{两个平面垂直的判定和性质}
\begin{Practice}
  \begin{question}
    \item 
    \item 
    \item 
  \end{question}
\end{Practice}
\begin{Exercise}
  \begin{question}
    \item 
    \item 
    \item 
    \item 
    \item 
    \item 
    \item 
    \item 
    \item 
    \item 
    \item 
    \item 
    \item 
  \end{question}
\end{Exercise}

\section*{小结}
\begin{enumerate}[C、,itemindent=4.5em]
  \item 
  \item 
  \item 
  \item 
  \item 
\end{enumerate}
\chapter*{复习参考题\chinese{chapter}}
\section*{A 组}
\begin{question}
  \item 
  \item 
  \item 
  \item 
  \item 
  \item 
  \item 
  \item 
  \item 
  \item 
  \item 
  \item 
  \item 
  \item 
\end{question}
\section*{B 组}
\begin{question}
  \item 
  \item 
  \item 
  \item 
  \item 
  \item 
  \item 
\end{question}
\chapter{多面体和旋转体}
\section{多面体}
\subsection{棱柱}
\subsubsection{棱柱的概念和性质}
\begin{Practice}
  \begin{question}
    \item 
    \item 
    \item 
  \end{question}
\end{Practice}
\subsubsection{长方体}
\begin{Practice}
  \begin{question}
    \item 
    \item 
    \item 
  \end{question}
\end{Practice}
\subsubsection{直棱柱直观图的画法}
\subsubsection{直棱柱的侧面积}
\begin{Practice}
  \begin{question}
    \item 
    \item 
  \end{question}
\end{Practice}
\begin{Exercise}
  \begin{question}
    \item 
    \item 
    \item 
    \item 
    \item 
    \item 
    \item 
    \item 
    \item 
    \item 
    \item 
    \item 
    \item 
  \end{question}
\end{Exercise}

\subsection{棱锥}
\subsubsection{棱锥的概念和性质}
\begin{Practice}
  \begin{question}
    \item 
    \item 
  \end{question}
\end{Practice}
\subsubsection{正棱锥的直观图的画法}
\subsubsection{正棱锥的侧面积}
\begin{Practice}
  \begin{question}
    \item 
    \item 
  \end{question}
\end{Practice}
\begin{Exercise}
  \begin{question}
    \item 
    \item 
    \item 
    \item 
    \item 
    \item 
    \item 
    \item 
    \item 
    \item 
    \item 
  \end{question}
\end{Exercise}

\subsection{棱台}
\subsubsection{棱台的概念和性质}
\begin{Practice}
  \begin{question}
    \item 
    \item 
  \end{question}
\end{Practice}
\subsubsection{正棱台的直观图的画法}
\subsubsection{正棱台的侧面积}
\begin{Practice}
  \begin{question}
    \item 
    \item 
  \end{question}
\end{Practice}
\subsubsection{多面体}
\begin{Practice}
  \begin{question}
    \item 
    \item 
  \end{question}
\end{Practice}

\begin{Exercise}
  \begin{question}
    \item 
    \item 
    \item 
    \item 
    \item 
    \item 
    \item 
    \item 
    \item 
    \item 
    \item 
    \item 
  \end{question}
\end{Exercise}

\section{旋转体}
\subsection{圆柱、圆锥、圆台}
\subsubsection{圆柱、圆锥、圆台的概念和性质}
\begin{Practice}
  \begin{question}
    \item 
    \item 
    \item 
  \end{question}
\end{Practice}
\subsubsection{圆柱、圆锥、圆台的直观图的画法}
\begin{Practice}
  画一个上底半径为 \qty{1.5}{cm},下底半径为 \qty{2.5}{cm},高为 \qty{4}{cm} 的圆台的直观图(比例尺取 $\frac{1}{2}$,不写画法)。
\end{Practice}
\subsubsection{圆柱、圆锥、圆台的侧面积}
\begin{Practice}
  \begin{question}
    \item 
    \item 
    \item 
  \end{question}
\end{Practice}

\begin{Exercise}
  \begin{question}
    \item 
    \item 
    \item 
    \item 
    \item 
    \item 
    \item 
    \item 
    \item 
    \item 
    \item 
    \item 
    \item 
    \item 
  \end{question}
\end{Exercise}

\subsection{球}
\subsubsection{球的概念和性质}
\subsubsection{球的直观图的画法}
\subsubsection{球的表面积}
\begin{Practice}
  \begin{question}
    \item 海面上,地球球心角 \ang{;1;} 所对的大圆弧长约为 1 海里,1 海里约是多少千米?
    \item 计算地球表面积是多少 \unit{km^2}。
  \end{question}
\end{Practice}
\subsection{球冠}
\subsubsection{球冠}
\begin{Practice}
  \begin{question}
    \item 
    \item 
  \end{question}
\end{Practice}
\subsubsection{旋转面和旋转体}
\begin{Practice}
  \begin{question}
    \item 举出一些旋转面和旋转体的实例。
    \item 圆柱和圆柱面、圆锥和圆锥面有何区别?
  \end{question}
\end{Practice}

\begin{Exercise}
  \begin{question}
    \item 
    \item 
    \item 
    \item 
    \item 
    \item 
    \item 
    \item 
    \item 
    \item 
    \item 
    \item 
    \item 
  \end{question}
\end{Exercise}

\section{多面体和旋转体的体积}
\subsection{体积的概念与公理}
\begin{Practice}
  \begin{question}
    \item 
    \item 
  \end{question}
\end{Practice}
\subsection{棱柱、圆柱的体积}
\begin{Practice}
  \begin{question}
    \item ;
    \item 。
  \end{question}
\end{Practice}
\begin{Exercise}
  \begin{question}
    \item 
    \item 
    \item 
    \item 
    \item 
    \item 
    \item 
    \item 
    \item 
    \item 
    \item 
    \item 
  \end{question}
\end{Exercise}
\subsection{棱锥、圆锥的体积}
\begin{Practice}
  \begin{question}
    \item ;
    \item 。
  \end{question}
\end{Practice}

\begin{Exercise}
  \begin{question}
    \item 
    \item 
    \item 
    \item 
    \item 
    \item 
    \item 
    \item 
    \item 
  \end{question}
\end{Exercise}

\subsection{棱台、圆台的体积}
\begin{Practice}
  已知上、下地面边长分别是 $a$、$b$,高是 $h$。求下列正棱台的体积:
  \begin{tasks}(2)
    \task 正四棱台;
    \task 正六棱台。
  \end{tasks}
\end{Practice}
\subsection{拟柱体及其体积}
\begin{Practice}
  已知拟柱体的下底面积为 \qty{20}{cm^2},上底面积为 \qty{6}{cm^2},中截面面积为 \qty{12}{cm^2},高为 \qty{15}{cm}。求这个拟柱体的体积。
\end{Practice}
\begin{Exercise}
  \begin{question}
    \item 
    \item 
    \item 
    \item 
    \item 
    \item 
    \item 
    \item 
    \item 
    \item 
    \item 
    \item 
    \item 
  \end{question}
\end{Exercise}

\subsection{球的体积}
\begin{Practice}
  \begin{question}
    \item 球面面积膨胀为原来的二倍,计算体积变为原来的几倍。
    \item 一个正方体的顶点都在球面上,它的棱长是 \qty{4}{cm}。求这个球的体积。
  \end{question}
\end{Practice}
\subsection{球缺的体积}
\begin{Practice}
  \begin{question}
    \item 
    \item 
  \end{question}
\end{Practice}
\begin{Exercise}
  \begin{question}
    \item 
    \item 
    \item 
    \item 
    \item 
    \item 
    \item 
    \item 
    \item 
    \item 
    \item 
  \end{question}
\end{Exercise}

\section*{小结}
\begin{enumerate}[C、,itemindent=4.5em]
  \item 
  \item 
  \item 
  \item 
  \item 
  \item 
\end{enumerate}
\chapter*{复习参考题\chinese{chapter}}
\section*{A 组}
\begin{question}
  \item 
  \item 
  \item 
  \item 
  \item 
  \item 
  \item 
  \item 
  \item 
  \item 
  \item 
  \item 
  \item 
  \item 
  \item 
  \item 
\end{question}
\section*{B 组}
\begin{question}
  \item 
  \item 
  \item 
  \item 
  \item 
  \item 
  \item 
\end{question}
\chapter{多面角和正多面体}
\section{多面角}
\subsection{多面角}
\begin{Practice}
  \begin{question}
    \item 
    \item 
  \end{question}
\end{Practice}
\subsection{多面角的性质}
\begin{Practice}
  \begin{question}
    \item 
    \item 
    \item 
  \end{question}
\end{Practice}
\begin{Exercise}
  \begin{question}
    \item 
    \item 
    \item 
    \item 
    \item 
    \item 
    \item 
    \item 
  \end{question}
\end{Exercise}

\subsection{正多面体、多面体变形}
\subsection{正多面体}
\begin{Practice}
  \begin{question}
    \item 
    \item 
  \end{question}
\end{Practice}
\subsection{多面体的变形}
\begin{Practice}
  \begin{question}
    \item 
    \item 
  \end{question}
\end{Practice}
\begin{Exercise}
  \begin{question}
    \item 
    \item 
    \item 
    \item 
    \item 
    \item 
    \item 
    \item 
  \end{question}
\end{Exercise}

\section*{小结}
\begin{enumerate}[C、,itemindent=4.5em]
  \item 
  \item 
  \item 
  \item 
\end{enumerate}
\chapter*{复习参考题\chinese{chapter}}
\section*{A 组}
\begin{question}
  \item 
  \item 
  \item 
  \item 
  \item 
\end{question}
\section*{B 组}
\begin{question}
  \item 
  \item 
  \item 
  \item 
\end{question}

\chapter*{总复习参考题}
\section*{A 组}
\begin{question}
  \item 
  \item 
  \item 
  \item 
  \item 
  \item 
  \item 
  \item 
  \item 
  \item 
  \item 
  \item 
  \item 
  \item 
  \item 
\end{question}
\section*{B 组}
\begin{question}
  \item 
  \item 
  \item 
  \item 
  \item 
  \item 
  \item 
  \item 
  \item 
  \item 
  \item 
\end{question}
\appendix
% % \input{contents/app1-2.tex} % ok
\end{document}