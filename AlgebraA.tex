\documentclass[colortheme=brown,txconfig=txmaths.cfg]{textbook}
\stylesetup{ 
  fullwidth-stop = catcode,
  boldemph = false,
}
\Booksetup{
  BookSeries  = 中学经典教材丛书, 
  BookTitle   = 高中代数(甲种本),
  BookTitle*  = {Textbook for Middle School Algebra},
  SubTitle    = 第一册,
  SubTitle*   = Volume I,
  BriefIntro    = 
    { 
      本书供六年制中学高中一年级选用。每周授课 3 课时。本书内容包括幂函数、指数函数和对数函数,三角函数,两角和与差的三角函数。本书习题共分三类:练习、习题、复习参考题。练习主要供课堂练习用;习题主要供课内外作业用;复习参考题分 A、B 两组。A 组供复习本章知识时使用;B 组题略带综合性、灵活性,仅供学有余力的学生参考使用。练习、习题及复习参考题 A 组题的题量较多,教学时可根据情况选用。本书在编写过程中,曾参考了中小学通用教材数学编写组编写的全日制十年制学校高中课本(试用本)《数学》第一册,大部分章、节内容是以该书为基础编写的。初稿编出后,曾向各省、市、自治区的教研部门、部分师范院校和中学教师征求意见,有的省、市还进行了试教,他们都提出了宝贵的意见。本书由人民教育出版社数学室编写。参加编写工作的有贾云山、蔡上鹤、饶汉昌、李琳等,全书由吕学礼校订。
    },
  DedicatedTo   = 奔赴高考的莘莘学子,
  % CoverGraph    = graphics/A.pdf,
  AuthorList    = {人民教育出版社数学室},
  ReleaseDate   = 2024-12-12,
  % Url           = https://www.tjad.cn,
  % ISBN          = 978-7-302-11622-6,
  % Publisher     = 同济极客出版社,
  % Logo          = graphics/logo.pdf,
  % Editor        = {张晨南},
  WrittenStyle  = 编,
}
\graphicspath{{figures/ALGA/}}
\begin{document}
\frontmatter
\tableofcontents
\mainmatter
\chapter{幂函数、指数函数和对数函数}
\section{集合}
\subsection{集合}
\begin{Practice}
  \begin{question}
    \item 
    \item 
    \item 
    \item 
    \item 
    \item 
    \item 
    \item 
    \item 
    \item 
    \item 
    \item 
    \item 
    \item 
  \end{question}
\end{Practice}
\subsection{子集、交集、并集、补集}
\subsubsection{子集}
\subsubsection{交集}
\begin{Practice}
  \begin{question}
    \item 
    \item 
    \item 
    \item 
    \item 
    \item 
    \item 
    \item 
    \item 
    \item 
    \item 
    \item 
  \end{question}
\end{Practice}

\subsubsection{并集}
\subsubsection{补集}
\begin{Practice}
  \begin{question}
    \item 
    \item 
    \item 
    \item 
    \item 
    \item 
    \item 
    \item 
    \item 
    \item 
    \item 
    \item 
  \end{question}
\end{Practice}

\begin{Exercise}
  \begin{question}
    \item 
    \item 
    \item 
    \item 
    \item 
    \item 
    \item 
    \item 
    \item 
    \item 
    \item 
    \item 
    \item 
    \item 
    \item 
    \item 
  \end{question}
\end{Exercise}

\section{映射与函数}
\subsection{映射}
\begin{Practice}
  \begin{question}
    \item 
    \item 
    \item 
    \item 
  \end{question}
\end{Practice}
\subsection{函数}
\begin{Practice}
  \begin{question}
    \item 
    \item 
    \item 
    \item 
  \end{question}
\end{Practice}

\begin{Exercise}
  \begin{question}
    \item 
    \item 
    \item 
    \item 
    \item 
    \item 
    \item 
    \item 
    \item 
    \item 
    \item 
    \item 
    \item 
  \end{question}
\end{Exercise}

\section{幂函数}
\subsection{幂函数}
\begin{Practice}
  \begin{question}
    \item 
    \item 
    \item 
    \item 
    \item 
  \end{question}
\end{Practice}
\subsection{函数的单调性}
\begin{Practice}
  \begin{question}
    \item 
    \item 
    \item 
    \item 
  \end{question}
\end{Practice}
\subsection{函数的奇偶性}
\begin{Practice}
  \begin{question}
    \item 
    \item 
    \item 
    \item 
    \item 
  \end{question}
\end{Practice}

\begin{Exercise}
  \begin{question}
    \item 
    \item 
    \item 
    \item 
    \item 
    \item 
    \item 
    \item 
    \item 
    \item 
    \item 
    \item 
  \end{question}
\end{Exercise}

\subsection{一一映射}
\begin{Practice}
  \begin{question}
    \item 
    \item 
    \item 
    \item 
  \end{question}
\end{Practice}
\subsection{逆映射}
\begin{Practice}
  \begin{question}
    \item 
    \item 
    \item 
  \end{question}
\end{Practice}
\subsection{反函数}
\begin{Practice}
  \begin{question}
    \item 
    \item 
  \end{question}
\end{Practice}

\subsection{互为反函数的函数图像间的关系}
\begin{Practice}
  \begin{question}
    \item 
    \item 
    \item 
    \item 
  \end{question}
\end{Practice}

\begin{Exercise}
  \begin{question}
    \item 
    \item 
    \item 
    \item 
    \item 
    \item 
    \item 
    \item 
    \item 
    \item 
    \item 
    \item 
  \end{question}
\end{Exercise}

\section{指数函数和对数函数}
\subsection{指数函数}
\begin{Practice}
  \begin{question}
    \item 
    \item 
    \item 
  \end{question}
\end{Practice}

\subsection{对数函数}
\begin{Practice}
  \begin{question}
    \item 
    \item 
    \item 
  \end{question}
\end{Practice}

\subsection{指数方程和对数方程}
\begin{Practice}
  \begin{question}
    \item 
    \item 
    \item 
    \item 
    \item 
    \item 
  \end{question}
\end{Practice}

\begin{Exercise}
  \begin{question}
    \item 
    \item 
    \item 
    \item 
    \item 
    \item 
    \item 
    \item 
    \item 
    \item 
    \item 
    \item 
    \item 
  \end{question}
\end{Exercise}

\section*{小结}
\begin{enumerate}[C、,itemindent=4.5em]
  \item 
  \item 
  \item 
  \item 
  \item 
  \item 
  \item 
\end{enumerate}
\chapter*{复习参考题\chinese{chapter}}
\section*{A 组}
\begin{question}
  \item 
  \item 
  \item 
  \item 
  \item 
  \item 
  \item 
  \item 
  \item 
  \item 
  \item 
  \item 
  \item 
  \item 
  \item 
  \item 
  \item 
  \item 
  \item 
  \item 
  \item 
  \item 
  \item 
  \item 
  \item 
  \item 
\end{question}
\section*{B 组}
\begin{question}
  \item 
  \item 
  \item 
  \item 
  \item 
  \item 
  \item 
  \item 
  \item 
  \item 
  \item 
  \item 
  \item 
  \item 
  \item 
  \item 
\end{question}
\chapter{三角函数}
\section{任意角的三角函数}
\subsection{角的概念的推广}
\begin{Practice}
  \begin{question}
    \item 
    \item 
    \item 
    \item 
  \end{question}
\end{Practice}
\subsection{弧度制}
\begin{Practice}
  \begin{question}
    \item 
    \item 
    \item 
    \item 
    \item 
    \item 
    \item 
    \item 
    \item 
    \item 
  \end{question}
\end{Practice}

\begin{Exercise}
  \begin{question}
    \item 
    \item 
    \item 
    \item 
    \item 
    \item 
    \item 
    \item 
    \item 
    \item 
    \item 
    \item 
    \item 
    \item 
    \item 
    \item 
    \item 
    \item 
    \item 
  \end{question}
\end{Exercise}

\subsection{任意角的三角函数}
\begin{Practice}
  \begin{question}
    \item 
    \item 
    \item 
    \item 
    \item 
    \item 
  \end{question}
\end{Practice}
\subsection{同角三角函数的基本关系式}
\begin{Practice}
  \begin{question}
    \item 
    \item 
    \item 
    \item 
  \end{question}
\end{Practice}


\begin{Practice}
  \begin{question}
    \item 
    \item 
    \item 
    \item 
  \end{question}
\end{Practice}

\begin{Exercise}
  \begin{question}
    \item 
    \item 
    \item 
    \item 
    \item 
    \item 
    \item 
    \item 
    \item 
    \item 
    \item 
    \item 
    \item 
    \item 
    \item 
    \item 
    \item 
    \item 
    \item 
    \item 
    \item 
    \item 
  \end{question}
\end{Exercise}

\subsection{诱导公式}
\begin{Practice}
  \begin{question}
    \item 
    \item 
    \item 
    \item 
  \end{question}
\end{Practice}

\begin{Practice}
  \begin{question}
    \item 
    \item 
    \item 
    \item 
  \end{question}
\end{Practice}

\subsection{已知三角函数值求角}
\begin{Practice}
  \begin{question}
    \item 
    \item 
    \item 
  \end{question}
\end{Practice}

\begin{Exercise}
  \begin{question}
    \item 
    \item 
    \item 
    \item 
    \item 
    \item 
    \item 
    \item 
    \item 
    \item 
  \end{question}
\end{Exercise}

\section{三角函数的图像和性质}
\subsection{用单位圆中的线段表示三角函数值}
\begin{Practice}
  \begin{question}
    \item 
    \item 
  \end{question}
\end{Practice}

\subsection{正弦函数、余弦函数的图像和性质}
\begin{Practice}
  用描点法作出正弦函数 $y=\sin x, x\in[0,2\uppi)$ 和余弦函数 $y=\cos x, x\in[0,2\uppi)$ 的图像。
\end{Practice}

\begin{Practice}
  \begin{question}
    \item 
    \item 
    \item 
    \item 
    \item 
    \item 
    \item 
  \end{question}
\end{Practice}

\subsection{函数\texorpdfstring{$y=A\sin(\omega x+\varphi)$}{}的图像}
\begin{Practice}
  \begin{question}
    \item 
    \item 
    \item 
    \item 
  \end{question}
\end{Practice}

\subsection{正切函数、余切函数的图像和性质}
\begin{Practice}
  \begin{question}
    \item 
    \item 
    \item 
    \item 
    \item 
  \end{question}
\end{Practice}

\begin{Exercise}
  \begin{question}
    \item 
    \item 
    \item 
    \item 
    \item 
    \item 
    \item 
    \item 
    \item 
    \item 
    \item 
    \item 
    \item 
    \item 
    \item 
    \item 
    \item 
    \item 
    \item 
    \item 
    \item 
    \item 
    \item 
  \end{question}
\end{Exercise}


\section*{小结}
\begin{enumerate}[C、,itemindent=4.5em]
  \item 
  \item 
  \item 
  \item 
  \item 
  \item 
  \item 
\end{enumerate}
\chapter*{复习参考题\chinese{chapter}}
\section*{A 组}
\begin{question}
  \item 
  \item 
  \item 
  \item 
  \item 
  \item 
  \item 
  \item 
  \item 
  \item 
  \item 
  \item 
  \item 
  \item 
  \item 
  \item 
  \item 
  \item 
  \item 
  \item 
  \item 
  \item 
  \item 
  \item 
  \item 
  \item 
  \item 
  \item 
  \item 
\end{question}
\section*{B 组}
\begin{question}
  \item 
  \item 
  \item 
  \item 
  \item 
  \item 
  \item 
  \item 
  \item 
  \item 
  \item 
\end{question}
\chapter{三角函数}
\phantomsection\pdfbookmark[1]{三角函数和差}{sum-trianglefunction}
\subsection{两角和与差的三角函数}
\subsubsection{两角和与差的余弦}
\begin{Practice}
  \begin{question}
    \item 
    \item 
    \item 
    \item 
    \item 
    \item 
  \end{question}
\end{Practice}
\subsubsection{两角和与差的正弦}
\begin{Practice}
  \begin{question}
    \item 
    \item 
    \item 
    \item 
    \item 
    \item 
  \end{question}
\end{Practice}
\subsubsection{两角和与差的正切}
\begin{Practice}
  \begin{question}
    \item 
    \item 
    \item 
    \item 
  \end{question}
\end{Practice}

\begin{Exercise}
  \begin{question}
    \item 
    \item 
    \item 
    \item 
    \item 
    \item 
    \item 
    \item 
    \item 
    \item 
    \item 
    \item 
    \item 
    \item 
  \end{question}
\end{Exercise}

\subsection{二倍角的正弦、余弦、正切}
\begin{Practice}
  \begin{question}
    \item 
    \item 
    \item 
    \item 
    \item 
    \item 
    \item 
  \end{question}
\end{Practice}
\subsection{半角的正弦、余弦、正切}
\begin{Practice}
  \begin{question}
    \item 
    \item 
    \item 
    \item 
    \item 
    \item 
  \end{question}
\end{Practice}

\begin{Exercise}
  \begin{question}
    \item 
    \item 
    \item 
    \item 
    \item 
    \item 
    \item 
    \item 
    \item 
    \item 
    \item 
    \item 
    \item 
    \item 
  \end{question}
\end{Exercise}

\subsection{三角函数的积化和差与和差化积}
\subsubsection{三角函数的积化和差}

\begin{Practice}
  \begin{question}
    \item 
    \item 
    \item 
  \end{question}
\end{Practice}

\subsubsection{三角函数的和差化积}

\begin{Practice}
  \begin{question}
    \item 
    \item 
    \item 
    \item 
  \end{question}
\end{Practice}

\begin{Exercise}
  \begin{question}
    \item 
    \item 
    \item 
    \item 
    \item 
    \item 
    \item 
    \item 
  \end{question}
\end{Exercise}

\section*{小结}
\begin{enumerate}[C、,itemindent=4.5em]
  \item 
  \item 
  \item 
\end{enumerate}

\chapter*{复习参考题\chinese{chapter}}
\section*{A 组}
\begin{question}
  \item 
  \item 
  \item 
  \item 
  \item 
  \item 
  \item 
  \item 
  \item 
  \item 
  \item 
  \item 
  \item 
  \item 
  \item 
  \item 
  \item 
  \item 
  \item 
  \item 
  \item 
  \item 
  \item 
  \item 
  \item 
  \item 
  \item 
  \item 
\end{question}
\section*{B 组}
\begin{question}
  \item 
  \item 
  \item 
  \item 
  \item 
  \item 
  \item 
  \item 
  \item 
  \item 
  \item 
  \item 
\end{question}
\appendix
% \input{contents/algebra-app-1-1.tex} % ok
\end{document}