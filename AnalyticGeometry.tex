\documentclass[colortheme=cyan,txconfig=txmaths.cfg]{textbook}
\stylesetup{ 
  fullwidth-stop = catcode,
  boldemph = false,
}
\Booksetup{
  BookSeries  = 中学经典教材丛书, 
  BookTitle   = 平面解析几何(甲种本),
  BookTitle*  = {Textbook for Middle School Analytic Geometry},
  SubTitle    = 全一册,
  % SubTitle*   = Volume I,
  BriefIntro    = 
    { 
      本书供六年制中学高中二年级选用,每周授课 2 课时。本书内容包括直线、圆锥曲线、坐标变换和极坐标四章。本书习题共分四类:练习、习题、复习参考题以及总复习参考题。练习主要供课堂练习用;习题主要供课内外作业用;复习参考题和总复习参考题都分 A、B 两组。复习参考题 A 组供复习本章知识时使用;总复习参考题 A 组供复习全书知识时使用;两类题中的 B 组综合性与灵活性较大,仅供学有余力的学生参考使用。习题及复习参考题、总复习参考题中的 A 组题的题量较多,约为学生通常所需题量的 1.5 倍,教学时可根据情况选用。本书是在中小学通用教材编写组编写的全日制十年制初中课本(试用本)《数学》第六册第六章“直线和圆的方程”和高中课本(试用本)《数学》第二册第六章“二次曲线”的基础上编写的。初稿编出后,曾向各省、市、自治区的教研部门、部分师范院校征求了意见,并向部分中学教师征求了意见,有的省还进行了试教。他们都提出了许多宝贵的意见。本书由人民教育出版社数学室编写。参加编写的有李慧君、鲍珑、许曼阁等,全书由孙福元校订。
    },
  DedicatedTo   = 奔赴高考的莘莘学子,
  CoverGraph    = graphics/AG.pdf,
  AuthorList    = {人民教育出版社数学室},
  ReleaseDate   = 2025-3-28,
  % Url           = https://www.tjad.cn,
  % ISBN          = 978-7-302-11622-6,
  % Publisher     = 同济极客出版社,
  % Logo          = graphics/logo.pdf,
  % Editor        = {张晨南},
  WrittenStyle  = 编,
}
\graphicspath{{figures/AG/}}
\begin{document}
\frontmatter
\chapter{引言}
我们在平面几何和立体几何里,所用的研究方法是以公理为基础,直接依据图形的点、线、面的关系来研究图形的性质。
在将要学习的平面解析几何里,所用的研究方法和平面几何、立体几何不同,它是在坐标系的基础上,用坐标表示点,用方程表示曲线(包括直线),通过研究方程的特征间接地来研究曲线的性质。
因此可以说,解析几何是用代数方法来研究几何问题的一门数学学科。

平面解析几何研究的主要问题是:
\begin{enumerate}
  \item 根据已知条件,求出表示平面曲线的方程;
  \item 通过方程,研究平面曲线的性质。
\end{enumerate}

解析几何的这种研究方法,在进一步学习数学、物理和其他科学技术中经常使用。

在十七世纪,法国数学家笛卡尔创始了解析几何。
解析几何的产生对数学发展,特别是对微积分的出现起了促进作用; 恩格斯对笛卡尔的这一发现给予了高度的评价。
\tableofcontents
\mainmatter
\chapter{直线}\label{chp:line}
\section{有向线段、定比分点}
\subsection{有向线段、两点的距离}
在初中,我们学过数轴,它是规定了原点、正方向和长度单位的直线。
任意一条直线,都可以规定两个相反的方向。
如果把其中一个作为正方向,那么相反的方向就是负方向。
规定了正方向的直线叫做有向直线。
在图中,有向直线 $l$ 的正方向用箭头表示(\cref{fig:1-1})。
例如,初中学过的直角坐标系中的 $x$ 轴、 $y$ 轴都是有向直线。
\begin{figure}
  \begin{minipage}[b]{0.48\linewidth}\centering
    \caption{}\label{fig:1-1}
  \end{minipage}
  \begin{minipage}[b]{0.48\linewidth}\centering
    \caption{}\label{fig:1-2}
  \end{minipage}
\end{figure}

一条线段也可以规定两个相反的方向。
如\cref{fig:1-2} 中的线段 $AB$,如果以 $A$ 为起点、$B$ 为终点,那么,它的方向是从 $A$ 到 $B$;相反,如果以 $B$ 为起点、$A$ 为终点,它的方向就是从 $B$ 到 $B$。
规定了方向,即规定了起点和终点的线段叫做有向线段。
表示有向线段时,要将表示起点的字母写在前面,表示终点的字母写在后面。
如以 $A$ 为起点、$B$ 为终点的有向线段记作 $\overline{AB}$。
\cref{fig:1-2} 中,点 $C$ 是线段 $AB$ 上的一点,$\overline{AB}$ 和 $\overline{AC}$ 是方向相同的有向线段,$\overline{AB}$ 和 $\overline{BC}$ 是方向相反的有向线段。

选定一条线段作为长度单位,我们可以量得一条线段的长度,线段 \({AB}\) 的长度,就是有向线段 $\overline{AB}$ 的长度,记作 $|AB|$。
如\cref{fig:1-2},设线段 $e$ 是长度单位,那么 $|AB|=3$。
因为有向线段的长度与它的方向无关,所以 $|AB|=|BA|$。
\begin{figure}
  \begin{minipage}[b]{0.48\linewidth}\centering
    \caption{}\label{fig:1-3}
  \end{minipage}
  \begin{minipage}[b]{0.48\linewidth}\centering
    \caption{}\label{fig:1-4}
  \end{minipage}
\end{figure}

如果有向线段在有向直线 $l$ 上或与 $l$ 平行,那么,它的方向与 $l$ 的正方向可能相同或相反。
例如\cref{fig:1-4} 中的 $\overline{AB}$ 与 $l$ 的方向相同,而 $\overline{BA}$ 与 $l$ 的方向相反.

根据 $\overline{AB}$ 与有向直线 $l$ 的方向相同或相反,分别把它的长度加上正号或负号,这样所得的数,叫做有向线段的\Concept{数量}(或\Concept{数值})。
有向线段 $\overline{AB}$ 的数量用 $AB$ 表示\footnote{在引入有向直线以后,线段 $AB$ 的长度一律用 $|AB|$ 表示。}、显然
\[AB = - BA.\]

数轴 $Ox$ 是有向直线,数轴上点 $P$ 的坐标 $x_0$ 实际上是以有向线段的数量来定义的。
点 $P$ 的坐标 $x_0$ 是以原点 $O$ 为起点、$P$ 为终点的有向线段 $\overline{OP}$ 的数量,$OP=x_0$。
例如,点 $A$、$B$ 的坐标分别是有向线段 $\overline{OA}$、$\overline{OB}$ 的数量,$OA=3$、$OB=-2$(\cref{fig:1-5})。
\begin{figure}
  \caption{}\label{fig:1-5}
\end{figure}

现在我们来研究,对于数轴上任意一条有向线段,怎样用它的起点坐标和终点坐标表示它的数量。

设 $\overline{AB}$ 是 $x$ 轴上的任意一条有向线段,$O$ 是原点。
先讨论两点 $A$、$B$ 与 $O$ 都不重合的情形。
如\cref{fig:1-6},它们的位置关系只可能有六种不同情形。
点 $O$、$B$ 的坐标分别用 $x_1$ 和 $x_2$ 表示,那么 $OA=x_1$,$OB=x_2$。

\begin{figure}
  \begin{minipage}{0.1\linewidth}
    \subcaption{}\label{fig:1-6a}
  \end{minipage}%
  \begin{minipage}{0.7\linewidth}

  \end{minipage}\par
  \begin{minipage}{0.1\linewidth}
    \subcaption{}\label{fig:1-6b}
  \end{minipage}%
  \begin{minipage}{0.7\linewidth}

  \end{minipage}\par
  \begin{minipage}{0.1\linewidth}
    \subcaption{}\label{fig:1-6c}
  \end{minipage}%
  \begin{minipage}{0.7\linewidth}

  \end{minipage}\par
  \begin{minipage}{0.1\linewidth}
    \subcaption{}\label{fig:1-6d}
  \end{minipage}%
  \begin{minipage}{0.7\linewidth}

  \end{minipage}\par
  \begin{minipage}{0.1\linewidth}
    \subcaption{}\label{fig:1-6e}
  \end{minipage}%
  \begin{minipage}{0.7\linewidth}

  \end{minipage}\par
  \begin{minipage}{0.1\linewidth}
    \subcaption{}\label{fig:1-6f}
  \end{minipage}%
  \begin{minipage}{0.7\linewidth}

  \end{minipage}\par
  \caption{}\label{fig:1-6}
\end{figure}

在\cref{fig:1-6a} 中,$AB=|AB|$,$OA=|OA|$,$OB=|OB|$,而 $|AB|=|OB|-|OA|$,

$\therefore AB=OB-OA$,即 $AB=x_2-x_1$。

在\cref{fig:1-6b} 中,$AB=|AB|$,$OA=-|OA|$,$OB=|OB|$,而 $|AB|=|OB|+|OA|$,

$\therefore AB=OB-OA$,即 $AB=x_2-x_1$。

同样可以证明,对于其他四种情况,这个等式也成立。
容易验证,当点 $A$ 或点 $B$ 与原点 $O$ 重合时这个等式同样成立。
因此,对于数轴上任意有向线段 $\overline{AB}$,它的数量 $AB$ 和起点坐标 $x_1$、终点坐标 $x_2$ 有如下关系:

\[ AB = x_2 - x_1 \]

根据这个公式可以得到,数轴上两点 $A$、$B$ 的距离公式

\[ |AB| = |x_2 - x_1| \]

下面,我们来求平面上任意两点的距离。

在直角坐标系中,已知两点 $P_1\,(x_1,y_1)$、$P_2\,(x_2,y_2)$(\cref{fig:1-7})。
从 $P_1$、$P_2$ 分别向 $x$ 轴和 $y$ 轴作垂线 $P_1M_1$、$P_1N_1$ 和 $P_2M_2$、$P_2N_2$,垂足分别为 $M_1\,(x_1,0)$、$N_1\,(0,y_1)$、$M_2\,(x_2,0)$、$N_2\,(0,y_2)$,其中直线 $P_1N_1$ 和 $P_2M_2$ 相交于点 $Q$。
\begin{figure}
  \caption{}\label{fig:1-7}
\end{figure}

在 Rt$\triangle P_1QP_2$ 中,
\begin{align*}
  |P_1P_2|^2 & = |P_1Q|^2+|QP_2|^2,\\
  \because \quad |P_1Q| &= |M_1M_2|=|x_2-x_1|,\\
  |QP_2| & = |N_1N_2|=|y_2-y_1|,\\
  \therefore \quad |P_1P_2|^2 &= |x_2-x_1|^2+|y_2-y_1|^2.
\end{align*}
由此得到两点 $P_1\,(x_1,y_1)$、$P_2\,(x_2,y_2)$ 的\Concept{距离公式}:

\[ |P_1P_2| = \sqrt{|x_2-x_1|^2+|y_2-y_1|^2}. \]

\begin{example}
已知数轴上三点 $A$、$B$、$C$ 的坐标分别是 $4$、$-2$、$-6$。
求 $\overline{AB}$、$\overline{BC}$、$\overline{CA}$ 的数量和长度(\cref{fig:1-8})。
\end{example}
\begin{figure}
  \caption{}\label{fig:1-8}
\end{figure}
\begin{solution}
  \begin{align*}
    AB & = (-2)-4=-6,   & |AB| &=|-6|=6;\\
    BC & = -6-(-2)=-4,  & |BC| &=|-4|=4;\\
    CA & = 4-(-6)=-10,  & |CA| &=|10|=10.
  \end{align*}
\end{solution}

\begin{example}
  $\triangle ABC$ 中,$AO$ 是 $BC$ 边上的中线(\cref{fig:1-9})。求证:
  \[ |AB|^2 + |AC|^2 =2(|AO|^2+|OC|^2)\]
\end{example}
\begin{figure}
  \caption{}\label{fig:1-9}
\end{figure}
\begin{proof}
  取线段 $BC$ 所在的直线为 $x$ 轴,点 $O$ 为原点建立直角坐标系。
  设点 $A$ 的坐标为 $(b,c)$,点 $C$ 的坐标为 $(a,0)$,则点 $B$ 的坐标为 $(-a,0)$。可得
  \begin{align*}
    |AB|^2 &=(a+b)^2+c^2, & |AC|^2&=(a-b)^2+c^2,\\
    |AO|^2 &=b^2+c^2, & |OC|^2&=a^2.\\
    \therefore\quad |AB|^2 +|AC|^2 &=2(a^2+b^2+c^2),\\
    |AO|^2 +|OC|^2 &=a^2+b^2+c^2.\\
    \therefore\quad |AB|^2 +|AC|^2 &=2(|AO|^2 +|OC|^2).
  \end{align*}
\end{proof}

\begin{Practice}
  \begin{question}
    \item 数轴上点 $A$ 的坐标为 2,点 $B$ 的坐标为 $-3$。验证公式 $AB=x_2-x_1$。
    \item 已知数轴 $x$ 上的点 $A$、$B$、$C$ 的坐标分别为 1、2、3。
    \begin{tasks}
      \task 求 $\overline{AB}$、$\overline{CB}$ 的数量;
      \task 如果在 $x$ 轴上还有两个点 $D$、$E$,且 $AD=2.5$,$CE=-3$。求点 $D$、$E$ 的坐标。
    \end{tasks}
    \item 求有下列坐标的两点距离:
    \begin{tasks}(2)
      \task  $(6,0)$、$(-2,0)$;
      \task  $(0,-4)$、$(0,-1)$;
      \task  $(6,0)$、$(0,-2)$;
      \task  $(2,1)$、$(5,-1)$;
      \task  $\left(\dfrac{\sqrt{3}}{2},-\dfrac{\sqrt{2}}{2}\right)$、$\left(-\dfrac{\sqrt{2}}{2},-\dfrac{\sqrt{3}}{2}\right)$;
      \task  $(ab^2,2abc)$、$(ac^2,0)$。
    \end{tasks}
    \item 已知点 $A\,(a,-5)$ 和 $A\,(0,10)$ 的距离是 17,求 $a$ 的值。
  \end{question}
\end{Practice}

\subsection{线段的定比分点}
有向直线 $l$ 上的一点 $P$,把 $l$ 上的有向线段 $\overline{P_1P_2}$ 分成两条有向线段 $\overline{P_1P}$ 和 $\overline{PP_2}$。$\overline{P_1P}$ 和 $\overline{PP_2}$ 数量的比叫做点 $P$ 分 $\overline{P_1P_2}$ 所成的比,通常用字母 $\lambda$ 来表示这个比值,

\[ \lambda = \frac{P_1P}{PP_2} \]

点 $P$ 叫做 $\overline{P_1P_2}$ 的定比分点。

如果点 $P$ 在线段 $\overline{P_1P_2}$ 上(\cref{fig:1-10a}),点 \(P\) 叫做 \(\overline{{P}_{1}{P}_{2}}\) 的内分点。
这时,无论 $l$ 的方向如何,$\overline{P_1P}$ 和 $\overline{PP_2}$ 的方向都相同,它们的数量的符号也相同,所以 $\lambda$ 为正值。
如果点 $P$ 在线段 $\overline{P_2P_1}$ 或 $\overline{P_1P_2}$ 的延长线上(\cref{fig:1-10b,fig:1-10c}),点 $P$ 叫做 $\overline{P_1P_2}$ 的外分点。
这时无论 $l$ 的方向如何,$\overline{P_1P}$ 和 $\overline{PP_2}$ 的方向都相反,它们的数量的符号也相反,所以 $\lambda$ 为负值。
\begin{figure}
  \begin{minipage}{0.1\linewidth}
    \subcaption{}\label{fig:1-10a}
  \end{minipage}%
  \begin{minipage}{0.7\linewidth}

  \end{minipage}\par
  \begin{minipage}{0.1\linewidth}
    \subcaption{}\label{fig:1-10b}
  \end{minipage}%
  \begin{minipage}{0.7\linewidth}

  \end{minipage}\par
  \begin{minipage}{0.1\linewidth}
    \subcaption{}\label{fig:1-10c}
  \end{minipage}%
  \begin{minipage}{0.7\linewidth}

  \end{minipage}
  \caption{}\label{fig:1-10}
\end{figure}

由于点 $P$ 分 $\overline{P_1P_2}$ 所成的比与它们所在的直线 $l$ 的方向无关,为了简便起见,在以后谈到点 $P$ 分 $\overline{P_1P_2}$ 所成的比时,一般不提它所在的有向直线的方向。

设 $\overline{P_1P_2}$ 的两个端点分别为 $P_1\,(x_1,y_1)$ 和 $P_2\,(x_2,y_2)$,点 $P$ 分 $\overline{P_1P_2}$ 所成的比为 $\lambda(\lambda\neq-1)$(\cref{fig:1-11}),求分点 $P$ 的坐标 $(x,y)$。

\begin{figure}
  \caption{}\label{fig:1-11}
\end{figure}

过点 $P_1$、$P_2$、$P$ 分别作 $x$ 轴的垂线 $P_1M_1$、$P_2M_2$、$PM$,则垂足分别为 $M_1\,(x_1,0)$、$M_2\,(x_2,0)$、$M\,(x,0)$。根据平行线分线段成比例定理,得
\[ \frac{|P_1P|}{|PP_2|}=\frac{|M_1M|}{|MM_2|}. \]

如果点 $P$ 在线段 $P_1P_2$ 上,那么点 $M$ 也在线段 $M_1M_2$ 上;如果点 $P$ 在线段 $P_1P_2$ 或 $P_2P_1$ 的延长线上,那么点 $M$ 也在线段 $M_1M_2$ 或 $M_2M_1$ 的延长线上。
因此 $\frac{|P_1P|}{|PP_2|}$ 与 $\frac{|M_1M|}{|MM_2|}$ 的符号相同,所以
\[ \frac{P_1P}{PP_2}=\frac{M_1M}{MM_2}. \]
\begin{align*}
  \because \quad M_1M&=x-x_1, \\
  MM_2&=x_2-x, \\
  \therefore \quad \lambda&=\frac{x-x_1}{x_2-x} 
\end{align*}
即 $(1+\lambda)x=x_1+\lambda x_2$,当 $\lambda\neq -1$ 时,得
\[ x= \frac{x_1+\lambda x_2}{1+\lambda}\]

同理可以求得
\[ \lambda=\frac{y-y_1}{y_2-y}, \quad y=\frac{y_1+\lambda y_2}{1+\lambda}.\]

因此,当已知两个端点为 $P_1\,(x_1,y_1)$、$P_2\,(x_2,y_2)$,点 $P\,(x,y)$ 分 $\overline{P_1P_2}$ 所成的比为 $\lambda$ 时,点 $P$ 的坐标是
\[ x= \frac{x_1+\lambda x_2}{1+\lambda}, \quad y=\frac{y_1+\lambda y_2}{1+\lambda}\,(\lambda\neq-1).\]

当点 $P$ 是线段 $\overline{P_1P_2}$ 的中点时,有 $P_1P=PP_2$,即 $\lambda = 1$。
因此线段 $\overline{P_1P_2}$ 中点 $P$ 的坐标是
\[ x=\frac{x_1+x_2}{2},\quad y=\frac{y_1+y_2}{2}.\]

\begin{example}
点 $P_1$ 和 $P_1$ 的坐标分别是 $(-1,-6)$ 和 $(3,0)$,点 $P$ 的横坐标为 $-\dfrac{7}{3}$。求点 $P$ 分 $\overline{P_1P_2}$ 所成的比 $\lambda$ 和点 $P$ 的纵坐标 $y$。
\end{example}
\begin{solution}
  由 $\lambda$ 的定义,可得
  \begin{gather*} 
    \lambda=\frac{x-x_1}{x_2-x}=\frac{-\dfrac{7}{3}-(-1)}{3-\left(-\dfrac{7}{3}\right)}=-\frac{1}{4}.\\
    y=\frac{y_1+\lambda y_2}{1+\lambda}=\frac{-6+ \left( -\dfrac{1}{4}\right)\cdot 0}{1+\left( -\dfrac{1}{4}\right)}=-8.
  \end{gather*}

  点 $P$ 分 $\overline{P_1P_2}$ 所成的比是 $-\dfrac{1}{4}$ ,点 $P$ 的纵坐标是 $-8$(\cref{fig:1-12})。
\end{solution}
\begin{figure}
  \caption{}\label{fig:1-12}
\end{figure}

\begin{example}
  已知三角形顶点是 $A\,(x_1,y_1)$、$B\,(x_2,y_2)$、$C\,(x_3,y_3)$。求 $\triangle ABC$ 的重心 $G$ 的坐标 $(x,y)$ (\cref{fig:1-13})。
\end{example}
\begin{solution}
  设 $BC$ 边的中点为 $D$,则点 $D$ 的坐标是
  \[ \left(\frac{x_2+x_3}{2},\frac{y_2+y_3}{2}\right).\]
  又因为 $AD$ 是中线,且 $\frac{AG}{GD} = 2$,所以点 $G$ 的坐标是
  \begin{align*}
    x&=\frac{x_1+2\times\dfrac{x_2+x_3}{2}}{1+2}, \\
    y&=\frac{y_1+2\times\dfrac{y_2+y_3}{2}}{1+2}, 
  \end{align*}
  整理后得重心 \(G\) 的坐标
  \[ x=\frac{x_1+x_2+x_3}{3},\quad y=\frac{y_1+y_2+y_3}{3}.\]
\end{solution}
\begin{figure}
  \caption{}\label{fig:1-13}
\end{figure}

\begin{Practice}
  \begin{question}
    \item 已知两点 $P_1\,(3,-2)$、$P_2\,(-9,4)$。求点 $P\,(x,0)$ 分 $\overline{P_1P_2}$ 所成的比 $\lambda$ 及 $x$ 的值。
    \item 点 $M$ 分有向线段 $\overline{M_1M_2}$ 的比为 $\lambda$,求点 $M$ 的坐标 $(x,y)$:
    \begin{tasks}
      \task 已知:$M_1\,(1,5)$、$M_2\,(2,3)$,$\lambda = \dfrac{1}{3}$;
      \task 已知:$M_1\,(1,5)$、$M_2\,(2,3)$,$\lambda = -2$;
      \task 已知:$M_1\,(1,5)$、$M_2\,(2,-3)$,$\lambda = -2$;
    \end{tasks}
    \item 已知 $\triangle ABC$ 的顶点 $A\,(2,3)$、$B\,(8,-4)$ 和重心 $G\,(2,-1)$。求点 $C$ 的坐标 $(x,y)$。
  \end{question}
\end{Practice}

\begin{Exercise}
  \begin{question}
    \item\label{exer:1-1} 如图,数轴上每一格等于一个长度单位,说出有向线段 $\overline{AB}$、$\overline{BC}$、$\overline{CD}$ 和 $\overline{EA}$ 的长度和数量。
    
    \begin{figurehere}
      \begin{minipage}{\linewidth}\centering
        \caption*{(第 \ref{exer:1-1} 题)}
      \end{minipage}
    \end{figurehere}
    \item 已知数轴上 $A$、$B$ 两点的坐标 $x_1$、$x_2$ 分别是:
    \begin{tasks}(2)
      \task $x_1=8$,$x_2=6$;
      \task $x_1=5$,$x_2=-3$;
      \task $x_1=-4$,$x_2=0$;
      \task $x_1=-9$,$x_2=-1$;
      \task $x_1=2a-b$,$x_2=a-2b$;
      \task $x_1=2+\sqrt{3}$,$x_2=3+\sqrt{2}$。
    \end{tasks}
    求 $\overline{AB}$ 和 $\overline{BA}$ 的数量。
    \item $A$、$B$ 是数轴上两点,点 $B$ 的坐标是 $x_2$。根据下列条件,求点 $A$ 的坐标 $x_1$:
    \begin{tasks}(2)
      \task $x_2=3$,$AB=5$;
      \task $x_2=-5$,$BA=-3$;
      \task $x_2=0$,$|AB|=2$;
      \task $x_2=-5$,$|AB|=2$。
    \end{tasks}
    \item 已知某零件一个面上有 3 个孔,孔中心的坐标分别为:$A\,(-10,30)$、$B\,(-2,3)$、$C\,(0,-1)$。求每两孔中心的距离。
    \item 已知点 $P\,(x,2)$、$Q\,(-2,-3)$、$M\,(1,1)$,且 $|PQ|=|PM|$。求 $x$。
    \item 解答:
    \begin{tasks}
      \task 求在 $x$ 轴上与点 $A\,(5,12)$ 的距离为 $13$ 的点的坐标;
      \task 已知点 $P$ 的横坐标是 7,点 $P$ 到点 $N\,(-1,5)$ 的距离等于 10,求点 $P$ 的纵坐标。
    \end{tasks}
    \item 设线段 $P_1P_2$ 长 \qty{5}{cm},写出点 $P$ 分 $\overline{P_1P_2}$ 所成的比 $\lambda$:
    \begin{tasks}
      \task 点 $P$ 在 $P_1P_2$ 上,$|P_1P|=\qty{1}{cm}$;
      \task 点 $P$ 在 $P_1P_2$ 的延长线上,$|P_2P|=\qty{10}{cm}$;
      \task 点 $P$ 在 $P_1P_1$ 的延长线上,$|PP_1|=\qty{1}{cm}$。
    \end{tasks}
    \item 求连结下列两点的线段的长度和中点坐标:
    \begin{tasks}(2)
      \task $A\,(7,4)$、$B\,(3,2)$;
      \task $A\,(6,-4)$、$B\,(-2,-2)$;
      \task $M\,(3,1)$、$N\,(2,1)$;
    \end{tasks}
    \item 一条线段的两个端点 $P_1$、$P_1$ 的坐标及点 $P$ 分 $\overline{P_1P_2}$ 所成的比如下,求分点 $P$ 的坐标:
    \begin{tasks}(2)
      \task $(2,1)$、$(3,-9)$,$\lambda=4$;
      \task $(5,-2)$、$(5,3)$,$\lambda=-\dfrac{2}{3}$;
      \task $(-4,1)$、$(5,4)$,$\lambda=\dfrac{5}{2}$;
      \task $(8,5)$、$(-13,-2)$,$\lambda=-\dfrac{4}{3}$。
    \end{tasks}
    \item 解答:
    \begin{tasks}
      \task 一条线段的两个端点坐标如下,求这条线段的两个三等分点的坐标:
      \begin{enumerate*}[i)]
        \item $(-1,2)$、$(10,-1)$;
        \item $(7,8)$、$(1,-6)$。
      \end{enumerate*}
      \task 已知点 $A\,(1,-1)$、$B\,(-4,5)$。将线段 $AB$ 延长至 $C$,使 $|AC|=3|AB|$。求点 $C$ 的坐标。
    \end{tasks}
    \item 三角形的三个顶点是 $A\,(2,1)$、$B\,(-2,3)$、$C\,(0,-1)$。求三条中线的长度。
    \item 已知点 $P_1$ 和 $P_2$ 的坐标分别是 $(4,-3)$ 和 $(-2,6)$,求适合下列条件的点 $P$ 的坐标:
    \begin{tasks}
      \task $\dfrac{|P_1P|}{|PP_2|}=2$,点 $P$ 在线段 $P_1P_2$ 上。
      \task $\dfrac{|P_1P|}{|PP_2|}=4$,点 $P$ 在线段 $P_1P_2$ 的延长线上。
      \task $\dfrac{|P_1P|}{|PP_2|}=\dfrac{4}{5}$,点 $P$ 在线段 $P_2P_1$ 的延长线上。
    \end{tasks}
    \item 解答:
    \begin{tasks}
      \task 已知三点 $A\,(x,5)$、$B\,(-2,y)$、$C\,(1,1)$,且点 $C$ 平分线段 $AB$。求 $x$、$y$。
      \task 已知三点 $A\,(3,-1)$、$B\,(2,1)$,求点 $A$ 关于点 $B$ 的对称点的坐标。
    \end{tasks}
    \item 已知三点 $A\,(1,-1)$、$B\,(3,3)$、$C\,(4,5)$。求证:三点在一条直线上。
    \item 证明:
    \begin{tasks}
      \task 直角三角形斜边的中点到三个顶点的距离相等;
      \task 三角形中位线等于底边的一半。
    \end{tasks}
  \end{question}
\end{Exercise}

\section{直线的方程}
\subsection{一次函数的图像与直线的方程}
初中研究一次函数时,在直角坐标系中,画出的一次函数图象是一条直线。
例如函数 $y=2x+1$ 的图象是直线 $l$(\cref{fig:1-14})。
这时,满足函数式 $y=2x+1$ 的每一对 $x$、$y$ 的值都是直线 $l$ 上的点的坐标,如数对 $(0,1)$ 满足函数式,在直线 $l$ 上就有一点 $A$,它的坐标是 $(0,1)$;而直线  $l$ 上每一点的坐标都满足函数式,如直线 $l$ 上点 $P$ 的坐标是 $(1,3)$,数对 $(1,3)$ 就满足函数式。
\begin{figure}
  \caption{}\label{fig:1-14}
\end{figure}

一般地,一次函数 $y=kx+b$ 的图象是一条直线,它是以满足 $y=kx+b$ 的每一对 $x$、$y$ 的值为坐标的点构成的。
由于函数 $y=kx+b$ 也可以看作二元一次方程,因此,我们也可以说,这个方程的解和直线上的点也存在这样的一一对应关系。

以一个方程的解为坐标的点都是某条直线上的点;反之,这条直线上点的坐标都是这个方程的解,这时,这个方程就叫做这条直线的方程,这条直线叫做这个方程的直线。

在解析几何里研究直线时,就是利用直线与方程的这种关系,建立直线的方程,并通过方程来研究直线的有关问题。

\begin{Practice}
  \begin{question}
    \item 在坐标平面上,画出下列方程的直线:
    \begin{tasks}(2)
      \task $y=x$;
      \task $2x+y=6$;
      \task $2x+3y+6=0$;
      \task $2x-3y+6=0$。
    \end{tasks}
    \item 解答:
    \begin{tasks}
      \task 用量角器测量上题中各直线向上的方向与 $x$ 轴的正方向所成的角的度数;
      \task 量出各直线与 $x$ 轴、$y$ 轴交点的坐标,并代入方程,看它们是不是方程的解。
    \end{tasks}
  \end{question}
\end{Practice}

\subsection{直线的倾斜角和斜率}
为了建立直角坐标系中的直线方程,需要研究直线的倾斜角和斜率。

一条直线 $l$ 向上的方向与 $x$ 轴的正方向所成的最小正角叫做这条直线的倾斜角,如\cref{fig:1-15} 中的 $\alpha$。特别地,当直线 $l$ 和 $x$ 轴平行时,我们规定它的倾斜角为 \ang{0}。因此,倾斜角的取值范围是 $\ang{0}\leqslant\alpha<\ang{180}$。
\begin{figure}
  \caption{}\label{fig:1-15}
\end{figure}

倾斜角不是 \ang{90} 的直线,它的倾斜角的正切叫做这条直线的\Concept{斜率}。
直线的斜率常用 $k$ 表示,即
\[ k = \tan\alpha. \]

倾斜角是 \ang{90} 的直线没有斜率;倾斜角不是 \ang{90} 的直线,都有斜率,并且是确定的,我们常用斜率来表示倾斜角不等于 \ang{90} 的直线对于 $x$ 轴的倾斜程度。

在坐标平面上,如果已知两点 $P_1\,(x_1,x_1)$、$P_2\,(x_2,x_2)$,那么直线 $P_1P_2$ 就是确定的,当直线 $P_1P_2$ 的倾斜角不等于 \ang{90} 时,这条直线的斜率也是确定的。
下面我们来研究怎样用两点的坐标来表示直线 $P_1P_2$ 的斜率。

设直线 $P_1P_2$ 的倾斜角是 $\alpha$,斜率是 $k$,$\overline{P_1P_2}$ 的方向是向上的方向。
从 $P_1$、$P_2$ 分别向 $x$ 轴作垂线 $P_1M_1$、$P_2M_2$,再作 $P_1Q \perp P_2M_2$,垂足分别是 $M_1$、$M_2$、$Q$。那么 $\alpha = \angle QP_1P_2$(\cref{fig:1-16a}),或 $\alpha = \angle PP_1P_2$(\cref{fig:1-16b})。
\begin{figure}
  \begin{minipage}[b]{0.48\linewidth}\centering
    \subcaption{}\label{fig:1-16a}
  \end{minipage}
  \begin{minipage}[b]{0.48\linewidth}\centering
    \subcaption{}\label{fig:1-16b}
  \end{minipage}
  \caption{}\label{fig:1-16}
\end{figure}

在\cref{fig:1-16a} 中,
\[ \tan\alpha =\tan QP_1P_2 =\frac{QP_2}{P_1Q}=\frac{y_2-y_1}{x_2-x_1}.\]

在\cref{fig:1-16a} 中,
\[ \tan\alpha =\tan PP_1P_2 =\frac{QP_2}{P_1Q}=\frac{y_2-y_1}{x_2-x_1}.\]

同样,对于 $\overline{P_2P_1}$ 方向向上的情形,
\[ \tan\alpha= \frac{y_1-y_2}{x_1-x_2}=\frac{y_2-y_1}{x_2-x_1}. \]

综上所述,我们得到经过点 $P_1\,(x_1,x_1)$、$P_2\,(x_2,x_2)$ 两点的直线的斜率公式:
\[ k=\frac{y_2-y_1}{x_2-x_1}. \]
\begin{example}
如\cref{fig:1-17},直线 $l_1$ 的倾斜角 $\alpha_1=\ang{30}$,直线 $l_2 \perp l_1$。求 $l_1$、$l_2$ 的斜率。
\end{example}
\begin{solution}
  $l_1$ 的斜率 $k_1=\tan\ang{30}=\dfrac{\sqrt{3}}{3}$,
  
  $\because\quad l_2$ 的倾斜角 $\alpha_2=\ang{90}+\ang{30}=\ang{120}$,

  $\therefore\quad l_2$ 的斜率 $k_2=\tan\ang{120}=-\tan\ang{60}=-\sqrt{3}$。
\end{solution}
\begin{figure}
  \caption{}\label{fig:1-17}
\end{figure}
\begin{example}
求经过 $A\,(-2,0)$、$B\,(-5,3)$ 两点的直线的斜率和倾斜角。
\end{example}
\begin{solution}
\[ k=\frac{3-0}{-5-(-2)}=-1\]
就是:$\tan \alpha=-1$。

$\because\quad \ang{0}\leqslant\alpha<\ang{180}$,

$\therefore\quad \alpha=\ang{135}$。

因此,这条直线的斜率是 $-1$,倾斜角是 \ang{135}。
\end{solution}

\begin{Practice}
  \begin{question}
    \item 已知直线的倾斜角,讨论这条直线的斜率的值:
    \begin{tasks}(2)
      \task $\alpha=\ang{0}$;
      \task $\ang{0}<\alpha<\ang{90}$;
      \task $\alpha=\ang{90}$;
      \task $\ang{90}\alpha<\ang{180}$。
    \end{tasks}
    \item 求经过下列每两个点的直线的斜率和倾斜角:
    \begin{tasks}(2)
      \task $C\,(10,8)$、$D\,(4,-4)$;
      \task $P\,(0,0)$、$Q\,(-1,\sqrt{3})$;
      \task $M\,(-\sqrt{3},\sqrt{2})$、$N\,(-\sqrt{2},\sqrt{3})$。
    \end{tasks}
    \item 已知:$a$、$b$、$c$ 是两两不等的实数。求经过下列每两个点的直线的倾斜角:
    \begin{tasks}(2)
      \task $A\,(a,c)$、$B\,(b,c)$;
      \task $C\,(a,b)$、$D\,(a,c)$;
      \task $P\,(b,b+c)$、$Q\,(c,c+a)$。
    \end{tasks}
    \item 证明: 已知三点 $A$、$B$、$C$。如果直线 $AB$、$AC$ 的斜率相同。那么这三点在同一条直线上。
  \end{question}
\end{Practice}

\subsection{直线方程的几种形式}
一条直线在直角坐标平面内的位置,可以由不同的条件来确定。
下面,我们来研究怎样根据所给的条件,求出直线的方程。

\subsubsection{点斜式}
已知直线 $l$ 的斜率是 $k$,并且经过点 $P_1\,(x_1,y_1)$,求直线 $l$ 的方程(图\cref{fig:1-18})。

设点 $P\,(x,y)$ 是直线 $l$ 上不同于点 $P_1$ 的任意一点。
根据经过两点的直线的斜率公式,得
\[k = \frac{y - y_1}{x - x_1} \]
可化为
\[y - y_1= k(x - x_1).\]

\begin{figure}
  \begin{minipage}[b]{0.48\linewidth}\centering
    \caption{}\label{fig:1-18}
  \end{minipage}
  \begin{minipage}[b]{0.48\linewidth}\centering
    \caption{}\label{fig:1-19}
  \end{minipage}
\end{figure}

可以验证,直线 $l$ 上的每个点的坐标都是这个方程的解;反过来,以这个方程的解为坐标的点都在直线 $l$ 上,所以这个方程就是过点 \({P}_{1}\) 、斜率为 $k$ 的直线 $l$ 的方程。

这个方程是由直线上一点和直线的斜率确定的,叫做直线方程的点斜式。

当直线 $l$ 的倾斜角为 \ang{0} 时(\cref{fig:1-19}),$\tan\ang{0}=0$,即 $k=0$。这时直线 $l$ 的方程就是
\[ y = y_1 \]

当直线 $l$ 的倾斜角为 \({90}^{ \circ }\) 时,直线没有斜率,这时直线 $l$ 与 \(y\) 轴平行或重合,它的方程不能用点斜式表示。但因 $l$ 上每一点的横坐标都等于 $x_1$(\cref{fig:1-20}),所以它的方程是
\[ x = x_1 \]

\begin{figure}
  \begin{minipage}[b]{0.48\linewidth}\centering
    \caption{}\label{fig:1-20}
  \end{minipage}
  \begin{minipage}[b]{0.48\linewidth}\centering
    \caption{}\label{fig:1-21}
  \end{minipage}
\end{figure}

\begin{example}
  一条直线经过点 $P_1\,(-2,3)$,倾斜角 $\alpha=\ang{45}$。求这条直线的方程,并画出图形。
\end{example}
\begin{solution}
  这条直线经过点 $P_1\,(-2,3)$,斜率是 
  \[k=\tan\ang{45}=1\]
  代入点斜式,得
  \[ y-3=x+2,\]
  即
  \[ x-y+5=0,\]
  这就是所求的直线方程,图形如\cref{fig:1-21}。
\end{solution}

如果已知直线 $l$ 的斜率是 $k$,与 $y$ 轴的交点是 $(0,b)$($b$ 是直线 $l$ 在 $y$ 轴上的截距),代入点斜式得直线 $l$ 的方程:
\[ y-b=k(x-0). \]
也就是
\[ y=kx+b. \]

这个方程是由直线 $l$ 的斜率和它在 $y$ 轴上的截距确定的,所以叫做直线方程的\Concept{斜截式}。

\begin{Practice}
  \begin{question}
    \item 写出下列直线的点斜式方程,并画出图形:
    \begin{tasks}
      \task 经过点 $A\,(2,5)$,斜率是 4 ;
      \task 经过点 $B\,(3,-1)$,斜率是 $\sqrt{2}$;
      \task 经过点 $C\,(-\sqrt{2},2)$,倾斜角是 \ang{30};
      \task 经过点 $D\,(0,3)$,倾斜角是 \ang{0};
      \task 经过点 $E\,(4,-2)$,倾斜角是 \ang{120}。
    \end{tasks}
    \item 已知下列直线的点斜式方程,求各直线经过的已知点、直线的斜率和倾斜角:
    \begin{tasks}(2)
      \task $y-2=x - 1$;
      \task $y-3=\sqrt{3}(x-4)$;
      \task $y+3=-(x-1)$;
      \task $y+2=-\dfrac{\sqrt{3}}{3}(x+1) $。
    \end{tasks}
    \item 写出下列直线的斜截式方程:
    \begin{tasks}
      \task 斜率是 $\dfrac{\sqrt{3}}{2}$,$y$ 轴上的截距是 $-2$;
      \task 倾斜角是 \ang{135},$y$ 轴上的截距是 3。
    \end{tasks}
  \end{question}
\end{Practice}

\subsubsection{两点式}
已知直线 $l$ 经过两点 $P_1\,(x_1,y_1)$、$P_2\,(x_2,y_2)$($x_1\neq x_2$),求直线 $l$ 的方程。

因为直线 $l$ 经过点 $P_1\,(x_1,y_1)$、$P_2\,(x_2,y_2)$,并且 $x_1\neq x_2$,所以它的斜率 
\[k = \frac{y_2-y_1}{x_2-x_1}.\] 
代入点斜式,得
\[ y-y_1 =\frac{y_2-y_1}{x_2-x_1}(x-x_1).\]
当 $y_2\neq y_1$ 时,方程可以写成:
\[ \frac{y-y_1}{y_2-y_1} = \frac{x-x_1}{x_2-x_1}.\]

这个方程是由直线上两点确定的,叫做直线方程的\Concept{两点式}。
\begin{example}
  已知直线 $l$ 在 $x$ 轴和 $y$ 轴上的截距分别是 $a$ 和 $b$ ($a\neq 0$,$b\neq 0$),求直线 $l$ 的方程。
\end{example}
\begin{solution}
  因为直线 $l$ 经过 $A\,(a,0)$ 和 $B\,(0,b)$ 两点,将这两点的坐标代入两点式,得
\[ \frac{y-0}{b-0} = \frac{x-a}{0-a} \]
就是
\[ \frac{x}{a} + \frac{y}{b} = 1 \]
\end{solution}

这个方程是由直线在 $x$ 轴和 $y$ 轴上的截距确定的,叫做直线方程的\Concept{截距式}。
\begin{example}
  三角形的顶点是 $A\,(-5,0)$、$B\,(3,-3)$、$C\,(0,2)$(\cref{fig:1-22}),求这个三角形三边所在直线的方程。
\end{example}
\begin{solution}
  直线 $AB$ 过 $A\,(-5,0)$、$B\,(3,-3)$ 两点。由两点式得
  \[ \frac{y-0}{-3-0}=\frac{x-(-5)}{3-(-5)},\]
  即
  \[ 3x + 8y + 15 = 0 .\]

  这就是直线 $AB$ 的方程。

  直线 $BC$ 在 $y$ 轴上的截距是 2,斜率是
  \[ k=\frac{2-(-3)}{0-3}=-\frac{5}{3}. \]
  即
  \[ 5x + 3y - 6 = 0 .\]

  这就是直线 $BC$ 的方程。

  直线 $AC$ 在 $x$ 轴、$y$ 轴上的截距分别是 $-5$、2。由截距式得
  \[ \frac{x}{-5} + \frac{y}{2} = 1,\]
  即
  \[ 2x - 5y + 10 = 0 .\]

  这就是直线 $AC$ 的方程。
\end{solution}
\begin{figure}
  \caption{}\label{fig:1-22}
\end{figure}

\begin{Practice}
  \begin{question}
    \item 在什么情况下直线方程可以表示成下列形式:
    \begin{tasks}(4)
      \task 点斜式;
      \task 斜截式;
      \task 两点式;
      \task 截距式。
    \end{tasks}
    \item 求过下列两点的直线的两点式方程,再化成斜截式方程:
    \begin{tasks}(3)
      \task $P_1\,(2,1)$、$P_2\,(0,-3)$;
      \task $A\,(0,5)$、$B\,(5,0)$;
      \task $C\,(-4,-5)$、$D\,(0,0)$。
    \end{tasks}
    \item 写出下列直线的截距式方程,并根据截距式方程作出直线:
    \begin{tasks}
      \task $x$ 轴上的截距是 2,$y$ 轴上的截距是 3;
      \task $x$ 轴上的截距是 $-5$,$y$ 轴上的截距是 6;
      \task $x$ 轴上的截距是 4,$y$ 轴上的截距是 $-3$;
      \task $x$ 轴上的截距和 $y$ 轴上的截距都是 $-\dfrac{1}{2}$。
    \end{tasks}
  \end{question}
\end{Practice}

\subsection{直线方程的一般形式}
上一节我们学习了直线方程的几种特殊形式,它们都是二元一次方程。
下面我们来进一步研究直线和二元一次方程的关系。

我们知道,在直角坐标系中,每一条直线都有倾斜角 $\alpha$。当 $\alpha\neq\ang{90}$ 时,它们都有斜率,方程可写成下面的形式:
\[y = kx + b\]

当 $\alpha=\ang{90}$ 时,它的方程可以写成 $x=x_1$ 的形式。
由于是在坐标平面上讨论问题,所以这个方程应认为是关于 $x$、$y$ 的二元一次方程,其中 $y$ 的系数是 0。

这样,对于每一条直线都可以求得它的方程,而且是二元一次方程。
就是说,\emph{直线的方程都是关于 $x$、$y$ 的一次方程}。

下面证明,任何关于 $x$、$y$ 的一次方程都表示一条直线。

$x$、$y$ 的一次方程的一般形式是
\begin{equation}
  \label{eq:linear_eqation}
  Ax+By+C=0
\end{equation}
其中 $A$、$B$ 不同时为零。
下面分 $B\neq 0$ 和 $B=0$ 两种情况加以研究。
\begin{enumerate}
  \item 当 $B\neq 0$ 时,\cref{eq:linear_eqation} 可化为
  \[ y=-\frac{A}{B}x-\frac{C}{B}.\]
  这就是直线的斜截式方程,它表示斜率为 $-\frac{A}{B}$、在 $y$ 轴上的截距为 $-\frac{C}{B}$ 的直线。
  \item 当 $B=0$ 时,由于  $A$、$B$ 不同时为零,必有 $A\neq 0$,\cref{eq:linear_eqation} 可化为
  \[x=-\frac{C}{A}\]
  它表示一条与 $y$ 轴平行或重合的直线。
\end{enumerate}
根据以上的讨论,我们又得到下面的结论:

关于 $x$ 和 $y$ 的一次方程都表示一条直线。

我们把方程
\[Ax+By+C=0.\]
(其中 $A$、$B$ 不全为零)叫做直线方程的\Concept{一般式}。

\begin{example}
  已知直线经过点 $A\,(6,-4)$,斜率为 $-\dfrac{4}{3}$,求直线的
  \begin{enumerate*} 
    \item 点斜式;
    \item 一般式;
    \item 截距式。
  \end{enumerate*}
\end{example}
\begin{solution}
  经过点 $A\,(6,-4)$ 并且斜率等于 $-\dfrac{4}{3}$ 的直线的点斜式是
  \[y+4=-\frac{4}{3}(x-6),\]
  化成一般式,得
  \[ 4x+3y-12= 0.\]
  把常数项移到等号的右边,再把方程的两边都除以 12 ,就得截距式
  \[ \frac{x}{3}+\frac{y}{4}=1.\]
\end{solution}

\begin{example}
  把直线 $l$ 的方程 $x-2y+6=0$ 化成斜截式,求出直线 $l$ 的斜率和在 $x$ 轴与 $y$ 轴上的截距,并画图。
\end{example}
\begin{solution}
  将原方程移项,得 $2y=x+6$。两边除以 2,得斜截式:
  \[ y = \frac{1}{2}x + 3.\]

  因此,直线 $l$ 的斜率 $k=\frac{1}{2}$,在 $y$ 轴上的截距是 3。
  在上面的方程中令 $y=0$,可得
  \[x = - 6,\]
  即直线 $l$ 在 $y$ 轴上的截距是 $-6$。
  
  画一条直线时,只要找出这条直线上的任意两点就可以了。通常是找出直线与两个坐标轴的交点。上面已经求得直线 \(l\) 与 \(x\) 轴、 \(y\) 轴的交点:
\[A\,(-6,0),\quad B\,(0,3).\]
过点 $A$、$B$ 作直线,就得直线 $l$(如\cref{fig:1-23})。
\end{solution}
\begin{figure}
  \caption{}\label{fig:1-23}
\end{figure}

\begin{Practice}
  \begin{question}
    \item 由下列各条件,写出直线的方程,并且化成一般式:
    \begin{tasks}
      \task 斜率是 $-\dfrac{1}{2}$,经过点 $A\,(8,-2)$;
      \task 经过点 $B\,(4,2)$,平行于 $x$ 轴;
      \task 经过点 $C\,(-\dfrac{1}{2},0)$,平行于 $y$ 轴;
      \task 在 $x$ 轴和 $y$ 轴上的截距分别是 $\dfrac{3}{2}$、$-3$;
      \task 经过两点 $P_1\,(3,-2)$、$P_2\,(5,- 4)$;
      \task $x$ 轴上的截距是 $-7$,倾斜角是 \ang{45}。
    \end{tasks}
    \item 已知直线 $Ax+By+C=0$,
    \begin{tasks}
      \task 当 $B\neq 0$ 时,斜率是多少? 当 $B = 0$ 时呢?
      \task 系数为什么值时,方程表示通过坐标原点的直线。
    \end{tasks}
    \item 求下列直线的斜率和在 $y$ 轴上的截距,并画出图形:
    \begin{tasks}(3)
      \task $3x+y-5=0$;
      \task $\dfrac{x}{4}-\dfrac{y}{5}=1$;
      \task $x+2y= 0$;
      \task $7x-6y+4=0$;
      \task $2y-7= 0$。
    \end{tasks}
  \end{question}
\end{Practice}

\subsection{二元一次不等式表示的区域}
前面,我们研究了二元一次方程和直线的关系,用同样的方法,也可以研究二元一次不等式和以它的解为坐标的点的集合(图形)的关系。

含有两个未知数,并且未知数的次数都是一次的不等式叫做二元一次不等式。
使不等式成立的未知数的值叫做它的解。

我们研究不等式
\begin{equation}
  \label{eq:inequality}
  y > 2x + 1
\end{equation}
的解,并把它在坐标平面上表示出来。

为了求\cref{eq:inequality} 的任何一个实数解,可任意选取 $x$ 的一个实数值,例如 $x=1$,把它看作一次方程,这个方程的图形是平行于 $y$ 轴的直线,它与直线 $l:y=2x+ 1$ 相交于点 $A\,(1,3)$(\cref{fig:1-24})。

在直线 $x=1$ 上,点 $A$ 上方的所有点,如 $B\,(1,4)$、$C\,(1,5)$ 、……的坐标都满足\cref{eq:inequality},它们都是\cref{eq:inequality} 的解。

在直线 $x=1$ 上,点 $A$ 下方的所有点,如 $B'\,(1,2)$、$C'\,(1,1)$ 、……的坐标都不满足\cref{eq:inequality},它们都不是\cref{eq:inequality} 的解。

可见,以\cref{eq:inequality} 的解为坐标的所有点的集合
\[P=\{M|y>2x+1\}\]
是直线 $l$ 上方的半平面所有的点,也就是\cref{fig:1-24} 中阴影所表示的平面部分,但不包括边界直线。
这种情况,直线 $l$ 在图中一般画成虚线。
\begin{figure}
  \caption{}\label{fig:1-24}
\end{figure}

以二元一次不等式的解为坐标的所有点的集合表示一个平面图形,我们把这个图形叫做不等式表示的区域。

由上例知道,$y>2x+1$ 表示的区域是直线 $l$ 上方的半平面;同理,容易求得 $y<2x+1$ 表示的区域是直线 $l$ 下方的半平面;而 $y=2x+1$ 就是边界直线 $l$。

一般地,$y=kx+b$ 的直线把平面分成两个半平面,$y>kx+b$ 表示的区域是直线上方的半平面;$y<kx+b$ 表示的区域是直线下方的半平面;直线 $y=kx+b$ 是两个半平面的边界线。

\begin{example}
  画出不等式  表示的区域。
\end{example}
\begin{solution}
  不等式 $y\leqslant-2x+3$ 的解集是
  \begin{align}
    \label{eq:equality_part}   y & = -2x+3\\
    \label{eq:inequality_part} y & < -2x+3
  \end{align}
  的解集的并集,所以它们表示的区域是由\cref{eq:equality_part,eq:inequality_part} 的图形合成的。

  因为\cref{eq:equality_part} 的图形是直线 $l$;(2) 式的图形是直线 $l$ 下方的半平面。
  所以已知不等式表示的区域是直线 $l$ 和它下方的半平面,也就是\cref{fig:1-25} 阴影部分并且包括直线。
  这种情况,直线 $l$ 在图中一般画成实线。

由上面的讨论容易想到,一般二元一次不等式
\[ Ax+By+C=0\]
表示的区域,一定是在直线 $Ax+By+C=0$ 的某一侧。
但要断定究竟是在哪一侧,并不需要将不等式化为 $y$ 的函数式,可以取直线 $Ax+By+C=0$ 一侧的一点,将它的坐标代入不等式,如果不等式成立,那么这一侧就是不等式表示的区域;如果不等式不成立,那么直线的另一侧是不等式表示的区域。

除选点代入不等式的方法外,也可以用 $y$ 的系数判断不等式表示的区域。如果 $B>0$ (或 $B<0$),那么不等式 $Ax+By+C>0$ 所表示的区域是直线 $Ax+By+C=0$ 的上(或下) 方的半平面;如果不等式写成  $Ax+By+C<0$ 的形式时,它表示的区域是直线下(或上)方的半平面。想一想,如果 $B=0$ 时,原不等式表示什么样的区域。
\end{solution}
\begin{figure}
  \caption{}\label{fig:1-25}
\end{figure}
\begin{example}\label{exp:1-13}
  求不等式
  \begin{equation} 
    \label{eq:inequality_2}
    x+2y-10<0
  \end{equation}
  表示的区域,并画出图形。
\end{example}

\begin{solution}
  先画出直线 $l:x+2y-10=0$。

  用选点代入\cref{eq:inequality_2} 的方法,例如将原点 $(0,0)$ 的坐标代入\cref{eq:inequality_2},得 $-10<0$,\cref{eq:inequality_2} 成立。
  所以坐标原点所在的半平面是\cref{eq:inequality_2} 表示的区域,即直线 $l$ 下方的半平面,如\cref{fig:1-26} 的阴影部分,但不包括直线 $l$。
\end{solution}
\begin{figure}
  \caption{}\label{fig:1-26}
\end{figure}

\cref{exp:1-13} 也可以用如下解法:

\begin{solution}
  用 $y$ 的系数判断\cref{eq:inequality_2} 表示的区域。

  $\because \; B=2>0$,
  
  $\therefore\; x+2y-10<0$ 表示的区域是直线 $x+2y-10=0$ 下方的半平面,但不包括直线。
\end{solution}

\begin{example}
  某工厂有一批长为 \qty{2.5}{m} 的条形钢材,要截成 \qty{60}{cm} 和 \qty{42}{cm} 两种规格的零件毛坯,找出最佳的下料方案并计算材料的利用率。
\end{example}

\begin{solution}
  设每根钢材可截成 \qty{60}{cm} 长的毛坯 $x$ 根和 \qty{42}{cm} 长的毛坯 $y$ 根。按题意得不等式
  \begin{equation}
    \label{eq:inequality_3}
    0.6x+0.42y\leqslant 2.5
  \end{equation}

  在坐标纸上画出
  \begin{equation}
    \label{eq:equality_3}
    0.6x+0.42y= 2.5
  \end{equation}
  的直线。如\cref{fig:1-27}。

  因为要截得的两种毛坯数的和必须是正整数,所以以\cref{eq:inequality_3} 的解为坐标的点一定是第一象限内的网格的交点。

如果直线~\eqref{eq:equality_3} 上有网格的交点,那么按直线上网格交点的坐标 $(x,y)$ 的值作为下料方案,这时材料全被利用,因此这个方案就是最佳方案。
但从\cref{fig:1-27}中可以看出,直线~\eqref{eq:equality_3} 不通过网格交点,在这种情况下,为了制订最佳下料方案,应该找靠近直线~\eqref{eq:equality_3} 的网格交点。

当然不能在直线~\eqref{eq:equality_3} 的上方半平面内找网格交点。
因为 $B=0.42>0$,上方半平面内任何网格交点的坐标都使 $0.6x+0.42y>2.5$,这时两种零件毛坯长度的和超过了原钢材长,这是不合理的。

这样,下料范围只能限制在 $0.6x+0.42y<2.5$ 表示的区域内。这个区域是直线~\eqref{eq:equality_3} 下方的半平面。
在直线~\eqref{eq:equality_3} 的下方半平面上找到最靠近直线的网格交点,得点 $M\,(2,3)$。

$x=2$,$y=3$ 就是所求的解,按这样截取毛坯,材料尽管没有被完全利用,但废料最少。

材料的利用率
\[ \frac{0.6 \times 2 + 0.42 \times 3}{2.5} =98.4\%.\]
答: 把每根条钢截成 2 根 \qty{60}{cm} 长和 3 根 \qty{42}{cm} 长的零件毛坯是最佳的下料方案。材料利用率为 98.4\%。
\end{solution}

\begin{figure}
  \caption{}\label{fig:1-27}
\end{figure}

\begin{Practice}
  \begin{question}
    \item 求下列不等式表示的区域:
    \begin{tasks}(3)
      \task $y<-2x-1$;
      \task $y\geqslant -x+3$;
      \task $x+2y-15<0$;
      \task $5x-3y+2\leqslant o$;
      \task $x\geqslant 4$;
      \task $y\leqslant -3$;
      \task $y\leqslant 3x$;
      \task $x\leqslant -2y$;
      \task $3x+2y<0$。
    \end{tasks}
    \item 已知 $y<kx+b$ 表示直线 $y=kx+b$ 下方的半平面,说明当 $B<0$ 时,不等式 $Ax+By+C>0$ 表示的区域是直线 $Ax+By+C=0$ 下方的半平面。
    \item 已知每袋面粉 \qty{25}{kg},每袋大米 \qty{105}{kg},用栽重量是 \qty{265}{kg} 的小推车运输,找出车运效率最高的装车方案。
  \end{question}
\end{Practice}

\subsection{直线型经验公式}
在生产和科学实验中,常常需要根据观察或实验所得的两个变量的一些对应的近似值(叫做实验数据),来求出这两个变量之间的函数关系的解析式。
这样通过实验数据得到的两个变量的关系式叫做经验公式。
如果所求得的经验公式是一次方程,它就叫做直线型经验公式。

下面举例说明求直线型经验公式的方法。
\begin{example}
  通过实验,测得某种合金的熔点 $y$(\unit{\celsius})和含铅量 $x$(\%)之间关系的数据如\cref{tab:1-1}:
  \begin{table}
    \caption{某种合金的熔点 $y$(\unit{\celsius})和含铅量 $x$(\%)之间关系}\label{tab:1-1}
    \begin{tblr}{colspec={X[c]*6{X[r]}},vline{2}=0.8pt}
      $x$(\%)              & 36.9 & 46.7 & 63.7 & 77.8 & 84.0 & 87.5 \\
      $y$(\unit{\celsius}) & 181  & 197  & 235  & 270  & 283  & 292  \\
    \end{tblr}
  \end{table}
  根据这些数据,求关于 $x$、$y$ 的经验公式。
\end{example}

\begin{solution}
  \begin{enumerate}
    \item 把各组对应的数据作为点的坐标,在坐标平面上画出这些点。
    例如,以 $(63.7,235)$ 为坐标的点就是 $P$(\cref{fig:1-28})。
    \item 观察这些点的位置,可以看出它们大致分布在一条直线上。
    用透明尺的边缘在这些点间移动,使它尽量靠近或通过大多数点,然后画出直线。
    \item 求方程:在这条直线上选出两点,例如 $A\,(46.7,197)$、$B\,(84.0,283)$。
    把它们的坐标代入直线的两点式方程,得
    \[ \frac{y-197}{283-197}=\frac{x-46.7}{84.0-46.7}\]
    化简后,就得经验公式
    \[ y=2.3x+89.3\]
  \end{enumerate}

  有了这个经验公式,就可以从这种合金的含铅量求得对应的熔点,或者反过来,从这种合金的熔点求得对应的含铅量。
\end{solution}
\begin{figure}
  \caption{}\label{fig:1-28}
\end{figure}


上面这种选择两点求直线型经验公式的方法,叫做\Concept{选点法}。

选点法虽然比较简便,但是由于选点的灵活性较大,一般精确度不高。下面再介绍一种精确度较高的方法,叫做\Concept{平均值法}。

我们仍用上面的例题来说明平均值法。
在确定经验公式是直线型的以后,可假定所求的经验公式为
\[y=kx+b\]

把表中 $x$、$y$ 的对应值代入这个方程,得到若干个关于 $k$、 $b$ 的一次方程。
为了确定 $k$、$b$ 的值,把它们分成两组,使两组中的方程个数相同或相差一个,再把各组方程两边分别相加,就得到关于 $k$、$b$ 的两个方程:
\[ 
  \begin{array}{rcrcr}
      181 &=& 36.9k&+&b \\
      197 &=& 46.7k&+&b \\
   +)\quad 235 &=& 63.7k&+&b \\
   \hline 
    613 &=& 147.3k&{}+&3b 
  \end{array}\qquad 
  \begin{array}{rcrcr}
    270 &=& 77.8k&+&b \\
    283 &=& 84.0k&+&b \\
 +)\quad 292 &=& 87.5k&+&b \\
 \hline 
  845 &=& 249.3k&{}+&3b 
\end{array}
\]
解方程组
\[ \begin{cases} 
  613=147.3k+3b,\\
  845=249.3+3b,
\end{cases}\]
得
\[ k\approx 2.27,\quad b\approx 92.9.\]

代入所设方程,就得到所求的经验公式
\[ y=2.27b+92.9.\]

\begin{Practice}
  \begin{question}
    \item 丙烯腈是合成人造羊毛的原料。在研究它的比重 $D$ 和温度 $T$(\unit{\celsius})的关系时,通过实验得到的数据如下:\par\noindent
    \begin{tablehere}
    \begin{tblr}{colspec={c*{7}{X[c]}},vline{2}=0.8pt}
      $T$(\unit{\celsius})& 0 & 5 & 10 & 15 & 20 & 25 & 30 \\
      $D$                   & 0.8287 &0.8282 & 0.8276 & 0.8271 &0.8266 & 0.8261& 0.8255 \\
    \end{tblr}
  \end{tablehere}
    用选点法求出 $D$ 与 $T$ 之间的经验公式。
    \item 根据下列实验数据,用平均值法求关于 $m$、$p$ 的经验公式:\par\noindent
    \begin{tablehere}
      \begin{tblr}{colspec={*{8}{X[c]}},vline{2}=0.8pt}
        $m$ & 0.03 & 0.06 & 0.10 & 0.14 & 0.17 & 0.20 & 0.25 \\
        $p$ & 0 & 0.42 & 0.93 & 1.60 & 2.03 & 2.41 & 3.31 \\
      \end{tblr}
    \end{tablehere}
  \end{question}
\end{Practice}

\begin{Exercise}
  \begin{question}
    \item 四边形 $ABCD$ 的四个顶点是 $A\,(2,3)$、$B\,(1,-1)$、$C\,(-1,-2)$、$D\,(-2,2)$。求四边所在直线的倾斜角和斜率。
    \item 根据下列条件写出直线的方程:
    \begin{tasks}
      \task 斜率是 $\dfrac{\sqrt{3}}{3}$,经过点 $A\,(8,-2)$;
      \task 过点 $B\,(-2,0)$,且与 $x$ 轴垂直;
      \task 斜率为 $-4$,在  $y$ 轴上的截距为 7;
      \task 经过两点 $A\,(-1,8)$、$B\,(4,-2)$;
      \task 在 $y$ 轴上的截距是 2,且与 $x$ 轴平行;
      \task 在 $x$ 轴、$y$ 轴上的截距分别是 4 与 $-3$。
    \end{tasks}
    \item 已知直线的斜率 $k=2$,$P_1\,(3,5)$、$P_2\,(x_2,7)$、$P_3\,(-1,y_3)$ 是这条直线上的三个点。求 $x_2$ 和 $y_3$。
    \item 已知两点 $M\,(2,2)$ 和 $N\,(5,-2)$,点 $P$ 在 $x$ 轴上,且 $\angle MPN =\ang{90}$。求点 $P$ 的坐标。
    \item 设 $A$、$B$ 两点的坐标分别是 $(x_1,y_1)$ 和 $(x_2,y_2)$,直线 $AB$ 的倾斜角是 $\alpha$。求证:
    \[|x_1-x_2|=\sqrt{(x_2-x_1)^2+(y_2-y_1)^2}|\cos\alpha|\]
    \item 一条直线经过点 $A\,(2,-3)$,它的倾斜角等于直线 $y=\dfrac{1}{\sqrt{3}}x$ 的倾斜角的 2 倍,求这条直线的方程。
    \item 一根弹簧,挂 \qty{4}{kg} 的物体时,长 \qty{20}{cm},在弹性限度内,所挂物体的重量每增加 \qty{1}{kg},弹簧伸长 \qty{1.5}{cm}。利用点斜式写出弹簧的长度 $l$(\unit{cm})和所挂物体重量 $F$(\unit{kg})之间关系的方程。
    \item 一条直线和 $y$ 轴相交于点 $P\,(0,2)$,它的倾斜角的正弦是 $\dfrac{4}{5}$。求这条直线的方程。这样的直线有几条?
    \item 证明:三点 $A\,(1,3)$、$B\,(5,7)$、$P\,(10,12)$ 在同一条直线上。
    \item 解答:
    \begin{tasks}
      \task 已知三角形的顶点是 $A\,(8,5)$、$B\,(4,-2)$、$C\,(-6,3)$,求经过每两边中点的三条直线的方程。
      \task $\triangle ABC$ 的顶点是 $A\,(0,5)$、$B\,(1,-2)$、$C\,(-6,4)$,求 $BC$ 边上的中线所在直线的方程。
    \end{tasks}
    \item 一根铁棒在 \qty{40}{\celsius} 时长 \qty{12.506}{m},在 \qty{80}{\celsius} 时长 \qty{12.512}{m},已知长度 $l$(\unit{m})和温度 $t$(\unit{\celsius})的关系可以用直线方程来表示,用两点式表示这个方程,并且根据这个方程,求这根铁棒在 \qty{100}{\celsius} 时的长度。
    \item 菱形的两条对角线长分别等于 8 和 6,并且分别放置在 $x$ 轴和 $y$ 轴上。求菱形各边所在的直线的方程。
    \item 求过点 $P\,(2,3)$,并且在两轴上的截距相等的直线方程。
    \item 油槽储油 \qty{20}{m^3},从一管道等速流出,\qty{50}{min} 流完。用截距式写出关于油槽里剩余的油量 $Q$(\unit{m^3})和流出的时间 $t$(\unit{min})的方程,并且画出图形(注意: $0\leqslant t \leqslant 50$)。
    \item 直线方程 $Ax+By+C=0$ 的系数 $A$、$B$、$C$ 满足什么关系时,这条直线
    \begin{tasks}(3)
      \task 与坐标轴都相交;
      \task 只与 $x$ 轴相交;
      \task 只与 $y$ 轴相交;
      \task 是 $x$ 轴;
      \task 是 $y$ 轴。
    \end{tasks}
    \item 设点 $P\,(x_0,y_0)$ 在直线 $Ax+By+C=0$ 上。求证:这条直线的方程可以写成 $A(x-x_0)+B(y-y_0)=0$。
    \item 求下列不等式表示的区域:
    \begin{tasks}(3)
      \task $y \geqslant x$;
      \task $x<2y$;
      \task $3y \leqslant 2x$;
      \task $3y-2 \leqslant 0$;
      \task $3x-5y+10>0$;
      \task $2x+3y-4 \geqslant 0$。
    \end{tasks}
    \item 某同学拿 5 元钱买纪念邮票,票面 4 分钱的每套 5 张,8 分钱的每套 4 张。如果每种至少买一套,共有几种买法,能否恰好将钱用完,怎样买剩钱最少?
    \item 有一滑轮组,举起的物重 \(W\) 与需用的力 \(F\) 之间的关系,由实验所得的数据如下表:\par
    \begin{tablehere}
      \begin{tblr}{colspec={*6{X[c]}},vline{2}={0.8pt}}
        $W$(\unit{kg}) & 20 & 40 & 60 & 80 & 100 \\
        $F$(\unit{kgf})& 4.35 & 7.55 & 10.40 & 13.80 & 16.80 \\
      \end{tblr}
    \end{tablehere}
    求适合以上关系的直线型经验公式。
    \item 一个弹簧的长度 $l$ 和它悬挂的重量 $W$ 间关系的实验数据如下:\par
    \begin{tablehere}
      \begin{tblr}{colspec={*7{X[c]}},vline{2}={0.8pt}}
        $W$ & 2 & 4 & 6 & 8 & 10 & 12 \\
        $l$ & 8.9 & 10.1 & 11.2 & 12.0 & 13.1 & 13.9 \\
      \end{tblr}
    \end{tablehere}
    分别用选点法和平均值法求关于 $l$、$W$ 的经验公式,并算出当 $l=9.2$ 时,$W$ 的值。
  \end{question}
\end{Exercise}

\section{两条直线的位置关系}
\subsection{两条直线的平行与垂直}
在平面几何中,我们研究过平面上两条直线互相平行或垂直的位置关系。
现在我们研究怎样通过直线的方程来判定两条直线平行或垂直。

设直线 $l_1$ 和 $l_2$ 的斜率为 $k_1$ 和 $k_2$,它们的方程分别是
\[ l_1:y =k_1x+b_1;\quad l_2:y =k_2x+b_2. \]

我们首先研究两条直线平行(不重合)的情形。

如果 $l_1\parallel l_2$(\cref{fig:1-29}),那么它们的倾斜角相等,$\alpha_1=\alpha_2$。

\begin{figure}
  \caption{}\label{fig:1-29}
\end{figure}

$\therefore \quad \tan\alpha_1=\tan\alpha_2$,即 $k_1=k_2$。

反过来,如果两条直线的斜率相等,$k_1=k_2$,也就是 $\tan\alpha_1=\tan\alpha_2$。

由于 $\ang{0}\leqslant\alpha_1<\ang{180}$,$\ang{0}\leqslant\alpha_2<\ang{180}$,

$\therefore \quad \tan\alpha_1=\tan\alpha_2$。

又因为两条直线不重合,

$\therefore l_1\parallel l_2$。

两条直线有斜率且不重合,如果它们平行,则斜率相等;反之,如果它们的斜率相等,则它们平行。即
\[ l_1\parallel l_2 \Leftrightarrow k_1=k_2.\]
\begin{example}
  已知两条不重合的直线
  \[ l_1:2x-4y+7=0,\quad l_2:x-2y+5=0. \]
  求证:$l_1\parallel l_2$。
\end{example}
\begin{proof}
  把 $l_1$、$l_2$ 的方程写成斜截式:
  \[ l_1:y=\frac{1}{2}x+\frac{7}{4},\quad l_2:y=\frac{1}{2}x+\frac{5}{2}. \]
  则 $l_1$ 的斜率 $k_1=\dfrac{1}{2}$,$l_2$ 的斜率 $k_1=\dfrac{1}{2}$,
  
  $\therefore\quad k_1=k_2.$

  又 $\because$ 两条直线不重合,

  $\therefore\quad l_1\parallel l_2.$
\end{proof}

\begin{example}
  求过点 $A\,(1,- 4)$,且与直线 $2x+3y+5=0$ 平行的直线方程。
\end{example}

\begin{solution}
已知直线的斜率是 $-\dfrac{2}{3}$,因为所求直线与已知直线平行,因此它的斜率也是 $-\dfrac{2}{3}$。

根据点斜式,得到所求直线的方程是
\[ y+4=-\frac{2}{3}(x-1),\]
即
\[ 2x+3y+10=0.\]
\end{solution}

现在研究两条直线垂直的情形。

如果 \(l_1 \perp l_2\) ,这时 $\alpha_1\neq\alpha_2$,(为什么?)

设 $\alpha_2<\alpha_1$(\cref{fig:1-30})。
根据三角形外角等于和它不相邻的两内角的和,得
\[ \alpha_1=\ang{90}+\alpha_2. \]
\begin{figure}
  \begin{minipage}[b]{0.48\linewidth}\centering
    \subcaption{}\label{fig:1-30a}
  \end{minipage}
  \begin{minipage}[b]{0.48\linewidth}\centering
    \subcaption{}\label{fig:1-30b}
  \end{minipage}
  \caption{}\label{fig:1-30}
\end{figure}

因为 $l_1$、$l_2$ 的斜率是 $k_1$、$k_2$,即 $\alpha_1\neq\ang{90}$,所以 $\alpha_2\neq\ang{0}$。
\[ \therefore \tan\alpha_1 = \tan(\ang{90}+\alpha_2) = - \frac{1}{\tan\alpha_2} \]
即
\[
k_1=-\frac{1}{k_2}\;\text{ 或 }\;k_1 \cdot k_2 = -1.
\]

反过来,如果 $k_1=-\frac{1}{k_2}$,即 $k_1 \cdot k_2 = -1$。设其中一个,例如 $k_2$ 是正数,则 $k_1$ 是负数。
那么 $\alpha_2$ 是锐角,$\alpha_1$ 是钝角。于是由
\[ \tan\alpha_1 = - \frac{1}{\tan\alpha_2} = \tan(\ang{90}+\alpha_2), \]
可以推出
\begin{gather*}
  \alpha_1 = \ang{90} + \alpha_2,\\
l_1 \perp l_2.
\end{gather*}

两条直线都有斜率,如果它们互相垂直,则它们的斜率互为负倒数;反之,如果它们的斜率互为负倒数,则它们互相垂直。即
\[ l_1 \perp l_2 \Leftrightarrow k_1=-\frac{1}{k_2} \]
或 $ l_1 \perp l_2 \Leftrightarrow k_1 \cdot k_2 = -1$。

\begin{example}
已知两条直线
\[l_1:2x-4y+7=0, \quad l_2:2x+y-5=0\]
求证:$l_1\perp l_2$
\end{example}
\begin{proof}
$l_1$ 的斜率 $k_1=\dfrac{1}{2}$,$l_2$ 的斜率 $k_2=-2$。

由于 $k_1\cdot k_2 =\dfrac{1}{2}\times(-2)=-1$,

$\therefore \quad l_1\perp l_2$。
\end{proof}

\begin{example}
求过点 $A\,(2,1)$,且与直线 $2x+y-10=0$ 垂直的直线方程。
\end{example}

\begin{solution}
  直线 $2x+y-10=0$ 的斜率是 $-2$。因为所求直线与已知直线垂直,所以它的斜率
  \[ k= -\frac{1}{-2}=\frac{1}{2}.\]

  根据点斜式,得到所求直线的方程是
\[ y-1=\frac{1}{2}(x-2), \]
就是
\[ x-2y= 0. \]
\end{solution}

\begin{Practice}
  \begin{question}
    \item 判别下列各对不重合的直线是否平行或垂直:
    \begin{tasks}(2)
      \task $y=3x+4$ 与 $2y-6x+1=0$;
      \task $y=x$ 与 $3x+3y-10=0$;
      \task $3x+4y=5$ 与 $6x-8y=7$;
      \task! $\sqrt{3}x-y-1=0$ 与 $\sqrt{3}x+3y+6=0$。
    \end{tasks}
    \item 求过点 $A\,(2,3)$,且分别适合下列条件的直线方程:
    \begin{tasks}(2)
      \task 平行于直线 $2x+y-5=0$;
      \task 垂直于直线 $x-y-2=0$。
    \end{tasks}
    \item 已知两条直线 $l_1$、$l_2$,其中一条没有斜率。这两条直线什么时候
    \begin{tasks}(2)
      \task 平行;
      \task 垂直。
    \end{tasks}
    逆命题成立吗?
  \end{question}
\end{Practice}

\subsection{两条直线所成的角}
两条直线 $l_1$ 和 $l_2$ 相交构成四个角,它们是两对对顶角。
为了区别这些角,我们把直线 $l_1$ 依逆时针方向旋转到与 $l_2$ 重合时所转的角,叫做 \Concept{$l_1$ 到 $l_2$ 的角},\cref{fig:1-31} 中,直线 $l_1$ 到 $l_2$ 的角是 $\theta_1$,$l_2$ 到 $l_1$ 的角是 $\theta_2$($\theta_1+\theta_2=\ang{180}$)。
\begin{figure}
  \caption{}\label{fig:1-31}
\end{figure}

现在我们来求斜率为 $k_1$、$k_2$ 的两条直线 $l_1$ 到 $l_2$ 的角 $\theta$。设已知直线的方程分别是
\begin{gather*}
  l_1: y=k_1x+b_1,\\
  l_2: y=k_2x+b_2.
\end{gather*}

如果 $1+k_1k_2=0$,那么 $\theta = \ang{90}$。

下面研究 $1+k_1k_2\neq0$ 的情形。

设 $l_1$、$l_2$ 的倾斜角分别是 $\alpha_1$ 和 $\alpha_2$(\cref{fig:1-32}),
\[ \tan\alpha_1=k_1,\quad \tan\alpha_2=k_2.\]
\begin{figure}
  \begin{minipage}[b]{0.48\linewidth}\centering
    \subcaption{}\label{fig:1-32a}
  \end{minipage}
  \begin{minipage}[b]{0.48\linewidth}\centering
    \subcaption{}\label{fig:1-32b}
  \end{minipage}
  \caption{}\label{fig:1-32}
\end{figure}

$\because\quad\theta =\alpha_2-\alpha_1$(\cref{fig:1-32a})或 $\theta = \uppi - (\alpha_1-\alpha_2) = \uppi+(\alpha_2-\alpha_1)$(\cref{fig:1-32b}),

$\therefore\quad\tan\theta = \tan(\alpha_2-\alpha_1)$ 或 $\tan\theta = \tan[\uppi+(\alpha_2-\alpha_1)]= \tan(\alpha_2-\alpha_1)$。可得
\[ \tan\theta=\frac{\tan\alpha_2-\tan\alpha_1}{1+\tan\alpha_2\tan\alpha_1},\]
即
\[ \tan\theta = \frac{k_2-k_1}{1+k_2k_1}.\]

从一条直线到另一条直线的角,可能不大于直角,也可能大于直角,但我们常常只需要考虑不大于直角的角(就是两条直线所成的角,简称\Concept{夹角})就可以了,这时可用下面的公式
\[ \tan\theta = \left|\frac{k_2-k_1}{1+k_2k_1}\right|.\]

\begin{example}
  求直线 $l_1:y=-2x+3$,$l_2:y=x-\dfrac{3}{2}$ 的夹角。
\end{example}
\begin{solution}
  两条直线的斜率为 $k_1= -2$,$k_2=1$,得
  \begin{gather*}
  \tan \theta = \left|\frac{k_2-k_1}{1+k_2k_1}\right| =\left|\frac{1-(-2)}{1+1\cdot(-2)k}\right|=3.\\
  \therefore\quad \theta=\arctan3\approx\ang{71;34;}.
  \end{gather*}
\end{solution}
\begin{example}
  求直线 $l_1:A_1x+B_1y+C_1=0$ 和 $l_2:A_2x+B_2y+C_2=0$($B_1\neq 0$、$B_2\neq 0$、$A_1A_2+B_1B_2\neq 0$),$l_1$ 到 $l_2$ 的角是 $\theta$。
  求证:
  \[ \tan\theta = \frac{A_1B_2-A_2B_1}{A_1A_2+B_1B_2}.\]
\end{example}
\begin{proof}
  设两条直线 $l_1$、$l_2$ 的斜率分别为 $k_1$、$k_2$,则
  \begin{gather*}
    k_1=-\frac{A_1}{B_1},\quad k_2=-\frac{A_2}{B_2},\\
    \therefore\quad \tan\theta=\frac{k_2-k_1}{1+k_2k_1}=\frac{-\dfrac{A_2}{B_2}-\left(-\dfrac{A_1}{B_1}\right)}{1+\left(-\dfrac{A_2}{B_2}\right)\left(-\dfrac{A_1}{B_1}\right)}=\frac{A_1B_2-A_2B_1}{A_1A_2+B_1B_2}.
  \end{gather*}
\end{proof}

\begin{example}
  等腰三角形一腰所在的直线 $l_1$ 的方程是 $x-2y-2=0$,底边所在的直线 $l_2$ 的方程是 $x+y-1=0$,点 $(-2,0)$ 在另一腰上,求这腰所在直线 $l_3$ 的方程。
\end{example}
\begin{solution}
  设 $l_1$、$l_2$、$l_3$ 的斜率分别为 $k_1$、$k_2$、$k_3$,$l_1$ 到 $l_2$ 的角是 $\theta_1$,$l_2$ 到 $l_3$ 的角是 $\theta_2$ 。则
  \begin{gather*} 
    k_1=\frac{1}{2},\quad k_2=-1,\\
    \tan\theta_1=\frac{k_2-k_1}{1+k_2k_1}=\frac{(-1)-\dfrac{1}{2}}{1+(-1)\cdot\dfrac{1}{2}}=-3.
  \end{gather*}
  因为 $l_1$、$l_2$、$l_3$ 所围成的三角形是等腰三角形,所以
  \begin{gather*} 
    \theta_1=\theta_2,\\
    \tan\theta_2=\tan\theta_2=-3.
  \end{gather*}
  即
  \begin{gather*} 
    \frac{k_3-k_2}{1+k_3k_2}=-3.\\
    \frac{k_3+1}{1-k_3}=-3.
  \end{gather*}
  解得
  \[ k_3=2.\]

  因为 $l_3$ 经过点 $(-2,0)$,斜率为 2,写出点斜式为
  \[ y=2[x-(-2)],\]
  得
  \[ 2x-y+4=0.\]

  这就是直线 $l_3$ 的方程。
\end{solution}

\begin{Practice}
  \begin{question}
    \item 求下列直线 $l_1$ 到 $l_2$ 的角与 $l_2$ 到 $l_1$ 的角:
    \begin{tasks}
      \task $l_1: y=\dfrac{1}{2}x+2,\quad l_2:y=3x+7$;
      \task $l_1: x-y=5,\quad l_2: x+2y-3=0$。
    \end{tasks}
    \item 求下列直线的夹角:
    \begin{tasks}(2)
      \task $y=3x-1,\quad y=-\dfrac{1}{3}x+4$;
      \task $x-y=5, \quad y=4$;
      \task $5x-3y=9, 6x+10y+7=0$。
    \end{tasks}
  \end{question}
\end{Practice}

\subsection{两条直线的交点}
设两条直线的方程是
\[l_1:A_1x+B_1y+C_1=0,\quad l_2:A_2x+B_2y+C_2=0.\]

如果这两条直线相交,由于交点同时在这两条直线上,交点的坐标一定是这两个方程的公共解;反之,如果这两个二元一次方程只有一个公共解,那么以这个解为坐标的点必是直线 $l_1$ 和 $l_1$ 的交点。
因此,两条直线是否有交点,就要看这两条直线的方程所组成的方程组
\begin{numcases}{}
  \label{eq:line_eq_1} A_1x+B_1y+C_1=0 \\ 
  \label{eq:line_eq_2} A_2x+B_2y+C_2=0  
\end{numcases}
是否有唯一解。

设 $A_1$、$A_2$、$B_1$、$B_2$ 全不为零。

解这个方程组,
\begin{align}
  \label{eq:eq1_times_B2} \text{\eqref{eq:line_eq_1}} \times B_2\ \text{得}\quad A_1B_2x+B_1B_2y+B_2C_1=0 \\
  \label{eq:eq2_times_B1} \text{\eqref{eq:line_eq_2}} \times B_1\ \text{得}\quad A_2B_1x+B_1B_2y+B_1C_2=0 
\end{align}

\cref{eq:eq1_times_B2}$-$\cref{eq:eq2_times_B1} 得 $(A_1B_2-A_2B_1)x+B_2C_1-B_1C_2=0$。

下面分两种情形进行讨论:
\begin{enumerate}
  \item 当 $A_1B_2-A_2B_1\neq 0$ 时,即 $\dfrac{A_1}{A_2} \neq \dfrac{B_1}{B_2}$ 时,方程有唯一解:
  \[ \begin{cases}
    x=\dfrac{B_1C_2-B_2C_1}{A_1B_2-A_2B_1}\\[10pt]
    y=\dfrac{C_1A_2-C_2A_1}{A_1B_2-A_2B_1}\\
  \end{cases}\]

  这时 $l_1$ 与 $l_2$ 相交,上面 $x$ 和 $y$ 的值就是交点的坐标。

  \medskip 因为直线 \eqref{eq:line_eq_1} 和 \eqref{eq:line_eq_2} 的斜率分别是 $-\dfrac{A_1}{B_1}$ 和 $-\dfrac{A_2}{B_2}$,由 $\dfrac{A_1}{B_1} \neq \dfrac{A_2}{B_2}$ 可得 $-\dfrac{A_1}{B_1}\neq -\dfrac{A_2}{B_2}$,也就是说,两条直线的斜率不相等,它们必相交于一点。

\medskip 例如,两条直线的方程是
\begin{align*}
  2x+3y-7 & = 0,\\
  5x-y-9 & = 0.
\end{align*}

由于 $\dfrac{2}{5} \neq \dfrac{3}{-1}$,这两个方程组成的方程组有唯一解,并且这个解是
\[\begin{cases}x=2\\y=1\end{cases}\]
就是说,这两条直线相交,交点的坐标是 $(2,1)$。

  \item 当 $A_1B_2-A_2B_1= 0$ 时:
  \begin{enumerate}
    \item 如果 $B_1C_2-B_2C_1\neq 0$,这时 $C_1$、$C_2$ 不能全为零。
    设 $C_2\neq 0$,有 $\dfrac{A_1}{A_2}=\dfrac{B_1}{B_2} \neq \dfrac{C_1}{C_2}$。方程组无解,也就是说这两条直线不相交,即两直线平行。

    \medskip 因为 $\dfrac{A_1}{A_2}=\dfrac{B_1}{B_2}\neq\dfrac{C_1}{C_2}$,即 $\dfrac{A_1}{B_1}=\dfrac{A_2}{B_2}$,$\dfrac{C_1}{B_1} \neq \dfrac{C_2}{B_2}$,所以直线 \eqref{eq:line_eq_1} 和 \eqref{eq:line_eq_2} 的斜率 $-\dfrac{A_1}{B_1}=- \dfrac{A_2}{B_2}$,截距 $\dfrac{C_1}{B_1} \neq \dfrac{C_2}{B_2}$,显然两条直线平行。

    \medskip 例如,两条直线的方程是
    \begin{align*}
      2x-3y+5 & = 0,\\
      4x-6y-7 & = 0.
    \end{align*}

    因为 $\dfrac{2}{4} = \dfrac{-3}{-6} \neq \dfrac{5}{-7}$,所以方程组没有解,两条直线平行。
    \item 如果 $B_1C_2-B_2C_1\neq 0$,这时 $C_1$、$C_2$ 或全为零,或全不为零$\biggl($当 $C_1$、$C_2$ 全不为零时,$\dfrac{A_1}{A_2}= \dfrac{B_1}{B_2}=\dfrac{C_1}{C_2}\biggr)$。
    两个方程是同解方程,因此它们有无穷多解。

    \medskip 这时两条直线的斜率 $-\dfrac{A_1}{B_1}=-\dfrac{A_2}{B_2}$,在 $y$ 轴上的截距 $-\dfrac{C_1}{B_1}=-\dfrac{C_2}{B_2}$(当 $C_1$、$C_2$ 全为零时,两条直线都通过原点),两方程的直线重合。

例如,两条直线的方程是
\begin{align*}
  2x-3y+5 & = 0,\\
  4x-6y+10 & = 0.
\end{align*}
由于 $\dfrac{2}{4} = \dfrac{-3}{-6} = \dfrac{5}{10}$,两个方程是同解方程,方程组有无穷多个解,两条直线重合。
  \end{enumerate}
\end{enumerate}

\bigskip 如果 $A_1$、$A_2$、$B_1$、$B_2$ 中有等于零的情形,方程比较简单,两条直线的交点很容易讨论。
\begin{example}
  求下列两条直线的交点。
  \begin{align*}
    l_1:&&3x+4y-2 & = 0,\\
    l_2:&&2x+y+2 & = 0.
  \end{align*}
\end{example}
\begin{solution}
  解方程组
  \[\begin{cases} 3x+4y=0,\\2x+y+2=0. \end{cases}\]
  得
  \[\begin{cases} x=-2,\\y=2. \end{cases}\]
  $\therefore \quad l_1$ 与 $l_2$ 的交点是 $M\,(-2,2)$,如\cref{fig:1-33} 所示。
\end{solution}
\begin{figure}
  \caption{}\label{fig:1-33}
\end{figure}
\begin{example}
  已知两条直线:
  \begin{align*}
    l_1:&&x+my+6 & = 0,\\
    l_2:&&(m-2)x+3y+2m & = 0.
  \end{align*}
  当 $m$ 为何值时,${l}_{1}$ 与 ${l}_{2}$ 
  \begin{tasks}(3)
    \task 相交;
    \task 平行;
    \task 重合。
  \end{tasks}
\end{example}
\begin{solution}
  将两直线的方程组成方程组
  \[\begin{cases} x+my+6=0,\\(m-2)x+3y+2m=0. \end{cases}\]

  这时,$\dfrac{A_1}{A_2}=\dfrac{1}{m - 2}$,$\dfrac{B_1}{B_2} = \dfrac{m}{3}$,$\dfrac{C_1}{C_2} = \dfrac{6}{2m}$。

  \medskip 当 $\dfrac{A_1}{A_2}=\dfrac{B_1}{B_2}$ 时,$\dfrac{1}{m-2} = \dfrac{m}{3}$,解得 $m=-1$ 或 $m=3$。

\medskip 当 $\dfrac{A_1}{A_2}=\dfrac{C_1}{C_2}$ 时,$\dfrac{1}{m-2} = \dfrac{6}{2m}$,解得 $m=3$。
\begin{enumerate}
  \item 当 $m\neq-1$ 且 $m\neq 3$ 时,$\dfrac{A_1}{A_2} \neq \dfrac{B_1}{B_2}$ ,方程组有唯一解,$l_1$ 与 $l_2$ 相交。
  \item 当 $m=-1$ 时,$\dfrac{A_1}{A_2}= \dfrac{B_1}{B_2}$,$\dfrac{A_1}{A_2} \neq \dfrac{C_1}{C_2}$,方程组无解,$l_1$ 与 $l_2$ 平行。
  \item 当 $m=3$ 时,$\dfrac{A_1}{A_2}= \dfrac{B_1}{B_2}= \dfrac{C_1}{C_2}$,方程组有无穷多解,$l_1$ 与 $l_2$  重合。
\end{enumerate}
\end{solution}

\begin{Practice}
  \begin{question}
    \item 求下列各对直线的交点,并画图:
    \begin{tasks}
      \task $l_1:2x+3y=12,\quad l_2:x-2y=4$;
      \task $l_1:x=2,\quad l_2:3x+2y-12=0$;
    \end{tasks}
    \item 判定下列各对直线的位置关系。如果相交,则求出交点的坐标:
    \begin{tasks}
      \task $l_1:2x-y=7,\quad l_2:4x+2y=1$;
      \task $l_1:2x-6y+4=0,\quad l_2:y=\dfrac{x}{3}+\dfrac{2}{3}$;
      \task $l_1:(\sqrt{2}-1)x+y=3,\quad l_2:x+(\sqrt{2}+1)y=2$;
    \end{tasks}
    \item $A$ 和 $C$ 取什么值时,直线 $Ax-2y-1=0$ 和直线 $6x-4y+C=0$
    \begin{tasks}(3)
      \task 平行;
      \task 重合;
      \task 相交。
    \end{tasks} 
  \end{question}
\end{Practice}

\subsection{点到直线的距离}
已知点 $P\,(x_0,y_0)$ 和直线 $l:Ax+By+C=0$,怎样求点 $P$ 到直线 $l$ 的距离呢?

根据定义,点 $P$ 到直线 $l$ 的距离是点 $P$ 到直线 $l$ 的垂线段的长(\cref{fig:1-34})。
\begin{figure}
  \begin{minipage}[b]{0.48\linewidth}\centering
    \subcaption{}\label{fig:1-34a}
  \end{minipage}
  \begin{minipage}[b]{0.48\linewidth}\centering
    \subcaption{}\label{fig:1-34b}
  \end{minipage}
  \caption{}\label{fig:1-34}
\end{figure}

设点 $P$ 到直线 $l$ 的垂线为 $l'$ ,垂足为 $Q$。由 $l'\perp l$ 可知 $l'$ 的斜率为 $\frac{B}{A}(A \neq 0)$,根据点斜式可写出 $l'$ 的方程,并由 $l$ 与 $l'$ 的方程求出点 $Q$ 的坐标;由此即可根据两点距离公式求出 $|PQ|$,这就是点 $P$ 到直线 $l$ 的距离。

这个方法虽然思路自然,但是运算很繁,下面介绍另一种求法。

设 $A\neq 0$,$B\neq 0$,直线 $l$ 的倾斜角为 $\alpha$。过点 $P$ 作 $PM\parallel Oy$,那么 $PM$ 与 $l$ 相交于点 $M\,(x_1,y_1)$(\cref{fig:1-34})。
\begin{gather*}
\because \quad PM\parallel Oy,\\
\therefore \quad x_1=x_0.
\end{gather*}
代入直线 $l$ 的方程可得
\begin{gather*}
  y_1=-\frac{Ax_0+C}{B}.\\
  \begin{split}
  \therefore\quad |PM| & =|y_0-y_1|=\left|y_0+\frac{Ax_0+C}{B}\right|\\
                  & = \frac{|Ax_0+By_0+C|}{|B|}
  \end{split}
\end{gather*}

当 $\alpha<\ang{90}$ 时(如\cref{fig:1-34a}),$\alpha_1=\alpha$;

当 $\alpha>\ang{90}$ 时(如\cref{fig:1-34b}),$\alpha_1= \uppi-\alpha$,

所以,在两种情况下都有
\begin{gather*}
\tan^2\alpha_1=\tan^2\alpha=\frac{A^2}{B^2}.\\
\because \quad \alpha_1< \ang{90},\\
\begin{split}
  \cos\alpha_1&=\frac{1}{\sqrt{1+\tan^2\alpha_1}}=\frac{1}{\sqrt{1+\dfrac{A^2}{B^2}}}\\
   &=\frac{\left| B\right| }{\sqrt{{A}^{2} + {B}^{2}}}.
\end{split}\\
\begin{split}
  \therefore\quad |PQ|&=|PM|\cos\alpha_1\\
   &=\frac{|Ax_0+By_0+C|}{|B|} \cdot \frac{|B|}{\sqrt{A^2+B^2}}\\
   &=\frac{|Ax_0+By_0+C|}{\sqrt{A^2+B^2}}.
\end{split}
\end{gather*}

这样,我们就得到平面内一点 $P\,(x_0,y_0)$ 到一条直线 $Ax+By+C=0$ 的距离公式:
\[ d = \frac{|Ax_0+By_0+C|}{\sqrt{A^2+B^2}}. \]

如果 $A=0$ 或 $B=0$,上面的距离公式仍然成立。
但这时不需要利用公式就可以直接求出距离。

\begin{example}
  求点 $P_0\,(-1,2)$ 到直线 
  \begin{tasks}(2)
    \task $2x+y-10=0$;
    \task $3x=2$。
  \end{tasks}
  的距离。
\end{example}
\begin{solution}
  \begin{enumerate}[a)]
    \item 根据点到直线的距离公式,得
    \[ d=\frac{|2\times(-1)+1\times2-10|}{\sqrt{2^2+1^2}}=2\sqrt{5}\]
    \item 因为直线 $3x=2$ 平行于 $y$ 轴,所以
    \[d=\left|\frac{2}{3}-(-1)\right|=\frac{5}{3}\]
  \end{enumerate}
\end{solution}

\begin{example}
  求平行线 $2x-7y+8=0$ 和 $2x-7y-6=0$ 的距离。
\end{example}
\begin{solution}
  在直线 $2x-7y-6=0$ 上任取一点,例如取 $P\,(3,0)$(\cref{fig:1-35}),则两平行线的距离就是点 $P\,(3,0)$ 到直线 
  \[2x-7y+8=0\]
  的距离。

  因此,
  \[\begin{split}
    d&=\frac{|2\times3-7\times0+8|}{\sqrt{2^2+7^2}}\\
    &=\frac{14}{\sqrt{53}}=\frac{14\sqrt{53}}{53}.
  \end{split}\]
\end{solution}
\begin{figure}
  \caption{}\label{fig:1-35}
\end{figure}

\begin{Practice}
  \begin{question}
    \item 求坐标原点到下列直线的距离:
    \begin{tasks}(2)
      \task $3x+2y-26=0$;
      \task $x=y$。
    \end{tasks}
    \item 求下列点到直线的距离:
    \begin{tasks}(2)
      \task $A\,(-2,3)$,$3x+4y+3=0$;
      \task $B\,(1,0)$,$\sqrt{3}x+y-\sqrt{3}=0$;
      \task $C\,(1,-2)$,$4x+3y=0$。
    \end{tasks}
    \item 求下列两条平行线的距离:
    \begin{tasks}
      \task $2x+3y-8=0$,$2x+3y+18=0$;
      \task $3x+4y=10$,$3x+4y=0$。
    \end{tasks}
  \end{question}
\end{Practice}

\begin{Exercise}
  \begin{question}
    \item 已知直线分别满足下列条件,求直线的方程:
    \begin{tasks}
      \task 经过点 $A\,(3,2)$,且与直线 $4x+y-2=0$ 平行;
      \task 经过点 $B\,(3,0)$,且与直线 $2x+y-5=0$ 垂直;
      \task 经过点 $C\,(2,-3)$,且平行于过两点 $M\,(1,2)$ 和 $N\,(-1,-5)$ 的直线。
    \end{tasks}
    \item 设有两点 $A\,(7,-4)$、$B\,(-5,6)$,求线段 $AB$ 的垂直平分线的方程。
    \item 三角形三个顶点是 $A\,(4,0)$、$B\,(6,7)$、$C\,(0,3)$。求这个三角形的三条高所在直线的方程。
    \item 已知直线分别满足下列条件,求直线的方程:
    \begin{tasks}
      \task 斜率为 $-2$,且过两条直线 $3x-y+4=0$ 和 $x+y-4=0$ 的交点;
      \task 过两条直线 $x-2y+3=0$ 和 $x+2y-9=0$ 的交点和原点;
      \task 经过两条直线 $2x-3y+10=0$ 和 $3x+4y-2=0$ 的交点,且垂直于直线 $3x-2y+4=0$;
      \task 经过两条直线 $2x+y-8=0$ 和 $x-2y+1=0$ 的交点,且平行于直线 $4x-3y-7=0$;
      \task 经过直线 $y=2x+3$ 和 $3x-y+2=0$ 的交点,且垂直于第一条直线。
    \end{tasks}
    \item 直线 $ax+2y+8=0$,$4x+3y=10$ 和 $2x-y=10$ 相交于一点。求 $a$ 的值。
    \item 不解方程组,判定下列两个方程的直线的位置关系。
    \begin{tasks}(2)
      \task $\begin{cases} 2x+y=11,\\x+3y=18; \end{cases}$
      \task $\begin{cases} 2x-3y=4,\\4x-6y=8; \end{cases}$
      \task $\begin{cases} 3x+10y=16,\\6x+20y=7; \end{cases}$
      \task $\begin{cases} 4x+10y=12,\\6x-15y=18. \end{cases}$
    \end{tasks}
    \item 已知两条直线 $l_1:(3+m)x+4y=5-3m$,$l_2:2x+(5+m)y=8$。$m$ 为何值时,$l_1$ 与 $l_2$
    \begin{tasks}(3)
      \task 相交;
      \task 平行;
      \task 重合。
    \end{tasks}
    \item 讨论两条直线 $l_1:A_1x+B_1y+C_1=0$,$l_2:A_2x+B_2y+C_2=0$ 的位置关系:
    \begin{tasks}(2)
      \task 当 $B_1=0$、 $B_2\neq0$ 时,
      \task 当 $B_1=B_2=0$ 时。
    \end{tasks}
    \item 三角形的三个顶点是 $A\,(6,3)$、$B\,(9,3)$、$C\,(3,6)$。求它的三个内角的度数。
    \item 已知直线 $l$ 经过点 $P\,(2,1)$ ,且和直线 $5x+2y+3=0$ 的夹角等于 \ang{45}。求直线 $l$ 的方程。
    \item 光线从点 $M\,(-2,3)$ 射到 $x$ 轴上一点 $P\,(1,0)$ 后被 $x$ 轴反射。求反射光线所在直线的方程。
    \item 求点 $P\,(-5,7)$ 到直线 $12x+5y-3=0$ 的距离。
    \item 点 $A\,(a,6)$ 到直线 $3x-4y=2$ 的距离
    \begin{tasks}(2)
      \task 等于 4;
      \task 大于 4。
    \end{tasks}
    分别求 $a$ 的值。
    \item 求证:两条平行线 $Ax+By+C_1=0$ 与 $Ax+By+C_2=0$ 的距离是
    \[ d = \frac{|C_1-C_2|}{\sqrt{A^2+B^2}}. \]
    \item 求两条平行线 $3x-2y-1=0$ 和 $6x-4y+2=0$ 的距离。
  \end{question}
\end{Exercise}

\section*{小结}
\begin{enumerate}[C、,itemindent=4.5em]
  \item 在这一章里,我们首先研究了有向线段、两点的距离及线段的定比分点;接着又研究了直线方程的各种形式,并利用这些方程讨论了两条直线的位置关系和两条直线所成的角、点到直线的距离。这种通过方程研究图形性质的方法就是解析法。解析法揭示了数学中“数”和“形”的内在联系。
  \item 有向线段的数量、两点的距离与定比分点等公式是解析几何的基本公式。在用解析法研究点的轨迹问题时,经常用到这些公式。
  \item 本章介绍了点斜式、斜截式;两点式、截距式等直线方程的特殊形式,并研究了直线方程的一般式,这些直线的方程都是二元一次方程。每一个二元一次方程都表示一条直线;反之,表示每一条直线的方程都是二元一次方程。但是表示同一条直线的方程形式却不是唯一的,不过它们都可以通过方程的同解变形互化,可以看作是同一个方程。就这个意义来说,直线和二元一次方程是一一对应的。
  \item 直线的斜率和截距是表示直线位置的重要特征数值,通过它们可以判定两条直线的位置关系:平行、相交(包括垂直)及重合。
  
  在斜截式 $y=kx+b$ 中,如果 $k$ 固定,$b$ 取不同的值时,我们得到一组平行的直线;在点斜式方程 $y-y_0=k(x-x_0)$ 中,$k$ 取不同的值时,则得到过定点 $A\,(x_0,y_0)$ 的一组直线。
  \item 二元一次不等式表示区域和直线型经验公式在生产与科学研究中都常常用到。利用二元一次不等式表示的区域,我们还可以确定一个点与直线的位置关系。把点 $P\,(x_0,y_0)$ 的坐标代入直线方程 $Ax+By+C=0$ 的左边,由 $Ax+By+C$ 的值的正负可以确定点 $P$ 是在该直线的上方还是下方。
  \item 在研究了直线方程的各种形式之后,本章还研究了两条直线的交点、夹角以及点到直线的距离等问题。我们除了要掌握这些问题的结论外,还应注意学习通过代数方程研究几何性质的方法。
\end{enumerate}

\chapter*{复习参考题\chinese{chapter}}
\section*{A 组}
\begin{question}
  \item 已知 $\triangle ABC$ 顶点的坐标是 $A\,(2,3)$、$B\,(5,3)$、$C\,(2,7)$。求 $\angle A$ 的平分线长及所在直线的方程。
  \item 已知矩形的顶点为 $O\,(0,0)$、$A\,(8,0)$、$B\,(0,5)$。试用两种方法求两条对角线所在直线的方程。
  \item 用两种方法证明:三点 $A\,(-2,12)$、$B\,(1,3)$、$A\,(4,-6)$ 在同一条直线上。
  \item 求直线 $2x-5y-10=0$ 和坐标轴所围成的三角形的面积。
  \item 求经过点 $A\,(-3,4)$,并且在两轴上截距的和等于 12 的直线的方程。
  \item 在 $x$,$y$ 的坐标都不小于 0 的整数点中,求满足 $x+y\leqslant 4$ 的点的个数。
  \item 一定量的气体,在一定压强下体积 $V$(\unit{cm^3})和温度 $T$(\unit{\celsius})之间关系的实验数据如下:
  \begin{tablehere}
    \begin{tblr}{colspec={c*{6}{X[c]}},vline{2}=0.8pt}
      $T$(\unit{\celsius})& 20 & 30 & 40 & 50 & 60 & 70 \\
      $V$(\unit{cm^3})    & 106.5 & 110.9 & 114.0 & 117.2 & 121.2 & 124.7 \\
    \end{tblr}
  \end{tablehere}
  用选点法求出 $V$ 与 $T$ 间的经验公式。
  \item 一个 \qty{10}{\ohm} 的电阻,使用不同的电压通电,量得电流数值如下:
  \begin{tablehere}
    \begin{minipage}{\linewidth}
    \begin{tblr}{colspec={c*{6}{X[c]}},vline{2}=0.8pt}
      电压 $V$(\unit{V})& 5 & 10 & 15 & 20 & 25 & 30 \\
      电流 $I$(\unit{A})& 0.55 & 1.14 & 1.45 & 2.10 & 2.45 & 3.10 \\
    \end{tblr}
    \end{minipage}
  \end{tablehere}
  用平均值法求出 $I$ 与 $V$ 间的经验公式。
  \item 已知平行四边形两条边所在直线的方程是
  \[ x+y-1=0,\quad 3x-y+4=0,\]
  它的对角线的交点是 $M\,(3,3)$。求这个平行四边形其他两边的方程。
  \item 直线 $(3a+2)x+(1-4a)y+8=0$ 和 $(5a-2)x+(a+4)y-7=0$ 互相垂直。求 $a$ 的值。
  \item $ABCD$ 是正方形,$E$、$F$、$G$、$H$ 分别是边 $AB$、$BC$、$CD$、$DA$ 的中点。求证:
  \begin{tasks}
    \task 线段 $AF$、$BG$、$CH$、$DE$ 围成一个正方形;
    \task 这个正方形的面积是原正方形面积的 $\dfrac{1}{5}$。
  \end{tasks}
  \item 解答:
  \begin{enumerate}[a),itemindent=1.5em]
    \item 当 $a$ 为何值时,下列两方程的直线平行:
    \begin{tasks}[label=\roman*)]
      \task $ax-5y=9$,$2x-3y=15$;
      \task $x+2ay-1=0$,$(3a-1)x-ay-1=0$;
      \task $2x+3y=a$,$4x+6y-3=0$。
    \end{tasks}
    \item 当 $m$、$n$ 各为何值时,下列两方程的直线重合:
    \begin{tasks}[label=\roman*)]
      \task $x+2y=4$,$mx+y=n$;
      \task $mx+10y=2$,$3x+(n-1)y=-1$。
    \end{tasks}
    \item 当 $a$ 为何值时,下列两方程的直线互相垂直:
    \begin{tasks}[label=\roman*)]
      \task $4ax+y=1$,$(1-a)x+y=-1$;
      \task $2x+ay=2$,$ax+2y=1$。
    \end{tasks}
  \end{enumerate}
  \item 已知两条直线 $l_1:x+(1+m)y=2-m$,$l_2:2mx+4y=-16$。$m$ 为何值时,$l_1$ 与 $l_2$
  \begin{tasks}(3)
    \task 相交;
    \task 平行;
    \task 重合。
  \end{tasks}
  \item 求两条平行直线 $3x+4y-12=0$ 和 $6x+8y+11=0$ 的距离。
  \item 求平行于直线 $x-y-2=0$ 且与它的距离为 $2\sqrt{2}$ 的直线方程。
  \item 正方形的中心在 $C\,(-1,0)$,一条边所在的直线方程是 $x+3y-5=0$,求其他三边所在的直线方程。
\end{question}
\section*{B 组}
\begin{question}[resume]
  \item 已知: 直线 $l:Ax+By+C=0$($A\neq 0,\,B\neq 0$),点 $M\,(x_0,y_0)$。求证:
  \begin{tasks}
    \task 经过点 $M$ 且平行于直线 $l$ 的直线方程是 $A(x-x_0)+B(y-y_0)=0$;
    \task 经过点 $M$ 且垂直于直线 $l$ 的直线的方程是 \[\frac{x-x_0}{A}=\frac{y-y_0}{B}.\]
  \end{tasks}
  \item 已知平行直线 $3x+2y-6=0$ 与 $6x+4y-3=0$。求与它们等距离的平行线的方程。
  \item 一条光线从点 $M_0\,(5,3)$ 射出后,被直线 $l:x+y=1$ 反射,入射光线到直线 $l$ 的角为 $\beta$。已知 $\tan\beta = 2$ ,求入射光线与反射光线所在直线的方程。
  \item 直线 $l_1$、$l_2$、$l_3$ 的方程分别是
  \begin{align*}
    l_1 &: A_1x+B_1y+C_1 =0; \\
    l_2 &: A_2x+B_2y+C_2 =0; \\
    l_3 &: A_1x+B_1y+C_1 + \lambda(A_2x+B_2y+C_2) =0. 
  \end{align*}
  如果当 
  \begin{tasks}(2)
    \task $l_1$ 和 $l_2$ 相交;
    \task $l_1$ 和 $l_2$ 平行
  \end{tasks}
  时,直线 $l_3$ 和 $l_1$、$l_2$ 的位置关系如何?
  \item\label{exec:1t-21} 山顶上有高 $h$ 的塔 $BC$,从塔顶 $B$ 测得地面上一点 $A$ 的俯角是 $\alpha$,从塔底 $C$ 测得点 $A$ 的俯角是 $\beta$,用解析法求山高 $H$。
  \begin{figurehere}
    \begin{minipage}{\linewidth}\centering
    \caption*{(第~\ref{exec:1t-21}~题)}
    \end{minipage}
  \end{figurehere}
  \item 过两点 $A\,(-3,2)$ 和 $B\,(6,1)$ 的直线与直线 $x+3y-6=0$ 交于点 $P$。求点 $P$ 分 $\overline{AB}$ 所成的比。
\end{question} % ok
\chapter{圆锥曲线}
\section{曲线和方程}
\subsection{曲线和方程}\label{subsec:curve_equation}
在\cref{chp:line} 里,我们研究过直线的各种方程,讨论了直线和二元一次方程的关系。下面,我们进一步研究一般曲线(包括直线)和方程的关系。

我们知道,两坐标轴所成的角在第一、三象限的平分线的方程是 $x-y= 0$,就是说,如果点 $M\,(x_0,y_0)$ 是这条直线上的任意一点,它到两坐标轴的距离一定相等,即 $x_0=y_0$,那么它的坐标 $(x_0,y_0)$ 是方程 $x-y=0$ 的解;反过来,如果 $(x_0,y_0)$ 是方程 $x-y=0$ 的解,即 $x_0=y_0$,那么以这个解为坐标的点到两轴的距离相等,它一定在这条平分线上。
这样,我们就说 $x-y=0$ 是这条平分线的方程。

又如,函数 $y=ax^2$ 的图象是关于 $y$ 轴对称的抛物线,这条抛物线是所有以方程 $y=ax^2$ 的解为坐标的点组成的。
这就是说,如果 $M\,(x_0,y_0)$ 是抛物线上的点,那么 $(x_0,y_0)$ 一定是这个方程的解;反过来,如果 $(x_0,y_0)$ 是方程 $y=ax^2$ 的解,那么以它为坐标的点一定在这条抛物线上。
这样,我们就说 $y=ax^2$ 是这条抛物线的方程。

一般地,在直角坐标系中,如果某曲线 $C$(看作适合某种条件的点的集合或轨迹)上的点与一个二元方程 $f(x,y)= 0$ 的实数解建立了如下的关系:
\begin{enumerate}[1.]
  \item 曲线上的点的坐标都是这个方程的解;
  \item 以这个方程的解为坐标的点都是曲线上的点,
\end{enumerate}
那么,这个方程叫做\Concept{曲线的方程};这条曲线叫做\Concept{方程的曲线(图形)}。
\begin{example}
  证明以坐标原点为圆心,半径等于 5 的圆的方程是 $x^2+y^2=25$,并判断点 $M_1\,(3,-4)$,$M_2\,(-2\sqrt{5},2)$ 是否在这个圆上。
\end{example}
\begin{proof}
  \begin{enumerate}
    \item 设 $M\,(x_0,y_0)$ 是圆上任意一点。因为点 $M$ 到坐标原点的距离等于 5,所以
    \[ \sqrt{x_0^2+y_0^2}=5,\]
    也就是
    \[ x_0^2+y_0^2=25.\]
    即 $(x_0,y_0)$ 是方程 $x^2+y^2=25$ 的解。
    \item 设 $(x_0,y_0)$ 是方程 $x^2+y^2=25$ 的解,那么
    \[ x_0^2+y_0^2=25,\]
    两边开方取算术根,得
    \[ \sqrt{x_0^2+y_0^2}=5.\]
    即点 $M\,(x_0,y_0)$ 到坐标原点的距离等于 5,点 $M\,(x_0,y_0)$ 是这个圆上的点。
  \end{enumerate}
  
  因此,方程 $x^2+y^2=25$ 是以坐标原点为圆心,半径等于 5 的圆的曲线方程。

  把 $M_1\,(3,-4)$ 的坐标代入方程 $x^2+y^2=25$,左右两边相等,$(3,-4)$ 是方程的解,所以点 $M_1$ 在这个圆上;

  把 $M_2\,(-2\sqrt{5},2)$ 的坐标代入方程 $x^2+y^2=25$,左右两边不等,$(-2\sqrt{5},2)$ 不是方程的解,所以点 $M_1$ 不在这个圆上(如\cref{fig:2-1})。
\end{proof}
\begin{figure}
  \caption{}\label{fig:2-1}
\end{figure}

\begin{Practice}
  \begin{question}
    \item 到两坐标轴距离相等的点组成的直线的方程是 $x-y=0$ 吗? 为什么?
    \item 已知等腰三角形三个顶点的坐标是 $A\,(0,3)$,$B\,(-2,0)$,$C\,(2,0)$。中线 $AO$ 的方程是 $x=0$ 吗?为什么?
  \end{question}
\end{Practice}

\subsection{求曲线的方程}
我们先看两个例子。
\begin{example}
  设 $A$、$B$ 两点的坐标是 $(-1,-1)$、$(3,7)$,求线段 $AB$ 的垂直平分线的方程。
\end{example}
\begin{solution}
  设 $M\,(x,y)$ 是线段 $AB$ 的垂直平分线上任意一点(\cref{fig:2-2}),也就是点 $M$ 属于集合
  \[ P = \bigl\{ M \bigm\vert |MA|= |MB| \bigr\}. \]

  由两点的距离公式,点 $M$ 所适合的条件可表示为
  \[ \sqrt{(x+1)^2+(y+1)^2}=\sqrt{(x-3)^2+(y-7)^2}.\]
  两边平方后,得
  \[(x+1)^2+(y+1)^2=(x-3)^2+(y-7)^2.\]
  即
  \begin{equation}
    \label{eq:mid_perp_line_equation}
    x+2y-7=0
  \end{equation}
\end{solution}
\begin{figure}
  \caption{}\label{fig:2-2}
\end{figure}

下面,我们证明\cref{eq:mid_perp_line_equation} 是线段 $AB$ 的垂直平分线的方程。
\begin{enumerate}[1 ]
  \item 由上面求方程的过程可知,垂直平分线上每一点的坐标都是\cref{eq:mid_perp_line_equation} 的解;
  \item 设点 $M_1$ 的坐标 $(x_1,y_1)$ 是方程\cref{eq:mid_perp_line_equation} 的解,即
  \begin{gather*}
    x_1+2y_1-7=0\\
    x_1=7-2y_1
  \end{gather*}
  点 $M_1$ 到 $A$、$B$ 的距离分别是
  \begin{align*}
    |M_1A| & = \sqrt{(x_1+1)^2+(y_1+1)^2}\\
           & = \sqrt{(8-2y_1)^2+(y_1+1)^2}\\
           & = \sqrt{5(y_1^2-6y_1+13)};
  \end{align*}
  \begin{align*}
    |M_1B| & = \sqrt{(x_1-3)^2+(y_1-7)^2}\\
           & = \sqrt{(4-2y_1)^2+(y_1-7)^2};\\
           & = \sqrt{5(y_1^2-6y_1+13)}.\\
    \therefore \quad |M_1A| & = |M_1B|,
  \end{align*}
  即点 $M_1$ 在线段 $AB$ 的垂直平分线上。
\end{enumerate}

由上述证明可知,方程\cref{eq:mid_perp_line_equation} 是线段 $AB$ 的垂直平分线的方程。
\begin{example}
  点 $M$ 与两条互相垂直的直线的距离的积是常数 $k$($k>0$),求点 $M$ 的轨迹方程。
\end{example}
\begin{figure}
  \caption{}\label{fig:2-3}
\end{figure}
\begin{solution}
  取已知两条互相垂直的直线为坐标轴,建立直角坐标系(\cref{fig:2-3})。

  设点 $M$ 的坐标为 $(x,y)$。点 $M$ 的轨迹就是与坐标轴的距离的积是常数 $k$ 的点的集合
  \[P = \bigl\{ M\bigm\vert |MR| \cdot |MQ|= k\bigr\},\]
  其中 $Q$、$R$ 分别是点 $M$ 到 $x$ 轴、$y$ 轴的垂线的垂足。

  因为点 $M$ 到 $x$ 轴、 $y$ 轴的距离,分别是它的纵坐标和横坐标的绝对值,所以条件 $\left| {MR}\right| \cdot \left| {MQ}\right| = k$ 可写成
\[|x| \cdot |y| = k\]
即
\begin{equation}
  \label{eq:inverse_proportion_equation} 
xy=\pm k.
\end{equation}
\end{solution}

下面我们证明\cref{eq:inverse_proportion_equation} 是所求轨迹的方程。
\begin{enumerate}[1 ]
  \item\label{itm:proof_1} 由上面求方程的过程可知,曲线上的点的坐标都是\cref{eq:inverse_proportion_equation} 的解;
  \item\label{itm:proof_2} 设点 $M_1$ 的坐标 $(x_1,y_1)$ 是\cref{eq:inverse_proportion_equation} 的解,那么
  \[ x_1y_1=\pm k,\]
  即
  \[ |x_1|\cdot|y_1|=k.\]
  而 $|x_1|$、$|y_1|$ 正是点 $M_1$ 到纵轴、横轴的距离,因此点 $M_1$ 到这两条直线的距离的积是常数 $k$,点 $M_1$ 在\cref{eq:inverse_proportion_equation} 的曲线上。
\end{enumerate}

由~\ref{itm:proof_1}、\ref{itm:proof_2} 可知,\cref{eq:inverse_proportion_equation} 是所求轨迹的方程。图形如\cref{fig:2-3}。

由上面的例子可以看出,求曲线 (图形) 的方程,一般有下面几个步骤:
\begin{enumerate}
  \item\label{itm:proof_step1} 建立适当的直角坐标系,用 $(x,y)$ 表示曲线上任意一点 $M$ 的坐标;
  \item\label{itm:proof_step2} 写出适合条件 $p$ 的点 $M$ 的集合 $P = \{ M \mid p(M) \}$;
  \item\label{itm:proof_step3} 用坐标表示条件 $p(M)$ ,列出方程 $f(x,y)=0$;
  \item\label{itm:proof_step4} 化方程 $f(x,y)=0$ 为最简形式;
  \item\label{itm:proof_step5} 证明以化简后的方程的解为坐标的点都是曲线上的点。
\end{enumerate}

除个别情况外,化简过程都是同解变形过程,步骤~\ref{itm:proof_step5} 可以省略不写,如有特殊情况,可适当予以说明。
另外,根据情况,也可以省略步骤~\ref{itm:proof_step2},直接列出曲线方程。

\begin{example}
  已知一条曲线在 $x$ 轴的上方,它上面的每一点,到点 $A\,(0,2)$ 的距离减去它到 $x$ 轴的距离的差都是 2,求这条曲线的方程。
\end{example}
\begin{solution}
  设点 $M\,(x,y)$ 是曲线上任意一点,$MB \perp x$ 轴,垂足是 $B$(\cref{fig:2-4}),那么点 $M$ 属于集合
  \[P = \bigl\{ M\bigm||MA|-|MB|= 2\bigr\}\]
  由距离公式,点 $M$ 适合的条件可表示为
  \begin{equation}
    \label{eq:parabolic_equation}
    \sqrt{x^2+(y-2)^2}-y=2.
  \end{equation}
  将\cref{eq:parabolic_equation} 移项后再两边平方,得
  \[x^2+(y-2)^2 =(y+2)^2,\]
  化简得:
  \[y=\frac{1}{8}x^2.\]

  因为曲线在 $x$ 轴的上方,$y>0$,虽然原点 $O$ 的坐标 $(0,0)$ 是这个方程的解,但不属于已知曲线,所以曲线的方程应是 $y=\dfrac{1}{8}x^2\,(x\neq 0)$,它的图形是关于 $y$ 轴对称的抛物线(\cref{fig:2-4}),但缺一个顶点。
\end{solution}

求出曲线方程以后,我们就可以根据曲线的方程,来研究曲线的几何性质,这个问题,我们将在后面结合各种具体的曲线方程来说明。
\begin{figure}
  \caption{}\label{fig:2-4}
\end{figure}

\begin{Practice}
  \begin{question}
    \item 求到坐标原点的距离等于 2 的点的轨迹方程。
    \item 已知点 $M$ 与 $x$ 轴的距离和它与点 $F\,(0,4)$ 的距离相等,求点 $M$ 的轨迹方程。
  \end{question}
\end{Practice}

\subsection{充要条件}
从\cref{subsec:curve_equation}我们知道,$y=ax^2$ 是一条抛物线的方程。
这就是说,如果点 $M$ 的坐标是方程 $y=ax^2$ 的解,那么点 $M$ 一定是这条抛物线上的点。

象这样,我们就说“点 $M$ 的坐标是方程 $y=ax^2$ 的解”是“点 $M$ 在这条抛物线上”的充分条件。

又如,如果一个三角形有两个角相等,那么这个三角形是等腰三角形。

同样,我们说“有两个角相等”是“三角形是等腰三角形”的充分条件。

一般地,如果 $A$ 成立,那么 $B$ 成立,即 $A \to B$,这时我们就说条件 $A$ 是 $B$ 成立的充分条件。
也就是说,为使 $B$ 成立,具备条件 $A$ 就足够了。

再举一些充分条件的例子:
\begin{enumerate}
  \item 如果不重合的两条直线 $l_1$、$l_2$ 的斜率 $k_1=k_2$,那么 $l_1\parallel l_2$。因此,$k_1=k_2$ 是 $l_1\parallel l_2$ 的充分条件。
  \item 如果 $x=y$,那么 $x^2=y^2$。因此,$x=y$ 是 $x^2=y^2$ 的充分条件。
  \item 如果两个三角形全等,那么这两个三角形面积相等。因此,两个三角形全等是两个三角形面积相等的充分条件。
\end{enumerate}

从\cref{subsec:curve_equation}我们还知道,如果点 $M$ 在方程 $y=ax^2$ 的曲线上,那么点 $M$ 的坐标一定是方程 $y=ax^2$ 的解。

象这样,我们就说“点 $M$ 的坐标是方程 $y = a{x}^{2}$ 的解”是“点 $M$ 在抛物线上”的必要条件。

又如,如果三角形是等腰的,那么它有两个角相等。

同样,我们说“有两个角相等”是“三角形是等腰三角形”的必要条件。

一般地,如果 $B$ 成立,那么 $A$ 成立,即 $B \to A$,或者,如果 $A$ 不成立,那么 $B$ 就不成立,这时我们就说,条件 $A$ 是 $B$ 成立的必要条件。
也就是说,要使 $B$ 成立,就必须 $A$ 成立。
因为“$B \to A$ ”和它的逆否命题“$\bar{A} \to \bar{B}$”是等价的,所以,如果 $A$ 不成立,那么 $B$ 就一定不成立,也就是说,要使 $B$ 成立,$A$ 就必须成立。

再举一些必要条件的例子:
\begin{enumerate}
  \item 如果两条有斜率的直线 $l_1\parallel l_2$,那么它们的斜率 $k_1=k_2$,也就是,如果 $k_1\neq k_2$,那么 $l_1$ 与 $l_2$ 不平行。
  因此,$k_1=k_2$ 是 $l_1\parallel l_2$ 的必要条件。
  \item 如果 $x=y$ ,那么 $x^2=y^2$。也就是,如果 $x^2\neq y^2$,那么 $x\neq y$。因此,$x^2=y^2$ 是 $x=y$ 的必要条件。
  \item 如果两个三角形全等,那么这两个三角形的面积相等,也就是,如果两个三角形的面积不相等,那么它们不能全等。因此,两个三角形面积相等是它们全等的必要条件。
\end{enumerate}

综上所述,我们看到,如果 $A \to B$,那么 $A$ 是 $B$ 成立的充分条件;如果 $B \to A$,那么 $A$ 是 $B$ 成立的必要条件。

有时既有 $A \to B$,又有 $B \to A$,那么 $A$ 既是 $B$ 成立的充分条件,又是 $B$ 成立的必要条件。
这时,我们就说 $A$ 是 $B$ 成立的\Concept{充分而且必要的条件},简称\Concept{充要条件}。

例如,如果 $f(x,y)=0$ 是曲线 $C$ 的方程,那么“点 $M$ 的坐标是方程 $f(x,y)=0$ 的解”就是“点 $M$ 在曲线 $C$ 上” 的充要条件;“有两个角相等”就是“三角形是等腰三角形”的充要条件;“两条有斜率的直线 $l_1$、$l_2$ 的斜率 $k_1=k_2$”就是“$l_1\parallel l_2$”的充要条件。

应该注意,对于某个结论来说,有的条件是充分条件,但不是必要条件;也有的条件是必要条件,但不是充分条件。

例如,“$x=y$”是“$x^2=y^2$”的充分条件,但不是必要条件。因为要使 $x^2=y^2$,不一定要有 $x=y$,有 $x=-y$ 也可以了。

又如,两个三角形面积相等是它们全等的必要条件,但不是充分条件。因为得出两个三角形全等,只有面积相等是不够的。

充要条件是进一步学习时常用的数学概念之一。
\begin{Practice}
  \begin{question}
    \item “$b=0$”是“直线 $y=kx+b$ 过原点”的什么条件,为什么?
    \item “四边相等”是“一个四边形是正方形”的什么条件,为什么?
    \item “$x-1=0$”是“$x^2-1=0$”的什么条件,为什么?
    \item “两条直线不相交”是“这两条直线异面”的什么条件,为什么?
  \end{question}
\end{Practice}

\subsection{曲线的交点}
由曲线方程的定义可知,两条曲线交点的坐标应该是两个曲线方程的公共实数解,即两个曲线方程组成的方程组的实数解;反过来,方程组有几个实数解,两条曲线就有几个交点,方程组没有实数解,两条曲线就没有交点。
即两条曲线有交点的充要条件是它们的方程所组成的方程组有实数解。
可见,求曲线的交点的问题,就是求由它们的方程所组成的方程组的实数解的问题。

\begin{example}
  求直线 $y=x+\dfrac{3}{2}$ 被曲线 $y=\dfrac{1}{2}x^2$ 截得的线段的长。
\end{example}
\begin{figure}
  \caption{}\label{fig:2-5}
\end{figure}
\begin{solution}
  先求交点。

  解方程组
  \[ \begin{cases} y=x+\dfrac{3}{2},\\y=\dfrac{1}{2}x^2, \end{cases}\]
  得
  \[ \begin{cases} x_1=-1,\\y_1=\frac{1}{2}; \end{cases} \quad \begin{cases} x_2=3,\\ y_2=\dfrac{9}{2}.\end{cases}\]
  所以交点 $A$、$B$ 的坐标分别是 $\left(-1,\dfrac{1}{2}\right)$、$\left(3,\dfrac{9}{2}\right)$。直线被曲线截得的线段长
  \[|AB|=\sqrt{(3+1)^2+\left(\frac{9}{2}-\frac{1}{2}\right)^2}=4\sqrt{2}.\]
\end{solution}

\begin{example}
  已知某圆的方程是 $x^2+y^2=2$。当 $b$ 为何值时,直线 $y=x+b$ 与圆有两个交点;两个交点重合为一点;没有交点?
\end{example}
\begin{solution}
  解方程组
  \begin{numcases}{}
    \label{eq:inter_line_equation} y=x+b,\\ \label{eq:circle_equation}x^2+y^2=2
  \end{numcases}
  把 \eqref{eq:inter_line_equation} 式代入 \eqref{eq:circle_equation} 式,得
  \begin{gather}
    x^2+(x+b)^2=2,\notag \\
    \label{eq:line_and_circle} 2x^2+2bx+b^2-2=0.
  \end{gather}
  \cref{eq:line_and_circle} 的根的判别式
  \[\begin{split} \Delta & = (2b)^2-4\times2(b^2-2) \\ &= 4(-b^2+4) \\ &=4(2+b)(2-b)\end{split}\]

  当 $-2<b<2$ 时,$\Delta>0$,这时方程组有两个不同的实数解,因此直线与圆有两个交点;

  当 $b=-2$ 或 $b=2$ 时,$\Delta=0$,这时方程组有两个相同的实数解,因此直线与圆的两个交点重合为一点;

  当 $b>2$ 或 $b<-2$ 时,$\Delta<0$,这时方程组没有实数解,因此直线与圆没有交点。
\end{solution}
\begin{figure}
  \caption{}\label{fig:2-6}
\end{figure}

实际上,上述三种情况,就是直线与圆相交、相切、相离(\cref{fig:2-6})。

\begin{Practice}
  求直线 $2x-5y+5=0$ 与曲线 $y=-\dfrac{10}{x}$ 的交点。
\end{Practice}

\begin{Exercise}
  \begin{question}
    \item 点 $A\,(1,-2)$、$B\,(2,-3)$、$C\,(3,10)$ 是否在方程
    \[ x^2-xy+2y+1=0\]
    的图形上?
    \item 解答:
    \begin{tasks}
      \task 在什么情况下,方程 $y=ax^2+bx+c$ 的曲线经过原点?
      \task 在什么情况下,方程 $(x-a)^2+(y-b)^2=r^2$ 的曲线经过原点?
    \end{tasks}
    \item 已知点 $M$ 到 $x$ 轴、$y$ 轴的距离的乘积等于 1。求点 $M$ 的轨迹方程。
    \item 点 $M$ 到点 $A\,(4,0)$ 和点 $B\,(-4,0)$ 的距离的和为 12,求点 $M$ 的轨迹方程。
    \item 一个点到点 $(4,0)$ 的距离等于它到 $y$ 轴的距离,求这个点的轨迹方程。
    \item 两个定点的距离为 6,点 $M$ 到这两个定点的距离的平方和为 26,求点 $M$ 的轨迹方程。
    \item 求与点 $O\,(0,0)$ 和 $A\,(c,0)$ 的距离的平方差为常数 $c$ 的点的轨迹方程。
    \item \label{exec:4-8}两根杆分别绕着定点 $A$ 和 $B$($AB=2a$)在平面内转动,并且转动时两杆保持相互垂直,求杆的交点 $P$ 的轨迹方程。
    \begin{figurehere}
      \begin{minipage}{\linewidth}\centering
        \caption*{(第 \ref{exec:4-8} 题)}
      \end{minipage}
    \end{figurehere}
    \item 在下列横线上填写:“充分条件”或“必要条件”或“充要条件”:
    \begin{tasks}
      \task “$m$ 是有理数”是“$m$ 是实数”的\CJKunderline[hidden]{\quad 充分条件\quad };
      \task “$x^2-1=0$”是“$x-1=0$”的\CJKunderline[hidden]{\quad 必要条件\quad };
      \task “$x=2$”是“$x^2-5x+6=0$”的\CJKunderline[hidden]{\quad 充分条件\quad };
      \task “$x<5$”是“$x<3$”的\CJKunderline[hidden]{\quad 充分条件\quad };
      \task “内错角相等”是“二直线平行”的\CJKunderline[hidden]{\quad 充要条件\quad };
      \task “ABCD是矩形”是“ABCD是平行四边形”的\CJKunderline[hidden]{\quad 充分条件\quad };
      \task “两边和夹角对应相等”是“三角形全等”的\CJKunderline[hidden]{\quad 充要条件\quad }。
    \end{tasks}
    \item 求直线 $4x-3y=20$ 和圆 $x^2+y^2=25$ 的交点。
    \item 求经过两条曲线 $x^2+y^2+3x-y=0$ 和 $3x^2+3y^2+2x+y=0$ 交点的直线的方程。
  \end{question}
\end{Exercise}

\section{圆}
\subsection{圆的标准方程}
我们知道,平面内与定点距离等于定长的点的集合(轨迹)是圆。定点就是圆心,定长就是半径。

根据圆的定义,我们来求圆心是 $C(a,b)$,半径是 $r$ 的圆的方程(\cref{fig:2-7})。
\begin{figure}
  \caption{}\label{fig:2-7}
\end{figure}

设 $M\,(x,y)$ 是圆上任意一点,根据定义,点 $M$ 到圆心 $C$ 的距离等于 $r$。圆就是集合
\[ P=\bigl\{ M \bigm\vert |MC|=r \bigr\}.\]

由两点的距离公式,点 $M$ 适合的条件可表示为
\begin{equation}
  \label{eq:circle_point_distance}
  \sqrt{(x-a)^2+(y-b)^2}=r.
\end{equation}

把\cref{eq:circle_point_distance} 两边平方,得
\begin{equation}
  \label{eq:standard_circle_equation}
  (x-a)^2+(y-b)^2=r^2.
\end{equation}

\cref{eq:standard_circle_equation} 就是圆心是 $C\,(a,b)$,半径是 $r$ 的圆的方程。
我们把它叫做\Concept{圆的标准方程}。 

如果圆心在坐标原点,这时 $a=0$,$b=0$,那么圆的方程就是
\[ x^2+y^2=r^2.\]

\begin{example}\end{example}
\begin{solution}\end{solution}

\begin{example}\end{example}
\begin{solution}\end{solution}

\begin{example}\end{example}
\begin{solution}\end{solution}

\begin{example}\end{example}
\begin{solution}\end{solution}

\begin{Practice}
  \begin{question}
    \item 
    \item 
    \item 
    \item 
    \item 
  \end{question}
\end{Practice}
\subsection{圆的一般方程}
\begin{Practice}
  \begin{question}
    \item 
    \item 
  \end{question}
\end{Practice}
\begin{Exercise}
  \begin{question}
    \item 
    \item 
    \item 
    \item 
    \item 
    \item 
    \item 
    \item 
    \item 
    \item 
    \item 
    \item 
    \item 
    \item 
    \item 
  \end{question}
\end{Exercise}
\section{椭圆}
\subsection{椭圆及其标准方程}
\begin{Practice}
  \begin{question}
    \item 
    \item 
    \item 
  \end{question}
\end{Practice}
\subsection{椭圆的几何性质}
\begin{Practice}
  \begin{question}
    \item 
    \item 
    \item 
  \end{question}
\end{Practice}
\begin{Exercise}
  \begin{question}
    \item 
    \item 
    \item 
    \item 
    \item 
    \item 
    \item 
    \item 
    \item 
    \item 
    \item 
    \item 
    \item 
    \item 
  \end{question}
\end{Exercise}
\section{双曲线}
\subsection{双曲线及其标准方程}
\begin{Practice}
  \begin{question}
    \item 
    \item 
  \end{question}
\end{Practice}
\subsection{双曲线的几何性质}
\begin{Practice}
  \begin{question}
    \item 
    \item 
    \item 
  \end{question}
\end{Practice}
\begin{Exercise}
  \begin{question}
    \item
    \item
    \item
    \item
    \item
    \item
    \item
    \item
    \item
    \item
    \item
    \item
    \item
    \item
    \item
    \item
    \item
    \item
  \end{question}
\end{Exercise}
\section{抛物线}
\subsection{抛物线及其标准方程}
\begin{Practice}
  \begin{question}
    \item 
    \item 
    \item 
  \end{question}
\end{Practice}
\subsection{抛物线的几何性质}
\begin{Practice}
  \begin{question}
    \item 
    \item 
    \item 
  \end{question}
\end{Practice}
\begin{Exercise}
  \begin{question}
    \item 
    \item 
    \item 
    \item 
    \item 
    \item 
    \item 
    \item 
    \item 
    \item 
    \item 
    \item 
    \item 
    \item 
  \end{question}
\end{Exercise}
\subsection{圆锥曲线的切线和法线}

\begin{Practice}
  \begin{question}
    \item 
    \item 
  \end{question}
\end{Practice}
\begin{Exercise}
  \begin{question}
    \item 
    \item 
    \item 
    \item 
    \item 
    \item 
    \item 
    \item 
    \item 
    \item 
  \end{question}
\end{Exercise}
\section*{小结}
\begin{enumerate}[C、,itemindent=4.5em]
  \item 本章在\cref{chp:line}直线方程的基础上,研究了直角坐标系中曲线和方程之间的一一对应关系,然后根据所求曲线的定义,得出了几种圆锥曲线的方程,并通过方程讨论了圆、椭圆、双曲线和抛物线的性质及应用。
  \item 曲线和方程的关系,反映了现实世界空间形式和数量关系之间的某种联系。我们把曲线看作适合某种条件 $p$ 的点 $M$ 的集合
  \[ P = \bigl\{ M \bigm\vert p(M) \bigr\}. \]

  在建立坐标系后,点集 $P$ 中任一元素 $M$ 都有一个有序数对 $(x,y)$ 和它对应,$(x,y)$ 是某个二元方程 $f(x,y)=0$ 的解,也就是说,它是解的集合
  \[ Q = \bigl\{ (x,y) \bigm\vert f(x,y)= 0 \bigr\}\]
  中的一个元素。
  反之,对于解集 $Q$ 中任一元素 $(x,y)$ 都有一点 $M$ 与它对应,点 $M$ 是点集 $P$ 中的一个元素。
  $P$ 和 $Q$ 的这种对应关系就是曲线和方程的关系。
  \item 根据圆的定义,求出了圆的标准方程,又由标准方程推出了圆的一般方程。
  圆的标准方程的优点,在于它明确地指出了圆心和半径,而圆的一般方程则突出了方程形式上的特点,它没有 $xy$ 项,并且 $x^2$、$y^2$ 项的系数相等。
  \item 由椭圆、双曲线、抛物线的几何条件求其标准方程,并通过分析标准方程研究这三种曲线的几何性质。三种曲线的标准方程(各取其中一种)和图形、性质如\cref{tab:2-1}:
  \begin{table}
    \caption{三种曲线的标准方程和图形、性质}\label{tab:2-1}
  \end{table}
  \item 圆、椭圆、双曲线、抛物线的统一性:
  \begin{enumerate}[(1)]
    \item 从方程的形式看: 在直角坐标系中,这几种曲线的方程都是二元二次的,所以把它们称为二次曲线。
    \item 除圆以外,从点的集合(或轨迹)的观点来看: 它们都是与定点和定直线距离的比是常数 $e$ 的点的集合(或轨迹),这个定点是它们的焦点,定直线是它们的准线。只是由于离心率 $e$ 的不同,而分为椭圆、双曲线和抛物线三种曲线。
    \item 从天体运行的轨道看: 天体运动的轨道是这四种曲线,例如,人造卫星、行星、彗星等由于运动的速度的不同,它们的轨道是圆、椭圆、抛物线或双曲线(\cref{fig:2-37})。
    \item 四种曲线又可以看作不同的平面截圆锥面所得到的截线,如\cref{fig:2-38}。因此,它们又统称圆锥曲线。
  \end{enumerate}
  \begin{figure}
    \begin{minipage}[b]{0.48\linewidth}\centering
      \caption{}\label{fig:2-37}
    \end{minipage}
    \begin{minipage}[b]{0.48\linewidth}\centering
      \caption{}\label{fig:2-38}
    \end{minipage}
  \end{figure}
  \item 圆锥曲线的切线定义和切线方程的求法,都是初等的。要注意这个方法不完全适用于一般曲线。一般曲线的切线定义和切线方程的求法,将在高中三年级的微积分课程中学习。
\end{enumerate}
\chapter*{复习参考题\chinese{chapter}}
\section*{A 组}
\begin{question}
  \item 
  \item 
  \item 
  \item 
  \item 
  \item 
  \item 
  \item 
  \item 
  \item 
  \item 
  \item 
  \item 
  \item 
  \item 
  \item 
  \item 
  \item 
  \item 
  \item 
  \item 
  \item 
\end{question}
\section*{B 组}
\begin{question}[resume]
  \item 
  \item 
  \item 
  \item 
  \item 
  \item 
  \item 
\end{question} % ok
\chapter{坐标变换}
\section{平移和旋转}
\subsection{坐标轴的平移}
点的坐标和曲线的方程是对一定的坐标系来说的。
例如,\cref{fig:3-1} 中 $\odot O'$ 的圆心 $O'$,在坐标系 $xOy$ 中的坐标是 $(3,2)$,$\odot O'$ 的方程是 $(x-3)^2+(y-2)^2=5^2$;如果取坐标系 $x'O'y'$($O'x'\parallel Ox,\ O'y'\parallel Oy$),那么在这个坐标系中,它们就分别变成 $(0,0)$ 和 $x^2+y^2=5^2$。
\begin{figure}
  \caption{}\label{fig:3-1}
\end{figure}

这就是说,对于同一点或者同一曲线,由于选取的坐标系不同,点的坐标或曲线的方程也不同。
从上面例子我们看出,把一个坐标系变换为另一个适当的坐标系,可以使曲线的方程简化,便于我们研究曲线的性质。

坐标轴的方向和长度单位都不改变,只改变原点的位置,这种坐标系的变换叫做\Concept{坐标轴的平移}。
简称\Concept{移轴}。

下面研究在平移情况下,同一个点在两个不同的坐标系中坐标之间的关系。

设 $O'$ 在原坐标系 $xOy$ 中的坐标为 $(h,k)$ ,以 $O'$ 为原点平移坐标轴,建立新坐标系 $x'O'y'$。
平面内 任意一点 $M$ 在原坐标系中的坐标为 $(x,y)$,在新坐标系中的坐标为 $(x',y')$,点 $M$ 到 $x$ 轴、$y$ 轴的垂线的垂足分别是 $M_1$、$M_2$。从\cref{fig:3-2} 可以看出,
\begin{figure}
  \caption{}\label{fig:3-2}
\end{figure}


\begin{Practice}
  \begin{question}
    \item 
    \item 
  \end{question}
\end{Practice}
\subsection{利用坐标轴的平移化简二元二次方程}
\begin{Practice}
  \begin{question}
    \item 
    \item 
  \end{question}
\end{Practice}
\begin{Exercise}
  \begin{question}
    \item 
    \item 
    \item 
    \item 
    \item 
  \end{question}
\end{Exercise}
\subsection{坐标轴的旋转}
\begin{Practice}
  \begin{question}
    \item 
    \item 
    \item 
  \end{question}
\end{Practice}
\subsection{利用坐标轴的旋转化简二元二次方程}
\begin{Practice}
  化简下列方程,并画图形:
  \begin{tasks}
    \task $x^2-2xy+y^2=12$;
    \task $4x^2+8xy-2y^2-7=0$。
  \end{tasks}
\end{Practice}
\begin{Exercise}
  \begin{question}
    \item 
    \item 
    \item 
    \item 
    \item 
    \item 
    \item 
  \end{question}
\end{Exercise}

\section{一般二元二次方程的讨论}
\subsection{化一般二元二次方程为标准式}
\begin{Practice}
  \begin{question}
    \item 化简方程 $3x^2-10xy+3y^2+26x-22y+35=0$。
    \item 化简方程 $4xy-3x^2+4=0$,画出它的图形。
  \end{question}
\end{Practice}
\subsection{一般二元二次方程的讨论}
\begin{Practice}
  判别下列方程的类型:
  \begin{question}
    \item $3x^2-7xy+5y^2+x-3y-3=0$;
    \item $5x^2+12xy+5y^2-18x-18y+9=0$;
    \item $2x^2+2xy+y^2+2x+2y-4=0$;
    \item $x^2+2xy+y^2+2x+2y-4=0$。
  \end{question}
\end{Practice}
\begin{Exercise}
  \begin{question}
    \item 
    \item 
    \item 
    \item 
    \item 
    \item 
  \end{question}
\end{Exercise}
\section*{小结}
\begin{enumerate}[C、,itemindent=4.5em]
  \item 
  \item 
  \item 
\end{enumerate}
\chapter*{复习参考题\chinese{chapter}}
\section*{A 组}
\begin{question}
  \item 
  \item 
  \item 
  \item 
  \item 
  \item 
  \item 
  \item 
  \item 
  \item 
  \item 
  \item 
\end{question}
\section*{B 组}
\begin{question}
  \item 
  \item 
  \item 
  \item 
  \item 
\end{question} % ok
\chapter{参数方程、极坐标}
\section{参数方程}
\subsection{曲线的参数方程}
\begin{Practice}
  \begin{question}
    \item 
    \item 
  \end{question}
\end{Practice}
\subsection{参数方程和普通方程的互化}
\begin{Practice}
  \begin{question}
    \item 
    \item 
  \end{question}
\end{Practice}
\subsection{圆的渐开线、摆线}

\subsubsection{圆的渐开线}
\begin{Practice}
  \begin{question}
    \item 
    \item 
  \end{question}
\end{Practice}
\subsubsection{摆线}
\begin{Practice}
  \begin{question}
    \item 
    \item 
  \end{question}
\end{Practice}
\begin{Exercise}
  \begin{question}
    \item 
    \item 
    \item 
    \item 
    \item 
    \item 
    \item 
    \item 
    \item 
    \item 
  \end{question}
\end{Exercise}

\section{极坐标}
\subsection{极坐标系}
\begin{Practice}
  \begin{question}
    \item 
    \item 
    \item 
  \end{question}
\end{Practice}
\subsection{曲线的极坐标方程}
\subsubsection{曲线的极坐标方程}
\begin{Practice}
  \begin{question}
    \item 
    \item 
    \item 
  \end{question}
\end{Practice}
\subsubsection{三种圆锥曲线的统一的极坐标方程}
\begin{Practice}
  \begin{question}
    \item 
    \item 
  \end{question}
\end{Practice}
\subsection{极坐标方程和直角坐标的互化}
\begin{Practice}
  \begin{question}
    \item 
    \item 
    \item 
    \item 
  \end{question}
\end{Practice}

\subsection{等速螺线}
在机械传动中,常常需要把旋转运动变成直线运动。
\cref{fig:4-20} 中的凸轮装置就是借助凸轮绕定轴旋转推动从动杆作上、下往复直线运动。
在设计中,根据对从动杆运动的要求不同,需要设计不同的凸轮轮廓线。
如果需要从动杆作等速运动,凸轮的轮廓线就要用等速螺线。
\begin{figure}
  \caption{}\label{fig:4-20}
\end{figure}

什么是等速螺线呢? 如\cref{fig:4-21},从点 $O$ 出发的射线 $l$,绕点 $O$ 作等角速度的转动,同时点 $M$ 沿 $l$ 作等速直线运动,点 $M$ 的轨迹叫做\Concept{等速螺线}或\Concept{阿基米德螺线}。
\begin{figure}
  \caption{}\label{fig:4-21}
\end{figure}

下面,我们来求等速螺线的极坐标方程。

取点 $O$ 为极点,以 $l$ 的初始位置为极轴,建立极坐标系(\cref{fig:4-21})。

设 $M_0\,(\rho_0,0)$ 是点 $M$ 的初始位置,$M$ 在 $l$ 上运动的速度为 $v$,$l$ 绕点 $O$ 转动的角速度为 $\omega$,经过时间 $t$ 后,$l$ 旋转了 $\theta$ 角,点 $M$ 到达位置 $(\rho,\theta)$。根据等速螺线的定义,得
\[ \rho-\rho_0=vt,\quad \theta=\omega t.\]

这是以时间 $t$ 为参数的极坐标参数方程。消去参数 $t$ ,得
\[ \rho-\rho_0=\frac{v}{\omega}\theta. \]
这就是所求的等速螺线的极坐标方程。

设 $\dfrac{v}{\omega}=a$($a\neq 0$),得
\[\rho=\rho_0+a\theta.\]
这是等速螺线的极坐标方程的一般形式,$\rho$ 是 $\theta$ 的一次函数。

在特殊情况下,当 $\rho_0=0$ 时,等速螺线的方程变为
\[\rho=a\theta.\]
这时方程中的 $\rho$ 和 $\theta$ 成正比例。

\begin{example}
  画出等速螺线 $\rho=a\theta$($a>0$)的图形。
\end{example}
\begin{solution}
和直角坐标系中的画图步骤一样,给出 $\theta$ 的一系列允许值,算出 $\rho$ 的对应值,再根据对应值表,描点画图。
\begin{table}
  \begin{tblr}{colspec={*{11}{X[c]}},vline{2}={0.8pt},rowsep=5pt}
    $\theta$ & 0 & $\dfrac{\uppi}{4}$  & $\dfrac{\uppi}{2}$ & $\dfrac{3\uppi}{4}$ & $\uppi$ & $\dfrac{5\uppi}{4}$ & $\dfrac{3\uppi}{2}$ & $\dfrac{7\uppi}{4}$ & $2\uppi$ & $\cdots$\\
    $\rho$ & 0 & $\dfrac{\uppi}{4}a$  & $\dfrac{\uppi}{2}a$& $\dfrac{3\uppi}{4}a$ & $\uppi a$ & $\dfrac{5\uppi}{4}a$ & $\dfrac{3\uppi}{2}a$ & $\dfrac{7\uppi}{4}a$ & $2\uppi a$ & $\cdots$\\
  \end{tblr}
\end{table}

在 $\theta=\dfrac{\uppi}{4}$ 的射线上,截取 $|OA|=\dfrac{\uppi}{4}a$,得到点 $A$;在 $\theta=\dfrac{\uppi}{2}$ 的射线上,截取 $|OB|=\dfrac{\uppi}{2}a=2\cdot \dfrac{\uppi}{4}a$,得到点 $B$;用同样方法可得到点 $C$、$D$、……将这些点连成平滑的曲线,就是 $\rho={a\theta }$ 的图形(\cref{fig:4-22})。

如果 $\rho$ 允许取负值,当 $\rho$、$\theta$ 是方程 $\rho=a\theta$ 的解时,$-\rho$、$-\theta$ 也是方程的解。
因为以 $(\rho,\theta)$ 和 $(-\rho,-\theta)$ 为坐标的点,关于过极点垂直于极轴的直线对称,所以 $\rho=a\theta$ 的图形也关于该直线对称。
\cref{fig:4-22} 中的实线表示 $\rho$、$\theta$ 取正值时的螺线部分,虚线表示 $\rho$、$\theta$ 取负值时的螺线部分。
\end{solution}
\begin{figure}
  \begin{minipage}[b]{0.48\linewidth}\centering
    \caption{}\label{fig:4-22}
  \end{minipage}
  \begin{minipage}[b]{0.48\linewidth}\centering
    \caption{}\label{fig:4-23}
  \end{minipage}
\end{figure}
\begin{example}
  \label{exp:cam}由于某种需要,设计一个凸轮,轮廓线如\cref{fig:4-23}。要求如下:

  凸轮依顺时针方向绕点 $O$ 转动,开始时从动杆接触点为 $A$,$|OA|=\qty{4}{cm}$。
  \begin{enumerate}
    \item 当从动杆接触轮廓线 $ABC$ 时,它被推向右方作等速直线运动。
    凸轮旋转角度 $\dfrac{11}{8}\uppi$ 时,有最大推程 \qty{14}{cm} 即 $|OC|=\qty{18}{cm}$;
    \item 当从动杆接触轮廓线 $CDA$ 时,它向左等速退回原位。
  \end{enumerate}
  求曲线 $ABC$ 及曲线 $CDA$ 的方程。
\end{example}

\begin{solution}
  
\end{solution}

\begin{Practice}
  \begin{question}
    \item 如果 $M_1\,(\rho_1,\theta_1)$、$M_2\,(\rho_2,\theta_2)$ 是等速螺线上的两点,那么 $\rho_2-\rho_1$ 与 $\theta_2-\theta_1$ 成正比例。
    \item\label{prac:9-3} 某自动机床上有一个凸轮,它的轮廓线 $ACB$ 是一段等速螺线(如图),$A$ 点到旋转中心 $O$ 的距离 $\rho_0= \qty{60}{mm}$,轴心角 $\angle AOB= \ang{30}$,工作时曲线 $ACB$ 能把从动杆推出 \qty{5}{mm}。求这段等速螺线的极坐标方程。
    \begin{figurehere}
      \begin{minipage}{\linewidth}\centering
        \caption*{(第 \ref{prac:9-3} 题)}
      \end{minipage}
    \end{figurehere}
    \item 当 $0\geqslant\theta\geqslant-\dfrac{5\uppi}{8}$ 时,求本节\cref{exp:cam} 中的曲线 $ADC$ 的方程。
  \end{question}
\end{Practice}

\begin{Exercise}
  \begin{question}
    \item 已知 $A\,(\rho,\theta)$、$B\,(\rho,-\theta)$、$C\,(-\rho,-\theta)$、$D\,(-\rho,\theta)$,点 $A$ 和 $B$、$C$、$D$ 分别有怎样的相互位置关系?
    \item 说明下列极坐标方程表示什么曲线,并画图。
    \begin{tasks}(2)
      \task $\rho=3$;
      \task $\theta=\dfrac{\uppi}{3}$。
    \end{tasks}
    \item 求下列各图形的极坐标方程:
    \begin{tasks}
      \task 经过点 $A\,(3,\dfrac{\uppi}{3})$,平行于极轴的直线;
      \task 经过点 $A\,(-2,\dfrac{\uppi}{4})$,垂直于极轴的直线;
      \task 圆心在点 $A\,(5,\uppi)$,半径等于 5 的圆;
      \task 经过点 $A\,(a,0)$ 和极轴相交成 $\alpha$ 角的直线。
    \end{tasks}
    \item 画出下列极坐标方程的图形:
    \begin{tasks}(2)
      \task $\rho\cos\theta=2$;
      \task $\rho=6\cos\theta$;
      \task $\rho=10\sin\theta$;
      \task $\rho=10(1+\cos\theta)$。
    \end{tasks}
    \item 从极点作圆 $\rho=2a\cos\theta$ 的弦,求各个弦的中点的轨迹方程。
    \item 从极点 $O$ 作直线和直线 $\rho\cos\theta=4$ 相交于点 $M$,在 $OM$ 上取一点 $P$,使 $OM\cdot OP=12$,求 $P$ 点的轨迹的方程,并且说明轨迹是什么曲线。
    \item 一颗彗星的轨道是抛物线,太阳位于这条抛物线的焦点上。已知这颗彗星距太阳 \qty{1.6e8}{km} 时,极径和轨道的轴成 $\dfrac{\uppi}{3}$ 角。求这颗彗星轨道的极坐标方程,并且求它的近日点离太阳的距离。
    \item 把下列直角坐标方程化成极坐标方程:
    \begin{tasks}(2)
      \task $x^2+y^2=16$;
      \task $xy=a$;
      \task $x^2+y^2+2y=0$;
      \task $x^2-y^2=a^2$。
    \end{tasks}
    \item 把下列极坐标方程化成直角坐标方程:
    \begin{tasks}(2)
      \task $\rho=5\tan\theta$;
      \task $\rho+6\cot\theta\cdot\csc\theta=0$;
      \task $\rho=\dfrac{5}{\cos\theta}$;
      \task $\rho=\dfrac{6}{1-2\cos\theta}$;
      \task $\rho(2\cos\theta-5\sin\theta)-3=0$。
    \end{tasks}
    \item 已知一个圆的方程是 $\rho=5\sqrt{3}\cos\theta-5\sin\theta$,求圆心和半径。
    \item 长为 $2a$ 的线段,其端点在两个直角坐标轴上滑动,从原点作这条线段的垂线,垂足为 $M$,求点 $M$ 的轨迹的极坐标方程($Ox$ 为极轴),再化为直角坐标方程。
    \item\label{exec:4-14-12} 一凸轮如图所示,当它按箭头方向等速转动时,要求:
    \begin{tasks}
      \task 从动杆接触 $ABC$ 时,从动杆不动;
      \task 从动杆接触 $CDE$ 时,从动杆等速向右移动。
    \end{tasks}
    试按图中尺寸写出该凸轮轮廓线 $ABC$ 和 $CDE$ 的极坐标方程。
    \begin{figurehere}
      \begin{minipage}{\linewidth}\centering
        \caption*{(第 \ref{exec:4-14-12} 题)}
      \end{minipage}
    \end{figurehere}
  \end{question}
\end{Exercise}

\section*{小结}
\begin{enumerate}[C、,itemindent=4.5em]
  \item 本章的主要内容是曲线的参数方程、极坐标系和曲线的极坐标方程,以及应用的初步知识。要注意,不仅在直角坐标系里可建立曲线的参数方程,在极坐标系里同样可以建立曲线的参数方程,不过这时是通过某个参数来表示 $\rho$ 和 $\theta$ 的关系。
  \item 在实际问题中,当我们求轨迹方程时,有时很难或不能找到曲线上点的坐标之间的直接关系。如果引进适当的参数,问题往往比较容易解决。研究运动的物体的轨迹时,常用时间作参数;研究旋转的物体的轨迹时,常用转角作参数。
  \item 化参数方程为普通方程的关键在于消去参数。反之,选择适当的参数也可以将普通方程化为参数方程。
  \item 和直角坐标系一样,极坐标系也是常用的一种坐标系。利用极坐标方程表示一些环绕一点作旋转运动的点的轨迹,比较方便。
  \item 极坐标和直角坐标可以互化。当把直角坐标系的原点作为极点,$x$ 轴的正半轴作为极轴时,点 $M$ 的直角坐标 $(x,y)$ 和极坐标 $(\rho,\theta)$ 有下面的关系:
  \[\begin{cases} x=\rho\cos\theta,\\y=\rho\sin\theta; \end{cases} \quad \begin{cases} \rho^2=x^2+y^2,\\\tan\theta=\dfrac{y}{x}\,(x\neq 0). \end{cases} \]
\end{enumerate}
\chapter*{复习参考题\chinese{chapter}}
\section*{A 组}
\begin{question}
  \item 叙述曲线和曲线的参数方程
  \[ \begin{cases} x=\varphi(t),\\y=\psi(t)  \end{cases}\]
  之间的对应关系。
  \item 设 $t$ 和 $\theta$ 是参数,化下列各参数方程为普通方程,并且画出它们的图形:
  \begin{tasks}(2)
    \task $\begin{cases} x=t^2-2t,\\ y=t^2+2;\end{cases}$
    \task $\begin{cases} x=5\cos\theta+2,\\ y=2\sin\theta-3.\end{cases}$
  \end{tasks}
  \item 把下列各方程按照所给条件化成参数方程 ($t$、$\theta$ 是参数):
  \begin{tasks}
    \task $x^2+2xy+y^2+2x-2y=0, \quad x=t-t^2$;
    \task $17x^2-16xy+4y^2-34x+16y+13=0, \quad x=1+2\cos\theta$。
  \end{tasks}
  \item 用描点法画出下列参数方程所表示的图形:
  \begin{tasks}(2)
    \task $\begin{cases} x=3t-5,\\ y=t^3-t; \end{cases}$
    \task $\begin{cases} x=5\cos\phi,\\ y=3\sin\phi; \end{cases}$
    \task $\begin{cases} x=\cos^3t,\\ y=\sin^3t; \end{cases}$
    \task $\begin{cases} x=t-\sin t,\\ y=1-\cos t. \end{cases}$
  \end{tasks}
  \item 已知弹道曲线的参数方程为
  \[ \begin{cases} x=v_0t\cos\alpha,\\ y=v_0t\sin\alpha-\dfrac{1}{2}gt^2,\end{cases} \]
  \begin{tasks}
    \task 求炮弹从发射到落回地面所需的时间;
    \task 求炮弹到达的最大高度。
  \end{tasks}
  \item 解答:
  \begin{enumerate}[itemindent=2em]
    \item 在 $\rho=\dfrac{3}{\cos\theta}$ 的图形上,求有下列极角的各点的坐标:
    \begin{tasks}(4)
      \task $\dfrac{\uppi}{3}$;
      \task $-\dfrac{\uppi}{3}$;
      \task $0$;
      \task $\dfrac{\uppi}{6}$。
    \end{tasks}
    \item 在 $\rho=\dfrac{1}{\sin\theta}$ 的图形上,求有下列极径的各点的坐标:
    \begin{tasks}(3)
      \task $1$;
      \task $2$;
      \task $\sqrt{2}$。
    \end{tasks}
  \end{enumerate}
  \item 求下列曲线的交点坐标,并画图:
  \begin{tasks}(2)
    \task $ \rho=4\sin\theta,\quad \rho=2$;
    \task $ \rho=\dfrac{3}{2-\cos\theta},\quad \rho=2$。
  \end{tasks}
  \item 说明下列方程表示什么曲线,并且画图:
  \begin{tasks}(2)
    \task $\rho=\dfrac{5}{1-\cos\theta}$;
    \task $\rho=\dfrac{5}{3-4\cos\theta}$;
    \task $\rho=\dfrac{1}{2-\cos\theta}$。
  \end{tasks}
  \item 求适合下列条件的点的轨迹的极坐标方程,并且画出它们的图形:
  \begin{tasks}
    \task 极径和极角成正比例;
    \task 极径和极角成反比例。
  \end{tasks}
  \item $O$ 是极点,$Ox$ 是极轴,点 $M$ 的坐标是 $(\rho,\theta)$,把 $Ox$ 绕点 $O$ 旋转角度 $\alpha$ 以后,$Ox$ 转到 $Ox'$ 的位置,对于新的极坐标系,点 $M$ 的坐标是 $(\rho',\theta')$,求点 $M$ 的新旧坐标之间的关系。
  \item 说明下列两条直线的位置关系:
  \begin{tasks}(2)
    \task $\theta=\alpha$ 和 $\rho\cos(\theta-\alpha)=\alpha$;
    \task $\theta=\alpha$ 和 $\rho\sin(\theta-\alpha)=a$。
  \end{tasks}
  \item 把下列各直角坐标方程化成极坐标方程:
  \begin{tasks}(2)
    \task $\left(x^2+y^2\right)^2=a^2\left(x^2-y^2\right)$;
    \task $x\cos\alpha+y\sin\alpha-p=0$;
    \task $x^2=2p\left(y+\dfrac{p}{2}\right)$。
  \end{tasks}
  \item 把下列极坐标方程化成直角坐标方程:
  \begin{tasks}(2)
    \task $\rho=64\sin^2\theta$;
    \task $\rho=-4\sin\theta+\cos\theta$;
    \task $\rho\cos\left(\theta-\dfrac{\uppi}{3}\right)=1$。
  \end{tasks}
  \item 等速螺线过点 $O\,(0,0)$ ,极角每增加 $\dfrac{\uppi}{3}$ 弧度,极径就增加 1.5。
  \begin{tasks}
    \task 求螺线的极坐标方程;
    \task 算出 $\theta=0,\ \dfrac{\uppi}{3},\ \dfrac{2\uppi}{3},\ \uppi,\ \dfrac{4\uppi}{3},\ \dfrac{5\uppi}{3},\ 2\uppi$ 所对应的 $\rho$ 的值;
    \task 描出上述各点,并作出 $0\leqslant \theta \leqslant 2\uppi$ 范围内螺线的图形。
  \end{tasks}
\end{question}
\section*{B 组}
\begin{question}[resume]
  \item \label{exec:4t-15} 如图,$OB$ 是机器上的曲柄,长是 $r$,绕点 $O$ 转动,$AB$ 是连杆,$M$ 是 $AB$ 上一点,$MA=a$,$MB=b$。当点 $A$ 在 $Ox$ 上作往返运动,点 $B$ 绕着点 $O$ 作圆运动时,求点 $M$ 的轨迹的参数方程。
  \begin{figure}
    \caption*{(第 \ref{exec:4t-15} 题)}
  \end{figure}
  \item \label{exec:4t-16}根据双曲线 $\dfrac{x^2}{a^2}-\dfrac{y^2}{b^2}=1$ 的图中所给出的 $\phi$,说明它的参数方程是
  \[\begin{cases} x=a\sec\phi,\\ y=b\tan\phi.\end{cases} \]
  并研究根据不同的 $\phi$,如何作出双曲线的一些点来画双曲线(图中的大圆半径是 $a$,小圆半径是 $b$)。
  \begin{figure}
    \caption*{(第 \ref{exec:4t-16} 题)}
  \end{figure}
  \item 从圆周上定点 $O$ 引直线 $OS$ ,交圆于 $Q$,在 $OS$ 上取点 $P$,使 $| QP|=b$(常数)。当 $OS$ 绕 $O$ 旋转时;点 $P$ 的轨迹称为帕斯卡蚶线。求它的极坐标方程。
\end{question}

\chapter*{总复习参考题}
\section*{A 组}
\begin{question}
  \item 已知直线 $3x+4y-10+\lambda(4x-6y+7)=0$ 通过点 $A\,(4,7)$,求 $\lambda$ 的值。
  \item 已知点 $P\,(2,0)$、$Q\,(8,0)$。点 $M$ 到点 $P$ 的距离是它到点 $Q$ 的距离的 $\dfrac{1}{5}$ ,求点 $M$ 的轨迹方程。
  \item 点 $M\,(x,y)$ 到两个定点 $M_1$、$M_2$ 距离的比是一个正数 $m$,求点 $M$ 的轨迹方程,并说明轨迹是什么图形(考虑 $m = 1$ 和 $m \neq 1$ 两种情形)。
  \item 求证:以 $A\,(4,1)$、$B\,(1,5)$、$C\,(-3,2)$、$D\,(0,-2)$ 为顶点的四边形是正方形。
  \item 求证:以 $A\,(-4,-2)$、$B\,(2,0)$、$C\,(8,6)$、$D\,(2,4)$ 为顶点的四边形是平行四边形。并求它两边的夹角。
  \item 已知: 四边形一组对边的平方和等于另一组对边的平方和。求证: 两条对角线互相垂直。
  \item 把函数 $y=f(x)$ 在 $x=a$ 及 $x=b$ 之间的一段图象近似地看作直线,设 $a\leqslant c\leqslant b$ ,证明 $f(c)$ 的近似值是
  \[ f(a)+\frac{c-a}{b-a}[f(b)-f(a)].\]
  \item 求证:过点 $P(a\cos^3\alpha,a\sin^3\alpha)$ 且与直线 $x\sec\alpha+y\csc\alpha=a$ 垂直的直线方程是 $x\cos\alpha-y\sin\alpha=a\cos2\alpha$。
  \item 判定两圆 $x^2+y^2-6x+4y+12=0$,$x^2+y^2-14x-2y+14=0$ 是否相切。
  \item 已知椭圆的方程是 $\frac{x^2}{16}+\frac{y^2}{9}=1$。求椭圆内接正方形的面积。
  \item 证明: 等轴双曲线上任意一点到中心的距离是它到两个焦点的距离的比例中项。
  \item 设抛物线的轴和它的准线相交于点 $A$,经过焦点垂直于轴的直线交抛物线于 $B$、$C$ 两点。求证:$BA\perp CA$。
  \item 把点 $A\,(a,b)$ 的坐标变成下列新坐标,求坐标轴旋转角的正弦或余弦值:
  \begin{tasks}(2)
    \task $(-a,-b)$;
    \task $(-a,b)$;
    \task $(a,-b)$;
    \task $(b,a)$。
  \end{tasks}
  \item 求曲线 $y^2=4-2x$ 上距离原点最近的点 $P$ 的坐标。
  \item 证明:$\sqrt{x}+\sqrt{y}=\sqrt{a}$($a>0$)是一段抛物线。求出它的顶点坐标和焦点坐标。
  \item 在直角坐标系中化简方程
  \[(1-e^2)x^2+y^2-2e^2px-e^2p^2=0\]
  当
  \begin{tasks}(3)
    \task $e<1$;
    \task $e>1$;
    \task $e=1$
  \end{tasks}
  时,求出它在原坐标系中的焦点坐标和准线方程。
  \item 把圆锥曲线的极坐标方程 $\rho=\dfrac{p}{2-\cos\theta}$ 化成直角坐标方程,再移轴化成标准方程。
  \item 定点 $M$ 的极坐标为 $(10,\ang{30})$,已知极点 $O'$ 在直角坐标系 $xOy$ 中的坐标为 $(2,3)$,极轴平行于 $x$ 轴,且极轴的正方向与 $x$ 轴的正方向相同,两个坐标系的长度单位也相同,求点 $M$ 的直角坐标。
  \item\label{exec:tt-19} 如图,设基圆半径为 $r$,渐开线的起点为 $A$,取圆心 $O$ 为极点,射线 $OA$ 为极轴。$M\,(\rho,\theta)$ 为渐开线上任一点,过 $M$ 作基圆的切线 $MB$,$B$ 是切点。设 $\angle BOM=\alpha$。试用 $\alpha$ 做参数,写出渐开线在极坐标系中的参数方程。
  \begin{figure}
    \caption*{(第 \ref{exec:tt-19} 题)}
  \end{figure}
  \item 从极点 $O$ 引一条直线和圆 $\rho^2-2a\rho\cos\theta+ a^2-r^2=0$ 相交于一点 $Q$,点 $P$ 分线段 $\overline{OQ}$ 成比 $m:n$,求点 $Q$ 在圆上移动时,点 $P$ 的轨迹方程,并画出图形。
\end{question}
\section*{B 组}
\begin{question}[resume]
  \item 某生产大队科学试验小组,为了提高玉米的产量,在试验田进行追施硝氨肥的试验,得到如下数据:
  \par\noindent%
  \begin{tablehere}%
  \begin{tblr}{colspec={c*{8}{X[r]}},row{1}={m,c},vline{2}={0.8pt}}
    追肥量 $x$(\unit{kg})& 1 & 2 & 3 & 4 & 5 & 6 & 7 & 8 \\
      产量 $y$(\unit{kg})& 120 & 123 & 125 & 128.5 & 131 & 133.5 & 136 & 139 \\
  \end{tblr}
  \end{tablehere}
  用平均值法求这范围内的 $y$ 与 $x$ 之间的经验公式。
  \item \label{exec:tt-22}图中表示凸模的一部分外形曲线,线段 $OQ$、$MP$ 分别和圆弧 $\overparen{QP}$ 切于 $Q$、$P$ 两点,建立如图所示的坐标系,试根据图中数据,求圆心 $O_1$ 的坐标和 $\alpha$ 角的大小。
  \begin{figure}
    \caption*{(第 \ref{exec:tt-22} 题)}
  \end{figure}
  \item 证明: $(A_1-C_1)B_2=(A_2-C_2)B_1\neq 0$ 时,二次曲线
  \begin{gather*}
    A_1x^2+B_1xy+C_1y^2+D_1x+E_1y+F_1=0,\\
    A_2x^2+B_2xy+C_2y^2+D_2x+E_2y+F_2=0
  \end{gather*}
  的交点在同一个圆上。
  \item 已知方程 $16x^2+ky^2=16k$,讨论当 $k$ 取不同数值时,它表示什么曲线?
  \item 直线 $y=x+b$ 与抛物线 $y=x^2-3x+5$ 相交于两点,求这两点连线的中点的轨迹方程。
  \item \label{exec:tt-26}一个半径是 $4r$ 的定圆 $O$ 和一个半径是 $r$ 的动圆 $C$ 相内切。当圆 $C$ 滚动时,求证圆 $C$ 上定点 $M$ (开始时在 $A$ 点)的轨迹的方程是
  \[\begin{cases} x=r(3\cos\phi+cos3\phi),\\y=r(3\sin\phi-\sin3\phi).  \end{cases}\]
  证明这个方程可以化成 
  \[ \begin{cases} x=4r\cos^3\phi,\\y=4r\sin^3\phi,\end{cases} \] 
  并且求出它的普通方程。
  \item \label{exec:tt-27}如图,$OA$ 是定圆的直径,它的长是 $2a$,直线 $OB$ 和圆相交于点 $M_1$,和经过点 $A$ 的切线相交于点 $B$。$MM_1\perp OA$,$MB\parallel OA$,$MM_1$ 与 $MB$ 相交于点 $M$。以点 $O$ 为原点,$OA$ 方向为 $x$ 轴的正向,求点 $M$ 的轨迹的参数方程(以 $\dot{\theta}$ 为参数)。
  \begin{figure}
    \begin{minipage}[b]{0.48\linewidth}\centering
      \caption*{(第 \ref{exec:tt-26} 题)}
    \end{minipage}
    \begin{minipage}[b]{0.48\linewidth}\centering
      \caption*{(第 \ref{exec:tt-27} 题)}
    \end{minipage}
  \end{figure}
\end{question}
\end{document}