\chapter{引言}
我们在平面几何和立体几何里,所用的研究方法是以公理为基础,直接依据图形的点、线、面的关系来研究图形的性质。
在将要学习的平面解析几何里,所用的研究方法和平面几何、立体几何不同,它是在坐标系的基础上,用坐标表示点,用方程表示曲线(包括直线),通过研究方程的特征间接地来研究曲线的性质。
因此可以说,解析几何是用代数方法来研究几何问题的一门数学学科。

平面解析几何研究的主要问题是:
\begin{enumerate}
  \item 根据已知条件,求出表示平面曲线的方程;
  \item 通过方程,研究平面曲线的性质。
\end{enumerate}

解析几何的这种研究方法,在进一步学习数学、物理和其他科学技术中经常使用。

在十七世纪,法国数学家笛卡尔创始了解析几何。
解析几何的产生对数学发展,特别是对微积分的出现起了促进作用; 恩格斯对笛卡尔的这一发现给予了高度的评价。