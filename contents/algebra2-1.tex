\chapter{反三角函数和简单三角方程}
\section{反三角函数}
\subsection{反正弦函数}
我们已经学习了正弦函数 $y=\sin x$ 和它的图象(\cref{fig:1-1})。从图象可以看到,对于 $x$ 在定义域 $(-\infty,+\infty)$ 上的每一个值,$y$ 都在 $[-1,1]$ 上有唯一的值和它对应。例如,对于 $x=\dfrac\uppi6$,有 $y=\sin\dfrac\uppi6=\dfrac12$ 和它对应。反过来,对于 $y$ 在 $[-1,1]$ 上的每一个值,$x$ 有无穷多个值和它对应。例如,对于 $y=\dfrac12$,$x$ 有 $\dfrac\uppi6,\dfrac{5\uppi}{6},\cdots$ 等无穷多个值和它对应。由此可见,确定函数 $y=\sin x$ 的映射不是定义域 $(-\infty,+\infty)$ 到值域 $[-1,1]$ 上的一一映射。函数 $y=\sin x\,(x\in(-\infty,+\infty))$ 没有反函数。

\begin{figure}
  \includegraphics{1-1.pdf}
  \caption{}\label{fig:1-1}
\end{figure}

但由\cref{fig:1-2} 可以看到,在正弦函数的单调区间 $\left[-\dfrac\uppi2,\dfrac\uppi2\right]$ 上,对于 $x$ 的每一个值,$y=\sin x$ 有唯一的值和 $x$ 对应;而对于 $x$ 的不同的值,$y=\sin x$ 有不同的值和 $x$ 对应,并且随着 $x$ 由 $-\dfrac\uppi2$ 增大到 $\dfrac\uppi2$,$y=\sin x$ 由 $-1$ 增大到 $+1$,取得 $[-1,1]$ 上的一切值。因此,确定函数 $y=\sin x\,\left(x\in\left[-\dfrac\uppi2,\dfrac\uppi2\right]\right)$ 的映射是区间 $\left[-\dfrac\uppi2,\dfrac\uppi2\right]$ 到 $[-1,1]$ 上的一一映射。所以这个映射有逆映射,函数 $y=\sin x\,\left(x\in\left[-\dfrac\uppi2,\dfrac\uppi2\right]\right)$ 有反函数。

\begin{figure}
  \includegraphics{1-2.pdf}
  \caption{}\label{fig:1-2}
\end{figure}

函数 $y=\sin x\,\left(x\in\left[-\dfrac\uppi2,\dfrac\uppi2\right]\right)$ 的反函数叫做\Concept{反正弦函数},记作 $x=\arcsin y$。

\medskip
习惯上用字母 $x$ 表示自变量,用 $y$ 表示函数,所以反正弦函数可以写成 $y=\arcsin x$,\footnote{有的书上把反正弦函数写作 $y=\sin^{-1}x$。同样,后面讲到的反余弦函数、反正切函数、反余弦函数也可写作 $\cos^{-1}x,\tan^{-1}x,\cot^{-1}x$。}它的定义域是 $[-1,1]$,它的值域是 $\left[-\dfrac\uppi2,\dfrac\uppi2\right]$。

这样,对于属于 $[-1.1]$ 的每一个 $x$ 值,$\arcsin x$ 就表示属于 $\left[-\dfrac\uppi2,\dfrac\uppi2\right]$ 的唯一确定的一个值,它的正弦正好等于已知的 $x$。也就是说,$\arcsin x$ 表示属于 $\left[-\dfrac\uppi2,\dfrac\uppi2\right]$ 的唯一确定的一个角(弧度制),这个角的正弦恰好等于 $x$。例如,对于 $x=\dfrac12,y=\arcsin\dfrac12$ 就表示 $\left[-\dfrac\uppi2,\dfrac\uppi2\right]$ 上使 $\sin y=\dfrac12$ 的唯一确定的一个角,这个角是 $\dfrac\uppi6$,因为根据正弦函数 $y=\sin x$ 在 $\left[-\dfrac\uppi2,\dfrac\uppi2\right]$ 上,除了 $\dfrac\uppi6$ 以外,其他任何角的正弦都不等于 $\dfrac12$。

\medskip
由此可以得到
\[ \sin\left(\arcsin\frac12\right)=\frac12.\]

一般地,根据反正弦函数的定义,可以得到
\[ \sin(\arcsin x)=x,\]
其中 $x\in[-1,1]$,$\arcsin x\in\left[-\dfrac\uppi2,\dfrac\uppi2\right]$。

\medskip
下面我们来研究反正弦函数的图像和性质,

根据互为反函数的图象的性质,容易知道,反正弦函数 $y=\arcsin x$ 的图象就是与正弦函数 $y=\sin x$ 在 $\left[-\dfrac\uppi2,\dfrac\uppi2\right]$ 上的一段图象关于直线 $y=x$ 对称的图形(\cref{fig:1-3})。

\begin{figure}
  \includegraphics{1-3.pdf}
  \caption{}\label{fig:1-3}
\end{figure}

从图象上可以看出,反正弦函数 $y=\arcsin x$ 有以下性质:
\begin{enumerate}
  \item 反正弦函数 $y=\arcsin x$ 在区间 $[-1,1]$ 上是增函数。
  \item 反正弦函数 $y=\arcsin x$ 的图象关于原点对称,这说明它是奇函数。也就是
  \[ \arcsin(-x)=-\arcsin x,\ x\in[-1,1].\]
\end{enumerate}

\begin{example}
  求下列反正弦函数值:
  \begin{tasks}[before-skip=5pt,after-skip=5pt,after-item-skip=7pt](2)
    \task $\arcsin\dfrac{\sqrt{2}}{2}$;
    \task $\arcsin 0.2672$;
    \task $\arcsin\left(-\dfrac{\sqrt{3}}{2}\right)$;
    \task $\arcsin(-1)$。
  \end{tasks}
\end{example}
\begin{solution}
  \begin{enumerate}
    \item 因为在 $\left[-\dfrac\uppi2,\dfrac\uppi2\right]$ 上,$\sin\dfrac\uppi4=\dfrac{\sqrt{2}}{2}$,所以
    \[ \arcsin \frac{\sqrt{2}}{2}=\frac\uppi4.\]

    注意:虽然 $\sin\dfrac{3\uppi}{4}=\dfrac{\sqrt{2}}{2}$,但是,$\dfrac{3\uppi}{4}\notin\left[-\dfrac\uppi2,\dfrac\uppi2\right]$,所以 $\arcsin\dfrac{\sqrt{2}}{2}\neq \dfrac{3\uppi}{4}$。同理,$\arcsin\dfrac{\sqrt{2}}{2}$ 也不等于其他值(如 $\dfrac{0\uppi}{4}$,$\dfrac{7\uppi}{4}$ 等),只能等于 $\dfrac{\uppi}{4}$。
    \item 查正弦函数表,得 $\sin\ang{15;30}=0.2672$。又因为 \ang{15;30} 的弧度数属于 $\left[-\dfrac\uppi2,\dfrac\uppi2\right]$,所以 
    \[ \arcsin 0.2672=\ang{15;30}.\]
    \item 因为在 $\left[-\dfrac\uppi2,\dfrac\uppi2\right]$ 上,$\sin\left(-\dfrac\uppi3\right)=-\dfrac{\sqrt{3}}{2}$,所以 
    \[ \arcsin \left(-\frac{\sqrt{3}}{2}\right)=-\frac{\uppi}{3}.\]
    \item 因为在 $\left[-\dfrac\uppi2,\dfrac\uppi2\right]$ 上,$\sin\left(-\dfrac\uppi2\right)=-1$,所以
    \[ \arcsin(-1)=-\dfrac\uppi2.\]
  \end{enumerate}
\end{solution}

\begin{example}
  求下列各式的值:
  \begin{tasks}[after-skip=5pt,before-skip=5pt](2)
    \task $\sin\left(\arcsin\dfrac23\right)$;
    \task $\sin\left[\arcsin\left(-\dfrac12\right)\right]$。
  \end{tasks}
\end{example}
\begin{solution}
  \begin{enumerate}
    \item $\because\qquad x=\dfrac23\in[-1,1]$,
    \[\therefore \qquad \sin\left(\arcsin\dfrac23\right)=\dfrac23.\]
    \item $\because\qquad x=-\dfrac12\in[-1,1]$,
    \[\therefore \qquad \sin\left[\arcsin\left(-\dfrac12\right)\right]=-\dfrac12.\]
  \end{enumerate}
\end{solution}

\begin{example}
  求下列各式的值:
  \begin{tasks}[after-skip=5pt,before-skip=5pt,after-item-skip=7pt](2)
    \task $\tan\left(\arcsin\dfrac{\sqrt{3}}{2}\right)$;
    \task $\cos\left(\arcsin\dfrac45\right)$;
    \task\label{tsk:exp1-3-3}$\cos(\arcsin x)$,$x\in[-1,1]$;
    \task $\sin\left(2\arcsin\dfrac35\right)$。
  \end{tasks}
\end{example}
\begin{solution}
  \begin{enumerate}
    \item $\tan\left(\arcsin\dfrac{\sqrt{3}}{2}\right)=\tan\dfrac{\uppi}{3}=\sqrt{3}$。
    \item 设 $\arcsin\dfrac45=\alpha$,则 $\sin\alpha=\dfrac45$。
    
    \medskip
    由 $\alpha\in\left[-\dfrac\uppi2,\dfrac\uppi2\right]$,得 $\cos\alpha\geqslant 0$,可知
    \begin{gather*}
      \cos\alpha=\sqrt{1-\sin^2\alpha}=\sqrt{1-\left(\dfrac45\right)^2}=\dfrac35.\\
      \therefore \qquad \cos\left(\arcsin\dfrac45\right)=\dfrac35.
    \end{gather*}
    \item 设 $\arcsin x=\alpha$,则 $\sin\alpha =x$,且 $\alpha\in\left[-\dfrac\uppi2,\dfrac\uppi2\right]$。
    \begin{gather*}
      \cos\alpha=\sqrt{1-\sin^2\alpha}=\sqrt{1-x^2},\\ 
      \therefore\qquad \cos(\arcsin x)=\sqrt{1-x^2}.
    \end{gather*}

    或:由 $x\in[-1,1]$,得 $\arcsin x\in\left[-\dfrac\uppi2,\dfrac\uppi2\right]$,可知 
    \begin{gather*}
      cos(\arcsin x)\geqslant 0,\\
      \therefore \quad \cos(\arcsin x)=\sqrt{1-[\sin(\arcsin x)]^2}=\sqrt{1-x^2}.
    \end{gather*}
    \item 利用倍角公式及本例题第~\ref{tsk:exp1-3-3}~题的结果,可知
    \begin{align*}
      \sin\left(2\arcsin\dfrac35\right)&=2\sin\left(\arcsin\dfrac35\right)\cos\left(\arcsin\dfrac35\right)\\ 
      &=2\times\dfrac35\times\sqrt{1-\left(\dfrac35\right)^2}\\ 
      &=2\times\dfrac35\times\dfrac45=\dfrac{24}{25}.
    \end{align*}
  \end{enumerate}
\end{solution}

\begin{example}\label{exp:1-4}
  求下列各式的值:
  \begin{tasks}[after-skip=5pt,before-skip=5pt](2)
    \task $\arcsin\left(\sin\dfrac\uppi4\right)$;
    \task $\arcsin\left(\sin\dfrac{2\uppi}{3}\right)$。
  \end{tasks}
\end{example}
\begin{solution}
  \begin{enumerate}
    \item $\arcsin\left(\sin\dfrac\uppi4\right)=\arcsin\dfrac{\sqrt{2}}{2}=\dfrac\uppi4$。
    \item\label{itm:exp1-4-2}$\arcsin\left(\sin\dfrac{2\uppi}{3}\right)=\arcsin\dfrac{\sqrt{3}}{2}=\dfrac\uppi3$。
  \end{enumerate}
\end{solution}

\medskip
由\cref{exp:1-4} 第~\ref{itm:exp1-4-2}~题可以看出,虽然 $\sin(\arcsin x)=x$,其中 $x\in[-1,1]$,但是 $\arcsin(\sin x)$ 不一定等于 $x$,而是等于在闭区间 $\left[-\dfrac\uppi2,\dfrac\uppi2\right]$ 上与 $x$ 有相同正弦的一个值。


\begin{Practice}
  \begin{question}
    \item 用反正弦的形式把下列各式中的 $x\ \left(x\in\left[-\dfrac\uppi2,\dfrac\uppi2\right]\right)$ 表示出来:
    \begin{tasks}[before-skip=5pt,after-skip=5pt,after-item-skip=7pt](2)
      \task $\sin x=\dfrac25$;
      \task $\sin x=-\dfrac13$;
      \task $\sin x=0.3147$;
      \task $\sin x=-\dfrac{\sqrt{3}}{4}$。
    \end{tasks}
    \item 解答:
    \begin{tasks}
      \task $\arcsin\sqrt{2}$ 有意义吗,为什么?
      \task $\sin\left(\arcsin\dfrac{\sqrt{5}}{2}\right)=\dfrac{\sqrt{5}}{2}$ 是否成立,为什么?
    \end{tasks}
    \item 写出下列函数的定义域、值域:
    \begin{tasks}[after-skip=5pt,before-skip=5pt,after-item-skip=7pt](2)
      \task $y=\arcsin2x$;
      \task $y=\dfrac12\arcsin x$;
      \task $y=3\arcsin\dfrac23x$;
      \task $y=2\arcsin(1-x)$。
    \end{tasks}
    \item 求下列反正弦函数值:
    \begin{tasks}[after-skip=5pt,before-skip=5pt,after-item-skip=7pt](2)
      \task $\arcsin\dfrac{\sqrt{3}}{2}$;
      \task $\arcsin\left(-\dfrac{\sqrt{2}}{2}\right)$;
      \task $\arcsin0.6959$;
      \task $\arcsin\left(-\dfrac13\right)$。
    \end{tasks}
    \item 求下列各式的值:
    \begin{tasks}[after-skip=5pt,before-skip=5pt](2)
      \task $\sin\left(\arcsin\dfrac45\right)$;
      \task $\sin\left[\arcsin\left(-\dfrac45\right)\right]$。
    \end{tasks}
    \item 求下列各式的值:
    \begin{tasks}[after-skip=5pt,before-skip=5pt,after-item-skip=7pt](2)
      \task $\cos\left(\arcsin\dfrac12\right)$;
      \task $\tan\left(\arcsin\dfrac35\right)$;
      \task $\tan(\arcsin x)$,$x\in(-1,1)$;
      \task $\cos\left(2\arcsin\dfrac45\right)$。
    \end{tasks}
    \item 求下列各式的值:
    \begin{tasks}[after-skip=5pt,before-skip=5pt](2)
      \task $\arcsin\left(\sin\dfrac{3\uppi}{4}\right)$;
      \task $\arcsin\left[\sin\left(-\dfrac{3\uppi}{4}\right)\right]$。
    \end{tasks}
  \end{question}
\end{Practice}

\subsection{反余弦函数}
从余弦函数的图象(\cref{fig:1-4})同样可以看到,余弦函数 $y=\cos x$($x\in(-\infty,+\infty)$)不存在反函数。但在单调区间 $[0,\uppi]$ 上,对于不同的 $x$ 值,$y$ 有不同的值和它对应,并且随着 $x$ 由 0 增大到 $\uppi$,$y$ 由 $1$ 减小到 $-1$,取得值域 $[-1.1]$ 上的一切值。因此,函数 $y=\cos x$($x\in[0,\uppi]$)有反函数。
\begin{figure}
  \includegraphics{1-4.pdf}
  \caption{}\label{fig:1-4}
\end{figure}

函数 $y=\cos x$($x\in[0,\uppi]$)的反函数叫做\Concept{反余弦函数},记作 $y=\arccos x$,它的定义域是 $[-1,1]$,值域是 $[0,\uppi]$。

这样,对于属于 $[-1,1]$ 的每一个 $x$ 值,$\arccos x$ 就表示属于 $[0,\uppi]$ 的唯一确定的一个值,它的余弦正好等于已知的 $x$,也可以说,$\arccos x$ 表示属于 $[0,\uppi]$ 的唯一确定的一个角(弧度数),这个角的余弦恰好等于 $x$。例如,对于 $x=\dfrac12$,$y=\arccos\dfrac12$ 就表示 $[0,\uppi]$ 上使 $\cos y=\dfrac12$ 的唯一确定的一个角,这个角是 $\dfrac\uppi3$,因为根据余弦函数 $y=\cos x$ 在 $[0,\uppi]$ 上的单调性可以知道,在 $[0,\uppi]$ 上,除了 $\dfrac\uppi3$ 外,其他任何角的余弦都不等于 $\dfrac12$。

由此可以得到:
\[ \cos\left(\arccos\frac12\right)=\frac12.\]

一般地,根据反余弦函数的定义,可以得到
\[ \cos(\arccos x)=x,\]
其中 $x\in[-1,1],\ \arccos x\in[0,\uppi].$

反余弦函数 $y=\arccos x$ 的图象如\cref{fig:1-5} 所示,它是与余弦函数 $y=\cos x$ 在 $[0,\uppi]$ 上的一段图象关于直线 $y=x$ 对称的图形。
\begin{figure}
  \includegraphics{1-5.pdf}
  \caption{}\label{fig:1-5}
\end{figure}

从图象上可以看出:\emph{反余弦函数 $y=\arccos x$ 在区间 $[-1,1]$ 上是减函数}。它既不是偶函数,也不是奇函数。

下面我们来证明:对于任意 $x\in[-1,1]$,有 
\[\arccos(-x)=\uppi-\arccos x.\]

\begin{proof}
  由 $-1\leqslant x\leqslant 1$,得 $1\geqslant -x\geqslant -1$,即 $-x$ 属于反余弦函数得定义域 $[-1,1]$。

根据诱导公式与反余弦函数的定义,得
\[\cos(\uppi-\arccos x)=-\cos(\arccos x)=-x,\]
因此,$\uppi-\arccos x$ 是余弦等于 $-x$ 的一个值。

又因 $0\leqslant \arccos x\leqslant \uppi$,所以 $0\geqslant -\arccos x\geqslant -\uppi$,由此可得 $\uppi\geqslant \uppi-\arccos x\geqslant 0$,即 $\uppi-\arccos x\in[0,\uppi]$。

因此,$\uppi-\arccos x$ 是属于 $[0,\uppi]$ 且它的余弦等于 $-x$ 的一个值。于是 
\[ \arccos(-x)=\uppi-\arccos x.\]
\end{proof}

\begin{example}
  求下列各式的值:
  \begin{tasks}[after-skip=5pt,after-item-skip=7pt,before-skip=5pt](2)
    \task $\arccos\dfrac{\sqrt{3}}{2}$;
    \task $\arccos\left(-\dfrac{\sqrt{2}}{2}\right)$;
    \task $\cos\left[\arccos\left(-\dfrac{\sqrt{2}}{3}\right)\right]$;
    \task $\arccos\left(\cos\dfrac{11\uppi}{6}\right)$;
  \end{tasks}
\end{example}
\begin{solution}
  \begin{enumerate}
    \item 因为在 $[0,\uppi]$ 上,$\cos\dfrac\uppi6=\dfrac{\sqrt{3}}{2}$,所以
    \[ \arccos\dfrac{\sqrt{3}}{2}=\frac\uppi6.\]
    \item 因为在 $[0,\uppi]$ 上,$\cos\dfrac{3\uppi}{4}=-\dfrac{\sqrt{2}}{2}$,所以
    \[\arccos\left(-\dfrac{\sqrt{2}}{2}\right)=\dfrac{3\uppi}{4}.\]
    或:$\arccos\left(-\dfrac{\sqrt{2}}{2}\right)=\uppi-\arccos\dfrac{\sqrt{2}}{4}=\uppi-\dfrac\uppi4=\dfrac{3\uppi}{4}$。
    \item $\because \qquad -\dfrac{\sqrt{2}}{3}\in[-1,1]$,
    \[\therefore\qquad \cos\left[\arccos\left(-\dfrac{\sqrt{2}}{3}\right)\right]=-\dfrac{\sqrt{2}}{3}.\]
    \item $\arccos\left(\cos\dfrac{11\uppi}{6}\right)=\arccos\dfrac{\sqrt{3}}{2}=\dfrac\uppi6$。
  \end{enumerate}
\end{solution}

\begin{example}
  求下列各式的值:
  \begin{tasks}[after-skip=5pt,after-item-skip=7pt,before-skip=5pt]
    \task $\sin\left[\arccos\left(-\dfrac45\right)\right]$;
    \task $\tan(\arccos x)$,$x\in[-1,1]$,且 $x\neq 0$;
    \task $\cos\left[\arccos\dfrac45+\arccos\left(-\dfrac{5}{13}\right)\right]$。
  \end{tasks}
\end{example}
\begin{solution}
  \begin{enumerate}
    \item 设 $\arccos\left(-\dfrac45\right)=\alpha$,则 $\cos\alpha=-\dfrac45$。
    
    由 $\alpha\in[0,\uppi]$,得 $\sin\alpha\geqslant 0$,可知
    \begin{gather*}
      \sin\alpha =\sqrt{1-\cos^2\alpha}=\sqrt{1-\left(-\dfrac45\right)^2}=\dfrac35.\\ 
      \therefore\qquad \sin\left[\arccos\left(-\dfrac45\right)\right]=\dfrac35.
    \end{gather*}
    \item 由 $\arccos x\in[0,\uppi]$,知 $\sin(\arccos x)\geqslant 0$。
    \begin{align*}
      \therefore\quad \tan(\arccos x)&=\frac{\sin(\arccos x)}{\cos(\arccos x)}\\
      &=\frac{\sqrt{1-[cos(\arccos x)]^2}}{\cos(\arccos x)}\\
      &=\frac{\sqrt{1-x^2}}{x}.
    \end{align*}
    \item 设 $\arccos\dfrac45=\alpha$,则 $\cos\alpha=\dfrac45$,$\alpha$ 是第一象限的角,
    \[\therefore \quad \sin\alpha=\sqrt{1-cos^2\alpha}=\frac35.\]

    又设 $\arccos\left(-\dfrac{5}{13}\right)=\beta$,则 $\cos\beta=-\dfrac{5}{13}$,$\beta$ 是第二象限的角,
    \[\therefore \quad \sin\beta=\sqrt{1-\cos^2\beta}=\frac{12}{13}.\]

    代入原式,得
    \begin{align*}
      & \cos\left[\arccos\dfrac45+\arccos\left(-\dfrac{5}{13}\right)\right]\\
      ={} &\cos(\alpha+\beta)=\cos\alpha\cos\beta-\sin\alpha\sin\beta\\
      ={} &\dfrac45\cdot\left(-\dfrac{5}{13}\right)-\dfrac35\cdot\dfrac{12}{13}=-\dfrac{56}{65}.
    \end{align*}
  \end{enumerate}
\end{solution}

\begin{Practice}
  \begin{question}
    \item 用反余弦的形式把下列各式中的 $x$($x\in[0,\uppi]$)表示出来:
    \begin{tasks}[after-item-skip=7pt](2)
      \task $\cos x=\dfrac23$;
      \task $\cos x=-\dfrac15$;
      \task $\cos x=0.8065$;
      \task $\cos x=a$($a\in[-1,1]$)。
    \end{tasks}
    \item 解答:
    \begin{tasks}[after-item-skip=7pt,after-skip=5pt]
      \task $\arccos 1.2$ 有有意义吗,为什么?
      \task $\cos\left(\arccos\dfrac{\sqrt{5}}{3}\right)=\dfrac{\sqrt{5}}{3}$ 是否成立,为什么?
    \end{tasks}
    \item 写出下列函数的定义域、值域:
    \begin{tasks}[after-item-skip=7pt,after-skip=5pt](2)
      \task $y=\arccos3x$;
      \task $y=-5arccos x$;
      \task $y=\dfrac12\arccos\dfrac{x}{4}$;
      \task $y=3\arccos(2-3x)$。
    \end{tasks}
    \item 求下列反余弦函数值:
    \begin{tasks}[after-item-skip=7pt,after-skip=5pt](2)
      \task $\arccos\dfrac{\sqrt{2}}{2}$;
      \task $arccos0$;
      \task $\arccos\left(-\dfrac34\right)$;
      \task $\arccos0.0471$。
    \end{tasks}
    \item 求下列各式的值:
    \begin{tasks}[after-item-skip=7pt,after-skip=5pt](2)
      \task $\cos(\arccos0.8795)$;
      \task $\arccos(\cos0.8795)$;
      \task $\cos\left[\arccos\left(-\dfrac14\right)\right]$;
      \task $\arccos\left[\cos\left(-\dfrac14\right)\right]$。
    \end{tasks}
    \item 求下列各式的值:
    \begin{tasks}[before-skip=5pt,after-skip=5pt,after-item-skip=7pt](2)
      \task $\sin\left(\arccos\dfrac27\right)$;
      \task $\cos\left(2\arccos\dfrac45\right)$;
      \task $\sin\left[\dfrac\uppi3+\arccos\left(-\dfrac14\right)\right]$;
      \task $\cot(\arccos x)$,$x\in(-1,1)$。
    \end{tasks}
  \end{question}
\end{Practice}

\subsection{反正切函数与反余切函数}
正切函数 $y=\tan x\ \left(x\in\left(-\dfrac\uppi2,\dfrac\uppi2\right)\right)$ 的反函数叫做\Concept{反正切函数},记作 $y=\arctan x$,它的定义域是 $(-\infty,+\infty)$,值域是 $\left(-\dfrac\uppi2,\dfrac\uppi2\right)$。

\medskip
余切函数 $y=\cot x\ (x\in(0,\uppi))$ 的反函数叫做\Concept{反余切函数},记作 $y=\arctan x$,它的定义域是 $(-\infty,+\infty)$,值域是 $(0,\uppi)$。

由反正切函数与反余切函数的定义,我们得到:
\[\tan(\arctan x)=x,\]
其中 $x\in(-\infty,+\infty)$,$\arctan x\in\left(-\dfrac\uppi2,\dfrac\uppi2\right)$;
\[\cot(\arccot x)=x,\]
其中 $x\in(-\infty,+\infty)$,$\arccot x\in(0,\uppi)$。

\begin{figure}
  \begin{minipage}[b]{0.48\linewidth}\centering
    \includegraphics{1-6.pdf}
    \caption{}\label{fig:1-6}
  \end{minipage}
  \begin{minipage}[b]{0.48\linewidth}\centering
    \includegraphics{1-7.pdf}
    \caption{}\label{fig:1-7}
  \end{minipage}
\end{figure}

\cref{fig:1-6,fig:1-7} 分别是反正切函数与反余切函数的图象。

从图象上可以看出:
\begin{enumerate}
  \item 反正切函数 $y=\arctan x$ 在区间 $(-\infty,+\infty)$ 上是增函数;反余切函数 $y=\arccot x$ 在区间 $(-\infty,+\infty)$ 上是减函数。
  \item 反正切函数 $y=\arctan x$ 是奇函数,即
  \[\arctan(-x)=-\arctan x,\ x\in(-\infty,+\infty).\]
  \item 反余切函数有下述关系:
  \[\arccot(-x)=\uppi-\arccot x,\ x\in(-\infty,+\infty).\]
\end{enumerate}

这个性质与反余弦函数是类似的。

反正弦函数、反余弦函数、反正切函数、反余切函数,都叫做\Concept{反三角函数}。\footnote{反三角函数还有反正割函数和反余割两种。这两种反三角函数在本书中不研究。}

\begin{example}
  求下列各式的值:
  \begin{tasks}(2)
    \task $\arctan0$;
    \task $\arctan(-2)$;
    \task $\arccot1$;
    \task $\arccot(-\sqrt{3})$。
  \end{tasks}
\end{example}
\begin{solution}
  \begin{enumerate}
    \item $\arctan0=0$;
    \item $\arctan(-2)=-\arctan2=\ang{-63;26}$;
    \item $\arccot1=\dfrac\uppi4$;
    \item $\arccot(-\sqrt{3})=\uppi-\arccot\sqrt{3}=\uppi-\dfrac\uppi6=\dfrac{5\uppi}{6}$。
  \end{enumerate}
\end{solution}

\begin{example}
  求证 $\arctan x+\arccot x=\dfrac\uppi2$。
\end{example}
\begin{proof}
  根据诱导公式与反余切函数的定义,得
  \[\tan\left(\frac\uppi2-\arccot x\right)=\cot(\arccot x)=x,\]

  因此,$\dfrac\uppi2-\arccot x$ 是正切等于 $x$ 的一个值。

  又因为 $0<\arccot x<\uppi$,所以 $0>-\arccot x>-\uppi$,由此可得 $\dfrac\uppi2>\dfrac\uppi2-\arccot x>-\dfrac\uppi2$,即 $\dfrac{\uppi}{2}-\arccot x\in\left(-\dfrac\uppi2,\dfrac\uppi2\right)$。
  
  \medskip
  因此,$\dfrac{\uppi}{2}-\arccot x$ 是属于 $\left(-\dfrac\uppi2,\dfrac\uppi2\right)$ 且它的正切等于 $x$ 的一个值。于是
\begin{gather*} 
  \arctan x=\dfrac\uppi2 -\arccot x,\\
  \therefore \quad \arctan x+\arccot x=\dfrac\uppi2.
\end{gather*}
\end{proof}

\begin{Practice}
  \begin{question}
    \item 用反正切或反余切的形式把下列各式中的 $x$ 表示出来:
    \begin{tasks}[before-skip=5pt,after-item-skip=7pt,after-skip=7pt](2)
      \task! $\tan x=0.6 \quad\left(-\dfrac\uppi2<x<\dfrac\uppi2\right)$;
      \task! $\tan x-\sqrt{5}=0\quad\left(-\dfrac\uppi2<x<\dfrac\uppi2\right)$;
      \task $\cot x=3\quad(0<x<\uppi)$;
      \task $3\cot x+1=0\quad(0<x<\uppi)$。
    \end{tasks}
    \item 写出下列函数的定义域、值域:
    \begin{tasks}[before-skip=5pt,after-item-skip=7pt,after-skip=7pt](2)
      \task $y=\arctan\dfrac{x}{2}$;
      \task $y=3\arccot(1-x)$。
    \end{tasks}
    \item 求下列各式的值:
    \begin{tasks}[before-skip=5pt,after-item-skip=7pt,after-skip=7pt](2)
      \task $\arctan\dfrac{\sqrt{3}}{3}$;
      \task $\arctan(-2.689)$;
      \task $\arccot\theta$;
      \task $\arccot(-1)$。
    \end{tasks}
    \item 求下列各式的值:
    \begin{tasks}[before-skip=5pt,after-item-skip=7pt,after-skip=7pt](2)
      \task $\tan(\arctan2.84)$;
      \task $\arctan\left(\tan\dfrac{4\uppi}{5}\right)$;
      \task $\cot\left[\arccot\left(-\dfrac12\right)\right]$;
      \task $\arccot\left[\cot\left(-\dfrac12\right)\right]$。
    \end{tasks}
    \item 求下列各式的值:
    \begin{tasks}[before-skip=5pt,after-item-skip=7pt,after-skip=7pt](2)
      \task $\cot(\arctan\sqrt{3})$;
      \task $\sin(\arccot2)$;
      \task $\tan\left(\arctan\dfrac14+\arctan\dfrac25\right)$;
      \task $\cos(2\arctan5) $。
    \end{tasks}
  \end{question}
\end{Practice}

\begin{Exercise}
  \begin{question}
    \item 求下列反正弦函数值:
    \begin{tasks}[after-skip=10pt,after-item-skip=7pt,before-skip=5pt](2)
      \task $\arcsin 0$;
      \task $\arcsin 0.7841$;
      \task $\arcsin\left(-\dfrac14\right)$;
      \task $\arcsin\dfrac{1+\sqrt{5}}{4}$。
    \end{tasks}
    \item 用反正弦的形式把下列各式中的 $x$ 表示出来:
    \begin{tasks}[after-skip=10pt,after-item-skip=7pt,before-skip=5pt](2)
      \task $\sin x=\dfrac{\sqrt{3}}{5}\quad\left(0<x<\dfrac\uppi2\right)$;
      \task $\sin x=-\dfrac{1}{4}\quad\left(-\dfrac\uppi2<x<\dfrac\uppi2\right)$;
      \task $\sin x=\dfrac{\sqrt{3}}{5}\quad\left(\dfrac\uppi2<x<\uppi\right)$;
      \task $\sin x=-\dfrac{1}{4}\quad\left(\uppi<x<\dfrac{3\uppi}2\right)$。
    \end{tasks}
    \item 求下列各式的值:
    \begin{tasks}[after-skip=10pt,after-item-skip=7pt,before-skip=5pt](2)
      \task $\sin\left[\arcsin\left(-\dfrac47\right)\right]$;
      \task $\cos\left(\arcsin\dfrac{\sqrt{5}}{4}\right)$;
      \task $\tan(\arcsin 0.8)$;
      \task $\sin\left(2\arcsin\dfrac16\right)$;
      \task $\cos(2\arcsin 0.5)$;
      \task $\sin\left[\dfrac\uppi3+\arcsin\left(-\dfrac{\sqrt{3}}{2}\right)\right]$;
      \task $\arcsin\left(\sin\dfrac{15\uppi}{4}\right)$;
      \task $\arcsin\left(\dfrac13+\sin\dfrac{\uppi}{6}\right)$。
    \end{tasks}
    \item 求下列函数的定义域、值域:
    \begin{tasks}[after-skip=10pt,after-item-skip=7pt,before-skip=5pt](2)
      \task $y=\arcsin3x$;
      \task $y=\dfrac13\arcsin(x-1)$;
      \task $y=\dfrac35\arcsin(2-x)$;
      \task $y=\dfrac\uppi2+\arcsin{x}{2}$。
    \end{tasks}
    \item 求下列各式的值:
    \begin{tasks}[after-skip=10pt,after-item-skip=7pt,before-skip=5pt](2)
      \task $\arccos 1$;
      \task $\arccos\left(-\dfrac12\right) $;
      \task $\arccos\left(-\dfrac{\sqrt{3}}{2}\right) $;
      \task $\arccos 0.6943$;
      \task $\arccos(-0.9178)$;
      \task $\arccos\dfrac{\sqrt{5}-1}{4}$。
    \end{tasks}
    \item 用反余弦的形式把下列各式中的 $x$ 表示出来:
    \begin{tasks}[after-skip=10pt,after-item-skip=7pt,before-skip=5pt](2)
      \task $\cos x=-\dfrac{1}{3}\quad\left(0<x<\dfrac\uppi2\right)$;
      \task $\cos x-\dfrac{3}{7}=0\quad\left(-\dfrac\uppi2<x<0\right)$。
    \end{tasks}
    \item 求下列各式的值:
    \begin{tasks}[after-skip=10pt,after-item-skip=7pt,before-skip=5pt](2)
      \task $\cos(\arccos0.2571)$;
      \task $\cos\left[\arccos\left(-\dfrac{\sqrt{5}}{13}\right)\right]$;
      \task $\sin\left(2\arccos\dfrac23\right)$;
      \task $\arccos\left[\cos\left(-\dfrac\uppi3\right)\right]$。
    \end{tasks}
    \item 求下列函数的定义域、值域:
    \begin{tasks}[after-skip=7pt,before-skip=7pt](2)
      \task $y=\arccos\left(\dfrac12-x\right)$;
      \task $y=\dfrac{1}{\arccos x}$。
    \end{tasks}
    \item 求下列各式的值:
    \begin{tasks}[after-skip=5pt,after-item-skip=7pt,before-skip=5pt](2)
      \task $\arctan(-\sqrt{3})$;
      \task $\arctan2.747$;
      \task $\arccot\left(-\dfrac14\right)$;
      \task $\arccot(-7.238)$;
      \task $\arctan\sqrt{3}+\arctan(\sqrt{2}+1)$;
      \task $\arccot(-6.460)+\arctan(-6.460)$。
    \end{tasks}
    \item 求下列各式的值:
    \begin{tasks}[after-skip=5pt,after-item-skip=7pt,before-skip=10pt](2)
      \task $\cot(\arctan 0.4)$;
      \task $\tan\left(\arctan\dfrac{\sqrt{2}}{2}+\arctan\dfrac{\sqrt{3}}{3}\right)$;
      \task $\tan(2\arccot x)$;
      \task $\cos[\arccot(-5)]$;
      \task $\arctan\left(\tan\dfrac{5\uppi}{6}\right)$;
      \task $\arccot\left(\tan\dfrac{\uppi}{3}\right)$。
    \end{tasks}
    \item 求下列函数的定义域、值域:
    \begin{tasks}[after-skip=5pt,before-skip=5pt](2)
      \task $y=\arctan\sqrt{x}$;
      \task $y=\sqrt{\arccot x}$。
    \end{tasks}
    \item 已知等腰三角形的高与底的比为 $4:3$,用反三角函数把它的三个内角表示出来。
    \item 求证:
    \begin{tasks}[after-skip=7pt,before-skip=7pt](2)
      \task $\cos(\arctan x)=\dfrac{1}{\sqrt{1+x^2}}$;
      \task $\sin(\arctan x)=\dfrac{1}{\sqrt{1+x^2}}$。
    \end{tasks}
  \end{question}
\end{Exercise}

\section{简单三角方程}
\subsection{三角方程}\label{subsec:triangle-equation}
我们看下面的问题:\cref{fig:1-8} 中的 $P,Q$ 分别是宽为 \qty{4}{cm}、\qty{8}{cm} 的钢板。现在要把它们焊接成 \ang{60} 角,下料时角 $x$ 应取多少度?
\par\medskip\noindent
\begin{minipage}{0.6\linewidth}\parindent2em
因为 $\angle CBD=\ang{60}$,从点 $A$ 画 $\angle CBD$ 两边的垂线 $AC,AD$,可以看出,
\[ AB\cdot\sin x=8,\quad AB\cdot\sin(\ang{60}-x)=4.\]
由此可得
\[ \sin x=2\sin(\ang{60}-x).\]
\end{minipage}\hfill
\begin{minipage}{0.35\linewidth}
\begin{figurehere}
  \includegraphics{1-8.pdf}
  \caption{}\label{fig:1-8}
\end{figurehere}
\end{minipage}\par\medskip

这是一个含有未知数的三角函数的方程。这种含有未知数的三角函数的方程叫做\Concept{三角方程}。

\subsection{最简单的三角方程}
在三角方程中,$\sin x=a, \cos x=a, \tan x=a, \cot x=a$ 是最简单的。其他的三角方程的求解,往往可以归结为求这种最简单的三角方程的解集。下面我们先研究这四个最简单的三角方程的解集。
\subsubsection{$\sin x=a$ 的解集}
因为 $|\sin x|\leqslant 1$,所以当 $|a|>1$ 时,方程 $\sin x=a$ 的解集为 $\vnothing$。

当 $|a|=1$ 时,方程 $\sin x=a$ 成为 $\sin x=1$ 或 $\sin x=-1$。由于 $y=\sin x$ 的周期为 $2\uppi$,而在长度为一个周期的区间 $\lbrack-\uppi,\uppi\rparen$ 上,方程 $\sin x=1$ 有唯一解 $x=\dfrac\uppi2$,方程 $\sin x=-1$ 有唯一解 $x=-\dfrac\uppi2$。因此,在$(-\infty,+\infty)$ 上,方程 $\sin x=1$ 的解集是
\[\left\{ x\biggm| x=2k\uppi+\frac\uppi2, k\in\mathbb{Z} \right\},\]
方程 $\sin x=-1$ 的解集是
\[\left\{ x\biggm| x=2k\uppi-\frac\uppi2, k\in\mathbb{Z} \right\}.\]
这就是说,当 $|a|=1$ 时,方程 $\sin x=a$ 的解集是
\[\left\{ x\bigm| x=2k\uppi+\arcsin a, k\in\mathbb{Z} \right\}.\]

当 $|a|<1$ 时,由反正弦函数的定义可知,方程 $\sin x=a$ 在单调区间 $\left\lbrack-\dfrac\uppi2,\dfrac\uppi2\right\rparen$ 上有唯一解 $x=\arcsin a$,而在单调区间
$\left\lbrack\dfrac{\uppi}{2},\dfrac{3\uppi}{2}\right\rparen$ 上又有唯一解 $x=\uppi-\arcsin a$。因此,在长度为一个周期的区间 $\left\lbrack-\dfrac\uppi2,\dfrac{3\uppi}{2}\right\rparen$ 上,方程 $\sin x=a$ 有两个解 $x=\arcsin a$,$x=\uppi-\arcsin a$。于是,当 $|a|<1$ 时,在 $(-\infty,+\infty)$ 上,方程 $\sin x=a$ 的解集是 
\[\left\{ x\bigm| x=2k\uppi+\arcsin a, k\in\mathbb{Z} \right\} \cup 
\left\{ x\bigm| x=(2k+1)\uppi-\arcsin a, k\in\mathbb{Z} \right\}.
\]

上面第一个集合中的元素 $x$ 等于 $\uppi$ 的偶数倍与 $\arcsin a$ 的和,第二个集合中的元素 $x$ 等于 $\uppi$ 的奇数倍与 $-\arcsin a$ 的和。因为当 $k$ 为偶数时 $(-1)^k=1$,$k$ 为奇数时 $(-1)^k=-1$,所以上述并集等于
\[\left\{ x\bigm| x=k\uppi+(-1)^k\arcsin a, k\in\mathbb{Z} \right\}.\] 

因此,方程 $\sin x=a$ 在 $(-\infty,+\infty)$ 上的解集如下表所示:
\begin{tablehere}
  \begin{tblr}{colspec={X[c]X[4,c]},hline{2}=0.8pt}
    $a$ 的取值范围 & 方程 $\sin x=a$ 的解集 \\ 
    $|a|>1$ & $\vnothing$ \\
    $|a|=1$ & $\{x \bigm| x=2k\uppi+\arcsin a, k\in\mathbb{Z}\}$\\
    $|a|<1$ & $\{x \bigm| x=k\uppi+(-1)^k\arcsin a, k\in\mathbb{Z}\}$\\
  \end{tblr}
\end{tablehere}

\subsubsection{$\cos x=a$ 的解集}
当 $|a|>1$ 时,方程 $\cos x=a$ 的解集为 $\vnothing$。

当 $|a|=1$ 时,在长度为一个周期的区间 $\lbrack0,2\uppi\rparen$ 上,方程 $\cos x=1$ 有唯一解 $x=0$;方程 $\cos x=-1$ 有唯一解 $x=\uppi$。因此,在 $(-\infty,+\infty)$ 上,方程 $\cos x=1$ 的解集是 $\{x\bigm| x=2k\uppi,\,k\in\mathbb{Z}\}$,方程 $\cos x=-1$ 的解集是 $\{x\bigm|x=2k\uppi+\uppi,\,k\in\mathbb{Z}\}$。就是说,在 $|a|=1$ 时,方程 $\cos x=a$ 的解集是
\[\{x\bigm| x=2k\uppi+\arccos a,\,k\in\mathbb{Z}\}.\]

当 $|a|<1$ 时,由反余弦函数的定义可知,方程 $\cos x=a$ 在单调区间 $\lbrack 0,\uppi\rparen$ 上有唯一解 $x=\arccos a$,在单调区间 $\lbrack-\uppi,0\rparen$ 上又有唯一解 $x=-\arccos a$。因此,在长度为一个周期的区间 $\lbrack-\uppi,\uppi\rparen$ 上,方程 $\cos x=a$ 有两个解 $x=\pm\arccos a$。于是,在 $(-\infty,+\infty)$ 上,方程 $\cos x=a$($|a|<1$)的解集是
\[ \{x\bigm|x=2k\uppi\pm\arccos a,\,k\in\mathbb{Z}\}.\]

方程 $\cos x=a$ 在 $(-\infty,+\infty)$ 上的解集如下表表示:
\begin{tablehere}
  \begin{tblr}{colspec={X[c]X[4,c]},hline{2}=0.8pt}
    $a$ 的取值范围 & 方程 $\cos x=a$ 的解集 \\ 
    $|a|>1$ & $\vnothing$ \\
    $|a|=1$ & $\{x \bigm| x=2k\uppi+\arccos a, k\in\mathbb{Z}\}$\\
    $|a|<1$ & $\{x \bigm| x=2k\uppi\pm\arccos a, k\in\mathbb{Z}\}$\\
  \end{tblr}
\end{tablehere}


\subsubsection{$\tan x=a$ 的解集}
由反正切函数的定义可知,在单调区间 $\left(-\dfrac\uppi2,\dfrac\uppi2\right)$ 上,不论 $a$ 为什么实数,方程 $\tan x=a$ 都有唯一解 $x=\arctan a$。因为 $y=\tan x$ 的周期是 $\uppi$,所以方程 $\tan x=a$ 的解集如下表所示:
\begin{tablehere}
  \begin{tblr}{colspec={X[c]},hline{2}=0.8pt}
    方程 $\tan x=a$ 的解集 \\ 
    $\{x\bigm| x=k\uppi+\arctan a, k\in\mathbb{Z}\}$\\
  \end{tblr}
\end{tablehere}

\subsubsection{$\cot x=a$ 的解集}
由反余切函数的定义可知,在单调区间 $(0,\uppi)$ 上,不论 $a$ 为什么实数,方程 $\cot x=a$ 有唯一解 $x=\arccot a$。因为 $y=\cot x$ 的周期是 $\uppi$,所以方程 $\cot x=a$ 的解集如下表所示:
\begin{tablehere}
  \begin{tblr}{colspec={X[c]},hline{2}=0.8pt}
    方程 $\cot x=a$ 的解集 \\ 
    $\{x\bigm| x=k\uppi+\arccot a, k\in\mathbb{Z}\}$\\
  \end{tblr}
\end{tablehere}

\begin{example}
解方程 $2\sin x+\sqrt{2}=0$。
\end{example}
\begin{solution}
原方程可化为
\[\sin x=-\dfrac{\sqrt{2}}{2}.\]

$\therefore\quad$ 解集是 $\left\{x \Bigm| x=k\uppi+(-1)^k\arcsin\left(-\dfrac{\sqrt{2}}{2}\right),\quad k\in\mathbb{Z}\right\}$,即
\[ \left\{x \Bigm| x=k\uppi+(-1)^k\cdot\left(-\dfrac\uppi4\right),\quad k\in\mathbb{Z}\right\} \]
\end{solution}

\begin{example}
解方程 $2\cos2x=1$。
\end{example}
\begin{solution}
原方程可化为
\begin{gather*}
  \cos2x=\frac12,\\
  2x=2k\uppi\pm\arccos\dfrac12\quad (k\in\mathbb{Z}),
\end{gather*}
即
\[ 2x=2k\uppi\pm\dfrac\uppi3\quad (k\in\mathbb{Z}).\]

$\therefore\quad$ 解集是 $\left\{x \Bigm| x=k\uppi\pm\dfrac\uppi6,\quad k\in\mathbb{Z}\right\}$。
\end{solution}

\begin{example}
  解方程 $\tan(x+\ang{15})+1=0$。
\end{example}
\begin{solution}
  原方程可化为
  \begin{gather*}
    \tan(x+\ang{15})=-1,\\ 
    x+\ang{15}=k\cdot\ang{180}+(-\ang{45})\quad(k\in\mathbb{Z}).
  \end{gather*}

  $\therefore\quad$ 解集是 $\{x\bigm| x=k\cdot\ang{180}-\ang{60},\quad k\in\mathbb{Z}\}$。
\end{solution}

\begin{example}
  求适合方程 $\sin(3x-\ang{105})=\dfrac12$ 且小于 \ang{360} 的正角。 
\end{example}
\begin{solution}
  由方程 $\sin(3x-\ang{105})=\dfrac12$,可得
\[ 3x-\ang{105}= k\cdot\ang{180}+(-1)^k\cdot\ang{30}\quad (k\in\mathbb{Z}).\]

$\therefore\quad$ 解集是 $\{x\bigm| x=k\cdot\ang{60}+(-1)^k\cdot\ang{10}+\ang{35},\quad k\in\mathbb{Z}\}$。

分别设 $k=0,1,2,3,4,5$,得适合方程且小于 \ang{360} 的正角是 \ang{45}、\ang{85}、\ang{165}、\ang{205}、\ang{285}、\ang{325}。
\end{solution}

\begin{Practice}
  \begin{question}
    \item 写出下列方程的解集:
    \begin{tasks}[before-skip=7pt,after-skip=7pt,after-item-skip=7pt](3)
      \task $\sin x=\dfrac{\sqrt{3}}{2}$;
      \task $\sin x=-\dfrac12$;
      \task $\cos x=\dfrac{\sqrt{3}}{2}$;
      \task $\cos x=-0.8475$;
      \task $\tan x=-\sqrt{3}$;
      \task $\cot x=\dfrac53$。
    \end{tasks}
    \item 解下列方程:
    \begin{tasks}[before-skip=7pt,after-skip=7pt,after-item-skip=7pt](2)
      \task $2\sin\dfrac{2x}{3}=1$;
      \task $2\cos(3x-\ang{15})+1=0$;
      \task $3\tan\dfrac{x+\ang{20}}{3}=\sqrt{3}$;
      \task $\cot\left(\dfrac{x}{4}+\ang{30}\right)+1=0$。
    \end{tasks}
    \item 求 \ang{0} 到 \ang{360} 的角 $x$,已知:
    \begin{tasks}[before-skip=5pt,after-skip=5pt](2)
      \task $\sin2x=-\dfrac12$;
      \task $\cos(3x+\ang{20})=0.95$。
    \end{tasks}
  \end{question}
\end{Practice}
\subsection{简单的三角方程}
有些比较简单的三角方程,可以通过三角恒等变形或利用代数中解方程的方法,把它化成一个或几个最简单的三角方程,从而求出它们的解。现举例如下:
\begin{example}
  解方程 $2\sin^2x+3\cos x=0$
\end{example}
\begin{solution}
  原方程可化为
  \[2(1-\cos^2x)+3\cos x=0.\]
  即
  \[2\cos^2x-3\cos x-2=0.\]

  解这个关于 $\cos x$ 的二次方程,得
  \[ \cos x=2,\qquad \cos x=-\dfrac12.\]

  $\cos x=2$ 的解集为 $\vnothing$;再由 $\cos x=-\dfrac12$,得
  \[ x=2k\uppi\pm\dfrac{2\uppi}{3},\quad(k\in\mathbb{Z}).\]

  所以原方程的解集是 $\left\{x\,\middle|\,x=2k\uppi\pm\dfrac{2\uppi}{3},\,(k\in\mathbb{Z})\right\}$。
\end{solution}

\begin{example}\label{exp:1-12}
  解方程 $\sin^2-\dfrac{2\sqrt{3}}{3}\sin x\cos x-\cos^2x=0$。
\end{example}
\begin{solution}
  显然,使 $\cos x=0$ 的 $x$ 值不可能满足原方程(因为 $\cos x=0$ 时,$\sin x=\pm1$),所以在方程的两边同除以 $\cos^2x$,得 
  \[\tan^2x-\frac{2\sqrt{3}}{3}\tan x-1=0.\]

  解这个关于 $\tan x$ 的二次方程,得 
  \[ \tan x=\sqrt{3},\qquad \tan x=-\dfrac{\sqrt{3}}{3}\]

  由 $\tan x=\sqrt{3}$,得 
  \[x=k\uppi+\frac\uppi3\quad(k\in\mathbb{Z});\]

  再由 $\tan x=-dfrac{\sqrt{3}}{3}$,得 
  \[x=k\uppi-\frac\uppi6\quad(k\in\mathbb{Z}).\]

  所以原方程的解集是 
  \begin{multline*}
    \left\{x\,\middle|\, x=k\uppi+\frac\uppi3,\,k\in\mathbb{Z}\right\}\cup\left\{x\,\middle|\, x=k\uppi-\frac\uppi6,\,k\in\mathbb{Z}\right\}=\\
    \left\{x\,\middle|\, x=k\uppi+\frac\uppi3,\,\text{或}\,x=k\uppi-\frac\uppi6,\,k\in\mathbb{Z}\right\}.
  \end{multline*}
\end{solution}

在\cref{exp:1-12} 中,方程的每一项关于 $\sin x$ 及 $\cos x$ 的次数都是相同的(这里都是二次)。像这样的方程叫做关于 $\sin x$ 及 $\cos x$ 的齐次方程。它的解法一般是先化为只含有未知数的正切函数的三角方程,然后求解。

\begin{example}\label{exp:1-13}
  解方程 $\sin x=2\sin(\ang{60}-x)$。
\end{example}
\begin{solution}
  将原方程变形,
  \begin{align*}
    \sin x&=2(\sin\ang{60}\cos x-\cos{60}\sin)\\
    \sin x&=2\left(\dfrac{\sqrt{3}}{2}\right).
  \end{align*}
  得
  \[2\sin x=\sqrt{3}\cos x.\]

  这是关于 $\sin x,\cos x$ 的齐次方程。在方程两边都除以 $2\cos x$,得
  \begin{align*}
    \tan x&=\frac{\sqrt{3}}{2}=0.8660.\\
    \arctan 0.8660&=\ang{40;54}.
  \end{align*}
  
  所以原方程的解集是
  \[\{x\bigm| x=k\cdot\ang{180}+\ang{40;54},\,k\in\mathbb{Z}\}.\]
\end{solution}

\cref{exp:1-13} 的方程就是\cref{subsec:triangle-equation}中焊接钢板问题所得的方程。它的解虽然有无穷多个,但是在这个实际问题中,要求 $\ang{0}<x<\ang{60}$,因此,只有当 $k=0$ 时,$x=\ang{40;54}$ 有意义。

\begin{example}
  解方程 $\sin x=\cos\dfrac{x}{2}$。
\end{example}
\begin{solution}
  利用倍角公式把原方程化为
  \begin{gather*}
    2\sin\dfrac{x}{2}\cos\dfrac{x}{2}=\cos\dfrac{x}{2},\\ 
    \cos\dfrac{x}{2}\left(2\sin\dfrac{x}{2}-1\right)=0,
  \end{gather*}
  得
  \[ \cos\dfrac{x}{2}=0,\ \text{或}\ \sin\dfrac{x}{2}=\dfrac12.\]

  由 $\cos\dfrac{x}{2}=0$,得 $\dfrac{x}{2}=2k\uppi\pm\dfrac\uppi2$,即 
  \[ x=4k\uppi\pm\uppi\quad(k\in\mathbb{Z});\]

  由 $\sin\dfrac{x}{2}=\dfrac12$,得 $\dfrac{x}{2}=k\uppi+(-1)^k\dfrac\uppi6$,即 
  \[x=2k\uppi+(-1)^k\frac\uppi3\quad(k\in\mathbb{Z}).\]

  所以原方程的解集是
  \begin{multline*}
    \{x\bigm| x=(4k\pm 1)\uppi,\,k\in\mathbb{Z}\} \cup \left\{x\,\middle|\, x=2k\uppi+(-1)^k\frac\uppi3,\,k\in\mathbb{Z}\right\} \\= \left\{x\,\middle|\,x=(4k\pm 1)\uppi\ \text{或}\ x=2k\uppi+(-1)^k\frac\uppi3,\,k\in\mathbb{Z}\right\}.
  \end{multline*}
\end{solution}

\begin{example}\label{exp:1-15}
  解方程 $\sin5x=\sin4x$。
\end{example}
\begin{solution}[解法一]
移项并运用三角函数的和差化积公式,得
\begin{gather*}
  \sin5x-\sin4x=0,\\
  2\cos\frac{9x}{2}\sin\frac{x}{2}=0,\\
  \cos\frac{9x}{2}=0\ \text{或}\ \sin\frac{x}{2}=0.
\end{gather*}

由 $\cos\dfrac{9x}{2}=0$,得 $\dfrac{9x}{2}=2k\uppi\pm\dfrac\uppi2\ (k\in\mathbb{Z})$,即 
\[x=\dfrac49k\uppi\pm\dfrac\uppi9\quad(k\in\mathbb{Z}).\]

由 $\sin\dfrac{x}{2}=0$,得 $\dfrac{x}{2}=k\uppi\ (k\in\mathbb{Z})$,即 
\[x=2k\uppi\quad(k\in\mathbb{Z}).\]

所以原方程的解集是
\begin{multline*}
  \left\{x\,\middle|\, x=\frac49k\uppi\pm\dfrac\uppi9,\,k\in\mathbb{Z}\right\} \cup \{x\bigm|x=2k\uppi,\,k\in\mathbb{Z}\} \\ 
  =\left\{x\,\middle|\, x=\frac49k\uppi\pm\dfrac\uppi9,\ \text{或}\ x=2k\uppi,\,k\in\mathbb{Z}\right\}.
\end{multline*}
\end{solution}

\begin{solution}[解法二]
  因为与 $\alpha$ 有相同得正弦值的弧度数 $x$ 的集合是 $\{x\bigm|x=k\uppi+(-1)^k\alpha,\,k\in\mathbb{Z}\}$,所以原方程可以化成
  \[5x=k\uppi+(-1)^k4x\quad(k\in\mathbb{Z}).\]

  当 $k$ 是偶数 $2n$($n\in\mathbb{Z}$)时,上式成为 $5x=2n\uppi+4x$,由此可得
  \[ x=2n\uppi\quad(n\in\mathbb{N}).\]

  当 $k$ 是奇数 $2n+1$($n\in\mathbb{Z}$)时,上式成为 $5x=(2n+1)\uppi-4x$,由此可得
  \[9x=(2n+1)\uppi\quad(n\in\mathbb{Z}),\]
  即
  \[x=\frac19(2n+1)\uppi\quad(n\in\mathbb{Z}).\]

  所以原方程的解集是
  \begin{multline*} 
    \{x\bigm|x=2n\uppi,\,n\in\mathbb{Z}\} \cup \left\{x\,\middle|\, x=\frac19(2n+1)\uppi,\,n\in\mathbb{Z}\right\} \\
    = \left\{x\,\middle|\,x=x=2n\uppi,\ \text{或}\  x=\frac19(2n+1)\uppi,\,n\in\mathbb{Z}\right\}
  \end{multline*}
\end{solution}

\cref{exp:1-15} 的两种解法,虽然得到的解集的表示形式不同,但因为当 $n$ 是偶数 $2k$ 时,$\dfrac19(2n+1)\uppi$ 成为 $\dfrac19(4k+1)\uppi$;当 $n$ 是奇数 $2k-1$ 时,$\dfrac19(2n+1)\uppi$ 成为 $\dfrac19(4k-1)\uppi$,所以实质上 $\left\{x\,\middle|\,x=\dfrac19(2n+1)\uppi,\,n\in\mathbb{Z}\right\}$ 与 $\left\{x\,\middle|\,x=\dfrac19(4k\pm1)\uppi,\,k\in\mathbb{Z}\right\}$ 是相等的集合。就是说,两种解法所得的解集是相同的。

\begin{example}\label{exp:1-16}
  解方程 $5\sin x-12\cos x=6.5$。
\end{example}
\begin{solution}
  在方程的两边都除以 $\sqrt{5^2+12^2}$,得
  \[\frac{5}{13}\sin x-\frac{12}{13}\cos x=\frac12.\]

  令 $\cos\theta=\dfrac{5}{13}$,$\sin\theta=\dfrac{12}{13}$,即令 $\tan\theta=\dfrac{12}{5}=2.4$,则满足这些式子的 $\theta$ 的一个值为 \ang{67;23}。由此得
  \begin{align*}
    \sin x\cos\ang{67;23}-\cos x\sin\ang{67;23}&=0.5,\\
    \sin(x-\ang{67;23})&=0.5,\\
    x-\ang{67;23}&=k\times \ang{180}+(-1)^k\times\ang{30}\, (k\in\mathbb{Z}).
  \end{align*}

  所以原方程的解集是
  \[\{x\bigm| x=k\times \ang{180}+\ang{67;23}+(-1)^k\times\ang{30}\, (k\in\mathbb{Z})\}\]
\end{solution}

在解\cref{exp:1-16} 的方程时,我们在方程的两边都除以 $\sqrt{5^2+12^2}$,其中被开方式 $5^2+12^2$ 是方程中 $\sin x$ 与 $\cos x$ 系数的平方和。一般说来,对于形如 $a\sin x+b\cos x=c$ 的三角方程,可先在方程的两边都除以 $\sqrt{a^2+b^2}$,然后令 $\cos\theta=\dfrac{a}{\sqrt{a^2+b^2}}$,$\sin\theta=\dfrac{b}{\sqrt{a^2+b^2}}$,则方程变形为
\[ \sin(x+\theta)=\frac{c}{\sqrt{a^2+b^2}},\]
当
\[\left|\frac{c}{\sqrt{a^2+b^2}}\right|\leqslant 1\]
时,方程有解。

\begin{Practice}
  解下列方程:
  \begin{question}[itemsep=5pt]
    \item $\sin^2x-2\sin x-3=0$。
    \item $4\cos^2x-4\sin x=1$。
    \item $2\sin x-5\cos x=0$。
    \item $3\sin^2x+2\sin x\cos x-5\cos^2x=0$。
    \item $4\cos\dfrac{x}{2}-5\cos x=5$。
    \item $\sin\dfrac{x}{2}-\sqrt{3}\cos\dfrac{x}{4}=0$。
    \item $\cos3x+\cos2x=0$。
    \item $6\sin x+8\cos x=5$。
  \end{question}
\end{Practice}

\begin{Exercise}
  解下列方程:
  \begin{question}[itemsep=7pt]
    \item $2\sin2x+1=0$。
    \item $\sin\left(\dfrac{x}{2}+\dfrac\uppi6\right)=\dfrac{\sqrt{3}}{2}$。
    \item $2\cos\left(\dfrac{x}{3}+\ang{45}\right)=1$。
    \item $\tan2x-\sqrt{3}=$。
    \item $\dfrac12\cot2(x+\ang{25})-2=0$。
    \item $2\cos^2x-3\cos x+1=$。
    \item $\sin^22x=sin2x$。
    \item $3\sin x-2\cos^2x=0$。
    \item $2\sin^2x=1$。
    \item $\sin^2x-7\sin x\cos x+6\cos^2x=0$。
    \item $\sin x\cos x=\dfrac14$。
    \item $3\sin^2x-\sin2x-\cos^2x=0$。
    \item $\sin3x=\sin x$。
    \item $\sin2x=\cos3x$。
    \item $\sin\left(2x+\dfrac\uppi3\right)+\sin\left(x+\dfrac\uppi6\right)=0$。
    \item $\sin\left(x+\dfrac\uppi6\right)+\cos\left(x+\dfrac\uppi6\right)=0$。
    \item $5\cos2x+2\sin2x=0$。
    \item $\cos^22x-3\sin^22x=0$。
    \item $\sin\dfrac{x}{2}-\cos\dfrac{x}{2}=1$。
    \item $4\sin x+3\cos x=3$。
  \end{question}
\end{Exercise}

\section*{小结}
\begin{enumerate}[C、,itemindent=4.5em]
  \item 本章主要内容是反三角函数的概念、图象、性质以及简单三角方程的解法。
  \item 本章学习的四种反三角函数的名称,函数式,定义域,值域,列表如下:
  \begin{tablehere}
    \begin{minipage}{\linewidth}
      \begin{tblr}{colspec={X[c]X[c]X[c]X[c]},hline{2}=0.8pt}
        名称       & 函数式        & 定义域              & 值域 \\
        反正弦函数 & $y=\arcsin x$ & $[-1,1]$            & $\left[-\dfrac\uppi2,\dfrac\uppi2\right]$ \\
        反余弦函数 & $y=\arccos x$ & $[-1,1]$            & $\left[0,\uppi\right]$ \\
        反正切函数 & $y=\arctan x$ & $(-\infty,+\infty)$ & $\left(-\dfrac\uppi2,\dfrac\uppi2\right)$ \\
        反余切函数 & $y=\arccot x$ & $(-\infty,+\infty)$ & $\left(0,\uppi\right)$ \\
      \end{tblr}
    \end{minipage}
  \end{tablehere}

  反正弦函数与反正切函数在它们的整个定义域内都是增函数,并且都是奇函数,所以具有以下关系:
  \begin{align*}
    \arcsin(-x) &=-\arcsin x,\\
    \arctan(-x) &=-\arctan x.
  \end{align*}

  反余弦函数与反余切函数在它们的整个定义域内都是减函数,并且具有以下关系:
  \begin{align*}
    \arccos(-x) &=\uppi-\arccos x,\\
    \arccot(-x) &=\uppi-\arccot x.
  \end{align*}
  由此可见,它们既不是奇函数,也不是偶函数。
  \item 最简单的三角方程的解集列表如下:
  \begin{tablehere}
    \begin{minipage}{\linewidth}
      \begin{tblr}{colspec={X[c]X[c]X[3,c]},hline{2}=0.8pt}
        \SetCell[c=2]{m,c}方程 & & 方程的解集 \\
        \SetCell[r=3]{m,c}$\sin x=a$ 
        & $|a|>1$ & $\vnothing$ \\
        & $|a|=1$ & $\{x \bigm| x=2k\uppi+\arcsin a, k\in\mathbb{Z}\}$ \\
        & $|a|<1$ & $\{x \bigm| x=k\uppi+(-1)^k\arcsin a, k\in\mathbb{Z}\}$ \\
        \SetCell[r=3]{m,c}$\cos x=a$ 
        & $|a|>1$ & $\vnothing$ \\
        & $|a|=1$ & $\{x \bigm| x=2k\uppi+\arccos a, k\in\mathbb{Z}\}$ \\
        & $|a|<1$ & $\{x \bigm| x=2k\uppi\pm\arccos a, k\in\mathbb{Z}\}$ \\
        \SetCell[c=2]{m,c} $\tan x=a$ & & $\{x \bigm| x=k\uppi+\arctan a, k\in\mathbb{Z} \}$ \\
        \SetCell[c=2]{m,c} $\cot x=a$ & & $\{x \bigm| x=k\uppi+\arccot a, k\in\mathbb{Z} \}$ \\
      \end{tblr}
    \end{minipage}
  \end{tablehere}
  \item 某些简单的三角方程,可以利用三角恒等变形或代数中解方程的方法,把它化成一个或几个最简单的三角方程,然后求解。
\end{enumerate}
\chapter*{复习参考题\chinese{chapter}}
\section*{A 组}
\begin{question}
  \item 解答:
  \begin{tasks}
    \task 怎样的函数可以有反函数?举出函数和它的反函数的例子。
    \task 函数和它的反函数的图象之间有什么关系?
  \end{tasks}
  \item 解答:
  \begin{tasks}
    \task 写出三角函数的诱导公式;
    \task 写出同角三角函数的基本关系式;
    \task 写出和角、差角、倍角、半角的三角函数的公式;
    \task 写出三角函数的和差化积及积化和差公式。
  \end{tasks}
  \item 画出 $y=x$ 及 $y=\sin(\arcsin x)$ 的图象,并比较两个图象的相同点及不同点。
  \item 求下列函数的反函数,并写出反函数的定义域、值域:
  \begin{tasks}[before-skip=5pt,after-skip=5pt](3)
    \task $y=\dfrac12\arcsin 3x$;
    \task $y=2\arccos\dfrac{x}{4}$;
    \task $y=\dfrac\uppi2+\arctan 2x$。
  \end{tasks}
  \item 求下列各式的值:
  \begin{tasks}[before-skip=5pt,after-item-skip=7pt,after-skip=5pt](2)
    \task $\sin\left(2\arcsin\dfrac14\right)$;
    \task $\cos\left[\dfrac12\arcsin\left(-\dfrac45\right)\right]$;
    \task $\sin\left(\arcsin\dfrac35-\arccos\dfrac12\right)$;
    \task $\cos\left(\arccos\dfrac35-\arcsin\dfrac{5}{13}\right)$。
  \end{tasks}
  \item 求出下列各式里的 $x$:
  \begin{tasks}[before-skip=5pt,after-item-skip=7pt,after-skip=5pt](2)
    \task $\arcsin\dfrac{20}{29}=\arccos x$;
    \task $\arcsin x=\arccos\dfrac{5}{12}$;
    \task $\arccot\dfrac{11}{60}=-\arctan x$。
  \end{tasks}
  \item 当 $a$ 取什么值时,下列三角方程的解集是空集?
  \begin{tasks}[before-skip=5pt,after-skip=5pt](2)
    \task $\sin x=\dfrac{1+a}{2}$;
    \task $\cos x=\dfrac{1-a}{2}$。
  \end{tasks}
  \item 解下列方程:
  \begin{tasks}[before-skip=5pt,after-item-skip=7pt,after-skip=5pt](2)
    \task! $4\sin^2x+(2\sqrt{3}-2)\cos x-(4-\sqrt{3})=0$;
    \task $\sec^2x=1+\tan x$;
    \task $\cos3\theta+2\cos \theta=0$;
    \task $\cos2\theta+\sin3\theta=$;
    \task $\tan3x=\tan4x$;
    \task $\dfrac{\sin2x}{\cos x}=\dfrac{\cos 2x}{\sin x}$;
    \task $5\cos x+12\sin x=13$;
    \task $4\sin x-3\cos x=\dfrac{5\sqrt{2}}{2}$。
  \end{tasks}
  \item 解下列方程:
  \begin{tasks}[before-skip=5pt,after-item-skip=7pt,after-skip=5pt](2)
    \task $\sin^4x-\cos^4x=\cos x+\sin x$;
    \task! $\cos\left(x+\dfrac\uppi4\right)+\sec\left(x+\dfrac\uppi4\right)+2=0$;
    \task $\sin\left(x+\dfrac\uppi4\right)\sin\left(x-\dfrac{\uppi}{12}\right)=\dfrac12$;
    \task $\cos2\theta=\cos\theta+\sin\theta$;
    \task! $5\sin^2x+7\sin x\cos x+4\cos^2x=1$;
    \task! $6\sin^2x+3\sin x\cos x-5\cos^2x=2$。
  \end{tasks}
  \item 圆的半径为 $R$,弦长为 $a$,用反正弦表示这条弦所对的圆周角,并确定 $R$ 与 $a$ 的取值范围。
  \item\label{exec:1t-11}如图,已知正三棱锥 $S\text{-}ABC$ 的各侧棱长都等于底面边长 $a$,又 $E,F$ 分别是 $AB,CS$ 的中点,求 $EF$ 和平面 $ABC$ 所成的角(用反三角函数表示)。
  \item 炮弹以初速度 $v_0$(\unit{m/s}) 沿与水平方向成 $\theta$ 角的方向向上射出,它的射程 $s$(\unit{m}) 可用下式表示:
  \[s=\frac{v_0^2\sin2\theta}{9.8}.\]
  已知 $v_0=\qty{630}{m/s}$,要使射程为 \qty{20}{km},$\theta$ 角应取多大?
  \begin{figurehere}
    \begin{minipage}{\linewidth}\centering
      \includegraphics{ex1t-11.pdf}
      \caption*{(第~\ref{exec:1t-11}~题图)}
    \end{minipage}
  \end{figurehere}
\end{question}
\section*{B 组}
\begin{question}[resume]
  \item 求下列各式的值:
  \begin{tasks}[before-skip=5pt,after-skip=5pt,after-item-skip=7pt]
    \task $\cos\left(2\arccos\dfrac13-\arccos\dfrac14\right)$;
    \task $\sin\left(\dfrac12\arcsin\alpha\right)\cdot\cos\left(\dfrac12\arcsin\alpha\right)$($|a|\leqslant 1$)。
  \end{tasks}
  \item 求证:
  \begin{tasks}[after-skip=5pt,after-item-skip=7pt]
    \task $\sin(2\arcsin x)=2x\sqrt{1-x^2}$($|x|\leqslant 1$);
    \task $\tan(2\arctan x)=\dfrac{2x}{1-x^2}$($|a|\leqslant 1$)。
  \end{tasks}
  \item 解下列方程:
  \begin{tasks}(2)
    \task $\sin6x\cos x=\sin4x\cos3x$;
    \task $\sin x\sin7x=\sin3x\sin5x$;
    \task $\sin5\theta-\sin3\theta=\sqrt{2}\cos4\theta$。
  \end{tasks}
  \item 求证:
  \begin{tasks}
    \task 方程 $\sin^2x=\sin^2\alpha$ 的解集是 $\{x\bigm| x=n\uppi\pm\alpha,\ n\in\mathbb{Z}\}$;
    \task 方程 $\cos^2x=\cos^2\alpha$ 的解集是 $\{x\bigm| x=n\uppi\pm\alpha,\ n\in\mathbb{Z}\}$;
    \task 方程 $\tan^2x=\tan^2\alpha$ 的解集是 $\{x\bigm| x=n\uppi\pm\alpha,\ n\in\mathbb{Z}\}$。
  \end{tasks}
  \item\label{exec:1t-17}如图,某种飞机机翼的曲边由线段和圆弧连接而成,切点为 $M$,求 $\theta$ 角。
  \begin{figurehere}
    \begin{minipage}[b]{0.45\linewidth}\centering
      \includegraphics{ex1t-17.pdf}
      \caption*{(第~\ref{exec:1t-17}~题图)}
    \end{minipage}
    \begin{minipage}[b]{0.45\linewidth}\centering
      \includegraphics{ex1t-18.pdf}
      \caption*{(第~\ref{exec:1t-18}~题图)}
    \end{minipage}
  \end{figurehere}
  \item\label{exec:1t-18}如图,有一块正方形钢板,一个角上有伤痕,要把它截成一块正方形钢板,面积是原钢板的 $\dfrac23$,应按怎样的角度 $x$ 来裁?
\end{question}