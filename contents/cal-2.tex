\chapter{导数和微分}
\section{导数概念}
\subsection{瞬时速度}
\begin{Practice}
  \begin{question}
    \item 
    \item 
  \end{question}
\end{Practice}
\subsection{导数}
\begin{Practice}
  \begin{question}
    \item 
    \item 
    \item 
  \end{question}
\end{Practice}
\subsection{导数的几何意义\texorpdfstring{\quad}{ }切线方程和法线方程}
\begin{Practice}
  \begin{question}
    \item 
    \item 
    \item 
  \end{question}
\end{Practice}
\subsection*{变化率举例}
\begin{Practice}
  \begin{question}
    \item 
    \item 
    \item 
  \end{question}
\end{Practice}
\subsection{函数的可导性与连续性的关系}
\begin{Practice}

\end{Practice}

\begin{Exercise}
  \begin{question}
    \item 
    \item 
    \item 
    \item 
    \item 
    \item 
    \item 
    \item 
    \item 
    \item 
    \item 
  \end{question}
\end{Exercise}

\section{求导方法}
\subsection{几种常见函数的导数}

\begin{Practice}
  (口答)求下列函数的导数:
  \begin{tasks}(2)
    \task $y=x^5$;
    \task $y=x^6$;
    \task $x=\sin t$;
    \task $u=\cos \varphi$。
  \end{tasks}
\end{Practice}

\subsection{函数的和、差、积、商的导数}
\begin{Practice}
  \begin{question}
    \item 
    \item 
    \item 
    \item 
    \item 
    \item 
  \end{question}
\end{Practice}

\subsection{复合函数的导数}
\begin{Practice}
  \begin{question}
    \item 
    \item 
    \item 
    \item 
  \end{question}
\end{Practice}

\begin{Exercise}
  \begin{question}
    \item 
    \item 
    \item 
    \item 
    \item 
    \item 
    \item 
    \item 
    \item 
    \item 
    \item 
    \item 
    \item 
    \item 
    \item 
    \item 
  \end{question}
\end{Exercise}

\subsection{三角函数的导数}
\begin{Practice}
  求下列函数的导数:
  \begin{tasks}(2)
    \task $f(\theta)=\dfrac{1+\cos\theta}{1-\cos\theta}$;
    \task $y=\cos x^2-\sin\sqrt{x}$;
    \task $f(\theta)=\tan\theta-\theta$;
    \task $y=\tan\frac{x}{2}-\cot\frac{x}{2}$。
  \end{tasks}
\end{Practice}

\subsection{反三角函数的导数}
\begin{Practice}
  求下列函数的导数:
  \begin{tasks}(2)
    \task $y=\arcsin\frac{x}{a}$;
    \task $y=x\arcsin x$;
    \task $y=2\arcsin x^2$;
    \task $y=\arccos\frac{x}{2}$;
    \task $y=\arctan\frac{x}{a}$;
    \task $y=(\arccot x)^2$。
  \end{tasks}
\end{Practice}

\subsection{对数函数的导数}
\begin{Practice}
  求下列函数的导数:
  \begin{tasks}(2)
    \task $y=x\ln x$;
    \task $y=\ln\dfrac{1+3x^2}{2-x^2}$;
    \task $y=\log_a(2x^3+3x^2)$;
    \task $y=\ln\sqrt{\dfrac{1+x}{1-x}}$;
    \task $y=\lg(1+\cos x)$;
    \task $y=\ln(\ln x)$。
  \end{tasks}
\end{Practice}

\subsection{指数函数的导数}
\begin{Practice}
  求下列函数的导数:
  \begin{tasks}(2)
    \task $y=e^x\sin x$;
    \task $y=\dfrac{e^x-1}{e^x+1}$;
    \task $y=x^ne^{-x}$;
    \task $y=\frac{a}{2}(e^{\frac{x}{a}}-e^{-\frac{x}{a}})$;
    \task $y=x^3+3^x$;
    \task $y=2^xe^x$;
    \task $y=e^{2x}\ln x$;
    \task $y=e^{x^2+1}$。
  \end{tasks}
\end{Practice}

\subsection{幂函数的导数}
\begin{Practice}
  求下列函数的导数:
  \begin{tasks}(2)
    \task $y=x^{\frac{2}{3}}-2x^{-\frac{1}{2}}+5x^{\frac{7}{6}}$;
    \task $y=\left(\dfrac{1}{\sqrt[3]{x^2}}+\dfrac{1}{\sqrt{x}}\right)^2$;
    \task $y=\sqrt[3]{(4-3x^2)^2}$;
    \task $y=\sqrt[3]{\dfrac{x-a}{x+a}}$。
  \end{tasks}
\end{Practice}

\begin{Exercise}
  \begin{question}
    \item 
    \item 
    \item 
    \item 
    \item 
    \item 
    \item 
    \item 
    \item 
  \end{question}
\end{Exercise}

\subsection{隐函数的导数}
\begin{Practice}
  \begin{question}
    \item 
    \item 
    \item 
    \item 
  \end{question}
\end{Practice}

\subsection{二阶导数}
\begin{Practice}
  \begin{question}
    \item 
    \item 
  \end{question}
\end{Practice}

\begin{Exercise}
  \begin{question}
    \item 
    \item 
    \item 
    \item 
    \item 
    \item 
    \item 
    \item 
  \end{question}
\end{Exercise}

\section{微分}
\subsection{微分概念}
\subsection{微分的运算}
\begin{Practice}
  \begin{question}
    \item 
    \item 
    \item 
    \item 
    \item 
    \item 
  \end{question}
\end{Practice}

\subsection{近似计算}
\begin{Practice}
  \begin{question}
    \item 
    \item 
    \item 
    \item 
  \end{question}
\end{Practice}

\begin{Exercise}
  \begin{question}
    \item 
    \item 
    \item 
    \item 
    \item 
    \item 
    \item 
    \item 
    \item 
  \end{question}
\end{Exercise}

\section*{小结}
\begin{enumerate}[C、,itemindent=4.5em]
  \item 
  \item 
  \item 
  \item 
  \item 
\end{enumerate}

\chapter*{复习参考题\chinese{chapter}}
\section*{A 组}
\begin{question}
  \item 
  \item 
  \item 
  \item 
  \item 
  \item 
  \item 
  \item 
  \item 
  \item 
  \item 
  \item 
  \item 
  \item 
  \item 
  \item 
  \item 
  \item 
  \item 
  \item 
  \item 
  \item 
  \item 
  \item 
\end{question}
\section*{B 组}
\begin{question}
  \item 
  \item 
  \item 
  \item 
  \item 
  \item 
\end{question}