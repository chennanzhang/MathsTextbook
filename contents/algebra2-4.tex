\chapter{行列式和线性方程组}
\phantomsection\pdfbookmark[1]{行列式和线性方程组}{determinant}
\subsection{二阶行列式和二元线性方程组}
\subsubsection{二阶行列式}
\begin{Practice}
  \begin{question}
    \item 
    \item 
  \end{question}
\end{Practice}

\subsubsection{二元线性方程组的解的行列式表示法}
\begin{Practice}
  利用二阶行列式解下列方程组:
  \begin{tasks}(2)
    \task 
    \task 
  \end{tasks}
\end{Practice}

\subsubsection{二元线性方程组的解的讨论}
\begin{Practice}
  解下列关于 $x$、$y$ 的方程组,并进行讨论:
  \begin{tasks}(2)
    \task 
    \task 
  \end{tasks}
\end{Practice}

\begin{Exercise}
  \begin{question}
    \item 
    \item 
    \item 
    \item 
    \item 
    \item 
    \item 
  \end{question}
\end{Exercise}

\subsection{三阶行列式}
\begin{Practice}
  \begin{question}
    \item 
    \item 
    \item 
  \end{question}
\end{Practice}

\subsection{三阶行列式的性质}
\begin{Practice}
  \begin{question}
    \item 
    \item 
    \item 
  \end{question}
\end{Practice}

\subsection{按一行(或一列)展开三阶行列式}
\begin{Practice}
  \begin{question}
    \item 
    \item 
    \item 
    \item 
  \end{question}
\end{Practice}

\subsection{三元线性方程组}
\begin{Practice}
  判断下列方程组是否有唯一解;如果有唯一解,根据克莱姆法则把解求出来。
  \begin{tasks}(2)
    \task 
    \task 
    \task 
    \task 
  \end{tasks}
\end{Practice}

\begin{Exercise}
  \begin{question}
    \item 
    \item 
    \item 
    \item 
    \item 
    \item 
    \item 
    \item 
    \item 
    \item 
    \item 
    \item 
    \item 
  \end{question}
\end{Exercise}

\subsection{三元齐次线性方程组}
\begin{Practice}
  下列齐次线性方程组有没有非零解?如果有,把解集求出来。
  \begin{tasks}(2)
    \task 
    \task 
  \end{tasks}
\end{Practice}

\subsection{四阶行列式和四元线性方程组}
\subsubsection{四阶行列式}
\begin{Practice}
  \begin{question}
    \item 
    \item 
  \end{question}
\end{Practice}

\subsubsection{四元线性方程组}
\begin{Practice}
  利用克莱姆法则求解方程组
  \[ \begin{cases} 2x-y+3z+2w=6,\\3x-3y+3z+2w=5, \\3x-y-z+2w=3, \\ 3x-y+3z-w=4. \end{cases}\]
\end{Practice}

\begin{Exercise}
  \begin{question}
    \item 
    \item 
    \item 
    \item 
    \item 
    \item 
  \end{question}
\end{Exercise}

\subsection{用顺序消元法解线性方程组}
\subsubsection{顺序消元法解线性方程组举例}
\subsubsection{顺序消元法解线性方程组的矩阵表示}
\begin{Practice}
  \begin{question}
    \item 
    \item 
  \end{question}
\end{Practice}


\begin{Exercise}
  \begin{question}
    \item 
    \item 
  \end{question}
\end{Exercise}

\section*{小结}
\begin{enumerate}[C、,itemindent=4.5em]
  \item 
  \item 
  \item 
  \item 
  \item 
  \item 
\end{enumerate}
\chapter*{复习参考题\chinese{chapter}}
\section*{A 组}
\begin{question}
  \item 
  \item 
  \item 
  \item 
  \item 
  \item 
  \item 
  \item 
  \item 
  \item 
  \item 
  \item 
  \item 
  \item 
  \item 
  \item 
  \item 
  \item 
  \item 
  \item 
  \item 
  \item 
  \item 
  \item 
  \item 
\end{question}
\section*{B 组}
\begin{question}[resume]
  \item 
  \item 
  \item 
  \item 
  \item 
\end{question}