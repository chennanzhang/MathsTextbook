\chapter{排列、组合、二项式定理}
\section{排列与组合}
\subsection{基本原理}
我们先看下面的问题:

从甲地到乙地,可以乘火车,也可以乘汽车,还可以乘轮船。
一天中,火车有 4 班,汽车有 2 班,轮船有 3 班。
那么一天中乘坐这些交通工具从甲地到乙地共有多少种不同的走法?

因为一天中乘火车有4种走法,乘汽车有 2 种走法,乘轮船有 3 种走法,每一种走法都可以从甲地到达乙地,因此,一天中乘坐这些交通工具从甲地到乙地共有
\[4+2+3=9\]
种不同的走法。

一般地,有如下原理:
\begin{Theorem}[加法原理]{原理}
  做一件事,完成它可以有 $n$ 类办法,在第一类办法中有 $m_1$ 种不同的方法,在第二类办法中有 $m_2$ 种不同的方法,……,在第 $n$ 类办法中有 $m_n$ 种不同的方法。那么完成这件事共有
\[N=m_1+m_2+\cdots+m_n\]
种不同的方法。
\end{Theorem}

我们再看下面的问题:
由 $A$ 村去 $B$ 村的道路有 3 条,由 $B$ 村去 $C$ 村的道路有 2 条(\cref{fig:2-1})。从 $A$ 村经 $B$ 村去 $C$ 村,共有多少种不同的走法?
\begin{figure}
  \includegraphics{2-1.pdf}
  \caption{}\label{fig:2-1}
\end{figure} 

这里,从 $A$ 村到 $B$ 村有 3 种不同的走法,按这 3 种走法中的每一种走法到达 $B$ 村后,再从 $B$ 村到 $C$ 村又有 2 种不同走法。因此,从 $A$ 村经 $B$ 村去 $C$ 村共有
\[ 3\times 2=6\]
种不同的走法。

一般地,有如下原理:
\begin{Theorem}[乘法原理]{原理}
  做一件事,完成它需要分成 $n$ 个步骤,做第一步有 $m_1$ 种不同的方法,做第二步有 $m_2$ 种不同的方法,……,做第 $n$ 步有 $m_n$ 种不同的方法。那么完成这件事共有
\[N=m_1\times m_2\times\cdots\times m_n\]
种不同的方法。
\end{Theorem}

\begin{example}
书架上层放有 6 本不同的数学书,下层放有 5 本不同的语文书。
\begin{tasks}
  \task 从中任取一本,有多少种不同的取法?
  \task 从中任取数学书与语文书各一本,有多少种不同的取法?
\end{tasks}
\end{example}
\begin{solution}
  \begin{enumerate}
    \item 从书架上任取一本书,有两类办法:第一类办法是从上层取数学书,可以从 6 本书中任取一本,有 6 种方法;第二类办法是从下层取语文书,可以从 5 本书中任取一本,有 5 种方法。根据加法原理,得到不同的取法的种数是
    \[ N=m_1+m_2=6+5=11.\]

    答:从书架上任取一本书,有 11 种不同的取法。
    \item 从书架上任取数学书与语文书各一本,可以分成两个步骤完成:第一步取一本数学书,有 6 种方法,第二步取一本语文书,有 5 种方法。根据乘法原理,得到不同的取法的种数是
    \[ N=m_1\times m_2=6\times 5=30.\]

    答:从书架上取数学书与语文书各一本,有 30 种不同的方法。
  \end{enumerate}
\end{solution}

\begin{example}
  由数字 $1,2,3,4,5$ 可以组成多少个三位数(各位上的数字允许重复)?
\end{example}
\begin{solution}
  要组成一个三位数可以分成三个步骤完成:第一步确定百位上的数字,从 5 个数字种任选一个数字,共有 5 种选法;第二步确定十位上的数字,由于数字允许重复,这仍有 5 种选法;第三步确定个位上的数字,同理,它也有 5 种选法。根据乘法原理得到可以组成的三位数的个数是
  \[ N=5\times 5\times 5=5^3=125.\]

  答:可以组成 125 个三位数。
\end{solution}

\begin{Practice}
  \begin{question}
    \item (口答)一件工作可以用两种方法完成。有 5 人会用第一种方法完成,另有 4 人会用第二种方法完成。选出一个人来完成这件工作,共有多少种选法?
    \item 在读书活动中,一个学生要从 2 本科技书、2 本政治书、3 本文艺书里任选一本,共有多少种不同的选法?
    \item 一名儿童做加法游戏。在一个红口袋中装着 20 张分别标有数 $1,2,\cdots,19,20$ 的红卡片,从中任抽一张,把上面的数作为被加数;在另一个黄口袋中装着 10 张分别标有数 $1,2,\cdots,9,10$ 的黄卡片,从中任抽一张,把上面的数作为加数。这名儿童一共可以列出多个加法式子?
    \item 乘积 $(a_1+a_2+a_3)(b_1+b_2+b_3+b_4)(c_1+c_2+c_3+c_4+c_5)$ 展开后共有多少项?
    \item\label{prac:3-1-5} 如图,从甲地到乙地有 2 条路可通,从乙地到丙地有 3 条路可通;从甲地到丁地有 4 条路可通,从丁地到丙地有 2 条路可通。从甲地到丙地共有多少种不同的走法?
    \begin{figurehere}
      \begin{minipage}{\linewidth}\centering
        \includegraphics{pr3-1-5.pdf}
        \caption*{(第~\ref{prac:3-1-5}~题图)}
      \end{minipage}
    \end{figurehere}
    \item 一个口袋内装有 5 个小球,另一个口袋内装有 4 个小球,所有这些小球的颜色互不相同。
    \begin{tasks}
      \task 从两个口袋内任取一个小球,有多少种不同的取法?
      \task 从两个口袋内各取一个小球,有多少种不同的取法?
    \end{tasks}
    \item\label{prac:3-1-7} 如图,从甲地到乙地有 2 条陆路可走,从乙地到丙地有 3 条陆路可走,又从甲地不经过乙地到丙地有 2 条水路可走。
    \begin{figurehere}
      \begin{minipage}{\linewidth}\centering
        \includegraphics{pr3-1-7.pdf}
        \caption*{(第~\ref{prac:3-1-7}~题图)}
      \end{minipage}
    \end{figurehere}
    \begin{tasks}
      \task 从甲地经乙地到丙地有多少种不同的走法?
      \task 从甲地到丙地共有多少种不同的走法?
    \end{tasks}
  \end{question}
\end{Practice}

\subsection{排列}\label{subsec:Permutation}
我们看下面的问题:
\begin{enumerate}[1.]
  \item 北京、上海、广州三个民航站之间的直达航线,需要准备多少种不同的飞机票?
  
  这个问题就是从北京、上海、广州三个民航站中,每次取出两个站,按照起点站在前、终点站在后的顺序排列,求一共有多少种不同的排法。

  首先确定起点站,在三个站中,任选一个站为起点站,有 3 种方法;其次确定终点站,当选定起点站后,终点站就只能在其余的两个站中去选,因此,有 2 种方法。那么,根据乘法原理,在三个民航站中,每次取两个,按起点站在前、终点站在后的顺序排列的不同方法共有 
\[3\times 2=6\]
种。也就是说,需要准备如下 6 种不同的飞机票:
  \begin{center}
    \begin{tblr}{colspec={cccc},hlines=0pt,vlines=0pt,rows={ht=5ex}}
      起点站 & 终点站 & \qquad\qquad & 飞机票 \\
      \SetCell[r=2]{m,c}{\tikz[remember picture,inner sep=0pt]{\node (n11){北京};}} & \tikz[remember picture,inner sep=0pt]{\node (n1){上海};} & \qquad\qquad & {北京——上海} \\
       & \tikz[remember picture,inner sep=0pt]{\node (n2){广州};} & \qquad\qquad & {北京——广州} \\
      \SetCell[r=2]{m,c}{\tikz[remember picture,inner sep=0pt]{\node (n22){上海};}} & \tikz[remember picture,inner sep=0pt]{\node (n3){北京};} & \qquad\qquad & {上海——北京} \\
       & \tikz[remember picture,inner sep=0pt]{\node (n4){广州};} & \qquad\qquad & {上海——广州} \\
      \SetCell[r=2]{m,c}{\tikz[remember picture,inner sep=0pt]{\node (n33){广州};}} & \tikz[remember picture,inner sep=0pt]{\node (n5){北京};} & \qquad\qquad & {广州——北京} \\
       & \tikz[remember picture,inner sep=0pt]{\node (n6){上海};} & \qquad\qquad & {广州——上海} \\
    \end{tblr}
    \tikz[overlay,remember picture]{
      \draw(n11.east)--(n1.west)(n11.east)--(n2.west);
      \draw(n22.east)--(n3.west)(n22.east)--(n4.west);
      \draw(n33.east)--(n5.west)(n33.east)--(n6.west);
    }
  \end{center}
  \item 由数字 $1,2,3,4$ 可以组成多少个没有重复数字的三位数?
  
  这个问题就是从 $1,2,3,4$ 这四个数字中,每次取出三个,按照百位、十位、个位的顺序排列起来,求一共有多少种不同的排法。

  第一步,先确定百位上的数字,在 $1,2,3,4$ 这四个数字种任取一个,有 4 种方法;

  第二步,确定十位上的数字,当百位上的数字确定以后,十位上的数字只能从余下的三个数字中去取,有 3 种方法;

  第三步,确定个位上的数字,当百位、十位上的数字都确定以后,个位上的数字只能从余下的两个数字中去取,有 2 种方法。

  根据乘法原理,从四个不同的数字中,每次取出三个排成一个三位数的方法共有
  \[ 4\times 3\times 2=24\]
  种。也就是说,可以排成 24 个不同的三位数。具体排法如下:
  \begin{center}
    \includegraphics{2-a.pdf}
  \end{center}
\end{enumerate}

我们把被取的对象(如上面问题中的民航站、数字)叫做\Concept{元素}。上面第一个问题,就是从 3 个不同的元素中,任取 2 个,然后按一定的顺序排成一列,求一共有多少种不同的排法;第二个问题,就是从 4 个不同的元素中,任取 3 个,然后按一定的顺序排成一列,求一共有多少种不同的排法。

一般地说,从 $n$ 个不同元素中,任取 $m$($m\leqslant n$)个元素(本章只研究被取出的元素各不相同的情况),按照一定的顺序排成一列,叫做从 $n$ 个不同元素中取出 $m$ 个元素的一个\Concept{排列}。

从排列的定义知道,如果两个排列相同,不仅这两个排列的元素必须完全相同,而且排列的顺序也必须完全相同。如果所取的元素不完全相同,例如问题 1 中的飞机票“上海——北京”和“上海——广州”,它们就是两个不同的排列。即使所取的元素完全相同,但排列顺序不同,也不是相同的排列。如问题 2 中的三位数“213”和“231”,虽然它们的元素相同,但排列顺序不同,也是两个不同的排列。

在实际问题中,有时需要写出某个排列问题的所有排列。例如,已知 $a,b,c,d$ 这 4 个元素,写出每次取出 3 个元素的所有排列,可以先列出下图(见\cref{fig:2-2}):
\begin{figure}
  \begin{minipage}{0.2\linewidth}\centering
    \includegraphics{2-2a.pdf}
    \subcaption{}\label{fig:2-2a}
  \end{minipage}
  \begin{minipage}{0.2\linewidth}\centering
    \includegraphics{2-2b.pdf}
    \subcaption{}\label{fig:2-2b}
  \end{minipage}
  \begin{minipage}{0.2\linewidth}\centering
    \includegraphics{2-2c.pdf}
    \subcaption{}\label{fig:2-2c}
  \end{minipage}
  \begin{minipage}{0.2\linewidth}\centering
    \includegraphics{2-2d.pdf}
    \subcaption{}\label{fig:2-2d}
  \end{minipage}
  \caption{}\label{fig:2-2}
\end{figure}

\noindent 由此可以写出所有的排列:
\begin{center}
  \begin{tabular}{c*{3}{@{\hspace{1em}}c}}
    $abc$ & $bac$ & $cab$ & $dab$ \\
    $abd$ & $bad$ & $cad$ & $dac$ \\
    $acb$ & $bca$ & $cba$ & $dba$ \\
    $acd$ & $bcd$ & $cbd$ & $dbc$ \\
    $adc$ & $bda$ & $cda$ & $dca$ \\
    $adc$ & $bdc$ & $cdb$ & $dcb$ \\
  \end{tabular}
\end{center}

\subsection{排列数公式}
从 $n$ 个不同元素中取出 $m$($m\leqslant n$)个元素的所有排列的个数,叫做从 $n$ 个不同元素中取出 $m$ 个元素的\Concept{排列数},用符号 $P_n^m$ 表示\footnote{$P$ 是英文 Permutation(排列)的第一个字母。}。

例如,从 6 个不同元素中取出 5 个元素的排列数表示为 $P_8^5$,从 7 个不同元素中取出 6 个元素的排列数表示为 $P_7^6$。

现在我们研究计算排列数的公式。

求排列数 $P_n^2$ 可以这样考虑:假定有排好顺序的 2 个空位(\cref{fig:2-3}),从 $n$ 个不同元素 $a_1,a_2,\cdots,a_n$ 中任意取 2 个去填空,一个空位填一个元素,每一种填法就得到一个排列;反过来,任一个排列总可以由这样的一种填法得到。因此,所有不同填法的种数就是排列数 $P_n^2$。
\begin{figure}
  \includegraphics{2-3.pdf}
  \caption{}\label{fig:2-3}
\end{figure}

现在我们计算有多少种不同的填法,完成这件事可分为两个步骤:

第一步,先排第一个位置的元素,可以从这 $n$ 个元素中任选一个填空,有 $n$ 种方法;

第二步,确定排在第二个位置的元素,可以从剩下的 $n-1$ 个元素中任选一个填空,有 $n-1$ 种方法。

于是,根据乘法原理,得到排列数为
\[ P_n^2 =n(n-1).\]

求排列数 $P_n^3$ 可以按依次填 3 个空位来考虑,得到
\[ P_n^3=n(n-1)(n-2).\]

同样,求排列数 $P_n^m$ 可以这样考虑:假定有排好顺序的 $m$ 个空位(\cref{fig:2-4}),从 $n$ 个不同元素 $a_1,a_2,\cdots,a_n$ 中任意取 $m$ 个去填空,一个空位填一个元素,每一种填法就得到一个排列;反过来,任一个排列总可以由一种填法得到。因此,所有不同填法的种数就是排列数 $P_n^m$。

现在我们计算共有多少中不同的填法(\cref{fig:2-4}):
\begin{figure}
  \includegraphics{2-4.pdf}
  \caption{}\label{fig:2-4}
\end{figure}

第一步,第 1 位可以从 $n$ 个元素中,任选一个填上,共有 $n$ 种填法;

第二步,第 2 位只能从余下的 $n-1$ 个元素中,任选一个填上,共有 $n-1$ 种填法;

第三步,第 3 位只能从余下的 $n-2$ 个元素中,任选一个填上,共有 $n-2$ 种填法;

依次类推,当前面的 $m-1$ 个空位都填上后,第 $m$ 位只能从余下的 $n-(m-1)$ 个元素中,任选一个填上,共有 $n-m+1$ 种填法。

根据乘法原理,全部填满 $m$ 个空位共有
\[n(n-1)(n-2)\cdots(n-m+1)\]
种填法。

所以得到公式
\[\tcbhighmath{P_n^m=n(n-1)(n-2)\cdots(n-m+1).}\]
这里 $n,m\in\mathbb{N}$,并且 $m\leqslant n$。这个公式叫做\Concept{排列数公式}。其中,公式右边种第一个因数是 $n$,后面的每个因数都比它前面一个因数少 1,最后一个因数为 $n-m+1$,共有 $m$ 个因数相乘。

例如,$P_8^3=8\times 7\times 6=336.$

排列数公式中,当 $m=n$ 时,有
\[ P_n^n=n\cdot(n-1)\cdot(n-2)\cdot\cdots\cdot 3\cdot 2\cdot 1.\]
这个公式指出,$n$ 个不同元素全部取出的排列数,等于自然数 1 到 $n$ 的连乘积。$n$ 个不同元素全部取出的一个排列,叫做 $n$ 个不同元素的一个\Concept{全排列}。自然数 1 到 $n$ 的连乘积,叫做 $n$ 的\Concept{阶乘},用 $n!$ 表示,所以 $n$ 个不同元素的全排列公式可以写成
\[ P_n^n=n!.\]

排列数公式可作如下变形:
\begin{align*}
  P_n^m&=n(n-1)(n-2)\cdots(n-m+1)\\
       &=\frac{n\cdot(n-1)\cdot(n-2)\cdot\cdots\cdot(n-m+1)\cdot(n-m)\cdot\cdots\cdot 2\cdot 1}{(n-m)\cdot\cdots\cdot 2\cdot 1}\\
       &=\frac{n!}{(n-m)!},
\end{align*}
因此,排列数公式还可写成
\[\tcbhighmath{P_n^m=\frac{n!}{(n-m)!}.}\]

\alertinfo{为了使这个公式在 $m=n$ 时也能成立,我们规定 \[0!=1.\]}

\begin{example}
  计算 $P_{16}^3$ 及 $P_6^6$。
\end{example}
\begin{solution}
  $P_{16}^3=16\times 15\times 14=3360$;

  $P_6^6=6!=720$。
\end{solution}

\begin{example}
  求证 $P_n^m+mP_n^{m-1}=P_{n+1}^m$。
\end{example}
\begin{proof}
  \begin{align*}
    P_n^m+mP_n^{m-1}&= \frac{n!}{(n-m)!}+\frac{n!}{[n-(m-1)]!} \\
    &= \frac{n!(n-m+1)}{(n-m+1)!}+\frac{m\cdot n!}{(n-m+1)} \\
    &= \frac{n!(n-m+1+m)}{(n-m+1)!}\\
    &=\frac{(n+1)!}{[(n+1)-m]!}=P_{n+1}^m.\\
    \therefore\quad P_n^m+mP_n^{m-1}&=P_{n+1}^m.
  \end{align*}
\end{proof}

\begin{example}
  某段铁路上有 12 个车站,共需要准备多少种普通客票?
\end{example}
\begin{solution}
  因为每一张车票对应着 2 个车站的一个排列,因此需要准备的车票种数,就是从 12 个车站中任取 2 个的排列数:
  \[ P_{12}^2=12\times 11=132 \text{(种)}.\]

  答:一共需要准备 132 种普通客票。
\end{solution}

\begin{example}
  某信号兵用红、黄、蓝三面旗从上到下挂在竖直的旗杆上表示信号,每次可以任挂一面、二面或三面,并且不同的顺序表示不同的信号,一共可以表示多少种不同的信号?
\end{example}
\begin{solution}
  如果把 3 面旗看成 3 个元素,则从 3 个元素里每次取出 1 个元素的一个排列,对应一种信号。于是,只用 1 面旗表示的信号是 $P_3^1$ 种。

  同样,只用二面旗表示的信号共有 $P_3^2$ 种,只用 3 面旗表示的信号共有 $P_3^3$ 种。

  根据加法原理,所求的信号种数是
  \[P_3^1+P_3^2+P_3^3=3+3\times 2+3\times 2\times 1=15 \text{(种)}.\]

  答:一共可以表示 15 种不同的信号。
\end{solution}

\begin{example}
  用 0 到 9 这十个数字,可以组成多少个没有重复数字的三位数?
\end{example}
\begin{analyze}[分析一]
  因为要用 0 到 9 这十个数字组成三位数,每一个三位数可以看成是从这十个数字中任取 3 个的一个排列(0 排在首位的除外),由于百位上的数字不能是 0,我们可以分成两个步骤考虑:先排百位上的数字,再排十位和个位上的数字。
\end{analyze}\par\medskip
\begin{solution}[解法一]
  百位上的数字只能从除 0 以外的 1 到 9 这九个数字中任选一个,有 $P_9^1$ 种;十位和个位上的数字,可以从余下的九个数字中任选两个,有 $P_9^2$ 种(如\cref{fig:2-5})。根据乘法原理,所求的三位数的个数是
\[P_9^1\cdot P_9^2=9\times 9\times 8=648.\]
\begin{figure}
  \includegraphics{2-5.pdf}
  \caption{}\label{fig:2-5}
\end{figure}
\end{solution}

\begin{analyze}[分析二]
从 0 到 9 这十个数字中任取三个数字的排列数,减去其中以 0 为排头的排列数,就是用这十个数字组成的没有重复数字的三位数的个数。
\end{analyze}\par\medskip
\begin{solution}[解法二]
  从 0 到 9 这十个数字中任取三个数字的排列数为 $P_{10}^3$,其中以 0 为排头的排列数为 $P_9^2$,因此所求的三位数的个数是
\[ P_{10}^3-P_9^2=10\times 9\times 8-9\times 8=648.\]
\end{solution}

\begin{solution}[解法三]
  如\cref{fig:2-6},符合条件的三位数可以分为三类:

  每一位数字都不是 0 的三位数有 $P_9^3$ 个;

  个位数字是 0 的三位数有 $P_9^2$ 个;

  十位数字是 0 的三位数有 $P_9^2$ 个。
  \begin{figure}
    \begin{minipage}{0.32\linewidth}\centering
      \includegraphics{2-6a.pdf}
      \subcaption{}\label{fig:2-6a}
    \end{minipage}
    \begin{minipage}{0.32\linewidth}\centering
      \includegraphics{2-6b.pdf}
      \subcaption{}\label{fig:2-6b}
    \end{minipage}
    \begin{minipage}{0.32\linewidth}\centering
      \includegraphics{2-6c.pdf}
      \subcaption{}\label{fig:2-6c}
    \end{minipage}
    \caption{}\label{fig:2-6}
  \end{figure}

  根据加法原理,符合条件的三位数的个数是
  \[ P_9^3+P_9^2+P_9^2=648.\]

  答:可以组成 648 个没有重复数字的三位数。
\end{solution}

\begin{Practice}
  \begin{question}
    \item 写出:
    \begin{tasks}
      \task 从四个元素 $a,b,c,d$ 中任取两个元素的所有排列;
      \task 从五个元素 $a,b,c,d,e$ 中任取两个元素的所有排列。
    \end{tasks}
    \item 计算:
    \begin{tasks}[after-item-skip=7pt,after-skip=5pt](4)
      \task $P_5^2$;
      \task $P_{15}^4$;
      \task $P_{100}^3$;
      \task $P_7^7$;
      \task $P_6^3$;
      \task $P_8^4-2P_8^2$;
      \task $\dfrac{P_{12}^8}{P_{12}^7}$。
    \end{tasks}
    \item 填写下面的阶乘表,并计算出各阶乘数:
    \begin{tablehere}
      \begin{minipage}{\linewidth}
        \begin{tblr}{colspec={*{8}{X[c]}},hline{2}=0.8pt}
          $n$  & 2 & 3 & 4 & 5 & 6 & 7 & 8 \\
          $n!$ &   &   &   &   &   &   &   \\
        \end{tblr}
      \end{minipage}
    \end{tablehere}
    \item 求证:
    \begin{tasks}[before-skip=5pt,after-skip=5pt](2)
      \task $n!=\dfrac{(n+1)!}{n+1}$;
      \task $P_8^8-8P_7^7+7P_6^6=P_7^7$。
    \end{tasks}
    \item 求 $n$:
    \[\frac{P_n^7-P_n^5}{P_n^5}=89\]
    \item 6 名同学排成一排照相,有多少种排法?
    \item 从 4 种蔬菜品种中选出 3 种,分别种植在不同土质的 3 块土地上进行试验,有多少种种植方法?
    \item 用 $1,2,3,4,5$ 这五个数字,可以组成多少个没有重复数字的四位数?其中有多少个四位数是 5 的倍数?
  \end{question}
\end{Practice}

\begin{Exercise}
  \begin{question}
    \item 计算:
    \begin{tasks}[after-skip=5pt,before-skip=5pt](3)
      \task $P_{10}^4$;
      \task $5P_5^3+4P_4^2$;
      \task $\dfrac{P_7^5-P_6^6}{7!+6!}$。
    \end{tasks}
    \item 求证:
    \begin{tasks}[after-skip=5pt,after-item-skip=5pt](2)
      \task  $P_n^m=nP_{n-1}^{m-1}$;
      \task  $P_{n+1}^{n+1}-P_n^n=n^2P_{n-1}^{n-1}$;
      \task! $\dfrac{(n+1)!}{k!}-\dfrac{n!}{(k-1)!}=\dfrac{(n-k+1)\cdot n!}{k!}$。
    \end{tasks}
    \item 求 $n$:
    \begin{tasks}[after-skip=5pt,before-skip=5pt](2)
      \task $P_{2n}^3=10P_n^3$;
      \task $\dfrac{P_n^5+P_n^4}{P_n^3}=4$。
    \end{tasks}
    \item 解答:
    \begin{tasks}
      \task 从多少个不同的元素中取出 2 个元素的排列数是 56?
      \task 已知从 $n$ 个不同的元素中取出 2 个元素的排列数等于从 $n-4$ 个不同的元素中取出 2 个元素的排列数的 7 倍,求 $n$。
    \end{tasks}
    \item 有 5 本不同的书,准备给 3 名同学,每人 1 本,共有多少种给法?
    \item 一个火车站有 8 股岔道,停放 4 列不同的火车,有多少种不同的停放方法(假定每股岔道只能停放一列火车)?
    \item 一部纪录影片在 4 个单位轮映,每一单位放映 1 场,可有几种轮映次序?
    \item 解答:
    \begin{tasks}
      \task 由数字 $1,2,3,4,5,6$ 可以组成多少个没有重复数字的五位数?
      \task 由数字 $0,1,2,3,4,5$ 可以组成多少个没有重复数字的五位数?
    \end{tasks}
    \item 解答:
    \begin{tasks}
      \task 由数字 $1,2,3,4,5$ 可以组成多少个没有重复数字的自然数?
      \task 由数字 $1,2,3,4,5$ 可以组成多少个没有重复数字,并且比 \num{13000} 大的自然数?
    \end{tasks}
    \item 7 个人并排站成一排:
    \begin{tasks}
      \task 如果甲必须站在正中间,有多少种排法?
      \task 如果甲、乙两人必须站在两端,有多少种排法?
    \end{tasks}
  \end{question}
\end{Exercise}

\subsection{组合}
我们看下面的问题:

在北京、上海、广州三个民航站之间的直达航线,有多少
种不同的飞机票价?

这个问题与\cref{subsec:Permutation}中计算飞机票种数的问题不同,飞机票的种数与起点站、终点站有关,从北京到上海和从上海到北京,飞机票是不同的,也就是与顺序有关;但飞机票价只与起点站和终点站之间的距离有关,从北京到上海和从上海到北京,飞机票价是相同的,也就是与顺序无关。

因此,\cref{subsec:Permutation}中计算飞机票种数的问题,是从三个不同的元素中任取两个,然后按照一定的顺序排列,求一共有多少种不同的排列方法,这是排列问题;而本节这个问题,是从三个不同的元素中任取两个,不管怎样的顺序并成一组,求一共有多少个不同的组,这就是要研究的组合问题。

一般地说,从 $n$ 个不同元素中,任取 $m(m\leqslant n)$ 个元素并成一组,叫做从 $n$ 个不同元素中取出 $m$ 个元素的一个\Concept{组合}。

上面问题中要确定有几种不同的飞机票价,就是要求从 3 个不同的元素中取出 2 个元素的所有组合的个数。
因为上海到广州和广州到上海的飞机票价是相同的,所以过两站间的飞机票价就是从北京、上海、广州这三个不同元素中取出上海、广州这两个元素的一个组合。

如果两个组合中的元素完全相同,不管元素的顺序如何,都是相同的组合;只有当两个组合中的元素不完全相同时,才是不同的组合。例如,从 $a,b,c$ 三个不同的元素中取出两个
元素的所有组合有 3 个,它们分别是:
\[ ab,\qquad ac,\qquad bc.\]
组合 $ab$ 与组合 $ba$ 是相同的组合,而组合 $ab$ 与组合 $ac$ 是不同的组合。

从排列和组合的定义可以知道,排列与元素的顺序有关,组合与顺序无关,例如 $ab$ 与 $ba$ 是两个不同的排列,但它们却是同一个组合。

在实际问题中,有时需要写出某个组合问题的所有组合。例如,已知 $a,b,c,d$ 这 4 个元素,写出每次取出 2 个元素的所有组合,可以先列出下图(\cref{fig:2-7}):
\begin{figure}
  \includegraphics{2-7.pdf}
  \caption{}\label{fig:2-7}
\end{figure}

如\cref{fig:2-7} 所表示的,先把 $a$ 从左到右依次与 $b,c,d$ 组合,再把 $b$ 依次与 $c,d$ 组合,再把 $c$ 与 $d$ 组合,由此可以写出所有的组合:
\[ ab,\qquad ac,\qquad ad,\qquad bc,\qquad bd,\qquad cd.\]

\subsection{组合数公式}
从 $n$ 个不同元素中取出 $m$($m\leqslant n$)个元素的所有组合的个数,叫做从 $n$ 个不同元素中取出 $m$ 个元素的\Concept{组合数},用符号 $C_n^m$ 表示\footnote{$C$ 是英文 Combination(组合)的第一个字母。}。

例如,从 8 个不同元素中取出 5 个元素的组合数表示为 $C_8^5$;从 7 个不同元素中取出 6 个元素的组合数表示为 $C_7^6$。

现在我们从研究组合数 $C_n^m$ 与排列数 $P_n^m$ 的关系入手,找出组合数 $C_n^m$ 的计算公式。

例如,从 4 个不同元素 $a,b,c,d$ 中取出 3 个元素的排列与组合的关系如下表所示:
\begin{center}
  \begin{tblr}{colspec={*{5}{X[c]}},hlines=0pt,vlines=0pt,
    hline{2,4,5,7,8,10,11,13}={3-5}{black,0.4pt},
    vline{3,6}={2-3,5-6,8-9,11-12}{black,0.4pt}
    }
    组合 & & \SetCell[c=3]{m,c} 排列 & & \\ 
    \SetCell[r=2]{m,c}\fbox{$a\quad b\quad c$} 
    &  \SetCell[r=2]{m,c} $\xlongrightarrow{}$ & $a\quad b\quad c$ & $b\quad a\quad c$ & $c\quad a\quad b$ \\ 
    &  & $a\quad c\quad b$ & $b\quad c\quad a$ & $c\quad b\quad a$ \\ 
    &&&&\\
    \SetCell[r=2]{m,c}\fbox{$a\quad b\quad d$} 
    &  \SetCell[r=2]{m,c} $\xlongrightarrow{}$ & $a\quad b\quad d$ & $b\quad a\quad d$ & $d\quad a\quad b$ \\ 
    &  & $a\quad d\quad b$ & $b\quad d\quad a$ & $d\quad b\quad a$ \\ 
    &&&&\\
    \SetCell[r=2]{m,c}\fbox{$a\quad c\quad d$} 
    &  \SetCell[r=2]{m,c} $\xlongrightarrow{}$ & $a\quad d\quad d$ & $c\quad a\quad d$ & $d\quad a\quad c$ \\ 
    &  & $a\quad d\quad c$ & $c\quad d\quad a$ & $d\quad c\quad a$ \\ 
    &&&&\\
    \SetCell[r=2]{m,c}\fbox{$b\quad c\quad d$} 
    &  \SetCell[r=2]{m,c} $\xlongrightarrow{}$ & $b\quad c\quad d$ & $c\quad b\quad d$ & $d\quad b\quad c$ \\ 
    &  & $b\quad d\quad c$ & $c\quad d\quad b$ & $d\quad c\quad b$ \\ 
  \end{tblr}
\end{center}

由表中可以看出,对于每一个组合都有 6 个不同的排列,因此,求从 4 个不同元素中取 3 个元素的排列数 $P_4^3$,可以分以下两步完成:

第一步,从 4 个不同元素中取出 3 个元素作组合,共有 $C_4^3$($=4$)个;

第二步,对每一个组合中的 3 个不同元素作全排列,各有 $P_3^3$($=6$)个。

根据乘法原理,得
\[ P_4^3=C_4^3\cdot P_3^3,\]

因此,
\[C_4^3 =\dfrac{ P_4^3}{ P_3^3}.\]

一般地,求从 $n$ 个不同元素中取出 $m$ 个元素得排列数 $P_n^m$,可分以下两步完成:

第一步,先求出从这 $n$ 个不同的元素中取出 $m$ 个元素的组合数 $C_n^m$;

第二步,求每一个组合中 $m$ 个元素的全排列数 $P_m^m$。

根据乘法原理,得到
\[ P_n^m=C_n^m\cdot P_m^m,\]
因此
\[\tcbhighmath{C_n^m=\frac{P_n^m}{P_m^m}=\frac{n(n-1)(n-2)\cdots(n-m+1)}{m!}.}\]
这里 $n,m\in\mathbb{N}$,并且 $m\leqslant n$。这个公式叫做\Concept{组合数公式}。

\medskip
因为
\[ P_n^m=\frac{n!}{(n-m)!},\]
所以,上面的组合数公式还可以写成
\[\tcbhighmath{C_n^m=\frac{n!}{m!(n-m)!}.}\]
这也是组合数的一个常用公式。

\begin{example}
  计算 $C_{10}^4$ 及 $C_7^3$。
\end{example}
\begin{solution}
  \begin{align*}
    C_{10}^4&=\frac{10\times 9\times 8\times 7}{4\times 3\times 2\times 1}=210;\\[7pt]
    C_7^3&=\frac{7\times 6\times 5}{3\times 2\times 1}=35.
  \end{align*}
\end{solution}

\begin{example}
求证 $C_n^m=\dfrac{m+1}{n-m}\cdot C_n^{m+1}$。
\end{example}
\begin{proof}
  \begin{align*}
    \because \qquad C_n^m&=\frac{n!}{m!(n-m)!},\\[7pt]
    \frac{m+1}{n-m}\cdot C_n^{m+1}&=\frac{m+1}{n-m}\cdot\frac{n!}{(m+1)!(n-m-1)!}\\[7pt]
    &=\frac{m+1}{(m+1)!}\cdot\frac{n!}{(n-m)(n-m-1)!}\\[7pt]
    &=\frac{n!}{m!(n-m)!}\\[7pt]
  \therefore\qquad  C_n^m&=\frac{m+1}{n-m}\cdot C_n^{m+1}.
  \end{align*}
\end{proof}


\subsection{组合数的两个性质}
\begin{Theorem}{定理 1}
  \[\tcbhighmath{C_n^m=C_n^{n-m}.}\]
\end{Theorem}
\begin{proof}
\begin{align*}
  \because \qquad C_n^m&=\frac{n!}{m!(n-m)!},\\[7pt]
  C_n^{n-m}&=\frac{n!}{(n-m)![n-(n-m)]!}=\frac{n!}{m!(n-m)!},\\
  \therefore \qquad C_n^m&=C_n^{n-m}.
\end{align*}
\end{proof}

这个性质也可以根据组合的定义得出。从 $n$ 个不同元素中取出 $m$ 个元素后,剩下 $n-m$ 个元素,也就是说,从 $n$ 个不同元素中取出 $m$ 个元素的每一个组合,都对应着从 $n$ 个不同元素中取出 $n-m$ 个元素的唯一的一个组合;反过来也是一样。因此,从 $n$ 个不同元素中取出 $m$ 个元素的组合数 $C_n^m$,等于从 $n$ 个不同元素中取出 $n-m$ 个元素的组合数 $C_n^{n-m}$,即
\[C_n^m=C_n^{n-m}.\]

当 $m>\frac{n}{2}$ 时,通常不直接计算 $C_n^m$,而是改为计算 $C_n^{n-m}$,这样比较简便。例如,$C_9^7$ 可以这样计算:
\[ C_9^7=C_9^{9-7}=C_9^2=\dfrac{9\times 8}{2!}=36.\]

\alertinfo{为了使这个公式在 $n=m$ 时也成立,我们规定 \[C_n^0=1.\]}

\begin{Theorem}{定理 2}
  \[\tcbhighmath{C_{n+1}^m=C_n^m+C_n^{m-1}.}\]
\end{Theorem}
\begin{proof}
\begin{align*}
  C_n^m+C_n^{m-1}&=\frac{n!}{m!(n-m)!}+\frac{n!}{(m-1)![n-(m-1)]!}\\[7pt]
  &=\frac{n!(n-m+1)+n!m}{m!(n-m+1)!}\\[7pt]
  &=\frac{(n-m+1+m)n!}{m!(n+1-m)!}\\[7pt]
  &=\frac{(n+1)!}{m![(n+1)-m]!}\\[7pt]
  &=C_{n+1}^m,\\
  \therefore C_{n+1}^m&=C_n^m+C_n^{m-1}.
\end{align*}
\end{proof}

这个性质也可以根据组合的定义和加法原理得出。从 $a_1,a_2,\cdots,a_{n+1}$ 这 $n+1$ 个不同的元素中取出 $m$ 个的组合数是 $C_{n+1}^m$,这些组合可以分成两类,一类含有 $a_1$,一类不含 $a_1$。含有 $a_1$ 的组合是从 $a_2,a_3,\cdots,a_{n+1}$ 这 $n$ 个元素中取出 $m-1$ 个元素与 $a_1$ 组成的,共有 $C_n^{m-1}$ 个;不含 $a_1$ 的组合是从 $a_2,a_3,\cdots,a_{n+1}$ 这 $n$ 个元素中取出 $m$ 个元素组成的,共有 $C_n^m$ 个。根据加法原理,得
\[ C_{n+1}^m=C_n^{m-1}+C_n^m.\]

\begin{example}
  计算 $C_{200}^{198}$ 及 $C_{99}^3+C_{99}^2$。
\end{example}
\begin{solution}
  由定理 1,得
  \[ C_{200}^{198}=C_{200}^{2}=\frac{200\times 199}{2\times 1}=19900;\]

  由定理 2,得
  \[C_{99}^3+C_{99}^2=C_{100}^3=\frac{100\times 99\times 98}{3\times 2\times 1}=161700.\]
\end{solution}

\begin{example}
  平面内有 12 个点,任何 3 点不在同一直线上,以每 3 点为顶点画一个三角形,一共可画多少个三角形?
\end{example}
\begin{solution}
  以平面内 12 个点中得每 3 个点为顶点画三角形,可画的三角形的个数,就是从 12 个不同的元素中取出 3 个元素的组合数,即
  \[ C_{12}^3=\frac{12\times 11\times 10}{3\times 2\times 1}=220.\]

  答:一共可画 220 个三角形。
\end{solution}

\begin{example}
  有 13 个队参加篮球赛,比赛时先分成两组,第一组 7 个队,第二组 6 个队。各组都进行单循环赛(即每队都要与本组其他各队比赛一场),然后由各组的前两名共 4 个队进行单循环赛决定冠军、亚军。共需要比赛多少场?
\end{example}
\begin{solution}
  根据题意,第一组是 7 个队,单循环赛的比赛场数是 $C_7^2$,第二组 6 个队,单循环赛的比赛场数是 $C_6^2$;各组的前两名共 4 个队再进行单循环赛时,还要比赛 $C_4^2$ 场。所以共需要比赛的场数是
  \[ C_7^2+C_6^2+C_4^2=21+15+6=42.\]

  答:这次篮球赛共需要比赛 42 场。
\end{solution}

\begin{example}
  在产品检验时,常从产品中抽出一部分进行检查。现在从 100 件产品中任意抽出 3 件:
  \begin{tasks}
    \task 一共有多少种不同的抽法?
    \task 如果 100 件产品中有 2 件次品,抽出的 3 件中恰好有 1 件是次品的抽法有多少种?
    \task 如果 100 件产品中有 2 件次品,抽出的 3 件中至少有 1 件是次品的抽法有多少种?
  \end{tasks}
\end{example}
\begin{solution}
\begin{enumerate}
  \item 所求的不同抽法的种数,就是从 100 件产品中取出 3 件的组合数:
  \[C_{100}^3=\frac{100\times 99\times 98}{3\times 2\times 1}=161700\]

  答:共有 161700 种抽法。
  \item 从 2 件次品种抽出 1 件次品的抽法有 $C_2^1$ 种,从 98 件合格品中抽出 2 件合格品的抽法有 $C_{98}^2$ 种,因此抽出的 3 件中恰好有 1 件是次品的抽法的种数是
  \[ C_2^1\cdot C_{98}^2= 2\times 4753=9506.\]
  
  答:3 件中恰好有 1 件是次品的抽法有 9506 种。
  \item 从 100 件产品种抽出 3 件,一共有 $C_{100}^3$ 种抽法,在这些抽法里,除掉抽出的 3 件都是合格品的抽法 $C_{98}^3$ 种,便得抽出的 3 件中至少有 1 件是次品的抽法的种数,即
  \[ C_{100}^3-C_{98}^3=161700-152096=9604.\]

  本小题也可以这样来解:

  从 100 件产品抽出的 3 件中至少有 1 件是次品的抽法,包括 1 件是次品的和 2 件是次品的,其中 1 件是次品的抽法有 $C_{98}^2\cdot C_2^1$ 种,2 件是次品的抽法有 $C_{98}^1\cdot C_2^2$ 种。因此,至少有 1 件是次品的抽法的种数为
  \[ C_{98}^2\cdot C_2^1+C_{98}^1\cdot C_2^2=9506+98=9604.\]

  答:3 件中至少有 1 件是次品的抽法有 9604 种。
\end{enumerate}
\end{solution}

\begin{Practice}
  \begin{question}
    \item 北京、上海、天津、广东四个足球队举行单循环赛:
    \begin{tasks}
      \task 列出所有各场比赛的双方;
      \task 列出所有冠亚军的可能情况。
    \end{tasks}
    \item 已知平面内不在同一直线上的三点 $A,B,C$:
    \begin{tasks}
      \task 写出连结任意两点的所有线段;
      \task 写出连结任意两点的所有有向线段。
    \end{tasks}
    \item\label{prac:3-3-3} 写出:
    \begin{tasks}
      \task\label{tsk:3-3-3-1}从五个元素 $a,b,c,d,e$ 中任取两个元素的所有组合;
      \task 从五个元素 $a,b,c,d,e$ 中任取三个元素的所有组合。
    \end{tasks}
    \item 利用第~\ref{prac:3-3-3}~题第~\ref{tsk:3-3-3-1}~小题的结果写出从五个元素 $a,b,c,d,e$ 中任取两个元素的所有排列。
    \item 计算:
    \begin{tasks}(2)
      \task $C_6^2$;
      \task $C_6^3$;
      \task $C_{100}^{96}$;
      \task $C_7^3-C_6^2$;
      \task $C_5^1+C_5^2+C_5^3+C_5^4+C_5^5$;
      \task $3C_8^3-2C_5^2$。
    \end{tasks}
    \item 从 $3,5,7,11$ 这四个质数中任取两个相乘,可以得到多少个不相等的积?
    \item 某校举行排球单循环赛,有 8 个队参加,共需要举行多少场比赛?
  \end{question}
\end{Practice}

\begin{Exercise}
  \begin{question}
    \item 计算:
    \begin{tasks}(2)
      \task $C_{15}^{2}$;
      \task $C_{200}^{197}$;
      \task $C_{8}^{3}\div C_8^4$;
      \task $C_{n+1}^{n}\cdot C_n^{n-2}$。
    \end{tasks}
    \item 求证:
    \begin{tasks}
      \task $C_{n+1}^{m}=C_{n}^{m-1}+C_{n-1}^{m}+C_{n-1}^{m-1}$;
      \task $C_{n}^{m+1}+C_{n}^{m-1}+2C_{n}^{m}=C_{n+2}^{m+1}$。
    \end{tasks}
    \item 圆上有 10 个点:
    \begin{tasks}
      \task 过每 2 点可画一条弦,一共可画多少条弦?
      \task 过每 3 点可画一个圆内接三角形,一共可画多少个圆内接三角形?
    \end{tasks}
    \item 解答:
    \begin{tasks}
      \task 凸五边形有多少条对角线?
      \task 凸 $n$ 边形有多少条对角线?
    \end{tasks}
    \item 壹分、贰分、伍分硬币各一枚,一共可以组成多少种币值?
    \item 从 9 个不相同的质数 $a_1,a_2,a_3,a_4,a_5,a_6,a_7,a_8,a_9$ 中任取 4 个相乘,可以得到多少个不同的积?
    \item 解答:
    \begin{tasks}
      \task 空间有 8 个点,没有 4 个点在同一平面内,过每 3 个点作一个平面,一共可以作多少个平面?
      \task 空间有 10 个点,其中任何 4 点不共面,以每 4 个点为顶点作一个四面体,一共可以作多少个四面体?
    \end{tasks}
    \item 某校高中一年级有 6 个班,二年级有 8 个班,三年级有 4 个班。各年级分别举行班与班的排球单循环赛,一共需要比赛多少场?
    \item 某班有 52 名学生,其中正副班长各 1 名,现选派 5 名学生参加某种课外活动:
    \begin{tasks}
      \task 如果班长和副班长必须在内,有多少种选派法?
      \task 如果班长和副班长必须有一人而且只有一人在内,有多少种选派法?
      \task 如果班长和副班长都不在内,有多少种选派法?
      \task 如果班长和副班长至少有一人在内,有多少种选派法?
    \end{tasks}
    \item 从 6 件不同的东西中任取 1 件、2 件、3 件、4 件、5 件、6 件,一共有多少种取法?
    \item 生产某种产品 200 件,其中由 2 件是次品,现在抽取 5 件进行检查:
    \begin{tasks}
      \task “其中恰有两件次品”的抽法有多少种?
      \task “其中恰有 1 件次品”的抽法有多少种?
      \task “其中没有次品”的抽法有多少种?
      \task “其中至少有 1 件次品”的抽法有多少种?
    \end{tasks}
    \item 从 $1,3,5,7,9$ 中任取三个数字,从 $2,4,6,8$ 中任取两个数字,组成没有重复数字的五位数,一共可以组成多少个数?
  \end{question}
\end{Exercise}

\section{二项式定理}
\subsection{二项式定理}
我们已经知道,
\begin{align*}
  (a+b)^2&=a^2+2ab+b^2,\\
  (a+b)^3&=a^3+3a^2b+3ab^2+b^3.
\end{align*}

现在研究 $(a+b)^n$ 的展开式,这里 $n\in\mathbb{N}$。

首先,研究 $(a+b)^4$ 的展开式的各项,即研究
\[ (a+b)^4=(a+b)(a+b)(a+b)(a+b)\]
的展开式的各项。

等号右边的积的展开式的每一项,是从四个括号种每个里任取一个字母的乘积,因而各项都是 4 次式,即展开式应有下面形式的各项:
\[ a^4,\quad a^3b,\quad a^2b^2,\quad ab^3,\quad b^4.\]

运用组合的知识,就可以得出展开式各项的系数规律:

在上面四个括号中,都不取 $b$,共有 1 种,即 $C_4^0$ 种,所以 $a^4$ 的系数是 $C_4^0$;

在四个括号中,恰有 1 个取 $b$,共有 $C_4^1$ 种,所以 $a^3b$ 的系数是 $C_4^1$;

在四个括号中,恰有 2 个取 $b$,共有 $C_4^2$ 种,所以 $a^2b^2$ 的系数是 $C_4^2$;

在四个括号中,恰有 3 个取 $b$,共有 $C_4^3$ 种,所以 $ab^3$ 的系数是 $C_4^3$;

在四个括号中,4 个都取 $b$,共有 $C_4^4$ 种,所以 $b^4$ 的系数是 $C_4^4$。

因此,
\[(a+b)^4=C_4^0a^4+C_4^1a^3b+C_4^2a^2b^2+C_4^3ab^3+C_4^4b^4.\]

一般地,有以下公式:
\[\tcbhighmath{(a+b)^n=C_n^0a^n+C_n^1a^{n-1}b^1+\cdots+C_n^ra^{n-r}b^r+\cdots+C_n^nb^n,\ (n\in\mathbb{N}).}\]

下面,我们用数学归纳法来证明这一公式。
\begin{proof}
  \begin{enumerate}
    \item\label{itm:Inductive1}当 $n=1$ 时,等式的左边是
    \[(a+b)^1=a+b;\]
    等式的右边是
    \[C_1^0a+C_1^1b=a+b.\]
    于是,当 $n=1$ 时等式成立。
    \item\label{itm:Inductive2}假设 $n=k$ 时等式成立,即
    \[(a+b)^k=C_k^0a^k+C_k^1a^{k-1}b^1+\cdots+C_k^ra^{k-r}b^r+\cdots+C_k^kb^k.\]
    现在证明当 $n=k+1$ 时等式也成立。

    由于
    \begin{align*}
      &(a+b)^{k+1}\\
     ={}&(a+b)^k(a+b)\\
     ={}&(C_k^0a^k+C_k^1a^{k-1}b^1+\cdots+C_k^ra^{k-r}b^r+\cdots+C_k^kb^k)(a+b)\\
     ={}&C_k^0a^{k+1}+C_k^1a^kb^1+\cdots+C_k^{r+1}a^{k-r}b^{r+1}+\cdots+C_k^kab^k\\
        &\phantom{C_k^0a^{k+1}}+C_k^0a^kb^1+\cdots+C_k^ra^{k-r}b^{r+1}+\cdots+C_k^{k-1}ab^k+C_k^kb^{k+1}\\
     ={}&C_k^0a^{k+1}+(C_k^1+C_k^0)a^kb^1+\cdots+(C_k^{r+1}+C_k^r)a^{k-r}b^{r+1}+\cdots\\
        &\phantom{(C_k^k+C_k^{k-1})ab^k}+(C_k^k+C_k^{k-1})ab^k+C_k^kb^{k+1},
    \end{align*}
    利用
    \begin{multline*}
      C_k^0=C_{k+1}^0,\ C_k^1+C_k^0=C_{k+1}^1,\ \cdots,\ C_k^{r+1}+C_k^r=C_{k+1}^{r+1},\ \cdots,\\ 
      C_k^k+C_k^{k-1}=C_{k+1}^k,\ C_k^k=C_{k+1}^{k+1},
    \end{multline*}
    则得到
    \begin{multline*}
      (a+b)^{k+1}=C_{k+1}^0a^{k+1}+C_{k+1}^1a^kb^1+\cdots+C_{k+1}^{r+1}a^{k-r}b^{r+1}+\cdots \\
      +C_{k+1}^ka^1b^k+C_{k+1}^{k+1}b^{k+1}.
    \end{multline*}
    这就是说,如果 $n=k$ 时等式成立,那么 $n=k+1$ 时等式也成立。
  \end{enumerate}

  根据~\ref{itm:Inductive1}~和~\ref{itm:Inductive2},可知对于任意自然数 $n$,公式都成立。
\end{proof}

这个公式所表示的定理叫做\Concept{二项式定理},右边的多项式叫做 $(a+b)^n$ 的\Concept{二项展开式},其中的系数 $C_n^r\,(r=0,1,\cdots,n)$ 叫做\Concept{二项式系数}。式中的 $C_n^ra^{n-r}b^r$ 叫做二项式展开式的\Concept{通项},用 $T_{r+1}$ 表示,即通项为展开式的第 $r+1$ 项:
\[\tcbhighmath{T_{r+1}=C_n^ra^{n-r}b^r.}\]

在二项式定理中,如果设 $a=1$,$b=x$,则得到公式:
\[ (1+x)^n=1+C_n^1x+C_n^2x^2+\cdots+C_n^rx^r+\cdots+x^n.\]

遇到 $n$ 是较小的正整数时,二项式系数也可以直接用下表计算:
\[\setcounter{MaxMatrixCols}{20}\setlength{\arraycolsep}{3pt}
\begin{NiceMatrix}
  (a+b)^1& \Cdots &   &   &  &   &  & 1 &  & 1 &  &   &  &   & \\
  (a+b)^2& \Cdots &   &   &   & & 1 & & 2 & & 1 & &   & & \\
  (a+b)^3& \Cdots &   &   &  & 1 &  & 3 &  & 3 &  & 1 &  &   & \\
  (a+b)^4& \Cdots &   &   & 1 & & 4 & & 6 & & 4 & & 1 & & \\
  (a+b)^5& \Cdots &   & 1 &  & 5 &  & 10 &  & 10 &  & 5 &  & 1 & \\
  (a+b)^6& \Cdots & 1 &   & 6 & & 15 & & 20 & & 15 & & 6 & & 1\\
\end{NiceMatrix}\]
表中每行两端都是 1,而且除 1 以外的每一个数都等于它肩上两个数的和。

类似这样的表,早在我国宋朝数学家杨辉 1261 年所著的《详解九章算法》一书里就已出现,这本书里记载着下面的表(\cref{fig:2-8}),我们称它为\Concept{杨辉三角}\footnote{在欧洲,人们认为这个表是法国数学家帕斯卡(Blaise Pascal,1623--1662 年)首先发现的,他们把这个表叫做帕斯卡三角。}。
\begin{figure}
  \includegraphics{2-8.pdf}
  \caption{}\label{fig:2-8}
\end{figure}

\begin{example}
  展开 $\left(1+\dfrac1x\right)^4$。
\end{example}
\begin{solution}
  $\left(1+\dfrac1x\right)^4=1+4\left(\dfrac1x\right)+6\left(\dfrac1x\right)^2+4\left(\dfrac1x\right)^3+\left(\dfrac1x\right)^4=1+\dfrac{4}{x}+\dfrac{6}{x^2}+\dfrac{4}{x^3}+\dfrac{1}{x^4}$。
\end{solution}

\begin{example}
  展开 $\left(2\sqrt{x}-\dfrac{1}{\sqrt{x}}\right)^6$。
\end{example}
\begin{solution}
  \begin{align*}
    \left(2\sqrt{x}-\frac{1}{\sqrt{x}}\right)^6&=\left(\frac{2x-1}{\sqrt{x}}\right)^6=\frac{1}{x^3}(2x-1)^6\\[5pt]
    &=\frac{1}{x^3}\Bigl[(2x)^6-C_6^1(2x)^5+C_6^2(2x)^4-C_6^3(2x)^3\\ &\phantom{=}\phantom{+C_6^4(2x)^2}+C_6^4(2x)^2-C_6^5(2x)+C_6^6\Bigr]\\
    &=\frac{1}{x^3}(64x^6-6\cdot 32x^5+15\cdot 16x^4-20\cdot 8x^3+15\cdot 4x^2-6\cdot 2x+1)\\
    &=64x^3-192x^2+240x-160+\frac{60}{x}-\frac{12}{x^2}+\frac{1}{x^3}. 
  \end{align*}
\end{solution}

\begin{example}
  求 $\left(x-\dfrac1x\right)^3$ 的展开式中 $x^3$ 的系数。
\end{example}
\begin{solution}
  展开式的通项是
  \[C_9^rx^{0-r}\left(-\dfrac1x\right)^r=(-1)^rC_9^rx^{0-2r}.\]
  根据题意,得
  \begin{align*}
    9-2r&=3,\\
    r&=3.
  \end{align*}
  因此,$x^3$ 的系数是
  \[(-1)^3C_9^3=-84.\]
\end{solution}

\alertwarning{展开式中第 $r+1$ 项的二项式系数 $C_n^r$ 与第 $r+1$ 项的系数不同,例如在 $(1+2x)^7$ 的展开式中,第四项为 $T_4=C_7^3\cdot 1^{7-3}\cdot(2x)^3$,其二项式系数是 $C_7^3=35$,而第四项(即含 $x^3$ 的项)的系数是 $C_7^3\cdot 2^3=280$。}

\begin{example}
  计算 $(0.997)^3$ 的近似值(精确到 0.001)。
\end{example}
\begin{solution}
  $(0.997)^3=(1-0.003)^3=1-3\times 0.003+3\times(0.003)^2-\cdots$。根据题中精确度的要求,从第三项起以后的各项都可以删去,所以
  \[(0.997)^3\approx 1-3\times 0.003=0.991.\]
\end{solution}

\begin{Practice}
  \begin{question}
    \item 写出 $(p+q)^7$ 的展开式。
    \item 求 $(2a+3b)^6$ 的展开式的第 3 项。
    \item 求 $(3b+2a)^6$ 的展开式的第 3 项。
    \item 写出 $\left(\sqrt[3]{x}-\dfrac{1}{2\sqrt[3]{x}}\right)$ 的展开式的第 $r+1$ 项。
    \item 求 $(x^3+2x)^7$ 的展开式的第 4 项的二项式系数,并求第 4 项的系数。
    \item 计算 $(1.002)^6$ 的近似值(精确到 0.001)。
  \end{question}
\end{Practice}

\subsection{二项式系数的性质}
我们已经知道,$(a+b)^n$ 的展开式的二项式系数是
\[C_n^0,\quad C_n^1,\quad C_n^2,\quad \cdots,\quad C_n^{n-1},\quad C_n^n\]

二项式系数有下列性质:
\begin{enumerate}[1.]
  \item 在二项展开式中,与首末两端“等距离”的两项的二项式系数相等。
  
  由已知公式
  \[C_n^m=C_n^{n-m},\]
  分别取 $m=0,1,\dots,k,\dots$,从而得
  \[C_n^0=C_n^n,\quad C_n^1=C_n^{n-1},,\quad C_n^2=C_n^{n-2},\quad\dots,\quad C_n^k=C_n^{n-k},\dots.\]
  \item 如果二项式的幂指数是偶数,中间一项的二项式系数最大;如果二项式的幂指数是奇数,中间两项的二项式系数相等并且最大。
  
  由于展开式各项的二项式系数顺次是
  \begin{gather*}
    C_n^0=1,\quad C_n^1=n,\quad C_n^2=\frac{n(n-1)}{1\cdot 2},\quad \dots,\\
    C_n^k=\frac{n(n-1)(n-2)\cdots(n-k+1)}{1\cdot 2\cdot \cdots \cdot k},\quad\dots,\quad C_n^n=1.
  \end{gather*}
  其中,后一个二项式系数的分子是前一个二项式系数的分子乘以逐次减小 1 的数(如 $n,n-1,n-2,\dots$),分母是乘以逐次增大的数(如 $1,2,3,\dots$),因而,各项的二项式系数从开始起是逐渐增大,又因为与首末两端“等距离”的两项的二项式系数相等,所以二项式系数增大到某一项时就逐渐减小,且二项式系数最大的项必在中间。

  当 $n$ 是偶数时,$n+1$ 是奇数,展开式共有 $n+1$ 项,所以展开式有中间一项,并且这一项的二项式系数最大。

  当 $n$ 是奇数时,$n+1$ 是偶数,展开式共有 $n+1$ 项,所以有中间两项,这两项的二项式系数相等并且最大。
\end{enumerate}

\begin{example}\label{exp:binomal}
  证明
  \[C_n^0+C_n^1+C_n^2+\cdots+C_n^k+\cdots+C_n^n=2^n.\]
\end{example}
\begin{proof}
  运用 $(1+x)^n$ 的展开式
  \[(1+x)^n=C_n^0+C_n^1x+C_n^2x^2+\cdots+C_n^kx^k+\cdots+C_n^nx^n,\]
  设 $x=1$,则
  \[2^n=C_n^0+C_n^1+C_n^2+\cdots+C_n^k+\cdots+C_n^n\]
\end{proof}

\cref{exp:binomal} 说明,$(a+b)^n$ 的展开式的所有二项式系数的和等于 $2^n$。

\begin{example}
  证明在 $(a+b)^n$ 的展开式中,奇数项的二项式系数的和等于偶数项的二项式系数的和。
\end{example}
\begin{proof}
  在展开式
  \[(a+b)^n=C_n^0a^n+C_n^1a^{n-1}b+C_n^2a^{n-2}b^2+\cdots+C_n^nb^n\]
  中,令 $a=1$,$b=-1$,则得
  \[ (1-1)^n=C_n^0-C_n^1+C_n^2-C_n^3+\cdots+(-1)^nC_n^n,\]
  就是
  \begin{gather*}
    0=(C_n^0+C_n^2+\cdots)-(C_n^1+C_n^3+\cdots)\\
    \therefore\quad C_n^0+C_n^2+\cdots=C_n^1+C_n^3+\cdots.
  \end{gather*}

  即 $(a+b)^n$ 的展开式中,奇数项的二项式系数的和等于偶数项的二项式系数的和。
\end{proof}

\begin{Practice}
  \begin{question}
    \item 求 $(1-x)^{13}$ 的展开式中的含 $x$ 的奇次项系数的和。
    \item 证明 $C_n^0+C_n^2+C_n^4+\cdots+C_n^n=2^{n-1}$,($n$ 是偶数)。
    \item 求 $C_{11}^1+C_{11}^3+\cdots+C_{11}^{11}$。
  \end{question}
\end{Practice}

\begin{Exercise}
  \begin{question}
    \item 用杨辉三角展开 $(a+b)^5$。
    \item 用二项式定理展开:
    \begin{tasks}[after-skip=5pt,before-skip=5pt](2)
      \task $(a+\sqrt[3]{b})^9$;
      \task $\left(\dfrac{\sqrt{x}}{2}-\dfrac{2}{\sqrt{x}}\right)^7$。
    \end{tasks}
    \item 化简:
    \begin{tasks}(2)
      \task $(1+\sqrt{x})^5+(1-\sqrt{x})^5$;
      \task $(2x^{\frac12}+3x^{-\frac12})^4-(2x^{\frac12}-3x^{-\frac12})^4$。
    \end{tasks}
    \item 解答:
    \begin{enumerate}[itemindent=2.4em]
      \item 求 $(1-2x)^{15}$ 的展开式中前四项;
      \item 求 $(2a^3-3b^2)^{10}$ 的展开式中第八项;
      \item 求 $\left(\dfrac{\sqrt{x}}{3}+\dfrac{3}{\sqrt{x}}\right)^{12}$ 的展开式中的中间一项;
      \item 求 $(x\sqrt{y}-y\sqrt{x})^{15}$ 的展开式中的中间两项。
    \end{enumerate}
    \item 求下列各式的二项展开式中指定各项的系数:
    \begin{tasks}[after-skip=5pt,before-skip=5pt](2)
      \task $\left(1-\dfrac{1}{2x}\right)^{10}$ 的含 $\dfrac{1}{x^5}$ 的项;
      \task $\left(2x^3-\dfrac{1}{2x^2}\right)^{10}$ 的常数项。
    \end{tasks}
    \item 求下列各数的近似值(精确到 0.001):
    \begin{tasks}(2)
      \task $(1.003)^5$;
      \task $(0.9998)^8$。
    \end{tasks}
    \item 用二项式定理证明:
    \begin{tasks}(2)
      \task $(n+1)^n-1$ 能被 $n^2$ 整除;
      \task $99^{10}-1$ 能被 1000 整除。
    \end{tasks}
    \item 证明:
    \begin{enumerate}[itemindent=2.4em]
      \item $\left(x-\dfrac1x\right)^{2n}$ 的展开式中常数项是
      \[(-2)^n\frac{1\cdot 3\cdot 5\cdot \cdots\cdot(2n-1)}{n!}\]
      \item $(1+x)^{2n}$ 的展开式的中间一项是
      \[\frac{1\cdot 3\cdot 5\cdot \cdots\cdot(2n-1)}{n!}(2x)^n\]
    \end{enumerate}
    \item 已知 $(1+x)^n$ 的展开式中第四项与第八项的二项式系数相等,求这两项的二项式系数。
    \item 求证:
    \[ 2^n-C_n^1\cdot 2^{n-1}+C_n^2\cdot 2^{n-2}+\cdots+(-1)^{n-1}C_n^{n-1}\cdot 2+(-1)^n=1.\]
  \end{question}
\end{Exercise}

\section*{小结}
\begin{enumerate}[C、,itemindent=4.5em]
  \item 本章主要内容是排列、组合、二项式定理。
  \item 加法原理与乘法原理是两个基本原理,它们不仅是推导排列数公式、组合数公式的基础,而且还常常需要直接运用它们去解某些问题。
  
  这两个原理的区别在于一个与分类有关,一个与分步有关。如果完成一件事有 $n$ 类办法,这 $n$ 类办法彼此之间是相互独立的,不论哪一类办法中的哪一种方法都能单独完成这件事,求完成这件事的方法种数,就用加法原理;如果完成一件事需分成 $n$ 个步骤,各步骤都不可缺少,需要依次完成所有的步骤,才能完成这件事,而完成每一个步骤各有若干方法,求完成这件事的方法种数就用乘法原理。
  \item 排列与组合是研究从一些不同的元素中,任取几个元素进行排列或并组有多少种方法的问题。本章所研究的主要是不同元素不允许重复的排列或组合。排列与组合的区别要看问题是否与顺序有关。与顺序有关就属于排列,与顺序无关就属于组合。
  
  在求应用题中的排列数或组合数时,注意防止重复与遗漏。
  \item 排列与组合的主要公式有:
  \begin{enumerate}[1.]
    \item 排列数公式
    \begin{align*}
      P_n^m&=n(n-1)(n-2)\cdots(n-m+1), \quad (m\leqslant n);\\
      P_n^m&=\frac{n!}{(n-m)!}, \quad (m\leqslant n);\\
      P_n^n&=n!=n(n-1)(n-2)\cdots 2\cdot 1;
    \end{align*}
    \item 组合数公式
    \[C_n^m=\frac{P_n^m}{P_m^m},\quad(m\leqslant n); \]
    \item 组合数性质
    \begin{align*}
      C_{n}^m  &=C_n^{n-m},\quad(m\leqslant n);\\
      C_{n+1}^m&=C_n^m+C_n^{m-1},\quad(m\leqslant n).
    \end{align*}
  \end{enumerate}
  \item 二项式定理通过公式的形式,表示出二项式的幂展开在项数、系数、各项中的指数等方面的联系。
  
  二项式定理为
  \[(a+b)^n=C_n^0a^n+C_n^1a^{n-1}b+\cdots+C_n^ra^{n-r}b^r+\cdots+C_n^nb^n,\]
  其中 $C_n^r$ 叫做第 $r+1$ 项的二项式系数,展开式的第 $r+1$ 项为
  \[ T_{r+1}=C_n^ra^{n-r}b^r.\]

  二项式系数的主要性质有:
  \begin{enumerate}[1.]
    \item 在二项展开式中,与首末两端“等距离”的两项的二项式系数相等。
    \item 如果二项式的幂指数是偶数,中间一项的二项式系数最大;如果二项式的幂指数是奇数,中间两项的二项式系数相等并且最大。
  \end{enumerate}
\end{enumerate}
\chapter*{复习参考题\chinese{chapter}}
\section*{A 组}
\begin{question}
  \item 求证:
  \begin{tasks}[after-item-skip=7pt]
    \task $\dfrac{(2n)!}{2^n\cdot n!}=1\cdot 3\cdot 5\cdot\cdots\cdot (2n-1)$;
    \task $1!+2\cdot 2!+3\cdot 3!+\cdots+n\cdot n!=(n+1)!-1$,(提示:考虑等式 $n\cdot n!=(n+1)!-n!$)。
  \end{tasks}
  \item 解答:
  \begin{tasks}[after-item-skip=7pt,after-skip=5pt]
    \task 已知 $\dfrac{1}{C_5^m}-\dfrac{1}{C_6^m}=\dfrac{7}{10\cdot C_7^m}$,求 $C_8^m$;
    \task 已知 $\dfrac{C_n^{m-1}}{2}=\dfrac{C_n^m}{3}=\dfrac{C_n^{m+1}}{4}$,求 $n$ 与 $m$。
  \end{tasks}
  \item 乘积 $\sum\limits_{i=1}^{m}a_i\sum\limits_{j=1}^{n}b_j$ 一共有多少项?
  \item 6 名同学站成一排,其中某一名不站在排头,也不站在排尾,共有多少种站法?
  \item 由数字 $1,2,3,4,5,6$ 可以组成多少个没有重复数字的自然数?
  \item 由数字 $1,2,3,4,5,6$ 可以组成多少个没有重复数字,并且比 \num{500000} 大的自然数?
  \item 一个集合由 8 个不同的元素组成,这个集合中含 3 个元素的子集有几个?
  \item 一个集合由 5 个不同的元素组成,其中含 1 个、2 个、3 个、4 个元素的子集共有几个?
  \item 解答:
  \begin{enumerate}[itemindent=2.4em]
    \item 平面内有 $n$ 条直线,其中没有两条互相平行,也没有三条相交于一点,一共有多少个交点?
    \item 空间有 $n$ 个平面,其中没有两个互相平行,也没有三个相交于一直线,一共有多少条交线?
  \end{enumerate}
  \item 100 件产品中有 97 件合格品,3 件次品,从中任意抽取 5 件进行检查。
  \begin{enumerate}[itemindent=2.4em]
    \item 抽出的 5 件都是合格品的抽法有多少种?
    \item 抽出的 5 件恰好有 2 件是次品的抽法有多少种?
    \item 抽出的 5 件至少有 2 件是次品的抽法有多少种?
  \end{enumerate}
  \item 书架上有 4 本不同的数学书,5 本不同的物理书,3 本不同的化学书,全部竖起排成一排,如果不使同类的书分开,一共有多少种排法?
  \item 当 $a$ 的绝对值与 1 相比很小时,$(1+a)^n$ 的近似值可以用公式 $(1+a)^n\approx 1+na$ 来计算。用这个近似公式计算:
  \begin{tasks}(2)
    \task $(1.002)^5$;
    \task $(0.997)^6$;
    \task $(1.005)^{10}$;
    \task $(0.9995)^9$;
  \end{tasks}
  \item 分别求当 $n=1,2,3,4$ 时,$\left(1+\dfrac1n\right)^n$ 的值。
  \item 解答:
  \begin{tasks}[after-item-skip=7pt]
    \task 求 $(a+\sqrt{b})^{12}$ 展开式中第 9 项;
    \task 求 $(1-2x)^5(1+3x)^4$ 展开式中按 $x$ 的升幂排列的前三项;
    \task 求 $\left(9x-\dfrac{1}{3\sqrt{x}}\right)^{18}$ 展开式的常数项;
    \task 求 $n$:已知 $(1+\sqrt{x})^n$ 的展开式中第 9 项、第 10 项、第 11 项的二项式系数成等差数列;
    \task 求 $(1+x+x^2)(1-x)^{10}$ 展开式中 $x^4$ 的系数。
  \end{tasks}
  \item 解答:
  \begin{enumerate}[itemindent=2.4em]
    \item 用二项式定理证明 $55^{55}+9$ 能被 8 整除; 
    \item 用二项式定理求 $89^{10}$ 除以 88 的余数。 
  \end{enumerate}
  \item 证明 $(1+x)^{2n}$ 展开式中 $x^n$ 的系数等于 $(1+x)^{2n-1}$ 展开式中 $x^n$ 的系数的 2 倍。
  \item 已知 $\left(\sqrt{x}+\dfrac{1}{\sqrt[3]{x}}\right)^n$ 展开式的二项式系数之和比 $(a+b)^{2n}$ 展开式的二项式系数之和小 240,求:
  \begin{tasks}[after-item-skip=7pt]
    \task $\left(\sqrt{x}+\frac{1}{\sqrt[3]{x}}\right)^n$ 展开式的第 3 项;
    \task $(a+b)^{2n}$ 展开式的中间项。
  \end{tasks}
\end{question}
\section*{B 组}
\begin{question}[resume]
  \item 求证:
  \begin{tasks}[after-item-skip=7pt]
    \task $1!\cdot 2!\cdot 3!\cdot\cdots\cdot n!=\dfrac{(n!)^{n-1}}{3\cdot 4^2\cdot 5^3\cdot \cdots\cdot n^{n-2}}$;
    \task $C_{n-1}^m+C_{n-2}^m+C_{n-3}^m+\cdots+C_{m+1}^m+C_m^m=C_{n}^{m+1}$。
  \end{tasks}
  \item 解答:
  \begin{enumerate}[itemindent=2.4em]
    \item 由数字 $1,2,3,4,5$ 可以组成多少个没有重复数字并且能被 5 整除的自然数?
    \item 由数字 $1,2,3,4,5$ 可以组成多少个没有重复数字,并且比 \num{30000} 小的自然数?
  \end{enumerate}
  \item 用数字 $0,1,2,3,4,5$ 组成没有重复数字的数:
  \begin{enumerate}[itemindent=2.4em]
    \item 能够组成多少个六位奇数?
    \item 能够组成多少个大于 \num{201345} 的自然数?
  \end{enumerate}
  \item 8 个不同的元素排成一行:
  \begin{enumerate}[itemindent=2.4em]
    \item 其中某 2 个元素必须排在一起,有多少种排法?
    \item 其中某 2 个元素不能排在一起,有多少种排法?
    \item 其中某 4 个元素要排在一起,另外 4 个元素也要排在一起,有多少种排法?
  \end{enumerate}
  \item 解答:
  \begin{enumerate}[itemindent=2.4em]
    \item 平面内有两组平行线,一组有 $m$ 条,另一组有 $n$ 条。这两组平行线相交,可以构成多少个平行四边形?
    \item 空间有三组平行平面,第一组有 $m$ 个,第二组有 $n$ 个,第三组有 $l$ 个。不同两组的平面都相交,且交线不都平行,可构成多少个平行六面体?
  \end{enumerate}
  \item 解答:
  \begin{enumerate}[itemindent=2.4em]
    \item 8 个不同的元素排成前后两排,每排 4 个元素,有多少种排法?
    \item 8 个不同的元素排成前后两排,每排 4 个元素,其中某 2 个元素要排在前排,某 1 个元素要排在后排,有多少种排法?
  \end{enumerate}
  \item 解答:
  \begin{enumerate}[itemindent=2.4em]
    \item 在 $\left(x\sqrt{x}+\dfrac{1}{x^4}\right)^n$ 的展开式中,第 3 项的二项式系数比第 2 项的二项式系数大 44,求展开式中不含字母 $x$ 的项。
    \item 在 $(1+x)^3+(1+x)^4+\cdots+(1+x)^{n+2}$ 的展开式中,求含 $x^2$ 项的系数。
  \end{enumerate}
  \item 证明:
  \begin{enumerate}[itemindent=2.4em]
    \item $(C_n^0)^2+(C_n^1)^2+\cdots+(C_n^n)^2=\dfrac{(2n)!}{n!\cdot n!}$,
    
    (提示:利用 $(1+x)^n\cdot(1+x)^n=(1+x)^{2n}$,并且比较等式两边的展开式中含 $x^n$ 项的二项式系数);
    \item $C_n^1+2C_n^2+3C_n^3+\cdots+nC_n^n=n\cdot 2^{n-1}$。
  \end{enumerate}
\end{question}