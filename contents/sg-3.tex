\chapter{多面角和正多面体}
\section{多面角}
\subsection{多面角}
相邻的两面墙壁和天花板所成的屋角,一些塔的塔顶都给我们以多面角的形象。

有公共端点并且不在同一平面内的几条射线,以及相邻两条射线间的平面部分所组成的图形,叫做\Concept{多面角}。

如\cref{fig:3-1,fig:3-2},都是多面角。
组成多面角的射线 $SA$、$SB$、……叫做\Concept{多面角的棱},这些射线的公共端点 $S$ 叫做\Concept{多面角的顶点},相邻两棱间的平面部分叫做\Concept{多面角的面},相邻两棱组成的角 $\angle ASB$、$\angle BSC$、……叫做\Concept{多面角的面角},相邻两个面组成的二面角 $E\text{-}SA\text{-}B$、$A\text{-}SB\text{-}C$、……叫做\Concept{多面角的二面角}。
\begin{figure}
  \begin{minipage}[b]{0.48\linewidth}\centering
    \caption{}\label{fig:3-1}
  \end{minipage}
  \begin{minipage}[b]{0.48\linewidth}\centering
    \caption{}\label{fig:3-2}
  \end{minipage}
\end{figure}


\begin{Practice}
  \begin{question}
    \item 二面角是不是多面角?多面角是不是棱锥?它们各有什么区别和联系?
    \item 求证:三个二面角都是直二面角的三面角是直三面角。
  \end{question}
\end{Practice}
\subsection{多面角的性质}
\begin{Practice}
  \begin{question}
    \item 下面各组面角能否构成三面角?为什么?
    \begin{tasks}(2)
      \task \ang{45}、\ang{65}、\ang{120};
      \task \ang{100}、\ang{90}、\ang{150}。
    \end{tasks}
    \item 下面各组面角能否构成四面角?为什么?
    \begin{tasks}(2)
      \task \ang{45}、\ang{65}、\ang{120}、\ang{95};
      \task \ang{175}、\ang{105}、\ang{120}、\ang{60}。
    \end{tasks}
    \item 证明:三面角的任何一个面角大于其他两个面角的差。
  \end{question}
\end{Practice}
\begin{Exercise}
  \begin{question}
    \item 
    \item 
    \item 
    \item 
    \item 
    \item 
    \item 
    \item 
  \end{question}
\end{Exercise}

\section{正多面体、多面体变形}
\subsection{正多面体}
\begin{Practice}
  \begin{question}
    \item 以正多面体的各面中心为顶点的多面体都是几面体?参照模型看它们是不是正多面体?
    \item 设正十二面体的棱长为 $a$,求它的表面积。
  \end{question}
\end{Practice}
\subsection{多面体的变形}
\begin{Practice}
  \begin{question}
    \item 检查六棱柱、五棱锥、四棱台,看它们的顶点数、棱数、面数是否适合欧拉定理。
    \item 简单多面体,凸多面体、正多面体、棱柱、棱锥、棱台的包含关系如何?用图表示。 
  \end{question}
\end{Practice}
\begin{Exercise}
  \begin{question}
    \item 求棱长为 $a$ 的正八面体的对角线长。
    \item 以四面体的高和棱为一边分别作正方形。求证:这两个正方形的面积比是 $2:3$.
    \item 正六面体各面中心是一个正八面体的顶点。求这个正六面体和正八面体的表面积的比。
    \item 正 $n$($n=4,8,20$)面体的棱长为 $a$,求它们表面积的共同公式。
    \item 正二十面体的棱长为 $a$,连结相对顶点的对角线为 $b$,求它的体积。
    \item 求证:平行于正四面体的相对棱的平面,截这个正四面体的截面是一个矩形。
    \item 就下面平面图形验证 $V+F-E=1$。
    \item 已知:凸多面体的各面都是四边形,求证:$F=V-2$。
  \end{question}
\end{Exercise}

\section*{小结}
\begin{enumerate}[C、,itemindent=4.5em]
  \item 本章的主要内容是多面角的概念及其主要性质,在此基础上研究正多面体,并用连续变形(也叫“拓扑变形”)的方法证明关于简单多面体的欧拉定理。
  \item 作为多面角的特殊情形,定义了凸多面角的概念,并指出三面角和直三面角又是凸多面角的特殊情形。然后研究三面角的性质,并在此基础上推出凸多面角的性质。
  \item 定义正多面体的概念,并根据多面角的性质推出正多面体只有物种:正四面体、正六面体(正方体)、正八面体、正十二面体、正二十面体。
  \item 研究简单多面体,推出它的表面在连续变形下的不变性质:欧拉定理($V+F-E=2$)。因为简单多面体包括凸多面体因而也包括正多面体,所以欧拉定理对于这些多面体也同样适用。
\end{enumerate}
\chapter*{复习参考题\chinese{chapter}}
\section*{A 组}
\begin{question}
  \item 解答:
  \begin{tasks}
    \task 用正三角形做面可以组成几种多面角?为什么?
    \task 用正方形做面可以组成几种多面角?为什么?
    \task 用正五边形做面可以组成几种多面角?为什么?
    \task 能否用正六边形做面组成三面角?为什么?
  \end{tasks}
  \item 在直三面角 $S\text{-}ABC$ 的三个棱上,取 $SA=SB=SC=a$。
  \begin{tasks}
    \task 求直线 $AB$ 与 $SC$ 的距离;
    \task 求点 $C$ 与直线 $AB$ 的距离;
    \task 求以 $SAB$、$CAB$ 为面的二面角的大小。
  \end{tasks}
  \item 已知正四面体所有面的中心,是一个正四面体的顶点,画出这个正四面体。
  \item 线段 $AC$、$BD$、$EF$ 相等,并且两两互相垂直平分于点 $O$。求证:$A$、$B$、$C$、$D$、$E$、$F$ 是正八面体的六个顶点。
  \item 已知:一个简单多面体的各个顶点都有三条棱。求证:$V=2F-4$。
\end{question}
\section*{B 组}
\begin{question}
  \item 在三面角 $S\text{-}ABC$ 中,$\angle ASB=\angle ASC=\ang{45}$,$\angle BSC=\ang{60}$。求证:$\angle BSC$ 所对的二面角是直二面角。
  \item 直三面角的三个面被任意平面所截。求证:截得的三角形的垂心是三面角的顶点在截面上的射影。
  \item 已知正十二面体的十二个面的中心是一个正二十面体的顶点,画出这个正二十面体。
  \item 对于图中多面体,求 $V+F-E$ 的值。
\end{question}

\chapter*{总复习参考题}
\section*{A 组}
\begin{question}
  \item 
  \item 
  \item 
  \item 
  \item 
  \item 
  \item 
  \item 
  \item 
  \item 
  \item 
  \item 
  \item 
  \item 
  \item 
\end{question}
\section*{B 组}
\begin{question}
  \item 
  \item 
  \item 
  \item 
  \item 
  \item 
  \item 
  \item 
  \item 
  \item 
  \item 
\end{question}