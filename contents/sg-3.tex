\chapter{多面角和正多面体}
\section{多面角}
\subsection{多面角}
相邻的两面墙壁和天花板所成的屋角,一些塔的塔顶都给我们以多面角的形象。

有公共端点并且不在同一平面内的几条射线,以及相邻两条射线间的平面部分所组成的图形,叫做\Concept{多面角}。

如\cref{fig:3-1,fig:3-2},都是多面角。
组成多面角的射线 $SA$、$SB$、……叫做\Concept{多面角的棱},这些射线的公共端点 $S$ 叫做\Concept{多面角的顶点},相邻两棱间的平面部分叫做\Concept{多面角的面},相邻两棱组成的角 $\angle ASB$、$\angle BSC$、……叫做\Concept{多面角的面角},相邻两个面组成的二面角 $E\text{-}SA\text{-}B$、$A\text{-}SB\text{-}C$、……叫做\Concept{多面角的二面角}。
\begin{figure}
  \begin{minipage}[b]{0.48\linewidth}\centering
    \includegraphics{3-1.pdf}
    \caption{}\label{fig:3-1}
  \end{minipage}
  \begin{minipage}[b]{0.48\linewidth}\centering
    \includegraphics{3-2.pdf}
    \caption{}\label{fig:3-2}
  \end{minipage}
\end{figure}

一个多面角的面数等于它的棱数、面角数、二面角数。
多面角最少应有三个面。
多面角依照它的面数分别叫做三面角、四面角、五面角、……。

多面角可以用表示它的顶点和冷的字母来表示,如\cref{fig:3-1} 中的多面角记作多面角 $S\text{-}ABCDE$;有时也用表示顶点的一个字母表示,记作多面角 $S$。

将多面角的任何一个面伸展成为平面,如果其他各面都在这个平面的同侧,这样的多面角叫做\Concept{凸多面角}。\cref{fig:3-1} 中的多面角就是一个凸多面角,\cref{fig:3-2} 中的多面角不是凸多面角。

用一个平面截凸多面角的所有面和棱,一定得到一个凸多边形,如\cref{fig:3-1}。用平面截\cref{fig:3-2} 中的多面角就不能得到凸多边形。

本章只研究凸多面角。
凸多面角中,最简单的是三面角,三个面角都是直角的三角形叫做\Concept{直三面角}。
例如屋角、箱角、砖角,都给我们以直三面角的形象。

直三面角 $O\text{-}XYZ$ 的一般画法如\cref{fig:3-3}。
\begin{example}
  求证:直三面角的各个二面角都是直二面角。

  已知:直三面角 $O\text{-}XYZ$(\cref{fig:3-3})。

  求证:二面角 $\beta\text{-}OX\text{-}\gamma$、$\gamma\text{-}OY\text{-}\alpha$、$\alpha\text{-}OX\text{-}\beta$ 都是直二面角。
\end{example}

\medskip\noindent
\begin{minipage}{0.6\linewidth}
\begin{proof}
  $\because\quad O\text{-}XYZ$ 是直三面角,
  
  $\therefore\quad \angle XOY=\angle YOZ =\angle ZOX= \text{Rt}\angle$,

  $\phantom{\therefore}\quad OZ\perp OX$,$OY\perp OX$。

  $\therefore\quad \angle YOZ$ 是二面角 $\beta\text{-}OX\text{-}\gamma$ 的平面角。

  $\because\quad \angle YOZ= \text{Rt}\angle$,

  $\therefore\quad$ 二面角 $\beta\text{-}OX\text{-}\gamma$ 是直二面角。

  同理,$\gamma\text{-}OY\text{-}\alpha$、$\alpha\text{-}OX\text{-}\beta$ 也是直二面角。
\end{proof}
\end{minipage}\hfill
\begin{minipage}{0.35\linewidth}
  \begin{figurehere}
    \includegraphics{3-3.pdf}
    \caption{}\label{fig:3-3}
  \end{figurehere}
\end{minipage}

\begin{Practice}
  \begin{question}
    \item 二面角是不是多面角?多面角是不是棱锥?它们各有什么区别和联系?
    \item 求证:三个二面角都是直二面角的三面角是直三面角。
  \end{question}
\end{Practice}

\subsection{多面角的性质}
在研究多面角的性质之前,先研究三面角的一个性质:
\begin{Theorem}{定理}
  三面角的任意两个面角的和大于第三个面角。
\end{Theorem}
已知:三面角 $S$--$ABC$(\cref{fig:3-4})。

求证:$\angle ASB+\angle BSC>\angle ASC$。

\medskip\noindent
\begin{minipage}{0.6\linewidth}\parindent2em
\begin{proof}
  当 $\angle ASC\leqslant \angle ASB$ 时,显然 $\angle ASB+\angle BSC>\angle ASC$。

  现在设 $\angle ASC>\angle ASB$。在面 $ASC$ 上作线段 $SD'$,使 $\angle ASD'=\angle ASB$,过 $D'$ 引与各棱都相交的平面 $A'B'C'$,使 $SB'=SD'$。这时,
  \[ \triangle A'SB' \cong \triangle A'SD',\] 
  $\therefore\quad A'B'=A'D'$.
\end{proof}
\end{minipage}\hfill
\begin{minipage}{0.35\linewidth}
  \begin{figurehere}
    \includegraphics{3-4.pdf}
    \caption{}\label{fig:3-4}
  \end{figurehere}
\end{minipage}

\medskip
在 $\triangle A'B'C'$ 中,$A'B'+B'C'>A'C'$,同时,$A'D'+D'C'=A'C'$,由此得到 $B'C'>D'C'$。

在 $\triangle B'SC'$ 和 $\triangle D'SC'$ 中,由余弦定理得 $\cos\angle B'SC'<\cos\angle D'SC'$。

所以,$\angle B'SC'>\angle D'SC'$,即 $\angle BSC>\angle D'SC$。

因此,$\angle ASB+\angle BSC>\angle ASD'+\angle D'SC$,

就是,$\angle ASB+\angle BSC>\angle ASC$。

\bigskip
这里我们注意到,如果使三面角的面与三角形的边对应,三面角的二面角与三角形的内角对应,那么三面角的一些性质与三角形类似。
因此,有些三面角的问题,常归结为三角形的问题来研究。

多面角由下面的性质:
\begin{Theorem}{定理}
  凸多面角所有面角的和小于四直角。
\end{Theorem}

已知:凸 $n$ 面角 $S\text{-}ABC\cdots E$(\cref{fig:3-5})。

求证:$\angle ASB+\angle BSC+\cdots+\angle ESA<4\cdot\mathrm{Rt}\angle$。

\begin{proof}
  用平面截已知多面角的所有面和棱,得到凸 $n$ 边形 $A'B'C'\cdots E'$。根据前面的定理,以 $A'$、$B'$、$C'$、$\cdots$、$E'$ 为顶点的各三面角的面角由下面的关系:
  \begin{gather*}
    \angle SA'E'+\angle SA'B' > \angle E'A'B',\\
    \angle SB'A'+\angle SB'C' > \angle A'B'C',\\
    \angle SC'B'+\angle SC'D' > \angle B'C'D'.\\
    \cdots\cdots
  \end{gather*}

  用 $\Sigma$ 表示已知多面角的所有面角的和,
  \[ \Sigma = \angle ASB+\angle BSC+\cdots+\angle ESA.\]

  将上面各不等式两边分别相加,左边是 $n$ 个三角形:$\triangle A'SB'$、$\triangle B'SC'$、$\cdots$、$\triangle E'SA'$ 内角的和 $2n\cdot\mathrm{Rt}\angle$ 减去已知多面角的和 $\Sigma$;右边是凸多边形 $A'B'C'\cdots E'$ 所有内角的和,它等于 $2(n-2)\cdot\mathrm{Rt}\angle$,因此,
  \[ 2n\cdot\mathrm{Rt}\angle -\Sigma >2(n-2)\cdot\mathrm{Rt}\angle,\]
  即
  \[\Sigma<4\cdot\mathrm{Rt}\angle.\]
\end{proof}

\bigskip
这个定理表明,如果沿凸多面角的一个棱剪开,把它展在平面上,那么它不能铺满一个周角,而是缺少一部分(\cref{fig:3-6})。
\begin{figure}
  \caption{}\label{fig:3-6}
\end{figure}

\begin{Practice}
  \begin{question}
    \item 下面各组面角能否构成三面角?为什么?
    \begin{tasks}(2)
      \task \ang{45}、\ang{65}、\ang{120};
      \task \ang{100}、\ang{90}、\ang{150}。
    \end{tasks}
    \item 下面各组面角能否构成四面角?为什么?
    \begin{tasks}(2)
      \task \ang{45}、\ang{65}、\ang{120}、\ang{95};
      \task \ang{175}、\ang{105}、\ang{120}、\ang{60}。
    \end{tasks}
    \item 证明:三面角的任何一个面角大于其他两个面角的差。
  \end{question}
\end{Practice}

\begin{Exercise}
  \begin{question}
    \item 在三面角中,如果有两个二面角是直二面角,那么它们所对的两个面角都是直角。为什么?
    \item 直三面角内有一点 $P$,它到各面的距离分别是 $x$、$y$、$z$。用 $x$、$y$、$z$ 表示点 $P$ 到三面角顶点 $O$ 的距离。
    \item 直三面角 $O\text{-}XYZ$ 内有一点 $P$,$OP$ 在三面角三个面上的射影长分别是 $a$、$b$、$c$。求证:$OP=\sqrt{\dfrac{a^2+b^2+c^2}{2}}$。
    \item 在三面角 $S\text{-}ABC$ 中,$\angle BSC=\ang{90}$,$\angle ASB=\angle ASC=\ang{60}$。
    \begin{tasks}
      \task 设 $SA=SB=SC$。求证:经过三点 $A$、$B$、$C$ 的平面垂直于 $\angle BSC$ 所在的平面;
      \task 求证:$SA$ 与 $\angle BSC$ 所在的平面成 \ang{45} 的角。
    \end{tasks}
    \item 下面各组面角能否构成三面角?为什么?
    \begin{tasks}(2)
      \task \ang{75}、\ang{45}、\ang{90};
      \task \ang{82}、\ang{56}、\ang{26};
      \task \ang{130}、\ang{85}、\ang{36}。
    \end{tasks}
    \item 求证:空间四边形每相邻两边所成的四个角的和小于四直角。
    \item 求证:凸多面角的任何一个面角小于其他面角的和。
    \item 下面各组面角能够构成四面角?为什么?
    \begin{tasks}(2)
      \task \ang{50}、\ang{70}、\ang{100}、\ang{150};
      \task \ang{150}、\ang{30}、\ang{70}、\ang{40}。
    \end{tasks}
  \end{question}
\end{Exercise}

\section{正多面体、多面体变形}
\subsection{正多面体}
常见的食盐的结晶(\cref{fig:3-7})、明矾的结晶(\cref{fig:3-8})都呈正多面体的形状。
\begin{figure}
  \begin{minipage}{0.45\linewidth}\centering
    \includegraphics{3-7.pdf}
    \caption{}\label{fig:3-7}
  \end{minipage}
  \begin{minipage}{0.45\linewidth}\centering
    \includegraphics{3-8.pdf}
    \caption{}\label{fig:3-8}
  \end{minipage}
\end{figure}

每个面都是由同数边的正多边形,在每个顶点都有同数棱的凸多面体,叫做\Concept{正多面体}。例如,正方体的所有面都是正方形,在各个顶点都有三条棱,而且是凸多面体,所以正方体就是一种正多面体。

我们知道,正多边形有无限种,那么正多面体能有多少种呢?现在来研究这个问题。

设正多面体的所有面都是正 $n$ 边形,在每个顶点的棱数都是 $m$,也就是说,每个顶点都是一个 $m$ 面角的顶点。

由于凸 $m$ 面角的面角都是正 $n$ 边形的内角,而正 $n$ 边形的内角和等于 $\dfrac{2(n-2)\cdot\text{Rt}\angle}{n}$,所以凸 $m$ 面角所有面角的和等于 $\dfrac{[2(n-2)\cdot\text{Rt}\angle]\cdot m}{n}$。根据凸多面角的性质定理,下面不等式成立:
\[ \dfrac{[2(n-2)\cdot\text{Rt}\angle]\cdot m}{n}<4\cdot \text{Rt}\angle.\]
化简得
\[m(n-2)<2n,\]
也就是
\[ m<\frac{2n}{n-2}\ \text{(}m\text{、}n\text{ 都是不小于 3 的正整数}\text{)。}\]
解这个不等式,

因为 $n>5$ 时,$m<3$ 不合题意,所以 $n$ 不能大于 5。

因此,我们只能得到关于 $m$、$n$ 的如下五个数对:
\[(3,3),\,(3,4),\,(3,5),\,(4,3),\,(5,3).\]
这就是说,正多面体只能有五种:用正三角形做面的正四面体、正八面体、正二十面体,在它们每个顶点的棱数分别是 3、4、5;用正方形做面的正六面体,在它每个顶点的棱数是 3;用正五边形做面的正十二面体,在它每个顶点的棱数是 3。这五种正多面体如\cref{fig:3-9} 所示。

\begin{figure}
  % \includegraphics{3-9.pdf}
  \caption{}\label{fig:3-9}
\end{figure}

五种正多面体的表面展开图如\cref{fig:3-10}。作为课外研究,可画它 10 倍大的展开图,然后粘成正多面体模型,加以观察。
\begin{figure}
  % \includegraphics{3-10.pdf}
  \caption{}\label{fig:3-10}
\end{figure}

数一数每种正多面体得顶点数、面数、棱数,分别填在下面表格内,研究每种正多面体的顶点数、面数与棱数,能否发现它们之间有什么共同关系?
\begin{table}
  \begin{tblr}{colspec={X[2,c]X[c]X[c]X[c]}}
    正多面体   & 顶点数 & 面数 & 棱数 \\
    正四面体   &        &      &      \\
    正六面体   &        &      &      \\
    正八面体   &        &      &      \\
    正十二面体 &        &      &      \\
    正二十面体 &        &      &      \\
  \end{tblr}
\end{table}
\begin{Practice}
  \begin{question}
    \item 以正多面体的各面中心为顶点的多面体都是几面体?参照模型看它们是不是正多面体?
    \item 设正十二面体的棱长为 $a$,求它的表面积。
  \end{question}
\end{Practice}
\subsection{多面体的变形}
我们考虑任意一个多面体,例如正六面体,假定它的面是用橡胶薄膜做成的。
如果充以气体,那么它就会连续(不破裂)变形,最后可以变为一个球面(\cref{fig:3-11})。
\begin{figure}
  % \includegraphics{3-11.pdf}
  \caption{}\label{fig:3-11}
\end{figure}

像这样,表面连续变形,可变形为球面的多面体叫做\Concept{简单多面体}。
棱柱、棱锥、棱台、正多面体、凸多面体都是简单多面体。

除简单多面体外,还有不是简单多面体的几何体,例如将正方体挖去一个洞所得的几何体(\cref{fig:3-12})。这样的几何题的表面连续变形后就不能变为一个球面,而能变为一个环面。
\begin{figure}
  % \includegraphics{3-12.pdf}
  \caption{}\label{fig:3-12}
\end{figure}

我们曾研究过五种正多面体,发现它们的顶点数 $V$、棱数 $E$ 和面数 $F$ 有下面关系:
\[ V+F-E=2. \]
现在用连续变形的方法研究简单多面体,看它的顶点数 $V$、棱数 $E$ 和面数 $F$ 是否也有这种关系。
以四面体 $ABCD$ 为例,将它的一个面 $BCD$ 去掉,再使它变形为平面图形(\cref{fig:3-13})。这时,四面体的顶点数 $V$、棱数 $E$ 与剩下的面数 $F_1$,变形后都没有变。因此,要研究 $V$、$E$、$F$ 之间的关系,研究平面图形即可。我们来研究
\[ V+F_1-E\]
的数值。可按下面两步进行:
\begin{enumerate}
  \item 去掉一条棱,就减少一个面。例如去掉 $BC$,就减少一个面 $ABC$。同理,去掉棱 $CD$、$BD$ 时,也都随着各减少一个面 $ACD$、$ABD$(\cref{fig:3-14}),由于 $F_1-E$、$V$ 的值都不变,因此,$V+F_1-E$ 的值不变。
  \begin{figure}
    % \includegraphics{3-14.pdf}
    \caption{}\label{fig:3-14}
  \end{figure}
  \item 再从剩下的树枝形,去掉一条棱,就减少一个顶点。例如去掉 $CA$,则减少一个顶点 $C$。同理,去掉棱 $DA$ 随着减少一个顶点 $D$,最后剩下 $AB$(\cref{fig:3-15})。在此过程中 $V-E$ 的值都不变。但此时因为面数 $F_1$ 都是 0,所以 $V+F_1-E$ 的值也不变。由于最后只剩下 $AB$,因此
  \[ V+F_1-E=2+0-1=1.\]
  \begin{figure}
    % \includegraphics{3-15.pdf}
    \caption{}\label{fig:3-15}
  \end{figure}
\end{enumerate}

最后,加上最初去掉的一个面,得到
\[ V+F-E=2.\]

因为对任意的简单多面体,应用这样的方法,最后都是只剩下一条线段,因而都得到上面的结果,所以可把它写成下面的定理:
\begin{Theorem}[欧拉定理]{定理}
  简单多面体的顶点数 $V$、棱数 $E$、面数 $F$,有下面的关系
  \[\tcbhighmath{V+F-E=2}.\]
\end{Theorem}

这个定理叫做\Concept{欧拉定理}。它表明 2 这个数是简单多面体表面在连续变形下不变的数。
\begin{example}
  一个简单多面体的面都是三角形。求证:$F=2V-4$。
\end{example}
\begin{proof}
  因为已知多面体的每个面有三条边,每相邻两个面的两条边重合为一条棱,所以棱数 $E=\dfrac{3F}{2}$,代入公式 $V+F-E=2$,得
  \[ V+F-\frac{3F}{2}=2.\]
  化简得
  \[F=2V-4.\]
\end{proof}
\begin{Practice}
  \begin{question}
    \item 检查六棱柱、五棱锥、四棱台,看它们的顶点数、棱数、面数是否适合欧拉定理。
    \item 简单多面体,凸多面体、正多面体、棱柱、棱锥、棱台的包含关系如何?用图表示。 
  \end{question}
\end{Practice}
\begin{Exercise}
  \begin{question}
    \item 求棱长为 $a$ 的正八面体的对角线长。
    \item 以四面体的高和棱为一边分别作正方形。求证:这两个正方形的面积比是 $2:3$.
    \item 正六面体各面中心是一个正八面体的顶点。求这个正六面体和正八面体的表面积的比。
    \item 正 $n$($n=4,8,20$)面体的棱长为 $a$,求它们表面积的共同公式。
    \item 正二十面体的棱长为 $a$,连结相对顶点的对角线为 $b$,求它的体积。
    \item 求证:平行于正四面体的相对棱的平面,截这个正四面体的截面是一个矩形。
    \item\label{exec:17-7} 就下面平面图形验证 $V+F-E=1$。
    \begin{figurehere}
      \begin{minipage}{\linewidth}
        % \includegraphics{ex17-7.pdf}
        \caption*{(第 \ref{exec:17-7} 题图)}
      \end{minipage}
    \end{figurehere}
    \item 已知:凸多面体的各面都是四边形,求证:$F=V-2$。
  \end{question}
\end{Exercise}

\section*{小结}
\begin{enumerate}[C、,itemindent=4.5em]
  \item 本章的主要内容是多面角的概念及其主要性质,在此基础上研究正多面体,并用连续变形(也叫“拓扑变形”)的方法证明关于简单多面体的欧拉定理。
  \item 作为多面角的特殊情形,定义了凸多面角的概念,并指出三面角和直三面角又是凸多面角的特殊情形。然后研究三面角的性质,并在此基础上推出凸多面角的性质。
  \item 定义正多面体的概念,并根据多面角的性质推出正多面体只有物种:正四面体、正六面体(正方体)、正八面体、正十二面体、正二十面体。
  \item 研究简单多面体,推出它的表面在连续变形下的不变性质:欧拉定理($V+F-E=2$)。因为简单多面体包括凸多面体因而也包括正多面体,所以欧拉定理对于这些多面体也同样适用。
\end{enumerate}
\chapter*{复习参考题\chinese{chapter}}
\section*{A 组}
\begin{question}
  \item 解答:
  \begin{tasks}
    \task 用正三角形做面可以组成几种多面角?为什么?
    \task 用正方形做面可以组成几种多面角?为什么?
    \task 用正五边形做面可以组成几种多面角?为什么?
    \task 能否用正六边形做面组成三面角?为什么?
  \end{tasks}
  \item 在直三面角 $S\text{-}ABC$ 的三个棱上,取 $SA=SB=SC=a$。
  \begin{tasks}
    \task 求直线 $AB$ 与 $SC$ 的距离;
    \task 求点 $C$ 与直线 $AB$ 的距离;
    \task 求以 $SAB$、$CAB$ 为面的二面角的大小。
  \end{tasks}
  \item 已知正四面体所有面的中心,是一个正四面体的顶点,画出这个正四面体。
  \item 线段 $AC$、$BD$、$EF$ 相等,并且两两互相垂直平分于点 $O$。求证:$A$、$B$、$C$、$D$、$E$、$F$ 是正八面体的六个顶点。
  \item 已知:一个简单多面体的各个顶点都有三条棱。求证:$V=2F-4$。
\end{question}
\section*{B 组}
\begin{question}
  \item 在三面角 $S\text{-}ABC$ 中,$\angle ASB=\angle ASC=\ang{45}$,$\angle BSC=\ang{60}$。求证:$\angle BSC$ 所对的二面角是直二面角。
  \item 直三面角的三个面被任意平面所截。求证:截得的三角形的垂心是三面角的顶点在截面上的射影。
  \item 已知正十二面体的十二个面的中心是一个正二十面体的顶点,画出这个正二十面体。
  \item\label{exec:3t-9} 对于图中多面体,求 $V+F-E$ 的值。
  \begin{figurehere}
    \begin{minipage}{\linewidth}
      % \includegraphics{3t-9.pdf}
      \caption*{(第 \ref{exec:3t-9} 题图)}
    \end{minipage}
  \end{figurehere}
\end{question}

\chapter*{总复习参考题}
\section*{A 组}
\begin{question}
  \item $AB$、$BC$、$CD$ 是不在同一平面内的线段。求证:经过它们中点的平面和 $AC$ 平行,也和 $BD$ 平行。
  \item\label{exec:tt-2} 如图,$AB$ 是圆 $O$ 的直径,$PA$ 垂直于圆 $O$ 所在的平面,$C$ 是圆周上的任意点。求证:$\triangle PAC$ 所在的平面垂直于 $\triangle PBC$ 所在的平面。
  \begin{figurehere}
    \begin{minipage}[b]{0.48\linewidth}\centering
      \includegraphics{tt-2.pdf}
      \caption*{(第 \ref{exec:tt-2} 题图)}
    \end{minipage}
    \begin{minipage}[b]{0.48\linewidth}\centering
      \includegraphics{tt-3.pdf}
      \caption*{(第 \ref{exec:tt-3} 题图)}
    \end{minipage}
  \end{figurehere}
  \item\label{exec:tt-3} 如图,$AB$ 和平面 $\alpha$ 所成的角是 $\theta_1$,$AC$ 在平面 $\alpha$ 内,$AC$ 和 $AB$ 的射影 $AB'$ 成角 $\theta_2$,设 $\angle BAC=\theta$。求证:
  \[ \cos\theta_1\cdot\cos\theta_2=\cos\theta. \]
  \item 在 \ang{60} 的二面角 $\alpha\text{-}AB\text{-}\beta$ 中,$AC\subset \alpha$,$BD\subset \beta$,且 $AC\perp AB$,$BD\perp AB$。已知 $AB=AC=BD=\alpha$,求 $CD$ 的长。
  \item 将正方体截去一个角。求证:截面是锐角三角形。
  \item 已知圆锥底面半径是 $r$,母线长是 $2r$,用平行于底面的平面把这个圆锥表面截成相等的两部分。求截下的圆锥的母线长。
  \item 要使电视卫星的电波,能直射到地球表面积的 $\dfrac{1}{3}$,卫星要发射到多高?
  \item 三棱锥 $S\text{-}ABC$ 中,侧棱 $SA$、$SB$、$SC$ 的长分别是 $a$、$b$、$c$,又 $\angle ASB=\ang{60}$,$\angle ASC=\angle BSC=\ang{90}$,求这个棱锥的体积。
  \item 圆柱的底面半径是 \qty{10}{cm},高是 \qty{15}{cm},平行于轴的截面在底面上截得的弦等于底面半径。求圆柱被截去部分的体积。 
  \item 圆台的两底面半径分别是 $a$ 和 $b$($a>b$),求这个圆台的体积与截得它的圆锥的体积的比。
  \item 面积为 \qty{2512}{cm^2} 的铝板,经冲压制成圆柱形铝桶,如果铝桶的面积和铝板面积相等,高是 \qty{10}{cm}。求这种铝桶的容积。
  \item\label{exec:tt-12} 长江大桥的钢梁结构使用铆钉铆合的(如图,单位:\unit{mm})。铆钉头是球缺形,钉身是圆柱形,铆钉插入两块钢板铆合后,两侧钉头大小相等。已知每块钢板厚 \qty{12}{mm},求钉身长度。
  \begin{figurehere}
    \begin{minipage}{\linewidth}\centering
      \includegraphics{tt-12.pdf}
      \caption*{(第 \ref{exec:tt-12} 题图)}
    \end{minipage}
  \end{figurehere}
  \item 分别以直角三角形的斜边、两直角边所在直线为轴,旋转这个直角三角形所得的三个旋转体体积为 $V$、$V_1$、$V_2$。求证:
  \[\frac{1}{V^2}=\frac{1}{V_1^2}+\frac{1}{V_2^2}.\]
  \item 正方体、等边圆柱(即底面直径与母线相等)、球的体积相等时,哪一个全面积最小?
  \item 从平面外一点像平面引两条斜线,其中一条与平面成 \ang{70} 角,另一条与平面成 \ang{15} 角。求这两条斜线组成的最大角、最小角各为多少度。
\end{question}
\section*{B 组}
\begin{question}[resume]
  \item 求证:过一点和一条直线垂直的所有直线都在一个平面内。
  \item 在\ang{120} 角的二面角 $\alpha\text{-}a\text{-}\beta$ 中,$A\in\alpha$,$B\in\beta$。
  \item 空气中有一气球,它和地面的距离是 $h$。在气球的东南 $A$ 处看气球时,仰角是 $\theta_1$;同时在气球的西南 $B$ 处看气球时,仰角是 $\theta_2$。$A$、$B$ 两地的距离是 $a$。求证:
  \[ h=\frac{a}{\sqrt{\cot^2\theta_1+\cot^2\theta_2}}.\] 
  \item 将半径为 $R$ 的四个球,两两相切地放在桌面上,求上面一个球的球心到桌面的距离。
  \item 要从半径为 \qty{210}{cm} 的圆形铁皮上剪下一些扇环做成漏斗。漏斗一端的直径是 \qty{40}{cm},另一端直径是 \qty{140}{cm},母线长 \qty{150}{cm}。计算这块铁皮能做几个漏斗,怎样剪法?
  \item 测定某些材料的硬度,可用标准钢球(直径 \qty{10}{mm})放在材料上,加上一定的压力 $P$\,\unit{kg},将材料表面压成球冠形凹痕。设凹痕的面积是 $S$\,\unit{mm^2},这时材料的硬度是 $P/S$\,\unit{kg/mm^2}。如果所加压力是 \qty{3e3}{kg},凹痕直径是 \qty{4.1}{mm},计算这材料的硬度。
  \item\label{exec:tt-22} 如图,将正方体的棱分成 4 等分,在 $\dfrac{1}{4}$ 处截取各棱角的到一个多面体,正方体体积减少几分之几(不证)?
  \begin{figurehere}
    \begin{minipage}[b]{0.48\linewidth}\centering
      \includegraphics{tt-22.pdf}
      \caption*{(第 \ref{exec:tt-22} 题图)}
    \end{minipage}
    \begin{minipage}[b]{0.48\linewidth}\centering
      \includegraphics{tt-23.pdf}
      \caption*{(第 \ref{exec:tt-23} 题图)}
    \end{minipage}
  \end{figurehere}
  \item\label{exec:tt-23} 求图中阴影部分绕轴 $l$ 旋转所成的旋转体的全面积和体积。
  \item 一个圆锥侧面的母线和底面直径相等(等边圆锥),有一内切球。已知圆锥底面直径为 $2r$,求球的体积。
  \item\label{exec:tt-25} 下图是正十二面体、正二十面体的表面去掉一个面后,连续变形所成的平面图形。数出它们的顶点数 $V$、棱数 $E$ 及面数 $F$ 来验证 $V+F-E=1$。
  \begin{figurehere}
    \begin{minipage}{\linewidth}\centering
      \includegraphics{tt-25.pdf}
      \caption*{(第 \ref{exec:tt-25} 题图)}
    \end{minipage}
  \end{figurehere}
  \item 求证:如果简单多面体的所有面都是奇数边的多边形,那么面数是偶数。
\end{question}