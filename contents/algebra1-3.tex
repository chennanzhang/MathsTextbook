\chapter{两角和与差的三角函数}
\phantomsection\pdfbookmark[1]{三角函数和差}{sum-trianglefunction}
\subsection{两角和与差的三角函数}
\subsubsection{两角和与差的余弦}
研究两角和与差的三角函数的问题,就是研究怎样利用角 $\alpha$ 与 $\beta$ 的三角函数,表示 $\alpha\pm\beta$ 的三角函数。关于 $\cos(\alpha+\beta)$ 怎样用 $\alpha,\beta$ 的三角函数来表示,我们有下面的重要公式:
\begin{equation}
\label{eq:Ca+b}
\cos(\alpha+\beta)=\cos\alpha\cos\beta-\sin\alpha\sin\beta \tag{$C_{\alpha+\beta}$\protect\footnotemark}
\end{equation}
\footnotetext{$C_{\alpha+\beta}$ 是简记符号,用来表示公式 \[\cos(\alpha+\beta)=\cos\alpha\cos\beta-\sin\alpha\sin\beta.\] 下面还有其他一些简记符号。}
\par\medskip\noindent
\begin{minipage}{0.62\linewidth}\parindent2em
\begin{proof}
  在直角坐标系 $xOy$ 内,作单位圆 $O$,并作 $\alpha,\beta$ 和 $-\beta$ 角;使 $\alpha$ 角的始边为 $Ox$,交圆 $O$ 于 $P_1$,终边交圆 $O$ 于 $P_2$;$\beta$ 角的始边为 $OP_2$,终边交圆 $O$ 于 $P_3$;$-\beta$ 角的始边为 $OP_1$,终边交圆 $O$ 于 $P_4$(\cref{fig:3-1})。这时 $P_1,P_2,P_3,P_4$ 的坐标分别是:
\begin{align*}
  &P_1\,(1,0); &
  &P_2\,(\cos\alpha,\sin\alpha);\\
  &P_3\,(\cos(\alpha+\beta),\sin(\alpha+\beta));&
  &P_4\,(\cos(-\beta),\sin(-\beta)).\\
\end{align*}
\end{proof}
\end{minipage}\hfill
\begin{minipage}{0.33\linewidth}
\begin{figurehere}
  \includegraphics{3-1.pdf}
  \caption{}\label{fig:3-1}
\end{figurehere}
\end{minipage}\par\medskip

由 $|P_1P_3|=|P_2P_4|$ 及两点间距离公式,得

\[[\cos(\alpha+\beta)-1]^2+\sin^2(\alpha+\beta)=[\cos(-\beta)-\cos\alpha]^2+[\sin(-\beta)-\sin\alpha]^2,\]
展开,整理得
\begin{gather*} 
  2-2\cos(\alpha+\beta)=2-2(\cos\alpha\cos\beta-\sin\alpha\sin\beta).\\ 
  \therefore\quad \cos(\alpha+\beta)=\cos\alpha\cos\beta-\sin\alpha\sin\beta.
\end{gather*}

上面的公式,对于任意的角 $\alpha$ 和 $\beta$ 都成立。

在上面的公式中,用 $-\beta$ 代替 $\beta$,就得到
\[\cos(\alpha-\beta)=\cos\alpha\cos(-\beta)-\sin\alpha\sin(-\beta),\]
即
\begin{equation}
  \label{eq:Ca-b}
  \cos(\alpha-\beta)=\cos\alpha\cos\beta+\sin\alpha\sin\beta. \tag{$C_{\alpha-\beta}$}
\end{equation}
\begin{example}
  不查表,求 $\cos\ang{105}$ 及 $\cos\ang{15}$ 的值。
\end{example}
\begin{solution}
\begin{align*}
  \cos\ang{105}&=\cos(\ang{60}+\ang{45})\\
  &=\cos\ang{60}\cos\ang{45}-\sin\ang{60}\sin\ang{45} \\
  &=\frac12\cdot\frac{\sqrt{2}}{2}-\frac{\sqrt{3}}{2}\cdot\frac{\sqrt{2}}{2}=\frac{\sqrt{2}-\sqrt{6}}{4};\\
  \cos\ang{15}&=\cos(\ang{45}-\ang{30})\\
  &=\cos\ang{45}\cos\ang{30}+\sin\ang{45}\sin\ang{30} \\
  &=\frac{\sqrt{2}}{2}\cdot\frac{\sqrt{2}}{2}+\frac{\sqrt{2}}{2}\cdot\frac{1}{2}=\frac{\sqrt{6}+\sqrt{2}}{4}.
\end{align*}
\end{solution}

\begin{example}
  已知 $\sin\alpha=\dfrac23$,$\alpha\in\left(\dfrac\uppi2,\uppi\right)$,$\cos\beta=-\dfrac34$,$\beta\in\left(\uppi,\dfrac{3\uppi}{2}\right)$,求 $\cos(\alpha-\beta)$ 的值。
\end{example}
\begin{solution}
由 $\sin\alpha=\dfrac23$,$\alpha\in\left(\dfrac\uppi2,\uppi\right)$,得
\[\cos\alpha=-\sqrt{1-\sin^2\alpha}=-\sqrt{1-\left(\frac23\right)^2}=-\frac{\sqrt{5}}{3};\]

又由 $\cos\beta=-\dfrac34$,$\beta\in\left(\uppi,\dfrac{3\uppi}{2}\right)$,得 
\[\sin\beta=-\sqrt{1-\cos^2\beta}=-\sqrt{1-\left(-\frac34\right)^2}=-\frac{\sqrt{7}}{4}.\]
\begin{align*}
  \therefore \quad \cos(\alpha-\beta) &=\cos\alpha\cos\beta+\sin\alpha\sin\beta\\
  &=\left(-\frac{\sqrt{5}}{3}\right)\cdot\left(-\frac{3}{4}\right)+\frac23\cdot\left(-\frac{\sqrt{7}}{4}\right)\\
  &=\frac{3\sqrt{5}-2\sqrt{7}}{12}.
\end{align*}
\end{solution}

\begin{example}
  证明:公式
  \begin{align*}
    \cos\left(\frac\uppi2-\alpha\right)&=\sin\alpha,\\
    \sin\left(\frac\uppi2-\alpha\right)&=\cos\alpha,
  \end{align*}
  当 $\alpha$ 为任意角时仍然成立。
\end{example}
\begin{proof}
  利用公\cref{eq:Ca-b},可得
\begin{gather*}
  \cos\left(\frac\uppi2-\alpha\right)=\cos\frac\uppi2\cos\alpha+\sin\frac\uppi2\sin\alpha\\ 
  \because \qquad \cos\frac\uppi2=0,\qquad \sin\frac\uppi2=1,\\ 
  \therefore\qquad \cos\left(\frac\uppi2-\alpha\right)=\sin\alpha.
\end{gather*}

因为上式中得 $\alpha$ 为任意角,如果把 $\left(\dfrac\uppi2-\alpha\right)$ 换成 $\alpha$,就得 
\[\cos\alpha=\sin\left(\frac\uppi2-\alpha\right),\]
即
\[\sin\left(\frac\uppi2-\alpha\right)=\cos\alpha.\]
\end{proof}

利用上述两式,不难证明下面两式在两边都有意义时成立:
\begin{align*}
  \tan\left(\frac\uppi2-\alpha\right)&=\cot\alpha,\\
  \cot\left(\frac\uppi2-\alpha\right)&=\tan\alpha,
\end{align*}

以上四个公式时当 $\alpha$ 为任意角时 $\left(\dfrac\uppi2-\alpha\right)$ 的诱导公式。如果把其中的 $\alpha$ 换成 $-\alpha$,就可得到当 $\alpha$ 为任意角时 $\left(\dfrac\uppi2+\alpha\right)$ 的诱导公式:
\begin{align*}
  \cos\left(\frac\uppi2+\alpha\right)&=-\sin\alpha,&
  \sin\left(\frac\uppi2+\alpha\right)&=\cos\alpha,\\
  \tan\left(\frac\uppi2+\alpha\right)&=-\cot\alpha,&
  \cot\left(\frac\uppi2+\alpha\right)&=-\tan\alpha,
\end{align*}

\begin{Practice}
  \begin{question}[itemsep=4pt]
    \item 等式 $\cos(\alpha+\beta)=\cos\alpha+\cos\beta$ 成立吗?用 $\alpha=\ang{60},\beta=\ang{30}$ 代入进行检验。 
    \item 不查表,求下列各式的值:
    \begin{tasks}[after-skip=5pt](3)
      \task $\cos\ang{75}$;
      \task $\cos\ang{165}$;
      \task $\cos\left(-\dfrac{61\uppi}{12}\right)$。
    \end{tasks}
    \item 已知 $\sin\alpha=\dfrac{15}{17}$,$\alpha\in\left(\dfrac\uppi2,\uppi\right)$,求 $\cos\left(\dfrac\uppi3-\alpha\right)$ 的值。
    \item 已知 $\cos\theta=-\dfrac{5}{13}$,$\alpha\in\left(\uppi,\dfrac{3\uppi}{2}\right)$,求 $\cos\left(\theta+\dfrac\uppi6\right)$ 的值。
    \item 把下列各三角函数化成 \ang{0}~\ang{45} 的角的三角函数:
    \begin{tasks}(3)
      \task $\cos\ang{1856}$;
      \task $\sin\ang{-1190}$;
      \task $\cot\ang{-310}$;
    \end{tasks}
    \item 不查表,求下列各式的值:
    \begin{tasks}(2)
      \task $\cos\ang{80}\cos\ang{20}+\sin\ang{80}\sin\ang{20}$;
      \task $\cos^2\ang{15}-\sin^2\ang{15}$。
    \end{tasks}
  \end{question}
\end{Practice}

\subsubsection{两角和与差的正弦}
因为 $\sin(\alpha+\beta)=\cos\left[\dfrac\uppi2-(\alpha+\beta)\right]$,而
\begin{align*}
  \cos\left[\dfrac\uppi2-(\alpha+\beta)\right]&=\cos\left[\left(\dfrac\uppi2-\alpha\right)-\beta\right]\\
  &=\cos\left(\dfrac\uppi2-\alpha\right)\cos\beta+\sin\left(\dfrac\uppi2-\alpha\right)\sin\beta\\
  &=\sin\alpha\cos\beta+\cos\alpha\sin\beta,
\end{align*}
所以
\begin{equation}
  \label{eq:Sa+b}
  \sin(\alpha+\beta)=\sin\alpha\cos\beta+\cos\alpha\sin\beta. \tag{$S_{\alpha+\beta}$}
\end{equation}

把公\cref{eq:Sa+b} 中的 $\beta$ 换成 $-\beta$,得 
\[\sin(\alpha-\beta)=\sin\alpha\cos(-\beta)+\cos\alpha\sin(-\beta),\]
即
\begin{equation}
  \label{eq:Sa-b}
  \sin(\alpha-\beta)=\sin\alpha\cos\beta-\cos\alpha\sin\beta. \tag{$S_{\alpha-\beta}$}
\end{equation}

\begin{example}
  不查表,求 $\sin\ang{75}$ 的值。
\end{example}
\begin{solution}
\begin{align*}
  \sin\ang{75}&=\sin(\ang{45}+\ang{30})\\ 
  &=\sin\ang{45}\cos\ang{30}+\cos\ang{45}\sin\ang{30}\\ 
  &=\frac{\sqrt{2}}{2}\cdot\frac{\sqrt{3}}{2}+\frac{\sqrt{2}}{2}\cdot\frac{1}{2}=\frac{\sqrt{6}+\sqrt{2}}{4}.
\end{align*}
\end{solution}

\begin{example}
已知 $\cos\varphi=\dfrac35$,$\varphi\in\left(0,\dfrac\uppi2\right)$,求 $\sin\left(\varphi-\dfrac\uppi6\right)$。
\end{example}
\begin{solution}
$\because\qquad \cos\varphi=\dfrac35$,$\varphi\in\left(0,\dfrac\uppi2\right)$,

$\therefore\qquad \sin\varphi=\sqrt{1-\left(\dfrac35\right)^2}=\dfrac45.$
\begin{align*}
\therefore\qquad \sin\left(\varphi-\dfrac\uppi6\right)&=\sin\varphi\cos\dfrac\uppi6-\cos\varphi\sin\dfrac\uppi6\\
&=\frac45\cdot\frac{\sqrt{3}}{2}-\frac35\cdot\frac12=\frac{4\sqrt{3}-3}{10}.
\end{align*}
\end{solution}

\begin{example}
求证:$\dfrac{\sin(\alpha+\beta)\sin(\alpha-\beta)}{\sin^2\alpha\cos^2\beta}=1-\cot^2\alpha\tan^2\beta.$
\end{example}
\begin{proof}
\begin{align*}
  \text{左边}&=\frac{(\sin\alpha\cos\beta+\cos\alpha\sin\beta)(\sin\alpha\cos\beta-\cos\alpha\sin\beta)}{\sin^2\alpha\cos^2\beta}\\ 
  &=\frac{\sin^2\alpha\cos^2\beta-\cos^2\alpha\sin^2\beta}{\sin^2\alpha\cos^2\beta}=1-\frac{\cos^2\alpha\cos^2\beta}{\sin^2\alpha\sin^2\beta}\\
  &=1-\cot^2\alpha\tan^2\beta=\text{右边}.
\end{align*}

$\therefore \quad$ 原式成立。
\end{proof}

\begin{example}
  求证 $\cos\alpha+\sqrt{3}\sin\alpha=2\sin\left(\dfrac\uppi6+\alpha\right)$。
\end{example}
\begin{solution}[证法一]
\begin{align*}
  \text{左边}&=2\left(\frac12\cos\alpha+\frac{\sqrt{3}}{2}\sin\alpha\right)\\ 
  &=2\left(\sin\frac\uppi6\cos\alpha+\cos\frac\uppi6\sin\alpha\right)\\
  &=2\sin\left(\frac\uppi6+\alpha\right)=\text{右边}.
\end{align*}

$\therefore \quad$ 原式成立。
\end{solution}

\begin{solution}[证法二]
\begin{align*}
  \text{右边}&=2\left(\sin\frac\uppi6\cos\alpha+\cos\frac\uppi6\sin\alpha\right)\\ 
  &=2\left(\frac12\cos\alpha+\frac{\sqrt{3}}{2}\sin\alpha\right)\\
  &=\cos\alpha+\sqrt{3}\sin\alpha=\text{左边}.
\end{align*}

$\therefore \quad$ 原式成立。
\end{solution}

\begin{Practice}
  \begin{question}[itemsep=3pt]
    \item 等式 $\sin(\alpha+\beta)=\sin\alpha+\sin\beta$ 成立吗?用 $\alpha=\ang{60},\beta=\ang{30}$ 代入进行检验。
    \item 不查表,求下列各式的值:
    \begin{tasks}[after-skip=5pt](3)
      \task $\sin\ang{105}$;
      \task $\sin\ang{15}$;
      \task $\sin\left(-\dfrac{5\uppi}{12}\right)$。
    \end{tasks}
    \item 已知 $\cos\theta=-\dfrac35$,$\theta\in\left(\dfrac\uppi2,\uppi\right)$,求 $\sin\left(\theta+\dfrac\uppi3\right)$ 的值。
    \item 已知 $\sin\alpha=\dfrac23$,$\cos\beta=-\dfrac34$,且 $\alpha,\beta$ 都是第二象限角,求 $\sin(\alpha-\beta)$ 及 $\cos(\alpha-\beta)$。
    \item 不查表,求下列各式的值:
    \begin{tasks}(2)
      \task $\sin\ang{13}\cos\ang{17}+\cos\ang{13}\sin\ang{17}$;
      \task $\sin\ang{70}\cos\ang{25}-\sin\ang{20}\sin\ang{25}$。
    \end{tasks}
    \item 求证:
    \begin{tasks}
      \task $\dfrac12\left(\cos\alpha+\sqrt{3}\sin\alpha\right)=\cos(\ang{60}-\alpha)$;
      \task $\sin\left(\dfrac{5\uppi}{6}-\varphi\right)+\sin\left(\dfrac{5\uppi}{6}+\varphi\right)=\cos\varphi$。
    \end{tasks}
  \end{question}
\end{Practice}

\subsubsection{两角和与差的正切}
由 $\tan(\alpha+\beta)=\dfrac{\sin(\alpha+\beta)}{\cos(\alpha+\beta)}=\dfrac{\sin\alpha\cos\beta+\cos\alpha\sin\beta}{\cos\alpha\cos\beta-\sin\alpha\sin\beta}$,把最后一个分式的分子、分母分别除以 $\cos\alpha\cdot\cos\beta$($\cos\alpha\neq 0$,$\cos\beta\neq 0$),得
\begin{equation}
\label{eq:Ta+b}
\tan(\alpha+\beta)=\frac{\tan\alpha+\tan\beta}{1-\tan\alpha\tan\beta}\tag{$T_{\alpha+\beta}$}
\end{equation}

把公\cref{eq:Ta+b} 中的 $\beta$ 换成 $-\beta$,得 
\begin{equation}
\label{eq:Ta-b}
\tan(\alpha-\beta)=\frac{\tan\alpha-\tan\beta}{1+\tan\alpha\tan\beta}\tag{$T_{\alpha-\beta}$}
\end{equation}

\alertwarning{在两角和与差的正切的公式中,$\alpha,\beta$ 的取值范围,应该是使 $\tan\alpha$,$\tan\beta$,$\tan(\alpha+\beta)$ 都存在的那些值,即 $\alpha$,$\beta$ 及 $\alpha+\beta$ 都不能取 $\dfrac\uppi2+n\uppi$($n\in\mathbb{Z}$)。例如,如果 $\alpha=\frac\uppi2$,$\beta=\frac\uppi3$,那么求 $\tan(\alpha+\beta)$ 的值,就不能用和角的正切公式,而应该用诱导公式。}

\begin{example}
已知 $\tan\alpha=\dfrac13$,$\tan\beta=-2$。
\begin{enumerate}
  \item 求 $\cot(\alpha-\beta)$;
  \item 求 $\alpha+\beta$ 的值(其中 $\ang{0}<\alpha<\ang{90}$,$\ang{90}<\beta<\ang{180}$)。
\end{enumerate}
\end{example}
\begin{solution}
\begin{enumerate}
  \item $\because\quad \tan\alpha=\dfrac13$,$\tan\beta=-2$,而 
  \[ \cot(\alpha-\beta)=\frac{1}{\tan(\alpha-\beta)},\]
  其中,
  \begin{gather*} 
    \tan(\alpha-\beta)=\frac{\tan\alpha-\tan\beta}{1+\tan\alpha\tan\beta}=\frac{\dfrac13+2}{1+\dfrac13\times(-2)}=7.\\
    \therefore \quad \cot(\alpha-\beta)=\frac17.
  \end{gather*}
  \item 由 $\tan\alpha=\dfrac13$,$\tan\beta=-2$,得
  \[\tan(\alpha+\beta)=\frac{\tan\alpha+\tan\beta}{1-\tan\alpha\tan\beta}=\frac{\dfrac13-2}{1-\dfrac13\times(-2)}=-1;\]
  又因 $\ang{0}<\alpha<\ang{90}$,$\ang{90}<\beta<\ang{180}$,所以
  \[\ang{90}<\alpha+\beta<\ang{270}.\]

  在 \ang{90} 与 \ang{270} 之间,只有 \ang{135} 的正切值为 $-1$,
  \[\therefore \qquad \alpha+\beta=\ang{135}.\]
\end{enumerate}
\end{solution}

\begin{example}
  计算 $\dfrac{1+\tan\ang{75}}{1-\tan\ang{75}}$ 的值。
\end{example}
\begin{analyze}
因为 $\tan\ang{45}=1$,所以原式可以看成是
\[\frac{\tan\ang{45}+\tan\ang{75}}{1-\tan\ang{45}\tan\ang{75}}.\]
这样,我们就可以运用两角和的正切公式,把原式化为
\[\tan(\ang{45}+\ang{75}),\]
而因 $\ang{45}+\ang{75}=\ang{120}$ 是特殊角,所以可以求得原式的值。
\end{analyze}

\begin{solution}
  $\because\qquad \tan\ang{45}=1$,
  \begin{align*}
    \therefore\qquad \frac{1+\tan\ang{75}}{1-\tan\ang{75}}&=\frac{tan\ang{45}+\tan\ang{75}}{1-\tan\ang{45}\tan\ang{75}}\\
    &=\tan(\ang{45}+\ang{75})\\ 
    &=\tan\ang{120}=-\sqrt{3}.
  \end{align*}
\end{solution}

\begin{example}
设 $\tan\alpha,\tan\beta$ 是一元二次方程 $ax^2+bx+c=0$($b\neq 0$)的两个根,求 $\cot(\alpha+\beta)$ 的值。
\end{example}
\begin{solution}
在一元二次方程 $ax^2+bx+c=0$ 中,$a\neq 0$。由一元二次方程根与系数的关系,得
\begin{gather*}
  \tan\alpha+\tan\beta=-\frac{b}{a},\\ 
  \tan\alpha\tan\beta=\frac{c}{a}.
\end{gather*}
而
\[\cot(\alpha+\beta)=\frac{1}{\tan(\alpha+\beta)}=\frac{1-\tan\alpha\tan\beta}{\tan\alpha+\tan\beta}.\]
由题设 $b\neq 0$,故 $\tan\alpha+\tan\beta\neq 0$,代入,得
\[\cot(\alpha+\beta)=\frac{1-\dfrac{c}{a}}{-\dfrac{b}{a}}=\frac{a-c}{-b}=\frac{c-a}{b}.\]
\end{solution}

\begin{Practice}
  \begin{question}[itemsep=3pt]
    \item 不查表,求下列各式的值:
    \begin{tasks}(3)
      \task $\tan\ang{75}$;
      \task $\tan\ang{15}$;
      \task $\cot\ang{105}$。
    \end{tasks}
    \item 不查表,求下列各式的值:
    \begin{tasks}[after-skip=5pt,before-skip=5pt](2)
      \task $\dfrac{\tan\ang{12}+\tan\ang{33}}{1-\tan\ang{12}\tan\ang{33}}$;
      \task $\dfrac{1-\tan\ang{15}}{1+\tan\ang{15}}$。
    \end{tasks}
    \item 已知 $\tan\alpha=2$,求 $\tan\left(\alpha-\dfrac\uppi4\right)$ 的值。
    \item 已知 $\tan\alpha=2$,$\tan\beta=3$,并且 $\alpha,\beta$ 都是锐角,求证:$\alpha+\beta=\ang{135}$。
  \end{question}
\end{Practice}

\begin{Exercise}
  \begin{question}
    \item 已知 $\sin\alpha=\dfrac{15}{17}$,$\cos\beta=-\dfrac{5}{13}$,并且 $\alpha,\beta$ 都是第二象限的角,求 $\cos(\alpha+\beta)$ 和 $\cos(\alpha-\beta)$ 的值。
    \item 在 $\triangle ABC$ 中,已知 $\cos A=-\dfrac45$,$\cos B=\dfrac{12}{13}$,求 $\cos C$ 的值。
    \item 求证:
    \begin{tasks}[before-skip=5pt,after-skip=5pt,after-item-skip=7pt](2)
      \task $\sin\left(\dfrac{3\uppi}{2}-\alpha\right)=-\alpha$;
      \task $\cos\left(\dfrac{3\uppi}{2}-\alpha\right)=-\sin\alpha$;
      \task $\tan\left(\dfrac{3\uppi}{2}-\alpha\right)=\cot\alpha$;
      \task $\sin\left(\dfrac{3\uppi}{2}+\alpha\right)=-\cos\alpha$;
      \task $\cos\left(\dfrac{3\uppi}{2}+\alpha\right)=\sin\alpha$;
      \task $\tan\left(\dfrac{3\uppi}{2}+\alpha\right)=-cot\alpha$。
    \end{tasks}
    \item 化简:
    \begin{tasks}[after-skip=5pt,before-skip=5pt,after-item-skip=7pt](2)
      \task $\sin(\ang{30}+\alpha)-\sin(\ang{30}-\alpha)$;
      \task $\sin\left(\dfrac\uppi3+\alpha\right)+\sin\left(\dfrac\uppi3-\alpha\right)$;
      \task $\cos\left(\dfrac\uppi4+\phi\right)-\cos\left(\dfrac\uppi4-\phi\right)$;
      \task $\cos(\ang{60}+\theta)+\cos(\ang{60}-\theta)$;
      \task $\sin\ang{58}\cos\ang{37}-\cos\ang{58}\sin\ang{37}$;
      \task $\cos\ang{24}\cos\ang{69}-\sin\ang{24}\sin\ang{69}$;
      \task $\sin\ang{14}\cos\ang{16}+\sin\ang{76}\cos\ang{74}$;
      \task $\sin\ang{21}\cos\ang{81}-\sin\ang{69}\cos\ang{9}$;
      \task! $\sin(\alpha-\beta)\cos\beta+\cos(\alpha-\beta)\sin\beta$;
      \task! $\cos(\alpha+\beta)\cos\beta+\sin(\alpha+\beta)\sin\beta$;
      \task! $\cos(\ang{36}+x)\cos(\ang{54}-x)-\sin(\ang{36}+x)\sin(\ang{54}-x)$
      \task! $\sin(\ang{70}+\alpha)\cos(\ang{10}+\alpha)-\cos(\ang{70}+\alpha)\sin(\ang{170}-\alpha)$。
    \end{tasks}
    \item 求证:
    \begin{tasks}[after-skip=5pt,before-skip=5pt,after-item-skip=7pt]
      \task $\dfrac{\sqrt{3}}{2}\sin\alpha-\dfrac12\cos\alpha=\sin\left(\alpha-\dfrac\uppi6\right)$;
      \task $\cos\theta-\sin\theta=\sqrt{2}\cos\left(\dfrac\uppi4+\theta\right)$;
      \task $\cos(\alpha+\beta)\cos(\alpha-\beta)=\cos^2\alpha-\sin^2\beta$;
      \task $\sin(\alpha+\beta)\sin(\alpha-\beta)=\sin^2\alpha-\sin^2\beta$;
      \task $\sin(\alpha+\beta)\cos(\alpha-\beta)=\sin\alpha\cos\alpha+\sin\beta\cos\beta$。
    \end{tasks}
    \item 解答:
    \begin{enumerate}[itemindent=2.4em,topsep=5pt]
      \item 已知 $\tan x=\dfrac14$,$\tan y=-3$,求 $\tan(x+y)$ 的值;
      \item 已知 $\tan\alpha=2k+1$,$\tan\beta=2k-1$,求 $\cot(\alpha-\beta)$ 的值。
    \end{enumerate}
    \item 已知 $\cos\theta=-\dfrac{12}{13}$,$\theta\in\left(\uppi,\dfrac{3\uppi}{2}\right)$,求 $\sin\left(\theta-\dfrac\uppi4\right)$,$\cos\left(\theta-\dfrac\uppi4\right)$ 和 $\tan\left(\theta-\dfrac\uppi4\right)$ 的值。
    \item 用 $\cot\alpha,\cot\beta$ 表示 $\cot(\alpha\pm\beta)$,并求 $\dfrac{1-\cot\ang{15}}{1+\cot\ang{15}}$ 的值。
    \item 化简:
    \begin{tasks}[after-skip=5pt,before-skip=5pt,after-item-skip=7pt](2)
      \task $\dfrac{\tan\ang{53}-\tan\ang{23}}{1+\tan\ang{53}\tan\ang{23}}$;
      \task $\dfrac{\tan2\theta-\tan\theta}{1+\tan2\theta\tan\theta}$;
      \task $\dfrac{1-\tan\ang{15}}{1+\cot\ang{75}}$;
      \task $\dfrac{1+\tan\theta}{1-\tan\theta}$。
    \end{tasks}
    \item 求证:
    \begin{tasks}[after-skip=5pt,before-skip=5pt,after-item-skip=7pt](2)
      \task! $\tan(x+y)\cdot\tan(x-y)=\dfrac{\tan^2x-\tan^2y}{1-tan^2x\tan^2y}$;
      \task  $\cot\left(\dfrac\uppi4+\theta\right)=\dfrac{1-\tan\theta}{1+\tan\theta}$;
      \task  $\dfrac{\tan x+\tan y}{\tan x-\tan y}=\dfrac{\sin(x+y)}{\sin(x-y)}$。
    \end{tasks}
    \item 已知 $\tan\theta=-\dfrac12$,$\tan\phi=\dfrac13$,并且 $\theta,\phi$ 都是锐角,求证:$\theta+\phi=\ang{45}$。
    \item 已知 $a\sin(\theta+\alpha)=b\sin(\theta+\beta)$,求证:
    \[\tan\theta=\frac{b\sin\beta-a\sin\alpha}{a\cos\alpha-b\cos\beta}.\]
    \item\label{exec:10-13}如图,在 $\triangle ABC$ 中,$AD\perp BC$,垂足为 $D$,且 $BD:DC:AD=2:3:6$,求 $\angle BAC$ 的度数。
    \item 已知 $\tan\alpha,\tan\beta$ 是方程 $x^2+6x+7=0$ 的两个根,求证
    \[\sin(\alpha+\beta)=\cos(\alpha+\beta)\]
    \begin{figurehere}
      \begin{minipage}{\linewidth}\centering
        \includegraphics{ex10-13.pdf}
        \caption*{(第~\ref{exec:10-13}~题图)}
      \end{minipage}
    \end{figurehere}
  \end{question}
\end{Exercise}

\subsection{二倍角的正弦、余弦、正切}\label{subsec:double-angle}
在公\cref{eq:Sa+b}、$(C_{\alpha+\beta})$、\eqref{eq:Ta+b} 中,当 $\alpha=\beta$ 时,就可以得出相应的二倍角的三角函数公式:
\begin{gather}
  \label{eq:S2a}\sin2\alpha=2\sin\alpha\cos\alpha; \tag{$S_{2\alpha}$}\\
  \label{eq:C2a}\cos2\alpha=\cos^2\alpha-\sin^2\alpha; \tag{$C_{2\alpha}$}\\
  \label{eq:T2a}\tan2\alpha=\frac{2\tan\alpha}{1-\tan^2\alpha}. \tag{$T_{2\alpha}$}
\end{gather}

因为 $\sin^2\alpha+\cos^2\alpha=1$,所以公\cref{eq:C2a} 可以变形为
\[ \cos2\alpha=2\cos^2\alpha-1,\]
或
\begin{equation}
  \label{eq:C2a'}\cos2\alpha=1-2\sin^2\alpha.\tag{$C'_{2\alpha}$}
\end{equation}

有了二倍角的三角函数公式,就可以用单角的三角函数来表示二倍角的三角函数。

\begin{example}
已知 $\sin\alpha=\dfrac{5}{13}$,$\alpha\in\left(\dfrac\uppi2,\uppi\right)$,求 $\sin2\alpha$,$\cos2\alpha$,$\tan2\alpha$ 的值。
\end{example}
\begin{solution}
$\because \sin\alpha=\dfrac{5}{13}$,$\alpha\in\left(\dfrac\uppi2,\uppi\right)$,
\begin{align*}
  \therefore \quad \cos\alpha &=-\sqrt{1-\sin^2\alpha}=-\sqrt{1-\left(\frac{5}{13}\right)^2}=-\frac{12}{13}.\\ 
  \therefore \quad \sin2\alpha &=2\sin\alpha\cos\alpha=2\times\frac{5}{13}\times\left(-\frac{12}{13}\right)=-\frac{120}{169},\\ 
  \cos2\alpha &=1-2\sin^2\alpha=1-2\times\left(\frac{5}{13}\right)^2=\frac{119}{169},\\ 
  \tan2\alpha&=\frac{\sin2\alpha}{\cos2\alpha}=\frac{-\dfrac{120}{169}}{\dfrac{119}{169}}=-1\frac{1}{119}.
\end{align*}
\end{solution}


\begin{example}\mbox{}\par
\begin{enumerate}
  \item 用 $\sin\theta$ 表示 $\sin3\theta$;
  \item 用 $\cos\theta$ 表示 $\cos3\theta$。
\end{enumerate}
\end{example}
\begin{solution}
\begin{enumerate}
  \item 
  \begin{align*}
    \sin3\theta&=\sin(2\theta +\theta)=\sin2\theta\cos\theta+\cos2\theta\sin\theta\\ 
    &=2\sin\theta\cos^2\theta+(1-2\sin^2\theta)\sin\theta \\
    &=2\sin\theta(1-\sin^2\theta)+\sin\theta-2\sin^3\theta \\
    &= 3\sin\theta-4\sin^3\theta.\\ 
    \therefore\quad \sin3\theta&=3\sin\theta-4\sin^3\theta.
  \end{align*}
  \item
  \begin{align*}
    \cos3\theta&=\cos(2\theta +\theta)= \cos2\theta\cos\theta-\sin2\theta\sin\theta\\ 
    &=(2\cos^2\theta-1)\cos\theta-2\sin^2\theta\cos\theta \\
    &=2\cos^3\theta-\cos\theta-2\cos\theta(1-\cos^2\theta) \\
    &=4\cos^3\theta-3\cos\theta.\\ 
    \therefore\quad \cos3\theta&=4\cos^3\theta-3\cos\theta.
  \end{align*}
\end{enumerate}
\end{solution}

\begin{example}
  求证
\[[\sin\theta(1+\sin\theta)+\cos\theta(1+\cos\theta)]\times[\sin\theta(1-\sin\theta)+\cos\theta(1-\cos\theta)]=\sin2\theta.\]
\end{example}
\begin{proof}
\begin{align*}
  \text{左边}&=(\sin\theta+\sin^2\theta+\cos^2\theta+\cos\theta)\times(\sin\theta-\sin^2\theta-\cos^2\theta+\cos\theta)\\ 
  &=(\sin\theta+\cos\theta+1)(\sin\theta+\cos\theta-1)\\ 
  &=(\sin\theta+\cos\theta)^2-1\\ 
  &=2\sin\theta\cos\theta\\
  &=\sin2\theta=\text{右边}.
\end{align*}

$\therefore\quad$ 原式成立。
\end{proof}


\begin{example}
  化简 $\sin\ang{50}(1+\sqrt{3}\tan\ang{10})$。
\end{example}
\begin{solution}
\begin{align*}
  \sin\ang{50}(1+\sqrt{3}\tan\ang{10})&=\sin\ang{50}\left(1+\frac{\sqrt{3}\sin\ang{10}}{\cos\ang{10}}\right)\\ 
  &=\sin\ang{50}\cdot\frac{2\left(\dfrac12\cos\ang{10}+\dfrac{\sqrt{3}}{2}\sin\ang{10}\right)}{\cos\ang{10}}\\ 
  &=2\sin\ang{50}\cdot\frac{\sin\ang{30}\cos\ang{10}+\cos\ang{30}\sin\ang{10}}{\cos\ang{10}}\\ 
  &=2\cos\ang{40}\cdot\frac{\sin\ang{40}}{\cos\ang{10}}\\ 
  &=\frac{\sin\ang{80}}{\cos\ang{10}}=\frac{\cos\ang{10}}{\cos\ang{10}}=1.
\end{align*}
\end{solution}


\begin{example}
  把一段半径为 $R$ 的圆木,锯成横截面为矩形的木料,怎样锯法才能使横截面的面积最大?
\end{example}
\begin{solution}
因为锯得的矩形横截面是圆内接矩形,所以它的对角线是圆的直径,其长度应为 $2R$。设对角线与一个边的夹角为 $\theta$(\cref{fig:3-2}),则矩形的长与宽分别为 $2R\cos\theta$,$2R\sin\theta$。因此,矩形的面积
\par\noindent
\begin{minipage}{0.55\linewidth}\parindent2em
\begin{align*}
  S&=2R\cos\theta\cdot 2R\sin\theta\\
   &=2R^2\cdot 2\sin\theta\cos\theta\\
   &=2R^2\cdot \sin2\theta.
\end{align*}

因为 $\sin2\theta\leqslant 1$,所以 $S\leqslant 2R^2$。

当 $\sin2\theta$ 取最大值 1 时,$S$ 取最大值 $2R^2$。
\end{minipage}\hfill
\begin{minipage}{0.4\linewidth}
  \begin{figurehere}
    \includegraphics{3-2.pdf}
    \caption{}\label{fig:3-2}
  \end{figurehere}
\end{minipage}\par\medskip

所以,当 $2\theta=\ang{90}$,即 $\theta=\ang{45}$ 时,圆内接矩形的面积最大,这时圆内接矩形为内接正方形。

答:以圆木的直径为对角线,锯成横截面为正方形的木料时,横截面的面积最大。
\end{solution}

\begin{Practice}
  \begin{question}[itemsep=5pt]
    \item 不查表,求下列各式的值:
    \begin{tasks}[after-item-skip=7pt,after-skip=5pt,before-skip=5pt](2)
      \task $2\sin\ang{67;30}\cos\ang{67;30}$;
      \task $\cos^2\dfrac\uppi8-\sin^2\dfrac\uppi8$;
      \task $2\cos^2\dfrac{\uppi}{12}-1$;
      \task $1-2\sin^2\ang{75}$;
      \task $\dfrac{2\tan\ang{22.5}}{1-\tan^2\ang{22.5}}$;
      \task $\sin\ang{15}\cos\ang{15}$;
      \task $1-2\sin^2\ang{750}$;
      \task $\dfrac{2\tan\ang{150}}{1-\tan^2\ang{150}}$。
    \end{tasks}
    \item 化简:
    \begin{tasks}[after-item-skip=7pt,after-skip=5pt,before-skip=5pt](2)
      \task $(\sin\alpha-\cos\alpha)^2$;
      \task $\sin\dfrac\theta2\cos\dfrac\theta2$;
      \task $\cos^4\varphi-\sin^4\varphi$;
      \task $\dfrac{1}{1-\tan\theta}-\dfrac{1}{1+\tan\theta}$。
    \end{tasks}
    \item 已知 $\sin\alpha=0.8$,$\alpha\in\left(0,\dfrac\uppi2\right)$,求 $\sin2\alpha$,$\cos2\alpha$ 的值。
    \item 已知 $\cos\alpha=-\dfrac{12}{13}$,$\alpha\in\left(\dfrac\uppi2,\uppi\right)$,求 $\cos2\alpha$,$\sin2\alpha$,$\tan2\alpha$,$\cot2\alpha$ 的值。
    \item 已知 $\tan\alpha=\dfrac12$,求 $\tan2\alpha$,$\cot2\alpha$ 的值。
    \item 写出由 $\tan\alpha$ 求 $\tan3\alpha$ 的公式。
    \item 证明下列恒等式:
    \begin{tasks}[after-item-skip=7pt,after-skip=5pt,before-skip=5pt](2)
      \task $\sin^2\theta=\dfrac{1-\cos2\theta}{2}$;
      \task $\cos^2\theta=\dfrac{1+\cos2\theta}{2}$;
      \task $2\sin(\uppi+\alpha)\cos(\uppi-\alpha)=\sin2\alpha$;
      \task $\cos^4\dfrac{x}{2}-\sin^4\dfrac{x}{2}=\cos x$;
      \task $1+2\cos^2\theta-\cos2\theta=2$;
      \task $\dfrac{1-\cos2\alpha}{\sin\alpha}=2\sin\alpha$;
      \task $\dfrac{2\cot\alpha}{\cot^2\alpha-1}=\tan2\alpha$;
      \task $\dfrac{\sin2\theta}{1-\cos2\theta}=\cot\theta$;
      \task $\cot\varphi-\cot2\varphi=\csc2\varphi$;
      \task $\dfrac{\sin\alpha\cos\alpha}{\sin^2\alpha-\cos^2\alpha}=-\dfrac12\tan2\alpha$。
    \end{tasks}
  \end{question}
\end{Practice}

\subsection{半角的正弦、余弦、正切}
\cref{subsec:double-angle} 我们研究了用单角的三角函数表示二倍角的三角函数。这一节我们研究如何用单角的三角函数表示半角(单角的一半)的三角函数。

由 $\cos2\alpha=1-2\sin^2\alpha=2\cos^2\alpha-1$,得
\[\cos\alpha =1-2\sin^2\frac\alpha2=2\cos^2\frac\alpha2-1,\]
即
\begin{align*}
  2\sin^2\frac\alpha2&=1-\cos\alpha;\\
  2\cos^2\frac\alpha2&=1+\cos\alpha.
\end{align*}
所以 
\begin{align}
  \label{eq:Sa/2}\sin\frac\alpha2&=\pm\sqrt{\frac{1-\cos\alpha}{2}};\tag{$S_{\frac{\alpha}{2}}$}\\
  \label{eq:Ca/2}\cos\frac\alpha2&=\pm\sqrt{\frac{1+\cos\alpha}{2}}.\tag{$C_{\frac{\alpha}{2}}$}
\end{align}
将这两个公式左边右边分别相除,又可得
\begin{equation}
  \label{eq:Ta/2}\tan\frac\alpha2=\pm\sqrt{\frac{1-\cos\alpha}{1+\cos\alpha}}.\tag{$T_{\frac{\alpha}{2}}$}
\end{equation}

这三个公式中根号前的符号,由 $\dfrac\alpha2$ 所在的象限来确定。如果没有给出限定符号的条件,根号前面应保持正负两个符号。

$\tan\dfrac\alpha2$ 还可以用 $\sin\alpha$,$\cos\alpha$ 的不带根号的式子来表示:
\[\tan\frac\alpha2=\frac{\sin\dfrac\alpha2}{\cos\dfrac\alpha2}=\frac{\sin\dfrac\alpha2\cdot2\cos\dfrac\alpha2}{\cos\dfrac\alpha2\cdot2\cos\dfrac\alpha2}=\frac{\sin\alpha}{1+\cos\alpha},\]
或
\[\tan\frac\alpha2=\frac{\sin\dfrac\alpha2}{\cos\dfrac\alpha2}=\frac{\sin\dfrac\alpha2\cdot2\sin\dfrac\alpha2}{\cos\dfrac\alpha2\cdot2\sin\dfrac\alpha2}=\frac{1-\cos\alpha}{\sin\alpha}.\]
即
\begin{equation}
  \label{eq:Ta/2'}\tan\frac{\alpha}{2}=\frac{\sin\alpha}{1+\cos\alpha}=\frac{1-\cos\alpha}{\sin\alpha}\tag{$T'_{\frac\alpha2}$}
\end{equation}

\begin{example}
已知 $\cos\alpha=\dfrac12$,求 $\sin\dfrac\alpha2$,$\cos\dfrac\alpha2$,$\tan\dfrac\alpha2$。
\end{example}
\begin{solution}
\begin{align*}
  \sin\frac{\alpha}{2}&=\pm\sqrt{\frac{1-\cos\alpha}{2}}=\pm\sqrt{\frac{1-\dfrac12}{2}}=\pm\frac12,\\
  \cos\frac{\alpha}{2}&=\pm\sqrt{\frac{1+\cos\alpha}{2}}=\pm\sqrt{\frac{1+\dfrac12}{2}}=\pm\frac{\sqrt{3}}{2},\\
  \tan\frac{\alpha}{2}&=\pm\frac{\dfrac12}{\dfrac{\sqrt{3}}{2}}=\pm\frac{\sqrt{3}}{3}.
\end{align*}
\end{solution}

\begin{example}
已知 $\cos\theta=-\dfrac35$,并且 $\ang{180}<\theta<\ang{270}$,求 $\tan\dfrac\theta2$。
\end{example}
\begin{solution}[解法一]
  因为 $\ang{180}<\theta<\ang{270}$,所以 $\ang{90}<\dfrac\theta2<\ang{135}$,即 $\dfrac\theta2$ 是第二象限角。

  $\therefore\quad \tan\dfrac\theta2=-\sqrt{\dfrac{1-\cos\theta}{1+\cos\theta}}=-\sqrt{\dfrac{1-\left(-\dfrac35\right)}{1+\left(-\dfrac35\right)}}=-2$。
\end{solution}

\medskip
\begin{solution}[解法二]
  $\ang{180}<\theta<\ang{270}$,即 $\theta$ 是第三象限角。
\begin{align*}
  \therefore\qquad \sin\theta&=-\sqrt{1-\cos^2\theta}=-\sqrt{1-\frac{9}{25}}=-\frac45.\\ 
  \therefore\qquad \tan\frac\theta2&=\frac{1-\cos\theta}{\sin\theta}=\frac{1-\left(-\dfrac35\right)}{-\dfrac45}=-2,\\ 
  \text{或}\qquad \tan\frac\theta2&=\frac{\sin\theta}{1+\cos\theta}=\frac{-\dfrac45}{1+\left(-\dfrac35\right)}=-2.
\end{align*}
\end{solution}

\begin{example}\label{exp:3-18}
  求证 $\dfrac{\cos^2\alpha}{\cot\dfrac\alpha2-\tan\dfrac\alpha2}=\dfrac14\sin2\alpha$。
\end{example}
\begin{solution}[证法一]
\begin{align*}
  \frac{\cos^2\alpha}{\cot\dfrac\alpha2-\tan\dfrac\alpha2}&=\frac{\cos^2\alpha}{\dfrac{1+\cos\alpha}{\sin\alpha}-\dfrac{1-\cos\alpha}{\sin\alpha}}=\frac{\cos^2\alpha\sin\alpha}{2\cos\alpha}\\
  &=\frac12\sin\alpha\cos\alpha=\frac14\sin2\alpha.
\end{align*}
\end{solution}

\begin{solution}[证法二]
\[
  \frac{\cos^2\alpha}{\cot\dfrac\alpha2-\tan\dfrac\alpha2}=\frac{\cos^2\alpha\tan\dfrac\alpha2}{1-\tan^2\dfrac\alpha2}=\dfrac12\cos^2\alpha\tan\alpha=\frac12\sin\alpha\cos\alpha=\frac14\sin2\alpha.
\]
\end{solution}

\begin{example}
用 $\tan\dfrac\alpha2$ 表示 $\sin\alpha$,$\cos\alpha$,$\tan\alpha$。
\end{example}
\begin{solution}
\begin{align*}
  \sin\alpha&=2\sin\dfrac\alpha2\cos\dfrac\alpha2=2\tan\dfrac\alpha2\cdot\cos^2\dfrac\alpha2=\frac{2\tan\dfrac\alpha2}{\sec^2\dfrac\alpha2}=\frac{2\tan\dfrac\alpha2}{1+\tan^2\dfrac\alpha2},\\
  \cos\alpha&=2\cos^2\dfrac\alpha2-1=\frac{2}{\sec^2\dfrac\alpha2}-1=\frac{2}{1+\tan^2\dfrac\alpha2}-1=\frac{1-\tan^2\dfrac\alpha2}{1+\tan^2\dfrac\alpha2},\\
  \tan\alpha&=\tan\left(2\cdot\frac{\alpha}{2}\right)=\frac{2\tan\dfrac\alpha2}{1-\tan^2\dfrac\alpha2}.
\end{align*}
\end{solution}

用 $\tan\dfrac\alpha2$ 分别表示 $\sin\alpha$,$\cos\alpha$,$\tan\alpha$ 的公式,即
\[ \sin\alpha=\frac{2\tan\dfrac\alpha2}{1+\tan^2\dfrac\alpha2},\quad \cos\alpha=\frac{1-\tan^2\dfrac\alpha2}{1+\tan^2\dfrac\alpha2},\quad \tan\alpha=\frac{2\tan\dfrac\alpha2}{1-\tan^2\dfrac\alpha2}\]
通常叫做\Concept{万能公式}。这是因为,不论 $\alpha$ 角的哪一种三角函数,都可以用这几个公式把它化为 $\tan\dfrac\alpha2$ 的有理式,这样就可以把问题转化为以 $\tan\dfrac\alpha2$ 为变量的一元有理函数,往往有助于问题的解决。

我们用万能公式来证明\cref{exp:3-18}:
\begin{align*}
  \text{右边}&=\dfrac12\cdot\frac{\tan\alpha}{1+\tan^2\alpha}=\dfrac12\tan\alpha\cdot\cos^2\alpha=\frac{\tan\dfrac\alpha2}{1-\tan^2\dfrac\alpha2}\cdot\cos^2\alpha,\\
  \text{左边}&=\cos^2\alpha\cdot\frac{\tan\dfrac\alpha2}{1-\tan^2\dfrac\alpha2}.
\end{align*}

$\therefore\quad$ 原式成立。

\begin{Practice}
  \begin{question}[itemsep=4pt]
    \item 已知 $\cos\alpha=\dfrac23$,求 $\sin\dfrac\alpha2$,$\cos\dfrac\alpha2$,$\tan\dfrac\alpha2$。
    \item 已知 $\sin\theta=-\dfrac45$,且 $\ang{270}<\theta<\ang{360}$,求 $\sin\dfrac\theta2$,$\cos\dfrac\theta2$,$\tan\dfrac\theta2$。
    \item 已知 $\cos A=\dfrac45$,且 $\dfrac32\uppi<A<2\uppi$,求 $\tan\dfrac{A}{2}$。
    \item 求证 $\tan\dfrac\uppi8=\sqrt{2}-1$。 
    \item 证明下列恒等式:
    \begin{tasks}[after-item-skip=7pt,after-skip=5pt,before-skip=5pt](2)
      \task $\sin^2\dfrac\uppi4=\dfrac{1-\cos\dfrac\alpha2}{2}$;
      \task $1+\sin\alpha=2\cos^2\left(\dfrac\uppi4-\dfrac\alpha2\right)$;
      \task $1-\sin\theta=2\cos^2\left(\dfrac\uppi4+\dfrac\theta2\right)$;
      \task $\dfrac{\cos A}{\cot\dfrac{A}{2}-\tan\dfrac{A}{2}}=\dfrac12\sin A$;
      \task $\dfrac{\sin2\alpha}{1+\cos2\alpha}\cdot\dfrac{\cos\alpha}{1+\cos\alpha}=\tan\dfrac\alpha2$;
      \task $\dfrac{4\sin\alpha(1-\tan^2\alpha)}{\sec\alpha(1+\tan^2\alpha)}=\sin4\alpha$。
    \end{tasks}
    \item 已知 $\tan\alpha=-3$,求 $2\alpha$ 的各三角函数值。
  \end{question}
\end{Practice}

\begin{Exercise}
  \begin{question}[itemsep=5pt]
    \item 已知等腰三角形一个底角的正弦等于 $\dfrac{5}{13}$,求这个三角形的顶角的正弦、余弦及正切。
    \item 已知 $\cos\phi=-\dfrac{\sqrt{3}}{3}$,并且 $\ang{180}<\phi<\ang{270}$,求 $\sin2\phi$,$\cos2\phi$,$\tan2\phi$ 的值。
    \item 证明下列恒等式:
    \begin{tasks}[after-item-skip=7pt,after-skip=5pt,before-skip=5pt](2)
      \task $\left(\sin\dfrac\alpha2+\cos\dfrac\alpha2\right)^2=1+\sin\alpha$;
      \task $\tan\theta-\cot\theta=-2\cot2\theta$;
      \task! $\tan\left(\alpha+\dfrac\uppi4\right)+\tan\left(\alpha-\dfrac\uppi4\right)=2\tan2\alpha$;
      \task $\dfrac{1+\sin2\phi}{\sin\phi+\cos\phi}=\sin\phi+\cos\phi$;
      \task $\sin\theta(1+\cos2\theta)=\sin2\theta\cos\theta$;
      \task! $2\sin\left(\dfrac\uppi4+\alpha\right)\sin\left(\dfrac\uppi4-\alpha\right)=\cos2\alpha$;
      \task $\dfrac{1+2\sin\alpha\cos\alpha}{\cos^2\alpha-\sin^2\alpha}=\dfrac{1+\tan\alpha}{1-\tan\alpha}$;
      \task $\dfrac{1+\sin2\theta-\cos2\theta}{1+\sin2\theta+\cos2\theta}=\tan\theta$。
    \end{tasks}
    \item 已知等腰三角形的顶角的余弦等于 $\dfrac{7}{25}$,求这个三角形的一个底角的正弦、余弦及正切。
    \item 已知 $\cos\phi=\dfrac13$,并且 $\ang{270}<\phi<\ang{360}$,求 $\sin\dfrac\phi2$,$\cos\dfrac\phi2$,$\tan\dfrac\phi2$ 的值。
    \item 已知 $2\alpha+\beta=\ang{90}$,且 $\alpha$ 是锐角,求证
    \[\sin\alpha=\sqrt{\frac{1-\sin\beta}{2}},\quad \cos\alpha=\sqrt{\frac{1+\sin\beta}{2}}.\]
    \item 已知圆心角的正弦等于 $\dfrac35$,求对同弧的圆周角的正弦、余弦及正切。
    \item 求证:
    \begin{tasks}[after-item-skip=7pt,after-skip=5pt,before-skip=5pt](2)
      \task $\sin\dfrac\uppi8=\dfrac12\sqrt{2-\sqrt{2}}$;
      \task $\cos\dfrac\uppi8=\dfrac12\sqrt{2+\sqrt{2}}$;
      \task $\tan\ang{67;30}=\sqrt{2}+1$。
    \end{tasks}
    \item 证明下列恒等式:
    \begin{tasks}[after-item-skip=7pt,after-skip=5pt,before-skip=5pt](2)
      \task $2\sin\theta+\sin2\theta=4\sin\theta\cos^2\dfrac\theta2$;
      \task $\dfrac{2\sin\alpha-\sin2\alpha}{2\sin\alpha+\sin2\alpha}=\tan^2\dfrac{\alpha}{2}$;
      \task $\tan\ang{15}+\cot\ang{15}=4$;
      \task $\dfrac{\csc^2\alpha-2}{\csc^2\alpha}=\cos2\alpha$;
      \task! $\sin(n\uppi+\theta)\cos(n\uppi-\theta)=\dfrac12\sin2\theta$,($n\in\mathbb{Z}$);
      \task! $\dfrac{1+\sin\phi}{\cos\phi}=\dfrac{\cos\phi}{1-\sin\phi}=\tan\left(\dfrac{\uppi}{4}+\dfrac{\phi}{2}\right)$;
      \task! $\cos\alpha(\cos\alpha-\cos\beta)+\sin\alpha(\sin\alpha-\sin\beta)=2\sin^2\dfrac{\alpha-\beta}{2}$;
      \task! $\dfrac{\cos\alpha}{\sec\dfrac\alpha2+\csc\dfrac\alpha2}=\dfrac12\sin\alpha\left(\cos\dfrac\alpha2-\sin\dfrac\alpha2\right)$;
      \task $\cos^4\theta=\dfrac14+\dfrac12\cos2\theta+\dfrac14\cos^22\theta$;
      \task $\sin^4x=\dfrac38-\dfrac12\cos2x+\dfrac18\cos4x$。
    \end{tasks}
    \item 解答:
    \begin{enumerate}[itemindent=2.4em,itemsep=5pt]
      \item 已知 $\tan\alpha=2$,求 $\sin2\alpha$,$\cos2\alpha$,$\tan2\alpha$ 的值;
      \item 已知 $\tan\theta=\dfrac{b}{a}$,求证 $a\cos2\theta+b\sin2\theta=a$;
      \item 已知 $\tan\dfrac\alpha2=\dfrac{m}{n}$,求 $m\cos\alpha+n\sin\alpha$ 的值。
    \end{enumerate}
    \item 设 $\sin\alpha$ 与 $\sin\dfrac\alpha2$ 的比为 $8:5$,求 $\cos\alpha$,$\cos\dfrac\alpha4$ 的值。
    \item 在一块半圆形的铁板中截出一块面积最大的矩形,应该怎样截取?求出这个矩形的面积。
    \item 已知 $\tan^2\theta=2\tan^2\phi+1$,求证 $\cos2\theta+\sin^2\phi=0$。
    \item 已知 $x+y=3-\cos4\theta$,$x-y=4\sin2\theta$,求证 $x^{\frac12}+y^{\frac12}=2$。
  \end{question}
\end{Exercise}

\subsection{三角函数的积化和差与和差化积}
在计算或化简的过程中,有时需要把三角函数的积的形式与和差的形式进行互化。下面我们就来研究这种互化。
\subsubsection{三角函数的积化和差}
将\cref{eq:Sa+b} 加上\cref{eq:Sa-b},得
\begin{gather} 
  \sin(\alpha+\beta)+\sin(\alpha-\beta)=2\sin\alpha\cos\beta,\notag\\ 
  \label{eq:SaCb}\therefore\quad \sin\alpha\cos\beta=\frac12[\sin(\alpha+\beta)+\sin(\alpha-\beta)]
\end{gather}

将\cref{eq:Sa+b} 减去\cref{eq:Sa-b},得
\begin{gather} 
  \sin(\alpha+\beta)-\sin(\alpha-\beta)=2\cos\alpha\sin\beta,\notag\\ 
  \label{eq:CaSb}\therefore\quad \cos\alpha\sin\beta=\frac12[\sin(\alpha+\beta)-\sin(\alpha-\beta)]
\end{gather}

将式~$(C_{\alpha+\beta})$ 加上\cref{eq:Ca-b},得
\begin{gather} 
  \cos(\alpha+\beta)+\cos(\alpha-\beta)=2\cos\alpha\cos\beta,\notag\\ 
  \label{eq:CaCb}\therefore\quad \cos\alpha\cos\beta=\frac12[\cos(\alpha+\beta)+\cos(\alpha-\beta)]
\end{gather}

将\cref{eq:Ca+b} 减去\cref{eq:Ca-b},得
\begin{gather} 
  \cos(\alpha+\beta)-\cos(\alpha-\beta)=-2\sin\alpha\sin\beta,\notag\\ 
  \label{eq:SaSb}\therefore\quad \sin\alpha\sin\beta=-\frac12[\cos(\alpha+\beta)-\cos(\alpha-\beta)]
\end{gather}

\cref{eq:SaCb,eq:CaSb,eq:CaCb,eq:SaSb} 这四个公式叫做\Concept{积化和差公式}。

\begin{example}
  不查表,求 $\sin\dfrac{5\uppi}{12}\cdot\cos\dfrac{\uppi}{12}$ 的值。
\end{example}
\begin{solution}[解法一]
  \begin{align*}
    \sin\frac{5\uppi}{12}\cdot\cos\frac{\uppi}{12} &=\frac12\left[\sin\left(\dfrac{5\uppi}{12}+\dfrac{\uppi}{12}\right)+\sin\left(\dfrac{5\uppi}{12}-\dfrac{\uppi}{12}\right)\right]\\
    &=\frac12\left(\sin\dfrac\uppi2+\sin\dfrac\uppi3\right)=\frac12+\frac{\sqrt{3}}{4}.
  \end{align*}
\end{solution}

\begin{solution}[解法二]
  \begin{align*}
    \sin\frac{5\uppi}{12}\cdot\cos\frac{\uppi}{12} &=\cos\dfrac{\uppi}{12}+\cos\dfrac{\uppi}{12}=\cos^2\dfrac{\uppi}{12}\\
    &=\dfrac{1+\cos\dfrac\uppi6}{2}=\dfrac{1+\dfrac{\sqrt{3}}{2}}{2}=\frac12+\frac{\sqrt{3}}{4}.
  \end{align*}
\end{solution}

\begin{example}
把下列各式化为和差的形式,然后查表求值:
\begin{tasks}[before-skip=5pt](2)
  \task $2\cos\ang{31}\sin\ang{14}$;
  \task $\cos\dfrac{2\uppi}{15}\cos\dfrac\uppi5$。
\end{tasks}
\end{example}
\begin{solution}
\begin{enumerate}
  \item 
  \begin{align*}
    2\cos\ang{31}\sin\ang{14}&=\sin(\ang{31}+\ang{14})-\sin(\ang{31}-\ang{14})\\
    &=\frac{\sqrt{2}}{2}-\sin\ang{17}=0.7071-0.2924=0.4147;
  \end{align*}
  \item 
  \begin{align*}
    \cos\frac{2\uppi}{15}\cos\frac\uppi5&=\dfrac12\left[\cos\left(\dfrac{2\uppi}{15}+\dfrac{\uppi}{5}\right)+\cos\left(\dfrac{2\uppi}{15}-\dfrac{\uppi}{5}\right)\right]=\frac12\left(\cos\dfrac\uppi3+\cos\dfrac{\uppi}{15}\right)\\
    &=\frac12\left(\frac12+\cos\ang{12}\right)=0.25+0.4891=0.7391.
  \end{align*}
\end{enumerate}
\end{solution}

\begin{example}
  求证:$\sin\ang{15}\cdot\sin\ang{30}\cdot\sin\ang{75}=\dfrac18$。
\end{example}
\begin{solution}[证法一]
\begin{align*}
\sin\ang{15}\cdot\sin\ang{30}\cdot\sin\ang{75}&=\frac12\sin\ang{15}\sin\ang{75}=-\frac14[\cos(\ang{15}+\ang{75})-\cos(\ang{15}-\ang{75})]\\
&=-\dfrac14(\cos\ang{90}-\cos\ang{60})=-\frac14\cdot\left(-\frac12\right)=\frac18.
\end{align*}
\end{solution}

\begin{solution}[证法二]
\begin{align*}
\sin\ang{15}\cdot\sin\ang{30}\cdot\sin\ang{75}&=\frac12\sin\ang{15}\sin\ang{75}=\frac12\sin\ang{15}\cos\ang{15}\\
&=\frac14\sin\ang{30}=\frac18.
\end{align*}
\end{solution}

\begin{example}
求 $\cos\ang{10}\cdot\cos\ang{30}\cdot\cos\ang{50}\cdot\cos\ang{70}$ 的值。
\end{example}
\begin{solution}
\begin{align*}
\text{原式}&=\cos\ang{10}\cdot\frac{\sqrt{3}}{2}\cdot\frac12[\cos(\ang{50}+\ang{70})+\cos(\ang{50}-\ang{70})]\\ 
&=\frac{\sqrt{3}}{4}\cos\ang{10}\left(-\dfrac12+\cos\ang{20}\right)\\ 
&=-\frac{\sqrt{3}}{8}\cos\ang{10}+\frac{\sqrt{3}}{4}\cos\ang{10}\cos\ang{20}\\ 
&=-\frac{\sqrt{3}}{8}\cos\ang{10}+\frac{\sqrt{3}}{8}(\cos\ang{30}+\cos\ang{10})\\ 
&=-\frac{\sqrt{3}}{8}\cos\ang{10}+\frac{\sqrt{3}}{8}\cdot\frac{\sqrt{3}}{2}+\frac{\sqrt{3}}{8}\cos\ang{10}\\ 
&=\frac{3}{16}.
\end{align*}
\end{solution}

\begin{example}
  求证 $\sin3\alpha\sin^3\alpha+\cos3\alpha\cos^3\alpha=\cos^32\alpha$。
\end{example}
\begin{proof}
\begin{align*}
\text{左边}&=\sin^2\alpha(\sin3\alpha\sin\alpha)+\cos^2\alpha(\cos3\alpha\cos\alpha)\\
&=\frac12[\sin^2\alpha(\cos2\alpha-\cos4\alpha)+\cos^2\alpha(\cos4\alpha+\cos2\alpha)]\\
&=\frac12[\cos2\alpha(\sin^2\alpha+\cos^2\alpha)+\cos4\alpha(\cos^2\alpha-\sin^2\alpha)]\\
&=\frac12(\cos2\alpha+\cos4\alpha\cos2\alpha)\\ 
&=\frac12\cos2\alpha(1+\cos4\alpha)\\
&=\frac12\cos2\alpha\cdot 2\cos^22\alpha=\cos^32\alpha=\text{右边}.
\end{align*}

$\therefore\quad$ 原式成立。 
\end{proof}

\begin{example}
已知 $\triangle ABC$ 中,$\sin B\sin C=\cos^2\dfrac{A}{2}$,求证这个三角形是等腰三角形。
\end{example}
\begin{proof}
由 $\sin B\sin C=\cos^2\dfrac{A}{2}$,知
\[ \sin B\sin C=\frac{1+\cos A}{2},\]
又由 $A+B+C=\ang{180}$,知
\begin{gather*} 
  \cos A=-\cos(B+C),\\
  \therefore \quad -\frac12[\cos(B+C)-\cos(B-C)]=\frac12[1-\cos(B+C)].
\end{gather*}
化简,得
\begin{gather*}
  \cos(B-C)=1.\\ 
  \because\qquad -\ang{180}<B-C<\ang{180},\\ 
  \therefore\qquad B-C=0.
\end{gather*}

由此,得 $B=C$,即 $\triangle ABC$ 是等腰三角形。
\end{proof}

\begin{Practice}
  \begin{question}
    \item 把下列各式化为和差的形式,然后查表求值:
    \begin{tasks}(2)
      \task $2\sin\ang{70}\cos\ang{20}$;
      \task $\cos\ang{80}\sin\ang{20}$;
      \task $\cos\ang{68}\sin\ang{52}$;
      \task $\sin\ang{121}\sin\ang{59}$。
    \end{tasks}
    \item 不查表,求下列各式的值:
    \begin{tasks}[after-skip=7pt,after-skip=5pt](2)
      \task $\sin\ang{105}\cos\ang{75}$;
      \task $2\cos\ang{37.5}\sin\ang{22.5}$;
      \task $2\cos\dfrac{9\uppi}{13}\cos\dfrac{\uppi}{13}+\cos\dfrac{5\uppi}{13}+\cos\dfrac{3\uppi}{13}$。
    \end{tasks}
    \item 证明下列各恒等式:
    \begin{tasks}[after-item-skip=7pt,before-skip=5pt,after-skip=5pt]
      \task $2\sin\left(\dfrac\uppi4+\alpha\right)\cdot\sin\left(\dfrac\uppi4-\alpha\right)=\cos2\alpha$;
      \task $2\sin(\ang{60}+\alpha)\cdot\cos(\ang{60}-\alpha)=\dfrac{\sqrt{3}}{2}+\sin2\alpha$;
      \task $\sin\ang{20}\cos\ang{70}+\sin\ang{10}\sin\ang{50}=\dfrac14$;
      \task $\cos2\alpha\cos\alpha-\sin5\alpha\sin2\alpha=\cos4\alpha\cos3\alpha$;
      \task $\cos4x\cdot\cos2x-\cos^23x=-\sin^2x$;
      \task $\tan\left(x+\dfrac\uppi4\right)+\tan\left(x-\dfrac\uppi4\right)=2\tan2x$。
    \end{tasks}
  \end{question}
\end{Practice}

\subsubsection{三角函数的和差化积}
在积化和差得公式中,如果令 $\alpha+\beta=\theta$,$\alpha-\beta=\varphi$,则
\[\alpha=\frac{\theta+\varphi}{2},\quad \beta=\frac{\theta-\varphi}{2}.\]

把 $\alpha$,$\beta$ 的值代入积化和差的公式~\eqref{eq:SaCb}~中,就有
\begin{multline*}
  \sin\dfrac{\theta+\varphi}{2}\cdot\cos\dfrac{\theta-\varphi}{2}=\frac12\left[\sin\left(\dfrac{\theta+\varphi}{2}+\dfrac{\theta-\varphi}{2}\right)\right.\\ {}+\left.\sin\left(\dfrac{\theta+\varphi}{2}-\dfrac{\theta-\varphi}{2}\right)\right]=\dfrac12(\sin\theta+\sin\varphi).
\end{multline*}
\[\therefore\quad \sin\theta+\sin\varphi=2\sin\frac{\theta+\varphi}{2}\cos\frac{\theta-\varphi}{2}.\]
同样可得,
\begin{align*}
  \sin\theta-\sin\varphi&=2\cos\frac{\theta+\varphi}{2}\cdot\sin\frac{\theta-\varphi}{2},\\
  \cos\theta+\cos\varphi&=2\cos\frac{\theta+\varphi}{2}\cdot\cos\frac{\theta-\varphi}{2},\\
  \cos\theta-\cos\varphi&=-2\sin\frac{\theta+\varphi}{2}\cdot\sin\frac{\theta-\varphi}{2}.
\end{align*}
这四个公式叫做\Concept{和差化积公式}。

\begin{example}
把下列各式化为积的形式:
\begin{tasks}(2)
  \task $\sin\ang{104}+\sin\ang{16}$;
  \task $\cos\left(\alpha+\dfrac\uppi4\right)+\cos\left(\alpha-\dfrac\uppi4\right)$。
\end{tasks}
\end{example}
\begin{solution}
\begin{enumerate}
  \item 
  \begin{align*}
    \sin\ang{104}+\sin\ang{16}&=2\sin\frac{\ang{104}+\ang{16}}{2}\cos\frac{\ang{104}-\ang{16}}{2}\\ 
    &=2\sin\ang{60}\cos\ang{44}=\sqrt{3}\cos\ang{44};
  \end{align*}
  \item 
  \begin{align*}
    &\cos\left(\alpha+\dfrac\uppi4\right)+\cos\left(\alpha-\dfrac\uppi4\right)\\ 
    ={}&2\cos\dfrac{\left(\alpha+\dfrac\uppi4\right)+\left(\alpha-\dfrac\uppi4\right)}{2}\cdot\cos\dfrac{\left(\alpha+\dfrac\uppi4\right)-\left(\alpha-\dfrac\uppi4\right)}{2}\\
    ={}&2\cos\alpha\cos\dfrac\uppi4=\sqrt{2}\cos\alpha.
  \end{align*}
\end{enumerate}
\end{solution}

\begin{example}
求 $\sin\ang{75}-\sin\ang{15}$ 的值。
\end{example}
\begin{solution}
  $\sin\ang{75}-\sin\ang{15}=2\cos\dfrac{\ang{75}+\ang{15}}{2}\sin\dfrac{\ang{75}-\ang{15}}{2}=2\cos\ang{45}\sin\ang{30}=\dfrac{\sqrt{2}}{2}$。
\end{solution}

\begin{example}
把下列各式化为积的形式: 
\begin{tasks}(2)
  \task $\cos x-\dfrac{\sqrt{3}}{2}$;
  \task $\sin x+\cos x$。
\end{tasks}
\end{example}
\begin{solution}
\begin{enumerate}
  \item 
  \begin{align*}
    \cos x-\frac{\sqrt{3}}{2}&=\cos x-\cos\frac\uppi6=-2\sin\frac{x+\dfrac\uppi6}{2}\sin\frac{x-\dfrac\uppi6}{2}\\
    &=-2\sin\left(\dfrac{x}{2}+\dfrac{\uppi}{12}\right)\sin\left(\dfrac{x}{2}-\dfrac{\uppi}{12}\right);
  \end{align*}
  \item 
  \begin{align*}
    \sin x+\cos x&=\sin x+\sin(\ang{90}-x)\\
    &=2\sin\ang{45}\cos(x-\ang{45})\\
    &=\sqrt{2}\cos(x-\ang{45}),\\
    \text{或}\qquad \sin x+\cos x&=\cos(\ang{90}-x)+\cos x\\
    &=2\cos\ang{45}\cos(\ang{45}-x)\\
    &=\sqrt{2}\cos(\ang{45}-x).
  \end{align*}

  因为 $\sqrt{2}\cos(\ang{45}-x)=\sqrt{2}\cos[-(x-\ang{45})]=\sqrt{2}\cos(x-\ang{45})$。所以两种解法的结果实际上是一样的。
\end{enumerate}
\end{solution}

\begin{example}
求 $\sin^2\ang{10}+\cos^2\ang{40}+\sin\ang{10}\cos\ang{40}$ 的值。
\end{example}
\begin{solution}
\begin{align*}
   &\sin^2\ang{10}+\cos^2\ang{40}+\sin\ang{10}\cos\ang{40}\\
  ={}&\frac{1-\cos\ang{20}}{2}+\frac{1+\cos\ang{80}}{2}+\frac12(\sin\ang{50}-\sin\ang{30})\\
  ={}&1+\frac12(\cos\ang{80}-\cos\ang{20})+\frac12\left(\sin\ang{50}-\frac12\right)\\ 
  ={}&1-\frac12\sin\ang{50}+\frac12\sin\ang{50}-\frac14\\
  ={}&\frac34.
\end{align*}
\end{solution}

\begin{example}
  在 $\triangle ABC$ 中,求证
\[ \sin A+\sin B+\sin C=4\cos\frac{A}{2}\cos\frac{B}{2}\cos\frac{C}{2}.\]
\end{example}
\begin{proof}
由 $A+B+C=\ang{180}$,得 
\[ C=\ang{180}-(A+B),\quad \frac{C}{2}=\ang{90}-\frac{A+B}{2}.\]
\begin{align*}
  \therefore\quad \sin A+\sin B+\sin C&=2\sin\dfrac{A+B}{2}\cos\dfrac{A-B}{2}+\sin(A+B)\\ 
  &=2\sin\frac{A+B}{2}\cos\frac{A-B}{2}+2\sin\frac{A+B}{2}\cos\frac{A+B}{2}\\
  &=2\sin\frac{A+B}{2}\left(\cos\frac{A-B}{2}+\cos\frac{A+B}{2}\right)\\ 
  &=2\sin\left(\ang{90}-\frac{C}{2}\right)\cdot\cos\frac{A}{2}\cos\left(-\frac{B}{2}\right)\\ 
  &=4\cos\frac{A}{2}\cos\frac{B}{2}\cos\frac{C}{2}.
\end{align*}
\end{proof}

\begin{example}
  把 $1+\sin\theta+\cos\theta$ 化成积的形式。
\end{example}
\begin{solution}
\begin{align*}
  1+\sin\theta+\cos\theta &= (1+\cos\theta)+\sin\theta=(1+\cos\theta)+\sin\theta \\ 
  &= 2\cos^2\frac{\theta}{2}+2\sin\frac{\theta}{2}\cos\frac{\theta}{2}=2\cos\frac{\theta}{2}\left(\cos\frac{\theta}{2}+\sin\frac{\theta}{2}\right)\\
  &=2\cos\frac{\theta}{2}\left[\sin\left(\ang{90}-\frac{\theta}{2}\right)+\sin\frac{\theta}{2}\right]\\
  &=2\cos\frac{\theta}{2}\cdot 2\sin\ang{45}\cos\left(\ang{45}-\frac{\theta}{2}\right)\\
  &=2\sqrt{2}\cos\frac{\theta}{2}\cos\left(\ang{45}-\frac{\theta}{2}\right).
\end{align*}
\end{solution}


\begin{example}
  化下列各式为一个角的一个三角函数的形式:
  \begin{tasks}[before-skip=5pt](3)
    \task $\dfrac{\sqrt{2}}{2}\sin\alpha+\dfrac{\sqrt{2}}{2}\cos\alpha$;
    \task $\sin\alpha-\sqrt{3}\cos\alpha$;
    \task $a\sin\alpha+b\cos\alpha$。
  \end{tasks}
\end{example}
\begin{solution}
\begin{enumerate}
  \item $\because\quad \cos\ang{45}=\frac{\sqrt{2}}{2},\quad\sin\ang{45}=\frac{\sqrt{2}}{2}$,
  \begin{align*}
    \therefore\qquad \text{原式}&=\sin\alpha\cos\ang{45}+\cos\alpha\sin\ang{45}\\ 
    &=\sin(\alpha+\ang{45});
  \end{align*}
  \item $\text{原式}=2\left(\dfrac12\sin\alpha-\dfrac{\sqrt{3}}{2}\cos\alpha\right)$,而
  \begin{gather*} 
    \cos\frac\uppi3,\quad \sin\frac\uppi3=\frac{\sqrt{3}}{2},\\
    \therefore\qquad \text{原式}=2\left(\sin\alpha\cos\frac\uppi3-\cos\alpha\sin\frac\uppi3\right)=2\sin\left(\alpha-\dfrac\uppi3\right);
  \end{gather*}
  \item 分析:如果 $a=x\cos\varphi,b=x\sin\varphi$,$\text{原式}=x(\sin\alpha\cos\varphi+\cos\alpha\sin\varphi)$,这样就可以把原式化为 $x\sin(\alpha+\varphi)$ 了。现在问题转变为 $x$ 与 $\varphi$ 应当怎样来确定。
  
  由 $\cos^2\varphi+\sin^2\varphi=1$,可得 $\left(\dfrac{a}{x}\right)^2+\left(\dfrac{b}{x}\right)^2=1$,
  \[\therefore\quad x^2=a^2+b^2.\]
  这样就得到 $x=\pm\sqrt{a^2+b^2}$,不妨取 $x=\sqrt{a^2+b^2}$,于是就得到 $\cos\varphi=\dfrac{a}{\sqrt{a^2+b^2}}$,$\sin\varphi=\dfrac{b}{\sqrt{a^2+b^2}}$,从而得 $\tan\varphi=\dfrac{b}{a}$。因为 $a,b$ 是已知的,所以 $\varphi$ 可以确定。

  \[ a\sin\alpha+b\cos\alpha=\sqrt{a^2+b^2}\left(\frac{a}{\sqrt{a^2+b^2}}\sin\alpha+\frac{b}{\sqrt{a^2+b^2}}\cos\alpha\right).\]

  令 $\cos\varphi=\dfrac{a}{\sqrt{a^2+b^2}}$,$\sin\varphi=\dfrac{b}{\sqrt{a^2+b^2}}$,则
  \[\text{原式}=\sqrt{a^2+b^2}(\sin\alpha\cos\varphi+\cos\alpha\sin\varphi)=\sqrt{a^2+b^2}\sin(\alpha+\varphi).\]

(其中 $\varphi$ 角所在象限由 $a,b$ 的符号确定,$\varphi$ 角的值由 $\tan\varphi=\dfrac{b}{a}$ 确定)。
\end{enumerate}
\end{solution}

\begin{Practice}
  \begin{question}
    \item 把下列各式化为积的形式:
    \begin{tasks}[after-item-skip=7pt,after-skip=5pt](2)
      \task $\sin\ang{24}+\sin\ang{21}$;
      \task $\sin(\ang{15}+\alpha)-\sin(\ang{15}-\alpha)$;
      \task $\cos3x+\cos2x$;
      \task $\cos\dfrac{\alpha+\beta}{2}\cos\dfrac{\alpha-\beta}{2}$。
    \end{tasks}
    \item 求下列各式的值:
    \begin{tasks}[after-skip=5pt,before-skip=5pt](2)
      \task $\dfrac{\sin\ang{20}-\cos\ang{50}}{\cos\ang{20}-\cos\ang{40}}$;
      \task $\sin\ang{20}+\sin\ang{40}-\sin\ang{80}$。
    \end{tasks}
    \item 求证:
    \begin{tasks}[after-item-skip=7pt,after-skip=5pt,before-skip=5pt]
      \task $\dfrac{\sin\alpha+\sin\beta}{\sin\alpha-\sin\beta}=\tan\dfrac{\alpha+\beta}{2}\cdot\cot\dfrac{\alpha-\beta}{2}$;
      \task $\dfrac{\sin x+\sin y}{\cos x-\cos y}=\cot\dfrac{y-x}{2}$。
    \end{tasks}
    \item 将下列各式化为一个角的一个三角函数的形式:
    \begin{tasks}[after-item-skip=7pt,after-skip=5pt,before-skip=5pt](2)
      \task $\dfrac{\sqrt{3}}{2}\sin x+\dfrac12\cos x$;
      \task $\dfrac{\sqrt{2}}{2}\sin\varphi-\dfrac{\sqrt{2}}{2}\cos\varphi$;
      \task $\sqrt{2}\cos\alpha+\sqrt{2}\sin\alpha$;
      \task $\cos\theta-\sqrt{3}\sin\theta$。
    \end{tasks}
  \end{question}
\end{Practice}

\begin{Exercise}
  \begin{question}
    \item 求下列各式的值:
    \begin{tasks}(2)
      \task $\sin\ang{45}\cos\ang{15}$;
      \task $\cos\ang{75}\cos\ang{15}$;
      \task $\cos\ang{157;30}\sin\ang{22;30}$;
      \task $\cos\ang{20}\cos\ang{40}\cos\ang{80}$;
      \task $\sin\ang{20}\sin\ang{40}\sin\ang{80}$;
      \task $\tan\ang{10}\tan\ang{50}\tan\ang{70}$。
    \end{tasks}
    \item 把下列各式化为和或差的形式:
    \begin{tasks}[before-skip=5pt,after-skip=5pt,after-item-skip=7pt](2)
      \task $2\sin\dfrac{3\alpha}{2}\cos\dfrac\alpha2$;
      \task $\sin ax\sin bx$;
      \task $\cos(\alpha+\beta)\cos(\alpha-\beta)$;
      \task $\cos\left(\dfrac{3\uppi}{4}+2\alpha\right)\sin\left(\dfrac\uppi4-\alpha\right)$;
      \task $\sin x\cos 3x$。
    \end{tasks}
    \item 把下列各式化为积的形式:
    \begin{tasks}[before-skip=5pt,after-skip=5pt,after-item-skip=7pt](2)
      \task $\sin\ang{28}+\cos\ang{17}$;
      \task $\cos\ang{54}-\sin\ang{54}$;
      \task $\cos\left(x-\dfrac\uppi4\right)-\cos\left(x+\dfrac\uppi4\right)$;
      \task $\cos\left(\dfrac\uppi6+\alpha\right)+\cos\left(\dfrac\uppi6-\alpha\right)$;
      \task $\dfrac{\sqrt{2}}{2}+\cos\alpha$;
      \task $1+\sin2A$;
      \task $1+\sqrt{3}\tan\alpha$;
      \task $3-4\sin^2\alpha$;
      \task $\sin^2\theta-\sin^2\varphi$;
      \task $\cos^2x-\cos^2y$。
    \end{tasks}
    \item 证明下列各恒等式:
    \begin{tasks}[before-skip=5pt,after-skip=5pt,after-item-skip=7pt](2)
      \task! $\dfrac{\sin A+\sin3A+\sin5A}{\sin3A+\sin5A+\sin7A}=\dfrac{\sin3A}{\sin5A}$;
      \task! $\dfrac{\sin \alpha+\sin3\alpha+\sin5\alpha}{\cos \alpha+\cos3\alpha+\cos5\alpha}=\tan3\alpha$;
      \task $\dfrac{\sin x-\sin y}{\sin(x+y)}=\dfrac{\sin\dfrac12(x-y)}{\sin\dfrac12(x+y)}$;
      \task $\dfrac{\cos2\alpha+\cos2\beta}{1+\cos2(\alpha+\beta)}=\dfrac{\cos(\alpha-\beta)}{\cos(\alpha+\beta)}$;
      \task! $\sec\left(\dfrac\uppi4+\alpha\right)\sec\left(\dfrac\uppi4-\alpha\right)=2\sec2\alpha$;
      \task! $\cos^2A+\cos^2(\ang{60}-A)+\cos^2(\ang{60}+A)=\dfrac32$;
      \task! $\sin\dfrac{\alpha}{2}\sin\dfrac{7\alpha}{2}\sin\dfrac{3\alpha}{2}\sin\dfrac{11\alpha}{2}=\sin2\alpha\sin5\alpha$;
      \task! $(\cos x+\sin x)(\cos2x+\sin2x)=\cos x+\sin3x$。
    \end{tasks}
    \item 求下列各式的值:
    \begin{tasks}[before-skip=5pt,after-skip=5pt,after-item-skip=7pt](2)
      \task $\cos\ang{20}+\cos\ang{100}+\cos\ang{140}$;
      \task $\dfrac{\cos\ang{80}-\cos\ang{20}}{\sin\ang{80}+\sin\ang{20}}$;
      \task! $\cos\ang{40}+\cos\ang{60}+\cos\ang{80}+\cos\ang{160}$;
      \task! $\cos\ang{40}\cos\ang{80}+\cos\ang{80}\cos\ang{160}+\cos\ang{160}\cos\ang{40}$;
      \task! $\sin\ang{20}\cdot\sin\ang{40}\cdot\sin\ang{60}\cdot\sin\ang{80}$。
    \end{tasks}
    \item 在 $\triangle ABC$ 中,求证:
    \begin{tasks}[before-skip=5pt,after-skip=5pt,after-item-skip=7pt]
      \task $\sin A+\sin B-\sin C=4\sin\dfrac{A}{2}\sin\dfrac{B}{2}\cos\dfrac{C}{2}$;
      \task $\cos A+\cos B+\cos C=1+4\sin\dfrac{A}{2}\sin\dfrac{B}{2}\sin\dfrac{C}{2}$;
      \task $\sin2A+\sin2B+\sin2C=4\sin A\sin B\sin C$;
      \task $\cos2A+\cos2B+\cos2C=-1-4\cos A\cos B\cos C$。
    \end{tasks}
    \item 将下列各式化为一个角的一个三角函数的形式:
    \begin{tasks}(2)
      \task $\cos\alpha-4\sin\alpha$;
      \task $5\sin\varphi+12\cos\varphi$;
      \task $4\sin t+3\cos t$;
      \task $7\sin2t-6\cos2t$。
    \end{tasks}
    \item 求下列各式的最大值和最小值:
    \begin{tasks}[before-skip=5pt,after-skip=5pt,after-item-skip=7pt](2)
      \task $\sin x\cos x$;
      \task $\sin\left(\alpha+\dfrac\uppi4\right)+\sin\left(\alpha-\dfrac\uppi4\right)$;
      \task $\cos\left(\dfrac\uppi3+2\theta\right)\sin\left(\dfrac\uppi3-2\theta\right)$;
      \task $6\cos\theta+8\sin\theta$。
    \end{tasks}
  \end{question}
\end{Exercise}

\section*{小结}
\begin{enumerate}[C、,itemindent=4.5em]
  \item 本章内容包括两角和与差的三角函数的公式,倍角、半角的三角函数公式,以及三角函数的积化和差与和差化积公式。这些公式主要用于三角函数式的计算和推导。它们在高等数学、电工学、力学、机械设计与制造等方面都有广泛的用用,要熟练地掌握,主要公式如下。
  \par\medskip
  \begin{tabular}{ll}
    两角和与差公式:& $\sin(\alpha\pm\beta)=\sin\alpha\cos\beta\pm\cos\alpha\sin\beta$;\\[5pt]
    & $\cos(\alpha\pm\beta)=\cos\alpha\cos\beta\mp\sin\alpha\sin\beta$;\\[10pt]
    & $\tan(\alpha\pm\beta)=\dfrac{\tan\alpha\pm\tan\beta}{1\mp\tan\alpha\tan\beta}$。\\[12pt]
    倍角公式:& $\sin2\alpha=2\sin\alpha\cos\alpha$;\\[5pt]
              & $\cos2\alpha=\cos^2\alpha-\sin^2\alpha=2\cos^2\alpha-1=1-2\sin^2\alpha$;\\[10pt]
              & $\tan2\alpha=\dfrac{2\tan\alpha}{1-\tan^2\alpha}$。\\[20pt]
    半角公式:& $\sin\dfrac\alpha2=\pm\sqrt{\dfrac{1-\cos\alpha}{2}}$;\\[12pt]
              & $\cos\dfrac\alpha2=\pm\sqrt{\dfrac{1+\cos\alpha}{2}}$;\\[12pt]
              & $\tan\dfrac\alpha2=\pm\sqrt{\dfrac{1-\cos\alpha}{1+\cos\alpha}}=\dfrac{1-\cos\alpha}{\sin\alpha}=\dfrac{\sin\alpha}{1+\cos\alpha}$。\\
  \end{tabular}

  \begin{tabular}{ll}
    积化和差公式:& $\sin\alpha\cos\beta=\dfrac12[\sin(\alpha+\beta)+\sin(\alpha-\beta)]$;\\[10pt]
    & $\cos\alpha\sin\beta=\dfrac12[\sin(\alpha+\beta)-\sin(\alpha-\beta)]$;\\[10pt]
    & $\cos\alpha\cos\beta=\dfrac12[\cos(\alpha+\beta)+\cos(\alpha-\beta)]$;\\[10pt]
    & $\sin\alpha\sin\beta=-\dfrac12[\cos(\alpha+\beta)-\cos(\alpha-\beta)]$。\\[20pt]
    和差化积公式:& $\sin\alpha+\sin\beta=2\sin\dfrac{\alpha+\beta}{2}\cos\dfrac{\alpha-\beta}{2}$;\\[10pt]
    & $\sin\alpha-\sin\beta=2\cos\dfrac{\alpha+\beta}{2}\sin\dfrac{\alpha-\beta}{2}$;\\[10pt]
    & $\cos\alpha+\cos\beta=2\cos\dfrac{\alpha+\beta}{2}\cos\dfrac{\alpha-\beta}{2}$;\\[10pt]
    & $\cos\alpha-\cos\beta=-2\sin\dfrac{\alpha+\beta}{2}\sin\dfrac{\alpha-\beta}{2}$。\\
  \end{tabular}

  此外,还有万能公式:
  \[
  \sin\alpha=\frac{2\tan\dfrac\alpha2}{1+\tan^2\dfrac{\alpha}{2}};\quad 
  \cos\alpha=\frac{1-\tan^2\dfrac\alpha2}{1+\tan^2\dfrac{\alpha}{2}};\quad 
  \tan\alpha=\frac{2\tan\dfrac\alpha2}{1-\tan^2\dfrac{\alpha}{2}}.
  \]
  \item 上述公式是以两角和的余弦公式为基础推导得出,这些公式的内在联系和推导的线索如\cpageref{fig:3-3}\cref{fig:3-3} 所示。
  
  \begin{figure}[p]
    \includegraphics{3-3.pdf}
    \caption{}\label{fig:3-3}
  \end{figure}
  
  掌握表中公式的内在联系及其推导的线索,能够帮助我们理解和记忆这些公式,这是学好本章内容的关键。
  \item 应注意的几个问题:
  \begin{enumerate}
    \item 凡使公式中某个式子没有意义的角,都不适合公式;
    \item 在半角公式(\cref{eq:Sa/2,eq:Ca/2,eq:Ta/2})中,根号前的符号由半角所在的象限来决定;
    \item 把 $a\sin\alpha+b\cos\alpha$ 化成 $\sqrt{a^2+b^2}\sin(\alpha+\varphi)$ 时,其中辅助角 $\varphi$ 在哪个象限,由 $a,b$ 的符号确定,$\varphi$ 的值由 $\tan\varphi=\dfrac{b}{a}$ 确定。
  \end{enumerate}
\end{enumerate}

\chapter*{复习参考题\chinese{chapter}}
\section*{A 组}
\begin{question}[itemsep=4pt]
  \item 以 $\alpha$ 角的顶点 $O$ 作原点,始边作 $x$ 轴的正半轴,建立直角坐标系。在终边上截取 $OP=1$,写出点 $P$ 的坐标。
  \item 在直角坐标系中,两点 $A,B$ 的坐标分别为 $A\,(x_1,y_1)$,$B\,(x_2,y_2)$,写出 $A,B$ 间的距离公式。
  \item 写出同角三角函数的基本关系式——倒数关系、商数关系、平方关系。
  \item 写出 $2k\uppi+\alpha$($k\in\mathbb{Z}$),$-\alpha$,$\uppi-\alpha$,$\uppi+\alpha$,$2\uppi-\alpha$ 的诱导公式,并用一句话加以概括。
  \item 写出函数 $y=\sin\alpha$ 和 $y=\cos\alpha$ 的定义域和值域,说出它们在什么时候取得最大值和最小值,并研究它们的单调性、奇偶性和周期性。
  \item 已知 $\cos\alpha=-\dfrac{9}{41}$,$\alpha\in\left(\uppi,\dfrac{3\uppi}{2}\right)$,求 $\tan\left(\dfrac\uppi4-\alpha\right)$。
  \item 如果 $\alpha$,$\beta$ 都是锐角,且 $\sin\alpha=\dfrac{\sqrt{5}}{5}$,$\sin\beta=\dfrac{\sqrt{10}}{10}$,求证 $\alpha+\beta=\dfrac\uppi4$。
  \item 解答:
  \begin{enumerate}[itemindent=2.4em]
    \item 已知 $A+B=\dfrac\uppi4$,求证 $(1+\tan A)(1+\tan B)=2$;
    \item 如果 $A,B$ 都是锐角,且 $(1+\tan A)(1+\tan B)=2$,求证  $A+B=\dfrac\uppi4$。
  \end{enumerate}
  \item\label{exec:3t-9}如图,三个相同的正方形相接,求证 $\alpha+\beta=\ang{45}$。
  \begin{figurehere}
    \begin{minipage}{\linewidth}\centering
      \includegraphics{ex3t-9.pdf}
      \caption*{(第~\ref{exec:3t-9}~题图)}
    \end{minipage}
  \end{figurehere}
  \item 如果 $\alpha$,$\beta$,$\gamma$ 都是锐角,并且它们的正切依次为 $\dfrac12$、$\dfrac15$、$\dfrac18$,求证 $\alpha+\beta+\gamma=\ang{45}$。
  \item 写出 $\dfrac\uppi2-\alpha$,$\dfrac\uppi2+\alpha$,$\dfrac{3\uppi}{2}-\alpha$,$\dfrac{3\uppi}{2}+\alpha$ 的诱导公式,并用一句话加以概括。
  \item 已知 $\cos2\alpha=\dfrac35$,求 $\sin^4\alpha-\cos^4\alpha$ 的值。
  \item 已知 $\tan x=\dfrac{7}{24}$,求 $\cos2x$,$\cot\left(2x-\dfrac\uppi4\right)$ 的值。
  \item 已知 $\sin\theta+\cos\theta =\dfrac23$,求 $\sin2\theta$ 的值。
  \item 已知 $\sin\varphi\cos\varphi=\dfrac{60}{169}$,且 $\dfrac\uppi4<\varphi<\dfrac\uppi2$,求 $\sin\varphi$,$\cos\varphi$ 的值。
  \item 在等腰三角形 $ABC$ 中,腰为底的 2 倍,求顶角 $A$ 的三角函数的值。
  \item 化下列各式为和差的形式:
  \begin{tasks}[after-item-skip=7pt,before-skip=5pt,](2)
    \task $2\sin\left(\dfrac\uppi4-x\right)\sin\left(\dfrac\uppi4+x\right)$;
    \task $\sin(n-1)x\cos(n+1)x$;
    \task $\cos(m-1)x\cos(m-3)x$;
    \task $\dfrac{2\sin(\ang{30}+\alpha)}{\cos\alpha}$。
  \end{tasks}
  \item 化下列各式为积的形式:
  \begin{tasks}[after-item-skip=7pt](2)
    \task $1+\sin2x-\cos2x$;
    \task $1+\cos\theta+\cos\dfrac{\theta}{2}$;
    \task $\sin\alpha+\sin2\alpha+\sin3\alpha$;
    \task $1-\dfrac14\sin^22\alpha-\sin^2\beta-\cos^4\alpha$。
  \end{tasks}
  \item 化简:
  \begin{tasks}(2)
    \task $\cos\ang{52;30}\cos\ang{7;30}$;
    \task $\dfrac{\sin2\alpha}{1+\cos2\alpha}\cdot\dfrac{\cos\alpha}{1+\cos\alpha}$;
    \task $\tan\ang{67;30}-\tan\ang{22;30}$;
    \task $\cos\ang{20}-\sin\ang{10}-\sin\ang{50}$;
    \task! $\sin(x+\ang{60})+2\sin(x-\ang{60})-\sqrt{3}\cos(\ang{120}-x)$;
    \task! $\cos\alpha\cdot\csc\alpha\cdot\sqrt{\sec^2\alpha-1}\ \left(\dfrac{3\uppi}{2}<\alpha<2\uppi\right)$。
  \end{tasks}
  \item 证明下列各式:
  \begin{tasks}[after-item-skip=7pt](2)
    \task! $\tan\ang{20}+\tan\ang{40}+\sqrt{3}\tan\ang{20}\tan\ang{40}=\sqrt{3}$;
    \task! $\sin x\left(1+\tan x\tan\dfrac{x}{2}\right)=\tan x$;
    \task! $\dfrac{\sin(2\alpha+\beta)}{\sin\alpha}-2\cos(\alpha+\beta)=\dfrac{\sin\beta}{\sin\alpha}$;
    \task $2\sin\alpha+\sin2\alpha=\dfrac{2\sin^3\alpha}{1-\cos\alpha}$;
    \task! $\sec\theta=\sqrt{\dfrac{\sec^4\theta-\tan^4\theta}{2\sin^2\theta+\cos^2\theta}}\ \left(0<\theta<\dfrac\uppi2\right)$;
    \task $\dfrac{\cot^2\alpha+1}{\cot^2\alpha-1}=\sec2\alpha$;
    \task $\sec\alpha-\tan\alpha=\tan\left(\dfrac\uppi4-\dfrac\alpha2\right)$;
    \task $\dfrac{3-4\cos2A+\cos4A}{3+4\cos2A+\cos4A}=\tan^4A$;
    \task! $\dfrac{1+\cos A+\cos2A+\cos3A}{2\cos^2A+\cos A-1}=2\cos A$;
    \task! $\tan3\theta-\tan2\theta-\tan\theta=\tan3\theta\tan2\theta\tan\theta$。
  \end{tasks}
  \item 求下列函数的最大值与最小值:
  \begin{tasks}(2)
    \task $y=\sin3x\cos3x$;
    \task $y=\sin(x-\ang{30})\cos x$;
    \task $y=\sin x-\sqrt{3}\cos x$;
    \task $y=\sin x+\cos x$;
    \task $y=a\sin x +b$。
  \end{tasks}
  \item 在 $\triangle ABC$ 中,求证:
  \begin{tasks}[after-skip=7pt]
    \task $\tan A+\tan B+\tan C=\tan A\cdot\tan B\cdot\tan C$;
    \task $\dfrac{\cos A}{\sin B\sin C}+\dfrac{\cos B}{\sin C\sin A}+\dfrac{\cos C}{\sin A\sin B}=2$。
  \end{tasks}
  \item 在 $\triangle ABC$ 中,求证:
  \begin{tasks}[before-skip=5pt,after-item-skip=7pt]
    \task $\dfrac{\cos2A}{a^2}-\dfrac{\cos2B}{b^2}=\dfrac{1}{a^2}-\dfrac{1}{b^2}$;
    \task $(a^2-b^2-c^2)\tan A+(a^2-b^2+c^2)\tan B=0$。
  \end{tasks}
  \item 在 $\triangle ABC$ 中,如果 $2\cos B\cdot\cos C=\sin A$,那么 $\triangle AC$ 为等腰三角形。
  \item $\triangle ABC$ 的三个内角 $A,B,C$ 的对边分别是 $a,b,c$,如果 $a^2=b(b+c)$,求证 $A=2B$。
  \item 发电厂发出的电是三相交流电,它的三根导线上的电流强度分别是时间 $t$ 的函数:
  \item 已知电流 $i=I_m\sin\omega t$,电压 $v=V_m\sin\left(\omega t+\dfrac\uppi2\right)$,求证电功率 $p=iv=\dfrac12V_mI_m\sin2\omega t$。
  \item\label{exec:3t-28}如图,在直角三角形中,$c$ 是斜边,$r$ 是内切圆半径,求证:
  \par\medskip\noindent
  \begin{minipage}{0.5\linewidth}
    \begin{enumerate}[itemsep=7pt]
      \item $r=\dfrac{c}{\cot\dfrac{A}{2}+\cot\left(\ang{45}-\dfrac{A}{2}\right)}$;
      \item $r\leqslant\dfrac{c}{2}(\sqrt{2}-1)$。
    \end{enumerate}
  \end{minipage}\hfill
  \begin{minipage}{0.45\linewidth}
    \begin{figurehere}
      \includegraphics{ex3t-28.pdf}
      \caption*{(第~\ref{exec:3t-28}~题图)}
    \end{figurehere}
  \end{minipage}
\end{question}
\section*{B 组}
\begin{question}[resume,itemsep=4pt]
  \item 证明下列各恒等式:
  \begin{enumerate}[itemindent=2.4em,topsep=5pt]
    \item $\dfrac{\cos\theta-\sin\theta}{\cos\theta+\sin\theta}=\sec2\theta-\tan2\theta$;
    \item $(\cos\alpha+\cos\beta)^2+(\sin\alpha+\sin\beta)^2=4\cos^2\left(\dfrac{\alpha-\beta}{2}\right)$;
    \item $\sin\alpha+\sin\beta+\sin\gamma-\sin(\alpha+
    \beta+\gamma)=4\sin\dfrac{\alpha+\beta}{2}\sin\dfrac{\beta+\gamma}{2}\sin\dfrac{\gamma+\alpha}{2}$;
    \item $\cos\alpha+\cos\beta+\cos\gamma+\cos(\alpha+
    \beta+\gamma)=4\cos\dfrac{\alpha+\beta}{2}\cos\dfrac{\beta+\gamma}{2}\cos\dfrac{\gamma+\alpha}{2}$。
  \end{enumerate}
  \item 把下列各式化成积的形式:
  \begin{tasks}[after-item-skip=5pt]
    \task $\sqrt{1-\cos\alpha}+\sqrt{1+\cos\alpha}$($\alpha$ 在第四象限);
    \task $\sqrt{\tan x+\sin x}+\sqrt{\tan x-\sin x}\ \left(\uppi<x<\dfrac{3\uppi}{2}\right)$。
  \end{tasks}
  \item 设 $\tan\dfrac{\alpha}{2}=t$,用含有 $t$ 的有理式表示下列各函数:
  \begin{tasks}(2)
    \task $\dfrac{1+\sin\alpha}{\sin\alpha(1+\cos\alpha)}$;
    \task $\dfrac{\sin\alpha}{\sin\alpha+\cos\alpha}$。
  \end{tasks}
  \item 设 $\sin\alpha$,$\sin\beta$ 是方程
  \[x^2-(\sqrt{2}\cos\ang{20})x+\left(\cos^2\ang{20}-\frac12\right)=0\]
  的两根,求 $\alpha$,$\beta$($\ang{0}<\alpha<\ang{90}$,$\ang{0}<\beta<\ang{90}$)。
  \item 如果方程 $x^2+px+q=0$ 的两个根分别是 $\tan\varphi$ 与 $\tan\left(\dfrac\uppi4-\varphi\right)$,而且两根的比是 $\dfrac32$,求 $p,q$ 的值。
  \item 已知三角形的最小内角为 \ang{30},它的对边的长为 \qty{2}{cm},另外两个内角的差为 \ang{60},求最大边的长(要准确值)。
  \item 设 $A,B,C$ 是一三角形的三个内角,且
  \[\lg\sin A-\lg\cos B-\lg\sin C=\lg 2,\]
  求证这个三角形是等腰三角形。
  \item 在 $\triangle ABC$ 中,
  \[\sin C=\frac{\sin A+\sin B}{\cos A+\cos B},\]
  求证这个三角形是直角三角形。
  \item 设 $R$ 是 $\triangle ABC$ 的外接圆半径,求证:
  \[a+b+c=8R\cos\frac{A}{2}\cos\frac{B}{2}\cos\frac{C}{2}\]
  \item 已知正 $n$ 变形的边长为 $a$,内切圆半径为 $r$,外接圆半径为 $R$,求证:
  \[R+r=\frac12a\cdot\cot\frac{\uppi}{2n}\]
  \item 半径分别为 $R,r$($R>r$)的两圆相外切,它们的两条外公切线的夹角为 $\theta$,求证:
  \[\sin\theta=\frac{4(R-r)\sqrt{Rr}}{(R+r)^2}\]
  \item 求证:
  \begin{tasks}[after-item-skip=7pt]
    \task $\sum\limits_{k=1}^{n}\sin kx=\dfrac{\sin\dfrac{n+1}{2}x\sin\dfrac{n}{2}x}{\sin\dfrac{x}{2}}$;
    \task $\sum\limits_{k=1}^{n}\cos kx=\dfrac{\cos\dfrac{n+1}{2}x\sin\dfrac{n}{2}x}{\sin\dfrac{x}{2}}$。
  \end{tasks}
\end{question}