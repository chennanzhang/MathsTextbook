\chapter{近似计算的法则}
在度量的时候,一般只能得到一个近似数。例如,用皮尺量教室一边的长,量得 \qty{6.8}{m},实际上这边的长可能比 \qty{6.8}{m} 略长一些或者略短一些。假定教室一边的长是 \qty{6.82}{m},\qty{6.82}{m} 就是教室一边长的准确数。这样,由皮尺量得教室一边的长(近似数),就比教室一边实际长(准确数)少 \qty{0.02}{m}。

一个近似数和它的准确数的差,叫做这个近似数的\Concept{误差}。

在计算的时候,我们已经知道可以把一个数进行四舍五入得到一个近似数。由四舍五入得到的一个近似数,它的误差的绝对值不超过这个近似数最末一位的单位的一半。

如果一个近似数的误差的绝对值不超过某一位的单位的一般,从左边第一个不是零的数字起,到这位数字止,所有的数字都叫做这个近似数的\Concept{有效数字}。如近似数 56.08 有四个有效数字 5、6、0、8,近似数 \num{0.0085} 有两个有效数字 8、5,近似数 \num{0.0390} 有三个有效数字 3、9、0。

在实际应用中,实数的运算往往取其近似数来进行。近似计算一般采用下面的法则。
\begin{Theorem}{法则 1}
  近似数相加减,所得结果的位数,通常只保留到各个已知数都有的最后一位为止。已知数中过多的位数,可以先四舍五入到这一位的下一位,再进行计算。
\end{Theorem}

\begin{example}
  \begin{enumerate}
    \item 作近似数的加减计算:$7.35-2.478-0.03419+18.6$
    \item 计算:$32+14\dfrac27+\sqrt{1880}$(精确到十分位)。
  \end{enumerate}
\end{example}
\begin{solution}
  \begin{enumerate}
    \item 因为 18.6 只精确到十分位,所以结果只要保留到十分位。可以先把位数过多的各数分别四舍五入到百分位后计算,得出的中间结果也都保留到百分位。因此,
    \begin{align*}
      &7.35-2.478-0.03419+18.6\\
      \approx{}&7.35-2.48-0.03+18.6\\ 
      ={}&23.44\approx 23.4.
    \end{align*}
    \item 题中各数都是准确数,它们可以精确到任意数位,因为结果只要求精确到十分位,所以在计算时,$14\dfrac27$ 和 $\sqrt{1880}$ 只要分别取精确到百分位的近似值 $14.29$ 和 $43.36$ 就可以了。因此,
    \begin{align*}
      32+14\dfrac27+\sqrt{1880}&\approx7.35-2.48-0.03+18.6\\ 
      &=89.65\approx 89.7.
    \end{align*}
  \end{enumerate}
\end{solution}

\begin{Theorem}{法则 2}
  近似数相乘除,所得结果的有效数字的个数,通常只保留到与已知数中有效数字个数最少的一个相同。已知数中过多的有效数字,可以先四舍五入到比结果应保留的有效数字的个数多一个,再进行计算。
\end{Theorem}

\begin{example}\mbox{}\par
  \begin{enumerate}
    \item 求近似数 24.78 与 0.32 的积;
    \item 求近似数 7.9 除以近似数 24.78 的商;
    \item 作近似数的乘除计算:$\dfrac{80.43\times1.05}{24\times7.146}$。
  \end{enumerate}
\end{example}
\begin{solution}
  \begin{enumerate*}
    \item 
    \begin{minipage}[t]{0.43\linewidth}\centering
      \renewcommand\arraystretch{0.9}
    \begin{tabular}[t]{cc@{}c@{}c@{}c@{}}
      &&2&4.&8\\
      $\times$&&0.&3&2\\
      \hline
      &&4&9&6\\
      &7&4&4&\\
      \hline
      &7.&9&3&6\\
    \end{tabular}
    \[\therefore 24.78\times 0.32\approx 7.9\]
    \end{minipage}
    \item 
    \begin{minipage}[t]{0.43\linewidth}\centering
    \renewcommand\arraystretch{0.9}
    \begin{tabular}[t]{r@{}r@{}l*3{@{}c}}
      &&0.&3&1&8\\
      \cline{2-6}
      248 )&7&9& & & \\
      &7&4&4& & \\
      \cline{2-6}
      &&4&6&0& \\
      &&2&4&8& \\
      \cline{2-6}
      &&2&1&2&0 \\
    \end{tabular}
    \[\therefore 7.9\div 24.78\approx 0.32\]
    \vskip10pt 
    \end{minipage}
    \item 因为 24 只有两个有效数字,所以结果只要保留两个有效数字。可以先把有效数字过多的各数分别四舍五入到有三个有效数字后计算,得出的中间结果也都保留三个有效数字。因此,
  \end{enumerate*}
  \[\frac{80.43\times1.05}{24\times7.146}\approx\frac{80.4\times1.05}{24\times7.15}\approx\frac{84.4}{172}\approx0.49\]
\end{solution}

\begin{Theorem}{法则 3}
  近似数平方或开平方,所得结果的有效数字的个数,通常只保留到与底数或被开方数的有效数字的个数相同。
\end{Theorem}

\begin{example}
  作近似数计算:
  \begin{tasks}(2)
    \task $12.8^2$;
    \task $\sqrt{0.049}$。
  \end{tasks}
\end{example}
\begin{solution}
  \begin{enumerate}
    \item $12.8^2=12.8\times 12.8\approx 164$。
    \item $\sqrt{0.049}\approx 0.22$。
  \end{enumerate}
\end{solution}

\begin{Theorem}{法则 4}
  近似数的混合计算,仍按照运算顺序进行计算,计算过程中得出的中间结果,一般要比按照法则 1、2、3 进行近似计算应保留的数字多一位。
\end{Theorem}

\begin{example}
  作近似数的计算:$3.28\times 2.15+4.8409\times 2.7$
\end{example}
\begin{solution}
  \begin{align*}
    &3.28\times 2.15+4.8409\times 2.7\\
    \approx{}&3.28\times 2.15+4.84\times 2.7\\
    \approx{}&7.052+13.1\approx 7.1+13.1\approx 20.
  \end{align*}
\end{solution}

注意:
\begin{enumerate}[1.]
  \item 在进行计算时,首先要看体重所给的数是近似数还是准确数。近似数用近似计算法则进行计算,准确数则用一般方法进行计算。
  \item 上述法则中所说的近似计算保留数位的方法,只是在一般情况下通常采用的方法,在实际问题中,也可以根据具体情况,比上述法则所说的多保留或少保留一位数字。
  
  我们知道,在近似计算的计算过程中,由于保留数位的不同或者计算次序的不同,虽然计算都正确,得数也可能稍有不同,但都应看做是正确的。
  \item 有些习题,如果近似计算所涉及的近似数的数字都是很简单的,或者两数相除能除尽的,开方能开尽的,并且所得结果的位数和用近似计算所得的位数相差不大时,可以不必用近似计算方法。此外,有些问题的答案,如果用分数或根式来表示比较方便时,也可用分数或根式来表示,不必采用近似计算方法。
\end{enumerate}

\begin{Practice}
\begin{question}
  \item 计算下列近似数的加、减法:
  \begin{tasks}[after-skip=10pt]
    \task $28.5+2.974+0.06429+5.73$;
    \task $140.0-8.3025$;
    \task $235.0-14.012-86.1254+43.007$;
    \task $2+\sqrt{2}+3\dfrac16+\sqrt{5}$(精确到百分位)。
  \end{tasks}
  \item 计算下列近似数的乘、除法:
  \begin{tasks}[before-skip=5pt,after-item-skip=7pt,after-skip=10pt](2)
    \task $12.7\times 56.9$;
    \task $0.078\times 3.14159265$;
    \task $7.84\div 2.46705$;
    \task $\dfrac{1.85\times 64.72\times 4.0}{17.9\times 284.3}$。
  \end{tasks}
  \item 计算下列近似数的平方和平方根:
  \begin{tasks}(2)
    \task $4.87^2$;
    \task $\sqrt{0.00565}$。
  \end{tasks}
  \item 计算下列近似数的混和运算:
  \begin{tasks}[before-skip=5pt,after-skip=5pt](2)
    \task $3.5\times 51.2+8.25\times 12.7$;
    \task $8.64\div 0.98 -33.2\times 0.57$;
    \task $4.58^2-\sqrt{165}+6.72$;
    \task $(16.7+32-18.64+5.976)\div 0.36$。
  \end{tasks}
\end{question}
\end{Practice}

\chapter{换底公式}
利用常用对数表,可以求得任意一个正数的以 10 为底的对数。
现在我们来说明以其他正数 $a$($a\neq 1$)为底的对数的求法,例如,求 $\log_35$。

设 $\log_35=x$,写成指数式,得
\[3^x=5.\]

两边取常用对数,得
\begin{gather*}
  x\lg3=\lg5,\\ 
  \therefore\quad x=\frac{\lg5}{\lg3}=\frac{0.6990}{0.4771}=1.465,
\end{gather*}
就是
\[ \log_35=1.465.\]

一般地,我们有下面的换底公式:
\[ \log_bN=\frac{\log_aN}{\log_ab}.\]
\begin{proof}
  设 $\log_bN=x$,写成指数式,得
  \[b^x=N.\]
  两边取以 $a$ 为底的对数,得
  \begin{gather*} 
    x\log_ab=\log_aN.\\
    \therefore\quad x=\frac{\log_aN}{\log_ab}.
  \end{gather*}
  所以
  \[ \log_bN=\frac{\log_aN}{\log_ab}.\]
\end{proof}

在科学技术中常常使用以无理数 $\upe=\num{2.71828}\cdots$ 为底的对数,以 $\upe$ 为底的对数叫做\Concept{自然对数},$\log_\upe N$ 通常记作 $\ln N$。根据对数换底公式,可以得到自然对数与常用对数之间的关系:
\[ \ln N=\frac{\lg N}{\lg\upe}=\frac{\lg N}{0.4343},\]
就是
\[ \ln N=2.303 \lg N.\]

\begin{example}
  求 $\log_89\cdot\log_{27}32$ 的值。
\end{example}
\begin{solution}
\end{solution}

\begin{example}
  求证 $\log_xy\cdot\log_yz=\log_xz$。
\end{example}
\begin{proof}
  把 $\log_yz$ 化成以 $x$ 为底的对数,则
  \[ \log_xy\cdot \log_yz=\log_xy\cdot\frac{\log_xz}{\log_xy}=\log_xz.\]
\end{proof}

\begin{Practice}
\begin{question}
  \item 利用常用对数表,求下列各对数的值:
  \begin{tasks}[before-skip=5pt,after-skip=7pt](2)
    \task $\log_2 1000$;
    \task $\log_5 0.5$;
    \task $\log_3 10$;
    \task $\log_{\frac12}\dfrac13$。
  \end{tasks}
  \item 利用常用对数表计算:
  \begin{tasks}[before-skip=7pt,after-skip=5pt](2)
    \task $\ln \uppi$;
    \task $\ln\dfrac{\sqrt{2}}{2}$;
    \task 已知 $\ln x=2.174$,求 $x$;
    \task 已知 $\ln x=-0.7103$,求 $x$。
  \end{tasks}
  \item 求下列各式的值:
  \begin{tasks}(2)
    \task $\ln\upe^2$;
    \task $\upe^{\ln x}$。
  \end{tasks}
  \item 不查表计算下列各题:
  \begin{tasks}[before-skip=5pt,after-skip=10pt,after-item-skip=7pt]
    \task $(\lg5)^2+\lg2\cdot\lg50$;
    \task 已知 $lg2=0.3010$,$\lg7=0.8451$,求 $\lg35$;
    \task $\log_2\dfrac{1}{25}\cdot\log_3\dfrac18\cdot\log_5\dfrac19$。
  \end{tasks}
  \item 利用换底公式证明:
  \begin{tasks}[before-skip=7pt,after-skip=5pt](2)
    \task $\log_ab=\dfrac{1}{\log_ba}$;
    \task $(\log_ab)(\log_bc)(\log_ca)=1$。
  \end{tasks}
\end{question}
\end{Practice}