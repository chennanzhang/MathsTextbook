\chapter{导数的应用}
\section{一阶导数的应用}
我们在初中研究了二次函数的极值问题。在高中学过可以根据函数的图象,说出函数 $f(x)$ 的单调区间。
现在我们学习了导数,就可以利用导数来直接判断一般函数的单调性和极值。为此,先讲预备知识。
\subsection{预备知识}
\begin{Theorem}[罗尔\footnotemark[1]中值定理]{定理 1}
  如果函数 $f(x)$ 在闭区间 $[a,b]$ 上连续,在开区间 $(a,b)$ 内可导,且在两端点的函数值相等,即 $f(a)=f(b)$,那么至少存在一点 $\xi \in (a,b)$,使 $f'(\xi)= 0$。
\end{Theorem}
\footnotetext[1]{罗尔(M.Rolle,1652--1719 年),法国数学家。}

罗尔定理的几何意义是明显的。若函数 $f(x)$ 满足罗尔定理的条件,这就告诉我们,在闭区间 $[a,b]$ 上有一条连续曲线 $y=f(x)$,且过曲线上每一点(端点除外)都可以作一条切线,当曲线两端点的纵坐标相等时,那么在曲线上至少能找到一点 $(\xi,f(\xi)),\ \xi \in (a,b)$,使曲线在该点的切线平行于 $x$ 轴(\cref{fig:3-1a})。

如果函数 $y=f(x)$ 在区间 $[a,b]$ 上罗尔定理的条件不全满足,那么在区间 $(a,b)$ 内就可能任何点的切线都不与 $x$ 轴平行(\cref{fig:3-1b})。
\begin{figure}
  \begin{minipage}{0.48\linewidth}\centering
    \includegraphics{3-1a.pdf}
    \subcaption{}\label{fig:3-1a}
  \end{minipage}
  \begin{minipage}{0.48\linewidth}\centering
    \includegraphics{3-1b.pdf}
    \subcaption{}\label{fig:3-1b}
  \end{minipage}
  \caption{}\label{fig:3-1}
\end{figure}

\begin{proof}
  令 $f(a)=f(b)=k$,分如下三种情况证明:
  \begin{enumerate}[1.]
    \item 在闭区间 $[a,b]$ 上,恒有 $f(x)=k$ 的情况
    
    这时,$f(x)$ 是 $[a,b]$ 上的常数函数,于是 $f'(x)=0$,因此罗尔定理对开区间 $(a,b)$ 内任何点皆成立。
    \item 在 $[a,b]$ 上有点 $x$,使得 $f(x)>k$ 的情况
    
    因为 $f(x)$ 为 $[a,b]$ 上的连续函数,根据连续函数性质 1,得知在 $[a,b]$ 上存在点 $(\xi_1,f(\xi_1))$ 为 $f(x)$ 在 $[a,b]$ 上的最大值,即 当 $a \leqslant x \leqslant b$ 时,
    \begin{equation}
      \label{eq:fx_vs_fxi}
      f(x)\leqslant f(\xi_1),
    \end{equation}
    又因在 $[a,b]$ 上有点 $x$,使得
    \begin{equation}
      \label{eq:fx_vs_k}
      f(x)>k,
    \end{equation}
    由\cref{eq:fx_vs_fxi,eq:fx_vs_k}得 $f(\xi_1)>k$,这说明点 $\xi _1$ 不可能是 $[a,b]$ 的端点,从而 $a<\xi_1<b$。现证
    \[f'(\xi_1)=0.\]

    当 $a\leqslant\xi_1+\Delta x\leqslant b$ 时,对于 $\Delta x> 0$(或 $\Delta x<0$)由\cref{eq:fx_vs_fxi} 总有
    \[\Delta y=f(\xi_1+\Delta x)-f(\xi_1)\leqslant 0,\]
    
    设 $\Delta x>0$,则 $\dfrac{\Delta y}{\Delta x}\leqslant 0$,于是 $\lim\limits_{\Delta x\to 0+}\dfrac{\Delta y}{\Delta x}\leqslant 0$。

    若 $\Delta x<0$,则 $\dfrac{\Delta y}{\Delta x}\leqslant 0$,于是 $\lim\limits_{\Delta x\to 0-}\dfrac{\Delta y}{\Delta x}\leqslant 0$。

    又因 $a<\xi_1<b$,根据定理的条件可知 $f'(\xi_1)$ 存在,所以 $f'(\xi_1)=0$。取 $\xi=\xi_1$,定理得证。
    \item 在 $[a,b]$ 上有点 $x$,使得 $f(x)<k$ 的情况
    
    由连续函数性质 1,得知在 $[a,b]$ 上存在点 $\xi_2$,$f(\xi _2)$ 为 $f(x)$ 在 $[a,b]$ 上的最小值。与情况 2 类似,容易证明
    \[ f'(\xi_2)=0,\]
    再取 $\xi=\xi_2$,定理证完。
  \end{enumerate}
\end{proof}

这个定理告诉我们,如果定理所要求的条件满足,那么方程
\[ f'(x)=0,\]
在 $(a,b)$ 内至少有一个实数根。
我们把方程 $f'(x)=0$ 的根称为函数 $f(x)$ 的\Concept{驻点}(\cref{fig:3-1a})。

\begin{example}
验证函数 $f(x)=x^2-4x$ 在区间 $[1,3]$ 上是否满足罗尔定理的条件,如果满足,求区间 $(1,3)$ 内满足罗尔定理的 $\xi$ 值。
\end{example}
\begin{solution}
  函数 $f(x)=x^2-4x$ 在闭区间 $[1,3]$ 上连续,在 $(1,3)$ 内可导,故满足罗尔定理的所有条件,且 $f'(x)=2x-4$,所以,根据定理至少存在一点 $\xi$,$\xi\in(1,3)$,为方程 $f'(x)=2x-4=0$ 的根。因此 $x=2$ 就是所求的 $\xi$。
\end{solution}

\begin{Theorem}[拉格朗日\footnotemark[1]中值定理]{定理 2}
  如果函数 $f(x)$ 在闭区间 $[a,b]$ 上连续,在开区间 $(a,b)$ 内可导,那么在 $(a,b)$ 内至少有一点 $\xi$,使得
  \[ \tcbhighmath{f'(\xi)=\frac{f(b)-f(a)}{b-a}.}\]
\end{Theorem}
\footnotetext[1]{拉格朗日(J.L.Lagrange,1736--1813 年),法国数学家。}

这个定理的几何意义也是明显的。
因为已知导数 $f'(\xi)$ 的几何意义是曲线 $y=f(x)$ 在点 $(\xi,f(\xi))$ 处切线的斜率,而 $\dfrac{f(b)-f(a)}{b-a}$ 表示 $AB$ 弦的斜率。
这样如果函数 $f(x)$ 满足拉格朗日定理的条件,这就告诉我们,在闭区间 $[a,b]$ 上有一连续曲线,且过曲线 $y=f(x)$ 上每一点(端点除外)都可以作一条切线,那么在曲线上至少有一点 $M\,(\xi,f(\xi))$,$\xi\in(a,b)$,使得过点 $M$ 的切线与弦 $AB$ 平行(\cref{fig:3-2})。

\par\bigskip\noindent
\begin{minipage}{0.55\linewidth}
\begin{analyze}
从罗尔定理和拉格朗日定理的条件与几何解释可以看出,罗尔定理是拉格朗日定理的特殊情形。
因为如果函数 $f(x)$ 在区间 $[a,b]$ 上满足拉格朗日定理条件,且 $f( x)$ 在两端点的函数值相等,即 $f(a)=f(b)$ 时,函数 $f(x)$ 在区间 $[a,b]$ 上也满足罗尔定理条件,所以这种情形的拉格朗日定理就是罗尔定理。
下边就利用罗尔定理来证明拉格朗日定理。
为此,我们作个辅助函数:
\end{analyze}
\end{minipage}\hfill
\begin{minipage}{0.4\linewidth}
  \begin{figurehere}
    \includegraphics{3-2.pdf}
    \caption{}\label{fig:3-2}
  \end{figurehere}
\end{minipage}
\[ \varphi(x)=f(x)-kx,\]
并适当选择待定系数 $k$,使函数 $\varphi(x)$ 满足罗尔定理的所有条件。

因 $f(x)$、$kx$ 在 $[a,b]$ 上连续,在 $(a,b)$ 内可导,所以函数 $\varphi(x)=f(x)-kx$ 在 $[a,b]$ 上连续,在 $(a,b)$ 内可导,为了使 $\varphi(a)=\varphi(b)$ 也成立,则必须且只须使
\[ f(a)-ka=f(b)-kb,\]
即
\[ k=\frac{f(b)-f(a)}{b-a}.\]
% \end{analyze}

\begin{proof}
  作辅助函数:
  \[ \varphi(x)=f(x)-\frac{f(b)-f(a)}{b-a}x,\]
  由前面分析可知,$\varphi(x)$ 在 $[a,b]$ 上连续,在 $(a,b)$ 内可导,且 $\varphi(a)=\varphi(b)$,即罗尔定理的条件皆满足,根据罗尔定理可知,至少存在一点 $\xi\in(a,b)$,使 $\varphi'(\xi)=0$,但
  \[ \varphi'(x)=f'(x)-\frac{f(b)-f(a)}{b-a},\]
  于是
  \[ \varphi'(\xi)=f'(\xi)-\frac{f(b)-f(a)}{b-a}=0,\]
  所以证得
  \[ f'(\xi)=\frac{f(b)-f(a)}{b-a}.\]
\end{proof}

这个定理告诉我们,如果定理所要求的条件已满足,那么,在 $(a,b)$ 内至少存在一个点 $\xi$,当 $x=\xi$ 时,等式
\[f'(\xi)=\frac{f(b)-f(a)}{b-a}\]
成立,也就是说方程
\[ f'(x)=\frac{f(b)-f(a)}{b-a}\]
在 $(a,b)$ 内至少有一个实根。

为了便于应用,拉格朗日定理的结论,通常写成如下形式:
\[f(b)-f(a)=f'(\xi)(b-a).\quad a<\xi<b.\]

\begin{example}
求函数 $f(x)=x^3$ 在 $(-1,2)$ 内,满足中值定理 $f(b)-f(a)=f'(\xi) (b-a)$ 的点 $\xi$ 的值。
\end{example}
\noindent
\begin{minipage}{0.5\linewidth}
\begin{solution}
  因为 $f'(x)=3x^2$,$f(2)=8$,$f(-1)=-1$,满足中值定理的 $\xi$ 应有,
  \[f(2)-f(-1)=3\xi^2[2-(-1)],\]
  即
  \[9=9\xi^2,\]
  得
  \[\xi=\pm1.\]
  \[
    \because 1\in(-1,2),\quad -1\notin (-1,2)
  \]
  所以,$\xi=1$。
\end{solution}
\end{minipage}\hfill
\begin{minipage}{0.45\linewidth}
  \begin{figurehere}
    \includegraphics{3-3.pdf}
    \caption{}\label{fig:3-3}
  \end{figurehere}
\end{minipage}

\begin{example}
  当 $x>1$ 时,证明不等式 $\upe^x>\upe x$ 成立。
\end{example}
\begin{solution}
  令 $f(x)=\upe^x$,则 $f'(x)=\upe^x$,由于函数 $f'(x)=\upe^x$ 在区间 $[1,x]$ 上满足拉格朗日定理条件,所以在 $(1,x)$ 内至少存在一点 $\xi$,使
  \begin{equation}
    \label{eq:ex-et}
    \upe^x-\upe^1=\upe^\xi(x-1)
  \end{equation}
  成立,又 $\upe^x$ 为增函数,且 $1<\xi<x$,于是
  \begin{equation}
    \label{eq:e1_vs_exi}
    \upe^1<\upe^\xi.
  \end{equation}
  将\cref{eq:e1_vs_exi} 代入\cref{eq:ex-et} 得 $\upe^x-e>e(x-1)$。
  即
  \[ \upe^x>\upe x.\]
\end{solution}

\begin{Practice}
  \begin{question}
    \item 说明下列函数在给定区间上,罗尔定理是否成立。如果成立,求 $\xi$ 值;如果不成立,说明理由:
    \begin{tasks}[before-skip=7pt,after-skip=7pt,after-item-skip=7pt](2)
      \task $f(x)=\sin x, x \in [0,2\uppi]$;
      \task $f(x)=|x|, x \in [-1,1]$;
      \task $f(x)=x^3, x \in [0,1]$;
      \task $f(x)=\begin{cases}x^2,x \in(0,1], \\ 1, x=0.\end{cases}$
    \end{tasks}
    \item 如果函数 $f(x)$ 满足下列条件之一,讨论拉格朗日中值定理是否成立:
    \begin{tasks}(2)
      \task 函数 $f(x)$ 在 $(a,b)$ 内可导;
      \task 函数 $f(x)$ 在 $[a,b]$ 上可导;
      \task 函数 $f(x)$ 在 $(a,b]$ 内可导;
      \task 函数 $f(x)$ 在 $[a,b)$ 内可导。
    \end{tasks}
    \item 求函数 $f(x)=x^3+2x$ 在下列区间内,满足拉格朗日中值定理的 $\xi$ 值。
    \begin{tasks}(3)
      \task $[0,1]$;
      \task $[1,2]$;
      \task $[-1,2]$。
    \end{tasks}
  \end{question}
\end{Practice}

\begin{Exercise}
  \begin{question}
    \item 对下列函数检验罗尔定理是否成立:
    \begin{tasks}[before-skip=7pt,after-skip=7pt,after-item-skip=5pt](2)
      \task $f(x)=(x-1)(x-2), x\in[1,2]$;
      \task $f(x)=|x|-1, x\in[-1,1]$;
      \task $f(x)=\sqrt[3]{x^2}, x\in[-1,1]$;
      \task $f(x)=x-[x], x\in[0,1]$。
    \end{tasks}
    \item 求出函数 $f(x)=x^3$ 在区间 $[a,b]$ 内,满足拉格朗日中值定理的 $\xi$ 值。
    \item 求下列函数满足 $f(b)-f(a)=f'(\xi)(b-a), a<\xi<b$ 的 $\xi$ 值:
    \begin{tasks}[before-skip=7pt,after-skip=7pt,after-item-skip=5pt](2)
      \task! $f(x)=x^2-2x+2,\quad a=-1,b=1$;
      \task $f(x)=x^3-x^2,\quad a=0,b=1$;
      \task $f(x)=\dfrac{6}{x},\quad a=2,b=3$。
    \end{tasks}
    \item 利用拉格朗日中值定理,证明下列不等式:
    \begin{tasks}[before-skip=7pt,after-skip=7pt,after-item-skip=5pt]
      \task $|\sin x_1-\sin x_2|\leqslant|x_1-x_2|$,
      \task $|\arctan x_1-\arctan x_2|\leqslant|x_1-x_2|$,
      \task 当 $0\leqslant a<b$ 时,$\upe^b-\upe^a>b-a.$
    \end{tasks}
    \item 证明:若函数 $f(x)$ 在 $[a,b]$ 上有二阶导数,且 $f(a)=f(b)=0$,$f(c)=0$($a<c<b$),则在 $(a,b)$ 内至少存在一点 $\xi$,使得 $f''(\xi)=0$。
  \end{question}
\end{Exercise}

\subsection{函数的单调性}\label{subsec:monotonicity}
我们已经学过增函数与减函数的概念,现在学习利用导数来判断函数的增减性(即函数的单调性)。有如下一个重要的定理。
\begin{Theorem}{定理}
  设函数 $f(x)$ 在区间 $(a,b)$ 内可导。如果在 $(a,b)$ 内 $f'(x)>0$,那么 $f(x)$ 在 $(a,b)$ 内是增函数;如果在 $(a,b)$ 内 $f'(x)<0$,那么 $f(x)$ 在 $(a,b)$ 内是减函数。

  如果在 $(a,b)$ 内恒有 $f'(x)=0$,那么 $f(x)$ 在 $(a,b)$ 内是常数。
\end{Theorem}
\begin{proof}
  在区间 $(a,b)$ 内任取两点 $x_1,x_2$,且使 $x_1<x_2$,根据拉格朗日中值定理,可得
  \begin{equation}
    \label{eq:lagrange}
    f(x_2)-f(x_1)=f'(\xi)(x_2-x_1),\quad (x_1<\xi<x_2).
  \end{equation}

  如果在区间 $(a,b)$ 内 $f'(x)> 0$,则\cref{eq:lagrange} 中的 $f'(\xi)>0$,而 $x_2-x_1>0$,因此由\cref{eq:lagrange} 可知 $f(x_2)>f(x_1)$。
  这就是说,$f(x)$ 在区间 $(a,b)$ 内是增函数。

  如果在区间 $(a,b)$ 内 $f'(x)< 0$,则\cref{eq:lagrange} 中的 $f'(\xi)<0$,而 $x_2-x_1>0$,因此由\cref{eq:lagrange} 可知 $f(x_2)<f(x_1)$。
这就是说,$f(x)$ 在区间 $(a,b)$ 内是减函数。

如果在区间 $(a,b)$ 内恒有 $f'(x)= 0$,则\cref{eq:lagrange} 中的 $f'(\xi)=$ 0,因此由\cref{eq:lagrange} 可知 $f(x_2)=f(x_1)$。这就是说,在区间 $(a,b)$ 内,任意两点的函数值相等。
因此,$f(x)$ 在区间 $(a,b)$ 内是常数。
\end{proof}

\begin{example}
  确定函数 $y=x^2-2x+4$ 在哪个区间内是增函数,哪个区间内是减函数。
\end{example}
\noindent
\begin{minipage}{0.55\linewidth}\parindent2em
  \begin{solution}
    将 $y$ 对 $x$ 求导,得
    \[ y'=2x-2.\]

    解不等式 $y'=2x-2>0$,得 $x>1$,因此 $y$ 在 $(1,+\infty)$ 内是增函数;

解不等式 $y'=2x-2<0$ ,得 $x<1$,因此 $y$ 在 $(-\infty,1)$ 内是减函数(\cref{fig:3-4})。
  \end{solution}
\end{minipage}\hfill
\begin{minipage}{0.4\linewidth}
  \begin{figurehere}
    \includegraphics{3-4.pdf}
    \caption{}\label{fig:3-4}
  \end{figurehere}
\end{minipage}

\par\medskip\noindent
\begin{minipage}{0.55\linewidth}\parindent2em
\begin{example}
  确定函数 $y= \dfrac{2}{x}$($x\neq 0$)的增减范围。
\end{example}
\begin{solution}
  $y'=-\dfrac{2}{x^2}$.

  由于 $x\neq 0$ 时,有 $-\dfrac{2}{x^2}<0$,$y$ 在 $(-\infty,0)$,$(0,+\infty)$ 内都是减函数(\cref{fig:3-5})。
\end{solution}
\end{minipage}\hfill
\begin{minipage}{0.4\linewidth}
  \begin{figurehere}
    \includegraphics{3-5.pdf}
    \caption{}\label{fig:3-5}
  \end{figurehere}
\end{minipage}

\begin{example}
  当 $x>0$ 时,证明不等式
  \[ \ln(1+x)>x-\frac{1}{2}x^2\]
  成立。
\end{example}
\begin{proof}
  设 $f(x)=\ln(1+x)-x+\dfrac{1}{2}x^2$,则
  \[ f'(x) = \frac{1}{1+x}-1+x=\frac{x^2}{1+x}.\]

  当 $x>0$ 时,$f'(x)>0$,因此 $f(x)$ 在 $(0,+\infty)$ 内为增函数,又 $f(x)$ 在点 $x=0$ 处连续,所以 $f(x)$ 在 $[0,+\infty)$ 上也是增函数。于是当 $x>0$ 时,$f(x)>f(0)$。
  \[ \because \quad f(0)=0,\qquad \therefore \quad f(x)>0,\]
  即
  \[ \ln(1+x)-x+\frac{1}{2}x^2>0.\]
  于是证得:
  \[ \ln(1+x)>x-\frac{1}{2}x^2.\]
\end{proof}


\begin{Practice}
  \begin{question}
    \item 根据三角函数在各象限内的值的符号,利用导数证明函数 $y=\cos x$ 在开区间 $\left(\dfrac{\uppi}{2},\uppi\right)$ 内是减函数。
    \item 确定下列函数的增减范围(不必画出图象):
    \begin{tasks}[before-skip=7pt,after-skip=7pt,after-item-skip=7pt](2)
      \task $y=\dfrac{1}{2x^2}\quad(x\neq 0)$;
      \task $y=x-x^3$;
      \task $y=\ln(2x-1)$;
      \task $y=-\upe^x$。
    \end{tasks}
    \item 已知函数 $y=ax^2$($a\neq0$)当 $x>0$ 时是减函数,利用求导数的方法确定 $a$ 的值的范围。
    \item 求下列函数的驻点:
    \begin{tasks}(2)
      \task $y=5x^2-4+1$;
      \task $y=x^3-27x$。
    \end{tasks}
    \item 当 $x>0$ 时,证明下列不等式成立:
    \begin{tasks}[before-skip=7pt,after-skip=7pt,after-item-skip=7pt](3)
      \task $\sin x>x-\dfrac{x^3}{6}$;
      \task $\cos x< 1-\dfrac{x^2}{2}+\dfrac{x^4}{24}$;
      \task $\upe^x>1+x+\dfrac{x^2}{2}$。
    \end{tasks}
  \end{question}
\end{Practice}

\subsection{函数的极大值与极小值}\label{subsec:peak}
如果函数 $y=f(x)$ 在点 $x_0$ 处连续,并且 $x_0$ 不是其定义区间的端点。若对 $x_0$ 附近的所有点 $x$($x\neq x_0$),都有
\[ f(x)<f(x_0) \qquad\qquad \text{(或} f(x)>f(x_0)\text{),}\]
我们就说函数 $f(x)$ 在点 $x_0$ 处取极大值(或极小值)。
也可以说 $f(x_0)$ 是函数 $f(x)$ 的一个\Concept{极大值}(或\Concept{极小值}),记作 $y_{\text{极大}}=f(x_0)$(或 $y_{\text{极小}}=f(x_0)$),并把点 $x_0$ 称为函数 $f(x)$ 的一个\Concept{极大点}(或\Concept{极小点})。
极大值与极小值统称为\Concept{极值}。
极大点与极小点统称为\Concept{极值点}。
\cref{fig:3-6,fig:3-7} 分别显示了可导函数在点 $x_0$ 处取极大值与极小值的情况。

\begin{figure}
  \begin{minipage}{0.48\linewidth}\centering
    \includegraphics{3-6.pdf}
    \caption{}\label{fig:3-6}
  \end{minipage}
  \begin{minipage}{0.48\linewidth}\centering
    \includegraphics{3-7.pdf}
    \caption{}\label{fig:3-7}
  \end{minipage}
\end{figure}

\medskip\noindent
\begin{minipage}{0.65\linewidth}\parindent2em
从\cref{fig:3-6,fig:3-7} 可以看出,可导函数 $f(x)$ 的曲线在它的极值点 $x_0$ 处的切线都平行于 $x$ 轴,即 $f'(x_0)=0$。
换句话说,可导函数的极值点一定是它的驻点。
但是,反过来,可导函数的驻点,却不一定是它的极值点。
如 $f(x)=x^3$ 的导数是 $f'(x)=3x^2$,在 $x=0$ 处有 $f'(0)=0$,即点 $x=0$ 是函数 $f(x)=x^3$ 的驻点,但并不是它的极值点(\cref{fig:3-8})。

因此,在求可导函数的极值时,除了 $f'(x)=0$ 的条件外,还要考虑 $f'(x)$ 在驻点 $x_0$ 两侧的正负情况:如果 $f'(x_0)= 0$,并且 $x$ 由小变大经过 $x_0$ 时,$f'(x)$ 由正变为负(或由负变为正),那么 $f(x)$ 在点 $x_0$ 必然取得极大值(或极小值)。
\end{minipage}\hfill
\begin{minipage}{0.3\linewidth}
  \begin{figurehere}
    \includegraphics{3-8.pdf}
    \caption{}\label{fig:3-8}
  \end{figurehere}
\end{minipage}

\medskip
综上所述,并根据\cref{subsec:monotonicity} 定理(参看\cref{fig:3-6,fig:3-7}),我们得到求可导函数 $f(x)$ 的极值的方法如下:
\begin{enumerate}[1.]
  \item 求导数 $f'(x)$;
  \item 求 $f(x)$ 在定义域内的驻点;
  \item 检查 $f'(x)$ 在驻点左右的符号,如果左正右负,那么 $f(x)$ 在这个驻点取极大值;
  
  如果左负右正,那么 $f(x)$ 在这个驻点取极小值;

  如果左右同号,那么 $f(x)$ 在这个驻点的函数值不是极值。
\end{enumerate}

\begin{example}
  求函数 $f(x)=\dfrac{1}{3}x^3-4x+4$ 的极值。
\end{example}
\begin{solution}
  $f(x)$ 是可导函数,令
  \[f'(x)=x^2-4=(x+2)(x-2)=0,\]
  解得驻点为 $x_1=-2$,$x_2=2$。

  当 $x$ 变化时,$f'(x)$,$f(x)$ 的变化状态如\cref{tab:3-1} 所示:
  \begin{table}
    \caption{$f'(x)$,$f(x)$ 的变化状态}\label{tab:3-1}
    \begin{tblr}{colspec={*6{X[c]}},vline{2}=0.8pt}
      $x$     & $(-\infty,-2)$ & $-2$ & $(-2,2)$ & $2$ & $(2,+\infty)$ \\
      $f'(x)$ & $+$ & $0$ & $-$ & $0$ & $+$ \\
      $f(x)$  & $\nearrow$ & {极大值 $9\dfrac13$} & $\searrow$ & {极小值 $-1\dfrac13$} & $\nearrow$ \\
    \end{tblr}
  \end{table}

  \medskip\noindent
  \begin{minipage}{0.6\linewidth}\parindent2em
  因此当 $x=-2$ 时,函数 $f(x)$ 有极大值,且
  \[ f(-2)=\frac13(-2)^3-4(-2)+4=9\frac13;\]
  而当 $x=2$ 时,函数 $f(x)$ 有极小值,且
  \[ f(2)=\frac13\cdot2^3-4\cdot2+4=-1\frac13.\]

  函数 $f(x)=\dfrac13x^3-4x+4$ 的图象如\cref{fig:3-9} 所示。
  \end{minipage}\hfill
  \begin{minipage}{0.35\linewidth}
    \begin{figurehere}
      \includegraphics{3-9.pdf}
      \caption{}\label{fig:3-9}
    \end{figurehere}
  \end{minipage}
\end{solution}

\begin{example}
  求函数 $f(x)=x+2\sin x$ 在区间 $[0,2\uppi]$ 内的极值。
\end{example}
\begin{solution}
  因 $f'(x)=1+2\cos x$,令 $1+2\cos x=0$,解得驻点为 $x_1=\dfrac23\uppi$,$x_2=\dfrac43\uppi$。

  当 $x$ 变化时,$f'(x)$,$f(x)$ 的变化状态如\cref{tab:3-2} 所示:
  \begin{table}
    \caption{$f'(x)$,$f(x)$ 的变化状态}\label{tab:3-2}
    \begin{tblr}{colspec={cX[c]X[c]X[3,c]cX[3,c]X[c]X[c]},vline{2}=0.8pt}
      $x$     & $0$ & $\cdots$ & $\dfrac23\uppi$ & $\cdots$ & $\dfrac43\uppi$ & $\cdots$ & $2\uppi$\\
      $f'(x)$ &     & $+$ & $0$ & $-$ & $0$ & $+$  &\\
      $f(x)$  & $0$ & $\nearrow$ & {极大值 $\dfrac23\uppi+\sqrt{3}$} & $\searrow$ & {极小值 $\dfrac43\uppi-\sqrt{3}$} & $\nearrow$ & $2\uppi$ \\
    \end{tblr}
  \end{table}

  \medskip\noindent
  \begin{minipage}{0.52\linewidth}\parindent2em
  因此,当 $x=\dfrac23\uppi$ 时,函数 $f(x)$ 有极大值,且
  \[f\left(\frac23\uppi\right)=\frac23\uppi+\sqrt{3};\]
  而当 $x=\dfrac43\uppi$ 时,函数 $f(x)$ 有极小值,且
  \[f\left(\frac43\uppi\right)=\frac43\uppi-\sqrt{3}.\]

  函数 $f(x)=x+2\sin x$ 的图像如\cref{fig:3-10} 所示。
  \end{minipage}\hfill
  \begin{minipage}{0.43\linewidth}\centering
    \begin{figurehere}
      \includegraphics{3-10.pdf}
      \caption{}\label{fig:3-10}
    \end{figurehere}
  \end{minipage}
\end{solution}

\begin{Practice}
  \begin{question}
    \item 填写下表:
    \begin{tablehere}
      \begin{minipage}{\linewidth}
        \begin{tblr}{colspec={*3{X[c]}},hline{2}=0.8pt,hline{3}=0pt}
          导数 $y'$ & 函数 $y$ & 极值\\
          由\CJKunderline[hidden]{\ 正\ }变为\CJKunderline[hidden]{\ 负\ }& 由\CJKunderline{\ 增\ }变为\CJKunderline{\ 减\ }  & 极\CJKunderline[hidden]{\quad 大\quad }值\\
          由\CJKunderline[hidden]{\ 负\ }变为\CJKunderline[hidden]{\ 正\ }& 由\CJKunderline{\ 减\ }变为\CJKunderline{\ 增\ }  & 极\CJKunderline[hidden]{\quad 小\quad }值\\
        \end{tblr}
      \end{minipage}
    \end{tablehere}
    \item 说明函数 $y=\ln x$,$y=ax+b$ 为什么没有极值。
    \item 求下列函数的极值,并根据极值情况画出草图:
    \begin{tasks}(2)
      \task $y=x^2-7x+6$;
      \task $y=3x^4-4x^3$;
      \task!$y=\dfrac{1}{2}x+\cos x \quad (-2\uppi<x<2\uppi)$。
    \end{tasks}
    \item 用求导数的方法证明二次函数 $ax^2+bx+c$($a\neq 0$)的极值点为 $x_0=-\dfrac{b}{2a}$,并讨论它的极值。
  \end{question}
\end{Practice}

\begin{Exercise}
  \begin{question}
    \item 确定下列函数的增减范围:
    \begin{tasks}(2)
      \task $y=-4x+2$;
      \task $y=(x-1)^2$;
      \task $y=x^2-2x+5$;
      \task $y=3x-x^3$;
      \task $y=x^2(x-3)$;
      \task $y=x^3-x^2-x$。
    \end{tasks}
    \item 证明下列函数的单调性:
    \begin{tasks}
      \task $p$ 为何值时,函数 $f(x)=\cos x-px+q$ 在整个数轴上是减函数?
      \task 证明 $y=2x+\sin x$ 在整个数轴上是增函数;
      \task 证明 $y=x+\dfrac{1}{x^2+1}$ 在整个数轴上是增函数;
      \task 研究函数 $y=\dfrac{1}{x^2++1}$ 的单调性。
    \end{tasks}
    \item 证明 $y=\sqrt{2x-x^2}$ 在区间 $(0,1)$ 内为增函数,在区间 $(1,2)$ 内为减函数。
    \item 求下列二次函数的极值:
    \begin{tasks}(2)
      \task $y=x^2-3x+10$;
      \task $y=-2x^2+4x-7$;
      \task $y=6x^2-x-2$;
      \task $y=-x^2-2x+3$;
      \task $y=2-x-x^2$;
      \task $y=\dfrac{1}{2}x^2-3x$。
    \end{tasks}
    \item 求下列函数的极值:
    \begin{tasks}(2)
      \task $y=6+12x-x^3$;
      \task $y=2x^2-x^4$;
      \task $y=2x^3-9x^2-24x-12$;
      \task $y=\dfrac{1}{3}x^3-\dfrac{1}{2}x^2-2x+2$;
      \task $y=15+9x-3x^2-x^3$;
      \task $y=4x^3-3x^2-6x+2$;
      \task $y=2x-\sin x$;
      \task $y=2\upe^x+\upe^{-x}$。
    \end{tasks}
  \end{question}
\end{Exercise}

\subsection{函数的最大值与最小值}
在生产中,常常会遇到要求在一定条件下使“强度最大”,“用料最省”,“功率最大”这样一类问题,在数学上,这类问题往往归结为求函数的最大值或最小值。

本节讨论函数的最大值与最小值,是导数应用的又一个方面。在\cref{subsec:peak}已知可导函数的极大值与极小值,就是函数在点 ${x}_{0}$ 附近的最大值与最小值。如\cref{fig:3-11} 所示。但是,就区间 $\left\lbrack {a,b}\right\rbrack$ 上的给定函数 $f(x)$ 来说,函数的极大值不一定是最大值,极小值也不一定是最小值。
\begin{figure}
  \includegraphics{3-11.pdf}
  \caption{}\label{fig:3-11}
\end{figure}

如果函数 $f(x)$ 在闭区间 $[a,b]$ 上连续,在开区间 $(a,b)$ 内可导,从连续函数性质 1 得知,函数 $f(x)$ 在 $[a,b]$ 上有最大值与最小值。
如\cref{fig:3-11} 中,最大值是区间端点 $a$ 的值 $f(a)$,最小值是几个极小值中最小的一个 $f(x_3)$。
这就是说求函数 $f(x)$ 的最大值,只要求得所有 $f(x)$ 的极大值与 $f(a)$、$f(b)$ 这些值中最大者就行了。
而求函数 $f(x)$ 的最小值,也只考虑 $f(x)$ 的所有极小值与 $f(a)$、$f(b)$ 中的最小者。

因此,求闭区间 $[a,b]$ 上的可导函数 $f(x)$ 在 $[a,b]$ 上的最大值与最小值,可以分两步来进行:
\begin{enumerate}[1.]
  \item 求 $f(x)$ 在 $(a,b)$ 内的驻点;
  \item 计算 $f(x)$ 在驻点和端点的函数值,并把这些值加以比较,其中最大的一个为最大值,最小的一个为最小值。
\end{enumerate}

因为二次函数在其定义域内只有一个极值点,所以在包含极值点的闭区间上,二次函数的极大值就是最大值,极小值就是最小值,如\cref{fig:3-12} 所示。

\begin{figure}
  \begin{minipage}{0.48\linewidth}\centering
  \includegraphics{3-12a.pdf}
    \subcaption{}\label{fig:3-12a}
  \end{minipage}
  \begin{minipage}{0.48\linewidth}\centering
  \includegraphics{3-12b.pdf}
    \subcaption{}\label{fig:3-12b}
  \end{minipage}
  \caption{}\label{fig:3-12}
\end{figure}

\begin{example}
  求函数 $y=x^4-2x^2+5$ 在区间 $[-2,2]$ 上的最大值与最小值。
\end{example}
\par\medskip\noindent
\begin{minipage}{0.75\linewidth}\parindent2em
\begin{solution}
  $y'=4x^3-4x$。

  令 $4x^3-4x=0$,求得驻点为
  \[x_1=-1,\quad x_2=0,\quad x_3=1.\]

  这些驻点的函数值为
  \[ y\vert_{x=0}=5,\quad y\vert_{x=\pm1}=4.\]

  区间端点的函数值为
  \[ y\vert_{x=\pm2}=13. \]

  将这些求出的函数值加以比较,知道最大值为 13 ,最小值为 4(\cref{fig:3-13})。
\end{solution}
\end{minipage}\hfill
\begin{minipage}{0.2\linewidth}\centering
  \begin{figurehere}
    \includegraphics{3-13.pdf}
    \caption{}\label{fig:3-13}
  \end{figurehere}
\end{minipage}

\medskip
\par\medskip\noindent
\begin{minipage}{0.45\linewidth}\parindent2em
  \begin{example}
  用边长为 \qty{60}{cm} 的正方形铁皮做一个无盖水箱,先在四角分别截去一个小正方形,然后把四边翻转 \ang{90},再焊接而成(\cref{fig:3-14})。
  问水箱底边的长应取多少,才能使水箱容积最大。
  最大容积是多少?
\end{example}
\end{minipage}\hfill
\begin{minipage}{0.5\linewidth} 
  \begin{figurehere}
    \includegraphics{3-14.pdf}
    \caption{}\label{fig:3-14}
  \end{figurehere}
\end{minipage}

\medskip
\begin{solution}
  设水箱底边长为 $n$(\unit{cm}),则水箱高
  \[h=\frac{60-x}{2}\]

  水箱容积
  \begin{align*}
    V&=V(x)=x^2h \\
     &=\frac{60x^2-x^3}{2}\quad(0<x<60).
  \end{align*}
\end{solution}

  由问题的实际情况来看,如果 $x$ 过小,水箱的底面积就很小,容积 $V$ 也就很小;如果 $x$ 过大,水箱的高就很小,容积 $V$ 也就很小。因此,其中必有一适当的 $x$ 值,使容积 $V$ 取得最大值。

  令
  \[V'(x)=60x-\frac{3}{2}x^2=0\]
  得两个根
  \[ x=0\text{ (不合题意,舍去),}x=40,\]
  从而在定义域 $(0,60)$ 内,函数 $V(x)$ 只有一个驻点
  \[ x=\qty{40}{cm}.\]
  代入函数式 $V(x)$,即得
  \[V_{\text{max}}=(40)^2\cdot\frac{60-40}{2}=\qty{16000}{cm^3}.\]

  答:水箱底边长取 \qty{40}{cm} 时,容积最大。最大容积为 \qty{16000}{cm^3}。

\bigskip
要注意,如果由问题的实际情况,可以断定可导函数在定义域开区间内存在最大(小)值,而且 $f(x)$ 在这个定义域开区间内又只有一个驻点,那么立即可以断定这个驻点的函数值就是最大(小)值。这一点在解决某些实际问题时很有用。

\begin{example}
  矩形横梁的强度同它断面的高的平方与宽的积成正比。要将直径为 $d$ 的圆木锯成强度最大的横梁,断面的宽和高应是多少?
\end{example}
\par\medskip\noindent
\begin{minipage}{0.65\linewidth}\parindent2em
\begin{solution}
  如\cref{fig:3-15} 所示,设断面宽为 $x$,高为 $h$,则
  \[h^2=d^2-x^2.\]

  横梁强度函数
  \begin{gather*}
  f(x)=kxh^2,\text{ (}k \text{ 为比例系数,}k>0\text{)。}\\
  \therefore \qquad f'(x)=kx(d^2-x^2)\quad(0<x<d).
  \end{gather*}
\end{solution}
\end{minipage}\hfill
\begin{minipage}{0.3\linewidth}
  \begin{figurehere}
    \includegraphics{3-15.pdf}
    \caption{}\label{fig:3-15}
  \end{figurehere}
\end{minipage}

\medskip
  从实际情况可知,横梁强度函数 $f(x)$ 在 $(0,d)$ 内一定有最大值。令
  \[f'(x)=k(d^2-3x^2)=0,\]
  即 $d^2-x^2=0$,此方程有两个根 $x=\pm\dfrac{\sqrt{3}}{3}d$,其中负根没有意义,舍去,从而在定义域 $(0,d)$ 内,函数 $f(x)$ 只有一个驻点
  \[x=\frac{\sqrt{3}}{3}d.\]
  $f(x)$ 在这一点的函数值,就是横梁强度的最大值。此时
  \[h=\sqrt{d^2-x^2}=\frac{\sqrt{6}}{3}d.\]

  答:当宽为 $\dfrac{\sqrt{3}}{3}d$,高为 $\dfrac{\sqrt{6}}{3}d$ 时,横梁强度最大。

\begin{example}
  某生产队要造一个可容纳 \qty{7.5}{t} 氨水(比重为 \qty{0.925}{t/m^3})的有盖有底圆柱形氨水槽(\cref{fig:3-16}),问氨水槽的底半径与圆柱高应取多少,才能使所用的材料最省?
\end{example}
\par\medskip\noindent
\begin{minipage}{0.65\linewidth}\parindent2em
\begin{solution}
  材料最省,就是使圆柱表面积最小。

  设槽底半径为 $R$,高为 $h$,则氨水槽的表面积
  \[S=2\uppi Rh+2\uppi R^2.\]
  
  由氨水槽的容积 $V=\uppi R^2h$,知 $h=\dfrac{V}{\uppi R^2}$,代入上式得
  \begin{align*}
  S=S(R)&=2\uppi R\frac{V}{\uppi R^2}+2\uppi R^2\\
        &=\frac{2V}{R}+2\uppi R^2.
  \end{align*}
\end{solution}
\end{minipage}\hfill
\begin{minipage}{0.3\linewidth}\centering
  \begin{figurehere}
    \includegraphics{3-16.pdf}
    \caption{}\label{fig:3-16}
  \end{figurehere}
\end{minipage}

\medskip
  从实际情况可知,表面积 $S$ 在其定义域内一定有最小值。令
  \[ S'(R)=-\frac{2V}{R^2}+4\uppi R=0,\]
  即 $2V-4\uppi R^3=0$,此方程的实根只有一个,就是 $R=\sqrt[\uproot{10}\leftroot{-3}3]{\dfrac{V}{2\uppi}}$,因此我们得到唯一驻点
  \[R=\sqrt[\uproot{10}\leftroot{-3}3]{\frac{V}{2\uppi}}.\]
  当 $R=\sqrt[\uproot{10}\leftroot{-3}3]{\dfrac{V}{2\uppi}}$ 时,
  \[ h=\frac{V}{\uppi R^2} =\frac{V}{\uppi\left(\sqrt[\uproot{10}\leftroot{-3}3]{\dfrac{V}{2\uppi}}\right)^2}=\sqrt[\uproot{10}\leftroot{-3}3]{\frac{4V}{\uppi}}=2\sqrt[\uproot{10}\leftroot{-3}3]{\frac{V}{2\uppi}}, \]
  这就是说,当 $h=2R$ 时,函数 $S(R)$ 取得最小值,这时所用的材料最省。
  
  现在我们可以得到
  \[ V=\frac{7.5}{0.925}\approx\qty{8.11}{m^3},\]
  于是
  \begin{gather*} 
  R=\sqrt[\uproot{10}\leftroot{-3}3]{\frac{V}{2\uppi}}\approx\sqrt[\uproot{10}\leftroot{-3}3]{\frac{8.11}{2\uppi}}\approx\qty{1.09}{m},\\
  h=2R\approx\qty{2.18}{m}.
  \end{gather*}
  这时建造氨水槽所用的材料最省。

  答:当氨水槽的底半径约为 \qty{1.09}{m},高约为 \qty{2.18}{m}时,所用的材料最省。

\begin{example}
  已知电源电压为 $E$,内电阻为 $r$(\cref{fig:3-17}),问当外电路负载电阻 $R$ 取什么值时,输出功率最大。
\end{example}
\par\medskip\noindent
\begin{minipage}{0.6\linewidth}\parindent2em
\begin{solution}
  由欧姆定律得电流强度
  \[ I=\frac{E}{R+r}\]

  在负载电阻 $R$ 上输出的功率
  \[ P=P(R)=I^2R=\frac{E^2R}{(R+r)^2}.\]
\end{solution}
\end{minipage}\hfill
\begin{minipage}{0.35\linewidth}  
  \begin{figurehere}
    \includegraphics{3-17.pdf}
    \caption{}\label{fig:3-17}
  \end{figurehere}
\end{minipage}

\medskip
  实验证明,当 $E$,$r$ 一定时,输出功率由负载电阻 $R$ 的大小决定。$R$ 很小时,电源功率大都消耗在内电阻 $r$ 上,输出功率可以变得很小;$R$ 很大时,电路中电流很小,输出功率也可以变得很小。因此,$R$ 一定有一个适当的数值,使得输出功率最大。令
  \[
    P'(R)=\left[\frac{E^2R}{(R+r)^2}\right]'=E^2\cdot\frac{(R+r)^2-2R(R+r)}{(R+r)^4}=E^2\cdot\frac{r-R}{(R+r)^3}=0
  \]
  即 $E^2(r-R)=0$,求得唯一驻点
  \[R=r.\]
  所以,当 $R=r$,即外电路负载电阻等于内电阻时,输出功率最大。

  答:当外电路负载电阻 $R$ 等于内电阻 $r$ 时,输出功率最大。

\begin{Practice}
  \begin{question}
    \item 已知函数 $y=3x^3-9x+5$,求函数在 $[-2,2]$ 上的最大值与最小值。
    \item 把长度为 $l$ 的线段分成两段,使得以这两段分别作为长与宽所得的举行的面积最大。
    \item 把长为 $l$ 的铁丝分成两段,各围成一个正方形,问怎样分法才能使它们的面积之和最小。
    \item 等腰三角形的周长为 $2p$,问绕这个三角形的底边旋转一周所成立体的体积为最大时,各边长分别是多少。
  \end{question}
\end{Practice}

\begin{Exercise}
  \begin{question}
    \item 证明函数 $y=2x^3+3x^2-12+1$ 在区间 $(-2,1)$ 内是减函数。
    \item 确定下列函数的增减范围:
    \begin{tasks}[before-skip=5pt,after-skip=5pt,after-item-skip=7pt](2)
      \task $y=(5-x)(1+x)$;
      \task $y=x^3-12x+2$;
      \task $y=x^3-9x^2+24x$;
      \task $y=x^4-2x^2-5$;
      \task $y=x\lg x$;
      \task $y=x\upe^x$。
    \end{tasks}
    \item 求下列函数的驻点:
    \begin{tasks}[before-skip=5pt,after-skip=5pt,after-item-skip=7pt](2)
      \task $y=x^3-2x^2-9x+31$;
      \task $y=6x^2-x^4$;
      \task $y=\dfrac{2}{1+x^2}$;
      \task $y=\sin x-\sqrt{3}\cos x$($0<x<2\uppi$)。
    \end{tasks}
    \item 讨论下列函数的增减性:
    \begin{tasks}[before-skip=5pt,after-skip=5pt,after-item-skip=7pt](2)
      \task $f(x)=7x^2+14x+1$;
      \task $f(x)=\dfrac{1}{3x}$;
      \task $f(x)=2x^3-6x^2-18x-7$;
      \task $f(x)=x^4-2x^2-5$;
      \task $y=x-\upe^x$;
      \task $y=x+\cos x$;
      \task $y=x-\sin x$;
      \task! 函数 $y=\arctan x-x$ 是整个定义域内的减函数。
    \end{tasks}
    \item 证明下列不等式成立:
    \begin{tasks}[before-skip=5pt,after-skip=5pt,after-item-skip=7pt](2)
      \task 当 $x>0$ 时,$\dfrac{5}{3}x^3-2x^2+x>0$;
      \task 当 $x>0$ 时,$x^5-\dfrac{4}{3}x^3+x>0$;
      \task! 如果 $0<x_1<x_2<\dfrac{\uppi}{2}$,那么 $\dfrac{\tan x_1}{x_1}<\dfrac{\tan x_2}{x_2}$。
    \end{tasks}
    \item 已知函数 $y=a(x^3-x)$($a\neq 0$),
    \begin{tasks}[before-skip=5pt,after-skip=5pt,after-item-skip=7pt]
      \task 如果 $x>\dfrac{\sqrt{3}}{3}$ 时,$y$ 是减函数,确定 $a$ 的值的范围;
      \task 如果 $x<-\dfrac{\sqrt{3}}{3}$ 时,$y$ 是减函数,确定 $a$ 的值的范围;
      \task 如果 $-\dfrac{\sqrt{3}}{3}<x<\dfrac{\sqrt{3}}{3}$ 时,$y$ 是减函数,确定 $a$ 的值的范围。
    \end{tasks}
    \item 求下列函数的极值:
    \begin{tasks}[before-skip=5pt,after-skip=5pt,after-item-skip=7pt](2)
      \task $y=x^3+12x^2+36x-50$;
      \task $y=4x^5-5x^4-40x^3$;
      \task $y=3x^5-5x^3+2$;
      \task $y=\dfrac{x^2}{x^2+3}$;
      \task $y=x^3-2x+\dfrac{8}{x}$;
      \task $y=(x^2-3)\upe^x$。
    \end{tasks}
    \item 求下列函数的极值:
    \begin{tasks}[before-skip=5pt,after-skip=5pt,after-item-skip=7pt](2)
      \task $y=\sin x\cos x$($0<x<\uppi$);
      \task $y=1-\sqrt{x^2-20+10}$;
      \task $y=1-\sqrt{6-x-x^2}$;
      \task $y=b+c(x-a)^{\frac{3}{2}}$;
      \task $y=\dfrac{x^3+x}{x^4-x^2+1}$;
      \task $y=x-\ln(1+x)$;
      \task $y=(x-5)^2\sqrt[3]{(x+1)^4}$;
      \task! $y=a\upe^{px}+b\upe^{-px}$($a$ 与 $b$ 同号,$p\neq 0$)。
    \end{tasks}
    \item 求下列函数在给定区间内的极值:
    \begin{tasks}[before-skip=5pt,after-skip=5pt,after-item-skip=7pt](2)
      \task $y=\cos\left(x+\dfrac{\uppi}{4}\right)$,$x$ 在 $(0,\uppi)$ 内;
      \task! $y=\cos x+\sin x$,$x$ 在 $\left(-\dfrac{\uppi}{2},\dfrac{\uppi}{2}\right)$ 内;
      \task $y=\dfrac{x}{1+x^2}$,$x$ 在 $\left(-\dfrac{3}{2},\dfrac{1}{2}\right)$ 内;
      \task $y=x-\sin2x$,$x$ 在 $(0,\uppi)$ 内;
      \task! $y=2\tan x-\tan^2 x$,$x$ 在 $(0,2\uppi)$ 内。
    \end{tasks}
    \item 求下列函数在给定区间内的最大值与最小值:
    \begin{tasks}[before-skip=5pt,after-skip=5pt,after-item-skip=7pt](2)
      \task $y=x^4-2x^2+5$,$[-2,2]$;
      \task $y=x+2\sqrt{x}$,$[0,4]$;
      \task $y=\dfrac{1-x+x^2}{1+x-x^2}$,$[0,1]$;
      \task $y=2\tan x-\tan^2 x$,$\left[0,-\dfrac{\uppi}{3}\right]$。
    \end{tasks}
    \item 将 36 分成两个因数,使其平方和最小。
    \item 求外切于半径为 $R$ 的球并且体积最小的圆锥的高。
    \item 在抛物线 $y^2=2px$ 的对称轴上,已知一个与顶点距离为 $a$ 的点 $M$(在 $y$ 轴右侧),求曲线上点 $N$ 的横坐标,使得 $|MN|$ 最小。
    \item\label{ex:11-14} 如图,已知一个正方形内接于另一个固定的正方形,问 $\alpha$ 取什么值时,内接正方形面积最小。(提示:用求导数的方法来解,可设小正方形边长为 $x$。)
    \begin{figurehere}
      \begin{minipage}{\linewidth}\centering
        \includegraphics{ex11-14.pdf}
        \caption*{(第 \ref{ex:11-14} 题图)}
      \end{minipage}
    \end{figurehere}
    \item 求下列函数在给定区间的最大值与最小值:
    \begin{tasks}
      \task $y=5-36x+3x^2+4x^3,\quad [-2,2]$;
      \task $y=4x^2(x^2-2),\quad [-2,2]$。
    \end{tasks}
    \item 将 8 分为两部分,使其立方和最小。
    \item\label{ex:11-17} 有木料长为 \qty{6}{m},要做一个如图的窗框,已知上框架与下框架的高之比为 $1:2$,问怎样利用木料,才能使光线通过的窗框面积最大(中间木档的面积可以忽略不计)。
    \begin{figurehere}
      \begin{minipage}{\linewidth}\centering
        \includegraphics{ex11-17.pdf}
        \caption*{(第 \ref{ex:11-17} 题图)}
      \end{minipage}
    \end{figurehere}
    \item 有根铁丝长 \qty{72}{cm},截成十二段,搭成一个正四棱柱的模型,要求占空间位置最大,问线段应该怎样截法。(提示:占空间位置是指铁丝所围成的正四棱柱的体积。)
    \item\label{ex:11-19} 如图,用半径为 $R$ 的圆铁皮,建一个圆心角为 $\alpha$ 的扇形,制成一个圆锥形的漏斗,问圆心角 $\alpha$ 取什么值时,漏斗容积最大。
    \begin{figurehere}
      \begin{minipage}{\linewidth}\centering
        \includegraphics{ex11-19.pdf}
        \caption*{(第 \ref{ex:11-19} 题图)}
      \end{minipage}
    \end{figurehere}
    \item 解答:
    \begin{tasks}
      \task 求内接于半径为 $R$ 的球并且体积最大的圆柱体的高;
      \task 求内接于半径为 $R$ 的球并且体积最大的圆锥体的高。
    \end{tasks}
    \item\label{ex:11-21} 如图,已知海岛 $A$ 到海岸公路 $BD$ 的距离 $AD$ 为 \qty{50}{km},$D$ 与工厂 $B$ 的距离为 \qty{200}{km},海上机船的速度为 \qty{25}{km/h},岸上卡车的速度为 \qty{50}{km/h}。问在海岸公路 $BD$ 上哪一处设立转运站 $C$,可以使从岛 $A$ 到工厂 $B$ 的运货时间最短(装货及卸货所用时间除外)。$B$、$D$ 的距离对 $C$ 点的位置有没有影响?
    \begin{figurehere}
      \begin{minipage}{\linewidth}\centering
        \includegraphics{ex11-21.pdf}
        \caption*{(第 \ref{ex:11-21} 题图)}
      \end{minipage}
    \end{figurehere}
    \item\label{ex:11-22} 如图,铁路线上 $AB$ 段长 \qty{100}{km},工厂 $C$ 到铁路的距离 $CA$ 为 \qty{20}{km}。现在要在 $AB$ 上某一点 $D$ 处,向 $C$ 修一条公路。已知铁路每吨公里与公路每吨公里的运费之比为 $3:5$,为了使原料从供应站 $B$ 运到工厂 $C$ 的运费最省,$D$ 点应选在何处?
    \begin{figurehere}
      \begin{minipage}{\linewidth}\centering
        \includegraphics{ex11-22.pdf}
        \caption*{(第 \ref{ex:11-22} 题图)}
      \end{minipage}
    \end{figurehere}
    \item\label{ex:11-23} 解答:
    \begin{tasks}
      \task 如图,已知防空洞的截面是矩形加半圆,周长为 $l$,底宽 $2x$ 取什么值时,截面面积最大?
      \task 如果上述防空洞截面积为 $S$,底宽 $2x$ 取什么值时,周长最小?
    \end{tasks}
    \begin{figurehere}
      \begin{minipage}[b]{0.33\linewidth}\centering
        \includegraphics{ex11-23.pdf}
        \caption*{(第 \ref{ex:11-23} 题图)}
      \end{minipage}%
      \begin{minipage}[b]{0.33\linewidth}\centering
        \includegraphics{ex11-24.pdf}
        \caption*{(第 \ref{ex:11-24} 题图)}
      \end{minipage}%
      \begin{minipage}[b]{0.33\linewidth}\centering
        \includegraphics{ex11-25.pdf}
        \caption*{(第 \ref{ex:11-25} 题图)}
      \end{minipage}
    \end{figurehere}
    \item\label{ex:11-24} 如图,在施工地点重心设计一灯架,上面挂一盏“太阳”灯,问灯离地面多高,可以使与工地中心距离为 $a$ 的圆形施工区域边上具有最大照度。(提示:照度 $J$ 与 $\cos\varphi$ 成正比,与光源距离 $r$ 的平方成反比。)
    \item\label{ex:11-25} 如图,在等腰梯形 $ABCD$ 中,底 $CD=40$,腰 $AD=40$,问 $AB$ 为多长时,等腰提醒的面积最大。(提示:可设 $\angle A=\theta$。)
  \end{question}
\end{Exercise}

\section{二阶导数的应用}
二阶导数的应用是导数应用的重要部分。
一是利用 $f''(x_0)<0$(或 $f''(x_0)>0$)来判定函数 $f(x)$ 的极大值 (或极小值);二是利用 $f''(x_0)>0$、$f''(x_0)<0$ 以及 $f''(x_0)=0$,来判定曲线 $y=f(x)$ 的凸向和拐点;三是绘制函数的图象。
为此,先讲预备知识。

\subsection{预备知识}\label{subsec:Preliminary}
首先引进连续函数的局部保号性质。

\begin{Theorem}{定理 1}
  若函数 $f(x)$ 在点 $x_0$ 处连续,且 $f(x_0)\neq 0$,则函数 $f(x)$ 在点 $x_0$ 附近必与 $f(x_0)$ 同号,即
  \begin{itemize}
    \item[] 当 $f(x_0)>0$ 时,$f(x)>0$(\cref{fig:3-18a});
    \item[] 当 $f(x_0)<0$ 时,$f(x)<0$(\cref{fig:3-18b})。
  \end{itemize}
\end{Theorem}
从\cref{fig:3-18} 容易看出连续函数具有局部保号性质。
\begin{figure}
  \begin{minipage}{0.48\linewidth}
    \includegraphics{3-18a.pdf}
    \subcaption{}\label{fig:3-18a}
  \end{minipage}
  \begin{minipage}{0.48\linewidth}
    \includegraphics{3-18b.pdf}
    \subcaption{}\label{fig:3-18b}
  \end{minipage}
  \caption{}\label{fig:3-18}
\end{figure}

其次,给出关于二阶导数的中值定理。
\begin{Theorem}{定理 2}
  如果函数 $f(x)$ 在闭区间 $[a,b]$ 上有二阶导数,那么至少有一点 $\xi\in(a,b)$,使得
  \begin{equation}
    \label{eq:theorem2}
    f(b)=f(a)+f'(a)(b-a)+\frac12f''(\xi)(b-a)^2.
  \end{equation}
\end{Theorem}
\begin{proof}
  令 
  \begin{equation}
    \label{eq:theorem2_K}
    f(b)=f(a)+f'(a)(b-a)+K(b-a)^2.
  \end{equation}
  若能证明 $K=\dfrac12f''(\xi)$,则定理的结论得证。

  \medskip
  为此,作辅助函数:
  \begin{equation}
    \label{eq:aux_func}
    \varphi(x)=f(b)-f(x)-f'(x)(b-x)-K(b-x)^2
  \end{equation}
  因 $f(x)$ 在 $[a,b]$ 上有二阶导数,故 $f(x)$、$f'(x)$ 在 $[a,b]$ 上连续,于是 $\varphi(x)$ 在 $[a,b]$ 上连续,且可导,又由\cref{eq:theorem2_K,eq:aux_func} 有
  \begin{align*}
    \varphi(a) &=f(b)-f(a)-f'(a)(b-a)-K(b-a)^2=f(b)-f(b)=0\\
    \varphi(b) &= f(b)-f(b)=0 \text{,即}\\
    \varphi(a) &=\varphi(b).
  \end{align*}
  根据罗尔定理可知,至少存在一点 $\xi\in(a,b)$,使得
  \[ \varphi'(\xi)=0.\]
  但对\cref{eq:aux_func} 求导得 $\varphi'(x)=-f''(x)(b-x)+2K(b-x)$,所以
  \[ \varphi'(\xi) =-f''(\xi)(b-\xi)+2K(b-\xi)=0.\]

  因 $b-\xi\neq 0$ 于是
  \[ K=\frac12f''(\xi).\]
  将 $K$ 代入\cref{eq:theorem2_K} 得证\cref{eq:theorem2} 成立,即
  \[ f(b)=f(a)+f'(a)(b-a)+\frac12f''(\xi)(b-a)^2.\]
\end{proof}

为了应用方便,本定理得结论也可写成如下形式:

令 $a=x_0$,$b=x\neq x_0$,则
\begin{equation}
  \label{eq:theorem2_other_form}
  f(x)=f(x_0)+f'(x_0)(x-x_0)+\frac12f''(\xi)(x-x_0)^2,
\end{equation}
其中 $\xi$ 在点 $x_0$ 与 $x$ 之间。

\subsection{函数极值的判定}
我们学过了用一阶导数判定函数极值的方法,下面进一步研究,怎样用二阶导数来判定函数的极大值与极小值问题。
\begin{Theorem}{定理}
  如果函数 $f(x)$ 在点 $x_0$ 附近有连续的导函数 $f'(x)$,且 $f'(x_0)=0$,$f''(x_0)\neq 0$。
  \begin{enumerate}
    \item 若 $f''(x_0)<0$,则函数 $f(x)$ 在点 $x_0$ 处取极大值;
    \item 若 $f''(x_0)>0$,则函数 $f(x)$ 在点 $x_0$ 处取极小值。
  \end{enumerate}
\end{Theorem}

\begin{proof}
  \begin{enumerate}
    \item 因 $f''(x_0)<0$,由连续函数的局部保号性质,在点 $x_0$ 附近有 $f''(x_0)<0$。
    
    根据\cref{subsec:Preliminary} 定理 2 中\cref{eq:theorem2_other_form} 有
    \begin{gather}
      f(x)=f(x_0)+f'(x_0)(x-x_0)+\frac{f''(\xi)}{2}(x-x_0)^2 \notag \\
      \because\quad f'(x_0)=0 \notag \\
      \label{eq:taylor-high}\therefore\quad f(x)-f(x_0)=\frac{f''(\xi)}{2}(x-x_0)^2
    \end{gather}
    又因 $\xi$ 在点 $x_0$ 与 $x$ 之间,即 $\xi$ 在点 $x_0$ 附近,于是 $f''(\xi)<0$。

    再从\cref{eq:taylor-high} 得 $f(x)<f(x_0)$,也就是说 $f(x_0)$ 为函数 $f(x)$ 在点 $x_0$ 处的极大值。
    \item 若 $f''(x_0)> 0$,同样可得 $f''(\xi)> 0$,再由\cref{eq:taylor-high} 则
    \[f(x)>f(x_0)\]

    也就是说,$f(x_0)$ 为 $f(x)$ 在点 $x_0$ 处的极小值。
  \end{enumerate}
\end{proof}

\begin{example}
  求函数 $f(x)=x^5-15x^3+3$ 的极值。
\end{example}
\begin{solution}
  $f'(x)=5x^4-45x^2$,$f''(x)=20x^3-90x$。令 $f'(x)=0$,求得驻点为 $x=-3,0,3$。

因为,$f''(-3)=-270<0$,故 $f(x)$ 在 $x=-3$ 取极大值,且 $f(-3)=165$。

因为 $f''(3)=270>0$,故 $f(x)$ 在 $x=3$ 取极小值,$f(3)=-159$。

\alertinfo{当 $f''(0)=0$ 时,$f(x)$ 在点 $x=0$ 处是否有极值不能判定,即使有极值,是极大值、极小值也不能判定。}

例如 $f_1(x)=(x-1)^3+1$ 在点 $x=1$ 处 $f''_1(1)= 0$,从 $f_1(x)$ 的图象可知 $f_1(x)$ 在 $x=1$ 处无极值。

又如 $f_2(x)=-(x-1)^4+1$ 在点 $x=1$ 处 $f''_2(1) = 0$,从 $f_2(x)$ 的图象可知 $f_2(x)$ 在 $x=1$ 处有极大值 1。

再如 $f_3(x)=(x-1)^4-1$ 在点 $x=1$ 处 $f''_3(1)=0$,从 $f_3(x)$ 的图象可知 $f_3(x)$ 在 $x=1$ 处有极小值 $-1$。
\end{solution}

\begin{example}
  求 $f(x)=(x^2-1)^2-1$ 的极值,并在下列区间 $(-\infty,\infty)$,$[-2,2]$,$(-1,1)$ 分别讨论其最大值、最小值。
\end{example}
\par\medskip\noindent
\begin{minipage}{0.65\linewidth}\parindent2em
\begin{solution}
  $f'(x)=4x(x^2-1)$,$f''(x)-4(3x^2-1)$,

  令 $f'(x)=4x(x^2-1)=0$,求得驻点:$x=-1,0,1$。
  
  因为 $f''(0)=-4<0$,故 $f(x)$ 在 $x=0$ 处取极大值。
  且极大值为 $f(0)=0$;

  又 $f''(-1)=f''(1)=8>0$,故 $f(x)$ 在 $x=\pm1$ 两点都有极小值。
  极小值为 $f(\pm1)=-1$。

  下面讨论函数 $f(x)$ 在不同区间内最大值、最小值是否存在问题。

  因当 $x\to\pm\infty$ 时,函数 $f(x)=(x^2-1)^2-1$ 也趋向于无穷大,故 $f(x)$ 在开区间 $(-\infty,\infty)$ 内无最大值。但从\cref{fig:3-19} 可知函数 $f(x)$,有最小值 $f(\pm1)=-1$;在 $[-2,2]$ 上 $f(x)$ 的最大值为 $f(\pm2)= 8$,最小值为 $f(\pm1)=-1$;在 $(-1,1)$ 内 $f(x)$ 的最大值为 $f(0)=0$,无最小值。
\end{solution}
\end{minipage}\hfill
\begin{minipage}{0.3\linewidth}
  \begin{figurehere}
    \includegraphics{3-19.pdf}
    \caption{}\label{fig:3-19}
  \end{figurehere}
\end{minipage}

\medskip
\alertinfo{一般的,在开区间 $(a,b)$ 内的连续函数不一定有最大值、最小值。}
\begin{Practice}
  应用二阶导数求下列函数的极值:
  \begin{tasks}(2)
    \task $f(x)=x^3+3x^2-9x+6$;
    \task $f(x)=x-2\sin x$($0\leqslant x \leqslant 2\uppi $);
    \task $g(x)=ax^2+bx+c$($a\neq 0$)。
  \end{tasks}
\end{Practice}

\subsection{曲线的凸向和拐点}
如果函数 $f(x)$ 的导函数 $f'(x)$ 在点 $x_0$ 处连续,同时 $f'(x_0)>0$(或 $f'(x_0)< 0$),根据连续函数的局部保号性质,则 $f'(x)$ 在点 $x_0$ 附近,必有 $f'(x)>0$(或 $f'(x)< 0$),这表示 $f(x)$ 在点 $x_0$ 附近是增函数(或减函数)。即函数 $f(x)$ 的图象在点 $x_0$ 附近是上升的(或下降的)。

当 $f'(x_0)>0$ 时,对于给定的函数 $f(x)$ 的图象在点 $P\,(x_0,f(x_0))$ 附近上升情况有四种(见\cref{fig:3-20})。
\begin{figure}
  \begin{minipage}{0.24\linewidth}\centering
    \includegraphics{3-20a.pdf}
    \subcaption{}\label{fig:3-20a}
  \end{minipage}
  \begin{minipage}{0.24\linewidth}\centering
    \includegraphics{3-20b.pdf}
    \subcaption{}\label{fig:3-20b}
  \end{minipage}
  \begin{minipage}{0.24\linewidth}\centering
    \includegraphics{3-20c.pdf}
    \subcaption{}\label{fig:3-20c}
  \end{minipage}
  \begin{minipage}{0.24\linewidth}\centering
    \includegraphics{3-20d.pdf}
    \subcaption{}\label{fig:3-20d}
  \end{minipage}
  \caption{}\label{fig:3-20}
\end{figure}

当 $f'(x_0)<0$ 时,函数 $f(x)$ 的图象在点 $P\,(x_0,f(x_0))$ 附近的下降情况也有四种(见\cref{fig:3-21})。

\begin{figure}
  \begin{minipage}{0.24\linewidth}\centering
    \includegraphics{3-21a.pdf}
    \subcaption{}\label{fig:3-21a}
  \end{minipage}
  \begin{minipage}{0.24\linewidth}\centering
    \includegraphics{3-21b.pdf}
    \subcaption{}\label{fig:3-21b}
  \end{minipage}
  \begin{minipage}{0.24\linewidth}\centering
    \includegraphics{3-21c.pdf}
    \subcaption{}\label{fig:3-21c}
  \end{minipage}
  \begin{minipage}{0.24\linewidth}\centering
    \includegraphics{3-21d.pdf}
    \subcaption{}\label{fig:3-21d}
  \end{minipage}
  \caption{}\label{fig:3-21}
\end{figure}

为了能确切地了解函数 $f(x)$ 在点 $P\,(x_0,f(x_0))$ 附近的变化情况,还应研究曲线的凸向与拐点。

设函数 $y=f(x)$ 在点 $x_0$ 处可导,则曲线 $y=f(x)$ 在点 $P\,(x_0,f(x_0))$ 处有切线。
若此切线位于切点附近曲线的下方,切点除外,则称曲线\Concept{在点 $x=x_0$ 处下凸}。
如\cref{fig:3-20a,fig:3-21a}。
若此切线位于切点附近曲线的上方,切点除外,则称曲线\Concept{在点 $x=x_0$ 处上凸}。
如\cref{fig:3-20b,fig:3-21b}。

如果曲线 $y=f(x)$ 在区间 $(a,b)$ 内所有点都下凸(或上凸),则称曲线\Concept{在区间 $(a,b)$ 内下凸(或上凸)}。

若曲线 $y=f(x)$ 在切点 $P\,(x_0,f(x_0))$ 的两侧改变了凸向,即左下凸右上凸,或左上凸右下凸,则称点 $P$ 为曲线的\Concept{拐点}。
如\cref{fig:3-20c,fig:3-21c,fig:3-20d,fig:3-21d}。

下面就给出应用二阶导数判定曲线的凸向和拐点的方法:

\begin{Theorem}{定理 1}
  设函数 $f(x)$ 在区间 $(a,b)$ 内有二阶导数 $f''(x)$,
  \begin{enumerate}
    \item 如果对所有点 $x \in (a,b)$,有 $f''(x)>0$,则曲线 $y=f(x)$ 在区间 $(a,b)$ 内下凸。
    \item 如果对所有点 $x \in (a,b)$,有 $f''(x)<0$,则曲线 $y=f(x)$ 在区间 $(a,b)$ 内上凸。
  \end{enumerate}
\end{Theorem}
\begin{proof}
  \begin{enumerate}
    \item 当 $x \in (a,b)$ 时,有 $f''(x)>0$,于是对于任取一点 $x_0\in(a,b)$,有 $f''(x_0)>0$,现证明曲线 $y=f(x)$ 在点 $x_0$ 处下凸。已知曲线 $y=f(x)$ 在点 $P\,(x_0,f(x_0))$ 处的切线为
    \[ y-f(x_0)=f'(x_0)(x-x_0),\]
    设 $x_1$ 为 $x_0$ 附近的任意一点,$x_1\neq x_0$,则切线上对应于点 $x_1$ 的纵坐标为
    \begin{equation}
      \label{eq:tangent_y1}
      y_1=f(x_0)+f'(x_0)(x_1-x_0).
    \end{equation}
    
    另外,因函数 $f(x)$ 在 $(a,b)$ 内有二阶导数,并且 $x_0,x_1\in (a,b)$,根据\cref{subsec:Preliminary} 定理 2 中\cref{eq:theorem2_other_form} 有
    \begin{equation}
      \label{eq:fx_1}
      f(x_1)=f(x_0)+f'(x_0)(x_1-x_0)+\frac{f''(\xi)}{2}(x_1-x_0)^2,
    \end{equation}
    其中 $\xi$ 在 $x_0$ 与 $x_1$ 之间。

    $f(x_1)$ 为对应于点 $x_1$ 的曲线上 $P_1$ 点的纵坐标(\cref{fig:3-22})。
    \begin{figure}
      \includegraphics{3-22.pdf}
      \caption{}\label{fig:3-22}
    \end{figure}

    由\cref{eq:fx_1} 减\cref{eq:tangent_y1},得
    \begin{equation}
      \label{eq:deltay}
      f(x_1)-y_1=\frac{f''(\xi)}{2}(x_1-x_0)^2.
    \end{equation}

    因为对所有点 $x\in(a,b)$,有 $f''(x)>0$,从而 $f''(\xi)>0$。
    
    由\cref{eq:deltay} 得
    \begin{equation}
      \label{eq:fx1_vs_y1}
      f(x_1)>y_1.
    \end{equation}
    由\cref{eq:fx1_vs_y1} 表明,曲线在点 $P$ 的切线位于曲线的下方,说明曲线 $y=f(x)$ 在点 $x=x_0$ 处下凸。

    因对区间 $(a,b)$ 内任意一点 $x_0$,\cref{eq:deltay} 皆成立,这样曲线 $y=f(x)$ 在区间 $(a,b)$ 内下凸得证。
    \item 当 $x\in(a,b)$,有 $f''(x)<0$ 的情况,利用\cref{eq:deltay},同样可证曲线 $y=f(x)$ 在区间 $(a,b)$ 内上凸。
  \end{enumerate}
\end{proof}

\begin{example}\label{exp:Convexity}
  判定曲线 $y=\dfrac{1}{x}$ 的凸向。
\end{example}
\begin{solution}
  由于 $f(x)=\dfrac{1}{x}$,$f'(x)=-\dfrac{1}{x^2}$,$f''(x)=\dfrac{2}{x^3}$。所以当 $x<0$ 时,$f''(x)<0$;当 $x>0$ 时,$f''(x)>0$。这表明曲线 $y=\dfrac1x$ 在 $(-\infty,0)$ 内为上凸;在 $(0,+\infty)$ 内为下凸。
\end{solution}

从\cref{exp:Convexity} 可知,曲线 $y=\dfrac{1}{x}$ 在点 $x=0$ 的两侧改变了凸向,可是曲线 $y=\dfrac{1}{x}$ 无拐点,因为 $x=0$ 时,$\dfrac{1}{x}$ 无意义。可见只有两侧改变了凸向的点,还不一定是拐点。那么拐点存在还应具备什么条件呢?

\emph{如果点 $P\,(x_0,f(x_0))$ 为曲线 $y=f(x)$ 的拐点,且 $f''(x_0)$ 存在,则横坐标 $x_0$ 必满足:$f''(x_0)=0$}。

事实上,假设 $f''(x_0)\neq 0$,必有 $f''(x_0)>0$ 或 $f''(x_0)<0$,根据定理 1 的证明可知曲线 $y=f(x)$ 在点 $x_0$ 处为下凸,或上凸。这样点 $P\,(x_0,f(x_0))$ 就不能是拐点,与点 $P\,(x_0,f(x_0))$ 为拐点的条件相矛盾。这说明拐点 $P\,(x_0,f(x_0))$ 的横坐标 $x_0$ 必须满足 $f''(x_0)=0$。

\begin{example}
  判定曲线 $y=\sin x$ 在区间 $(-\uppi,\uppi)$ 内的凸向与拐点。
\end{example}
\begin{solution}
  因 $f''(x)=-\sin x$。

  当 $x\in(-\uppi,0)$ 时,$f''(x)=-\sin x>0$,则曲线 $y=\sin x$ 在 $(-\uppi,0)$ 内下凸。

  当 $x\in(0,\uppi)$ 时,$f''(x)=-\sin x<0$,则曲线 $y=\sin x$ 在 $(0,\uppi)$ 内上凸。
  \begin{figure}
    \includegraphics{3-23.pdf}
    \caption{}\label{fig:3-23}
  \end{figure}

  从\cref{fig:3-23} 看出曲线 $y=\sin x$ 在原点 $(0,0)$ 的两侧改变了凸向,点 $(0,0)$ 为曲线的拐点。且 $f''(0)=0$。

  但是,如果只有 $f''(x_0)=0$,那么点 $P\,(x_0,f(x_0))$ 也不一定是拐点。
\end{solution}

\begin{example}
  讨论曲线 $y=x^4-1$ 的凸向与拐点。
\end{example}
\begin{solution}
  因 $f''(x)=12x^2$,且 $f''(0)=0$,当 $x\neq 0$ 时,$f''(x)>0$。

  因此,曲线 $y=x^4-1$ 在 $(-\infty,\infty)$ 内下凸,虽然 $f''(0)=0$,但点 $(0,-1)$ 不是拐点。
\end{solution}

\begin{Theorem}{定理 2}
  设函数 $f(x)$ 在点 $x_0$ 附近有二阶导函数,满足下列条件:
  \begin{enumerate}
    \item $f''(x_0)=0$;
    \item 在 $x=x_0$ 的两侧 $f''(x)$ 变号,
  \end{enumerate}
  则点 $P\,(x_0,f(x_0))$ 必为曲线 $y=f(x)$ 的拐点。
\end{Theorem}

\begin{example}
  判定曲线 $y=x^3-6x^2+9x-1$ 的凸向与拐点。
\end{example}
\begin{solution}
  因 $f'(x)=3x^2-6x+9$,$f''(x)=6x-12=6(x-2)$,所以,$f(x)$ 在 $(-\infty,2)$ 内上凸,在 $(2,\infty)$ 内下凹,点 $(2,1)$ 为拐点(\cref{fig:3-24})。
  \begin{figure}
    \begin{minipage}{0.55\linewidth}
      \begin{tblr}{colspec={*4{X[c]}},hline{2}=0.8pt}
        $x$ & $x<2$ & 2 & $2<x$ \\
        $f''(x)$ & $-$ & 0 & $+$ \\
        $f(x)$ & 上凸 & 1 & 下凸 \\
      \end{tblr}
    \end{minipage}\hfill
    \begin{minipage}{0.4\linewidth}
      \includegraphics{3-24.pdf}
      \caption{}\label{fig:3-24}
    \end{minipage}
  \end{figure}
\end{solution}

\begin{Practice}
  \begin{question}
    \item 判定下列曲线的凸向与拐点:
    \begin{tasks}[before-skip=5pt,after-skip=5pt,after-item-skip=7pt](2)
      \task $f(x)=x^4-2x^3+1$;
      \task $f(x)=x^2+\dfrac{1}{x}$;
      \task $f(x)=x-\sin x$。
    \end{tasks}
    \item 讨论下列曲线的凸向与拐点:
    \begin{tasks}[before-skip=5pt,after-skip=5pt](3)
      \task $f(x)=\upe^{-2x^2}$;
      \task $f(x)=\dfrac{x}{x+1}$;
      \task $f(x)=\ln(x^2+1)$。
    \end{tasks}
  \end{question}
\end{Practice}

\subsection{函数的图像}
设函数 $f(x)$ 已由某个式子给定,在平面解析几何中我们可以利用描点法作出函数的图象,这种图象一般是粗糙的,在一些关键性点的附近函数的变化状态,不一定能确切地反映出来。
现在我们学习了导数及其应用,就可以利用函数的一、 二阶导数及其某些性质,给出较准确地描述函数动态。
一般的,描绘函数图象的步骤如下:
\begin{enumerate}[1.]
  \item 确定函数的定义域,及其某些性质:
  \begin{itemize}
    \item[]由函数的定义域,找出其图象范围;
    \item[]由函数的奇偶性、周期性,缩小研究范围;
    \item[]找出函数图象与两坐标轴的交点。
  \end{itemize}
  \item 计算 $f'(x)$,求方程 $f'(x)=0$ 在研究范围内的所有实根。找出 $f(x)$ 的增减区间、驻点、极值点。
  \item 计算 $f''(x)$,求方程 $f''(x)=0$ 在研究范围内的所有实根,找出曲线 $y=f(x)$ 的凸向区间和拐点。
  \item 计算驻点、拐点及有关点的函数值,列出表格、描绘图象。
\end{enumerate}

\begin{example}
  描绘函数 $y=\dfrac{x}{x^2+1}$ 的图象。
\end{example}
\begin{solution}
  \begin{enumerate}[1.]
    \item 函数 $f(x)$ 的定义域为 $(-\infty,+\infty)$;
    
    又因 $f(-x)=-f(x)$,所以 $f(x)$ 为奇函数,只须研究区间 $[0,\infty)$ 上的图象,再描绘它关于原点的对称图形即得 $(-\infty,+ \infty)$ 上 $f(x)$ 的图象;

另外,因 $x=0$ 时,$y=0$,所以,函数 $f(x)$ 的图象过坐标原点 $(0,0)$。
    \item $f'(x)=\dfrac{1-x^2}{(x^2+1)^2}$,解方程 $f'(x)=0$,得驻点为 $x=\pm 1$;
    
    在 $(-\infty,-1)$、$(1,+\infty)$ 内 $f'(x)<0$,$f(x)$ 为减函数;

在 $(-1,+1)$ 内 $f'(x)>0$,$f(x)$ 为增函数。
    \item $f''(x)= \dfrac{2x(x^2-3)}{(x^2+1)^2}$,解方程 $f''(x)=0$,得根为 $x=-\sqrt{3}$,0,$\sqrt{3}$。
    
    在区间 $(-\infty,-\sqrt{3})$、$(0,\sqrt{3})$ 内 $f''(x)< 0$,$f(x)$ 上凸;

在区间 $(-\sqrt{3},0)$、$(\sqrt{3},+\infty)$ 内 $f''(x)>0$,$f( x)$ 下凸;

在点 $\left(-\sqrt{3},-\dfrac{\sqrt{3}}{4}\right)$、$(0,0)$、$\left(\sqrt{3},\dfrac{\sqrt{3}}{4}\right)$ 处为拐点。
    \item 因 $\lim\limits_{x\to\infty}f(x)=0$,$f(0)=0$,$f(1)=\dfrac{1}{2}$,$f(\sqrt{3})=\dfrac{\sqrt{3}}{4}$。$f'(x)$,$f''(x)$,$f(x)$ 的变化状态如\cref{tab:3-3}:
    \begin{table}
      \caption{$f'(x)$,$f''(x)$,$f(x)$ 的变化状态}\label{tab:3-3}
      \begin{tblr}{colspec={*7{X[c]}},hline{2}=0.8pt}
        $x$      & 0    & $(0,1)$    & 1          & $(1,\sqrt{3})$ & $\sqrt{3}$             & $(\sqrt{3},+\infty)$\\
        $f'(x)$  & 1    & $+$        & 0          & $-$            & $-\dfrac18$            & $-$ \\
        $f''(x)$ & 0    & $-$        & $-$        & $-$            & 0                      & $+$ \\
        $f(x)$   & 0    & $\nearrow$ & $\dfrac12$ & $\searrow$     & $-\dfrac{\sqrt{3}}{4}$ & $\nearrow$ \\
        $y=f(x)$ & 拐点 & 上凸       & 极大值     & 上凸           & 拐点                   & 下凸 \\
      \end{tblr}
    \end{table}
  \end{enumerate}
\end{solution}

根据以上的讨论,函数的图象描绘如下(\cref{fig:3-25})。
\begin{figure}
  \includegraphics{3-25.pdf}
  \caption{}\label{fig:3-25}
\end{figure}

\begin{example}
  描绘函数 $y=\upe^{-\frac{x^2}{2}}$ 的图象。
\end{example}
\begin{solution}
  \begin{enumerate}[1.]
    \item 函数 $f(x)=\upe^{-\frac{x^2}{2}}$ 的定义域为 $(-\infty,+\infty)$;
    
    因为 $f(-x)=f(x)$,所以 $f(x)$ 为偶函数,只研究区间 $[0,\infty)$ 上的图象,再利用它关于 $y$ 轴对称性即得 $(-\infty,+\infty)$ 上的 $f(x)$ 的图象;

    因当 $x=0$ 时,$y=1$,故与 $y$ 轴交点为 $(0,1)$;

    又当 $y=0$ 时,方程 $0=\upe^{-\frac{x2}{2}}$ 无解,故与 $x$ 轴无交点。
    \item $f'(x)=-x\upe^{-\frac{x}{2}}$,解方程 $f'(x)=0$,得驻点为 $x=0$。
    
    在区间 $(0,+\infty)$ 内 $f'(x)<0$,$f(x)$ 为减函数。
    \item $f''(x)=\upe^{-\frac{x^2}{2}}$,解方程 $f''(x)=0$,得根 $x=1$。
    
    在区间 $(0,1)$ 内 $f''(x)<0$,$f(x)$ 上凸;

    在区间 $(1,+\infty)$ 内 $f''(x)>0$,$f(x)$ 下凸;

    点 $\left(1,\dfrac{1}{\sqrt{\upe}}\right)$ 为拐点。
    \item 因 $\lim\limits_{x\to+\infty} f(x)=0$,$f(0)=1$,$f(1)=\dfrac{1}{\sqrt{\upe}}$。$f'(x)$,$f''(x)$,$f(x)$ 的变化状态如\cref{tab:3-4}:
    \begin{table}
      \caption{$f'(x)$,$f''(x)$,$f(x)$ 的变化状态}\label{tab:3-4}
      \begin{tblr}{colspec={*5{X[c]}},hline{2}=0.8pt}
        $x$      &  0     & $(0,1)$ & 1   & $(1,+\infty)$\\
        $f'(x)$  &  0     &   $-$   & $-$ & $-$\\
        $f''(x)$ & $-$    &   $-$   & 1   & $+$\\
        $f(x)$   &  1     & $\searrow$ & $\dfrac{1}{\sqrt{\upe}}$ & $\searrow$\\
        $y=f(x)$ & 极大值 &   上凸  & 拐点 & 下凸 \\
      \end{tblr}
    \end{table}
  \end{enumerate}

  根据以上的讨论,函数的图象描绘如下(\cref{fig:3-26})。
\begin{figure}
  \includegraphics{3-26.pdf}
  \caption{}\label{fig:3-26}
\end{figure}
\end{solution}

\begin{Exercise}
  \begin{question}
    \item 求下列函数的极值:
    \begin{tasks}[before-skip=7pt,after-skip=7pt,after-item-skip=7pt](2)
      \task $f(x)=2x^3-6x^2-18x+7$;
      \task $f(x)=2x^3+6x^2-18x+120$;
      \task $f(x)=x^4-2x^2$;
      \task $f(x)=\dfrac{1}{2}x^2-3x$;
      \task $f(x)=\dfrac{x^2+x+1}{x^2-x+1}$;
      \task $f(x)=\dfrac{x+1}{x^2-2x+1}$。
    \end{tasks}
    \item 确定下列函数的增减区间,并求极值:
    \begin{tasks}[before-skip=7pt,after-skip=7pt,after-item-skip=7pt](2)
      \task $f(x)=2x+\dfrac{1}{x^2}$;
      \task $g(x)=\dfrac{x+1}{\sqrt{x^2+1}}$;
      \task $h(x)=\dfrac{\ln x}{x^2}$。
    \end{tasks}
    \item 讨论下列曲线的凸向与拐点:
    \begin{tasks}[before-skip=7pt,after-skip=7pt,after-item-skip=7pt](2)
      \task $f(x)=3x^2-x^3$;
      \task $f(x)=2x^3-3x^2-36x+25$;
      \task $g(x)=-x^2+2x-1$;
      \task $g(x)=\dfrac{1}{3}x^3-x^2-3x+2$;
      \task $y=x+\dfrac{1}{x}$;
      \task $y=\ln x$;
      \task $y=\dfrac{1}{2}(\upe^x+\upe^{-x})$。
    \end{tasks}
    \item 考察下列函数的增减性和极值,并画出图象:
    \begin{tasks}[before-skip=7pt,after-skip=7pt,after-item-skip=7pt](2)
      \task $f(x)=x^3-3x^2+2$;
      \task $f(x)=\dfrac{1}{3}x^3-\dfrac{1}{2}x^2-2x$;
      \task $y=x^4-2x+10$;
      \task $y=\dfrac{8}{x^2+4}$;
      \task $g(x)=\sqrt{x-1}$。
    \end{tasks}
    \item 求曲线 $y=x^3-3x+3$ 上的切线并与直线 $y=3x$ 平行的切点的坐标。
    \item 求与曲线 $y=x^3+3x^2-5$ 相切,且与直线 $2x-6y-1=0$ 垂直的直线方程。
  \end{question}
\end{Exercise}

\section*{小结}
\begin{enumerate}[C、,itemindent=4.5em]
  \item 本章主要内容是一阶导数与二阶导数的应用。
  \item 中值定理是微分学的基本定理,罗尔定理、拉格朗日定理是导数应用的理论基础。它们在微分学的公式推导、不等式的证明以及利用导数研究函数的性质中有着广泛的应用。
  \item 利用一阶导数可以讨论函数的单调性和极值,导数 $y'$ 符号的变化与函数 $y$ 的增减情况以及极值的关系是:
  
  \begin{tablehere}
    \begin{minipage}{\linewidth}
      \begin{tblr}{colspec={*{4}{X[c]}},hline{2}=0.8pt}
        变量 $x$ & 导数 $y'$ & 函数 $y$ & 极值 \\
        由小变大 & 由正变为负 & { 由增变为减 \\ $\nearrow \quad \searrow$ } & { $y'=0$ 时\\ 函数 $y$ 有极大值}\\
        由大变小 & 由负变为正 & { 由减变为增 \\ $\searrow \quad \nearrow$} & { $y'=0$ 时\\ 函数 $y$ 有极小值}\\
      \end{tblr}
    \end{minipage}
  \end{tablehere}
  \item 从极值与最大、最小值的定义可知: 极值是指某一点附近函数值的比较。因此,同一函数在某一点的极大(小)值,可以比另一点的极小 (大)值小(大);而最大、最小值是指闭区间 $[a,b]$ 上所有函数值的比较。因而在一般情况下,两者是有区别的。极大(小)值不一定是最大(小)值,最大(小)值也不一定是极大 (小)值。但如果连续函数在区间 $(a,b)$ 内只有一个极值,那么极大值就是最大值,极小值就是最小值。
  
  在闭区间 $[a,b]$ 上连续,在开区间 $(a,b)$ 内可导的函数 $f(x)$,它的极值可以通过检查导数 $f'(x)$ 在每一个驻点两旁的符号来求得。
  而 $f(x)$ 在 $[a,b]$ 上的最大(小)值,则可以通过将驻点与端点的函数值加以比较来求得,其中最大(小)的一个即为最大(小)值。

  在生产建设与科学技术中,要求 “用料最省”,“功率最大” “体积最小”等实际问题,一般地,都可以用求函数的最大值与最小值方法来解决。
  \item 利用二阶导数可以判定函数的极值、凸向和拐点,并可描绘函数的图象。
  
  设 $f'(x_0)=0$,若 $f''(x_0)>0$(或 $f''(x_0)<0$),则 $f(x_0)$ 为极小值(或极大值)。

  在区间 $(a,b)$ 内二阶可导函数 $f(x)$,若对所有点 $x \in (a,b)$ 有 $f'(x)>0$(或 $f'(x)<0$),则曲线 $y=f(x)$ 在区间 $(a,b)$ 内下凸 (或上凸)。

如果点 $P\,(x_0,f(x_0))$ 为曲线 $y = f(x)$ 的拐点,且 $f''(x_0)$ 存在,则 $f''(x_0)= 0$。
因此,$f''(x_0)=0$ 为曲线 $y=f(x)$ 有拐点的必要条件。
如果 $f''(x_0)=0$,且在 $x=x_0$ 的两侧 $f'(x)$ 变号,则点 $P\,( x_0,f(x_0))$ 必为曲线 $y=f(x)$ 的拐点。
\end{enumerate}

\chapter*{复习参考题\chinese{chapter}}
\section*{A 组}
\begin{question}
  \item 如果多项式的导函数 $P'(x)=0$ 在 $(-\infty,+\infty)$ 中无实根,求证 $P(x)=0$ 在 $(-\infty,+\infty)$ 中至多有一个实根。
  \item 利用拉格朗日中值定理,证明下列不等式:
  \begin{tasks}[before-skip=7pt,after-skip=7pt,after-item-skip=7pt]
    \task 当 $h>0$ 时,$\dfrac{h}{1+h^2}<\arctan h<h$
    \task 当 $b>a>0$ 时,$\dfrac{b-a}{b}<\ln\dfrac{b}{a}<\dfrac{b-a}{a}$;
    \task 当 $x>0$ 时,$\upe^x>1+x$。
  \end{tasks}
  \item 考察下列函数的增减范围:
  \begin{tasks}[before-skip=7pt,after-skip=7pt,after-item-skip=7pt](2)
    \task $f(x)=x^3-3x-1$;
    \task $f(x)=\dfrac{x^3}{3}-\dfrac{5x^2}{2}+6x+4$;
    \task $f(x)=3x^4+2x^3-3x^2-2$;
    \task $f(x)=3x^4-4x^3-12x^2+15$;
    \task $f(x)=x^3-6x^2+9x+16$;
    \task $f(x)=x^3+3x^2+3x-4$;
    \task $f(x)=5-x^3-12x^2-48x$;
    \task $f(x)=3x+x^3$。
  \end{tasks}
  \item 求下列函数在指定范围内的单调区间:
  \begin{tasks}[before-skip=7pt,after-skip=7pt,after-item-skip=7pt](2)
    \task $y=\sin x,\quad x\in(0,2\uppi)$;
    \task $y=\tan x,\quad x\in\left(-\dfrac{\uppi}{2},\dfrac{\uppi}{2}\right)$;
    \task $y=t+\cos t,\quad t\in(0,2\uppi)$。
  \end{tasks}
  \item 求下列函数的单调区间:
  \begin{tasks}[before-skip=7pt,after-skip=7pt,after-item-skip=7pt](2)
    \task $f(t)=\dfrac{t+2}{t}$;
    \task $g(t)=\dfrac{t-1}{t+1}$;
    \task $h(t)=\dfrac{t^2+1}{t}$;
    \task $h(x)=\dfrac{x}{x^2-9}$;
    \task $f(x)=\dfrac{x^2+1}{x^2-1}$;
    \task $g(x)=\dfrac{x^3-2x+2}{x}$;
    \task $f(t)=\sqrt{t}+t$;
    \task $g(t)=\dfrac{1}{t}-\dfrac{1}{t^2}+3$;
    \task $f(x)=2\sqrt{x}+\dfrac{\sqrt{8}}{x}$;
    \task $h(t)=t^{\frac{3}{2}}+\sqrt{t}$;
    \task $f(x)=x^3-3x^2-105x-2$。
  \end{tasks}
  \item 证明下列不等式:
  \begin{tasks}[before-skip=7pt,after-skip=7pt,after-item-skip=7pt]
    \task 当 $x>0$ 时,$x-\dfrac{x^2}{2}<\ln(1+x)<x$;
    \task 当 $x>0$,$\alpha>1$ 时,$(1+x)^\alpha>1+\alpha x$;
    \task 当 $h>0$ 时,$\dfrac{h}{1+h}<\ln(1+h)<h$。
  \end{tasks}
  \item 求下列函数的极值:
  \begin{tasks}[before-skip=7pt,after-skip=7pt,after-item-skip=7pt](2)
    \task $f(x)=8x^3-12x^2+6x+1$;
    \task $f(x)=x^4-2x^2+3$;
    \task $f(x)=x^3+3x^2-24x+12$;
    \task $f(x)=x^5-5x^4+5x^3+1$;
    \task $h(x)=\dfrac{x}{3}+\dfrac{3}{x}$;
    \task $g(x)=x^2-\dfrac{1}{2}x^4$;
    \task $f(x)=\dfrac{x^2-7x+6}{x-10}$;
    \task $y=x^2+\dfrac{16}{x}$;
    \task $y=x^2+\dfrac{1}{x^2}$;
    \task $y=\dfrac{6x}{x^2+1}$;
    \task $y=\dfrac{x^2+x+1}{x}$。
  \end{tasks}
  \item 求下列函数的最大值与最小值:
  \begin{tasks}[before-skip=7pt,after-skip=7pt,after-item-skip=7pt](2)
    \task $f(x)=\dfrac{4x}{x^2+1}$;
    \task $g(x)=x\sqrt{1-x^2}$。
  \end{tasks}
  \item 求下列函数在给定区间上的最大值与最小值:
  \begin{tasks}[before-skip=7pt,after-skip=7pt,after-item-skip=7pt]
    \task $f(x)=x^5-5x^4+5x^3+1,\quad -1\leqslant x\leqslant 2$;
    \task $g(x)=\dfrac{x-1}{x^2+1},\quad 0\leqslant x\leqslant 4$;
    \task $h(x)=x+\dfrac{1}{x},\quad 0.01\leqslant x\leqslant 100$;
    \task $f(x)=x+2\sqrt{x},\quad 0\leqslant x\leqslant 4$;
    \task $f(x)=\sin2x-x,\quad -\dfrac{\uppi}{2}\leqslant x\leqslant \dfrac{\uppi}{2}$;
  \end{tasks}
  \item 用长为 $2l$ 的线段围成矩形,问长和宽各为多长时矩形面积最大?
  \item 讨论下列曲线的凸向与拐点:
  \begin{tasks}[before-skip=7pt,after-skip=7pt,after-item-skip=7pt](2)
    \task $f(x)=2x^3+3x^2+x+2$;
    \task $f(x)=x^4-6x^2-7$;
    \task $g(x)=2x^4-6x^2+1$;
    \task $g(x)=x^4-4x^3+16$。
  \end{tasks}
  \item 已知曲线 $y=x^3+ax^2-9x+4$ 在 $x=1$ 处有拐点,
  \begin{tasks}
    \task 试确定系数 $a$;
    \task 求曲线的拐点与凸向区间。
  \end{tasks}
  \item 研究下列函数,并作出图象:
  \begin{tasks}[before-skip=7pt,after-skip=7pt,after-item-skip=7pt](2)
    \task $y=8+2x^2-x^4$;
    \task $y=x(x-2)^2$;
    \task $f(x)=x^2+\dfrac{1}{x}$;
    \task $g(x)=\dfrac{(x-1)^2}{x^2+1}$。
  \end{tasks}
\end{question}
\section*{B 组}
\begin{question}[resume]
  \item 试证:
  \begin{tasks}
    \task 方程 $x^3+x-1=0$ 只有一个实根,(提示:用反证法);
    \task 方程 $x^4+3x^2-5x-4=0$ 只有两个实根;
    \task 对任意常数 $c$,在 $[0,1]$ 上,方程
    \[ x^3-3x+c=0\]
    不可能有两个不同的根,(提示:用反证法)。
  \end{tasks}
  \item 证明实系数多项式 $f(x)$ 有重根 $a$ 的充要条件为
  \[ f(a)=0, f'(a)=0.\]
  \item 证明下列不等式:
  \begin{tasks}
    \task 设 $f(x)=\frac{1}{x^n}$($n$ 为正整数),当 $b>a>0$ 时,
    \[ \frac{n}{a^{n+1}}(a-b)<f(b)-f(a)<\frac{n}{b^{n+1}}(a-b).\]
    \task 当 $0<a<c<b$ 时,
    \[ \frac{\ln c-\ln a}{c-a}>\frac{\ln b-\ln c}{b-c}.\]
    \task 如果函数 $f(x)$ 在 $[a,b]$ 上的导数 $f'(x)$ 是有界的,那么函数 $f(x)$ 在 $[a,b]$ 上满足利普希茨 (Lipschitz) 条件,即存在常数 $L$,使对任意两点 $x_1$,$x_2$ 有
    \[|f(x_1)-f(x_2)|\leqslant L|x_1-x_2|,\quad x_1,x_2\in[a,b].\]
  \end{tasks}
  \item 确定下列函数的增减范围:
  \begin{tasks}[before-skip=10pt,after-skip=10pt,after-item-skip=7pt](3)
    \task*(2) $y=x^4+\dfrac{4}{3}x^3-2x^2-4x$;
    \task $y=\sqrt{2x-x^2}$;
    \task $y=\dfrac{2x}{1+x^2}$;
    \task $y=\dfrac{x^2-1}{x}$;
    \task $y=2x^2-\ln x$;
    \task $y=\dfrac{\upe^x}{x}$;
    \task* $y=x-2\sin x$($0\leqslant x\leqslant 2\uppi$)。
  \end{tasks}
  \item 设质点作直线运动,运动规律为
  \[ s=\frac{1}{4}t^4-4t^3+10t^2\quad(t>0).\]
  问:
  \begin{tasks}
    \task 何时速度为 $0$?
    \task 何时作前进($s$ 增加)运动?
    \task 何时作后退($s$ 减少)运动?
  \end{tasks}
  \item 在某化学反应过程中,反应物在时刻 $t$ 的浓度是 $x=x_0\upe^{-kt}$,其中,$x_0$,$k$ 都是正数,问浓度是减少还是增加?
  \item 在指定区间内,证明下列不等式:
  \begin{tasks}
    \task $x\in\left(0,\dfrac{\uppi}{2}\right)$ 时,$\tan x>x- \dfrac{x^3}{3}$;
    \task 当 $0<\alpha<1$,$x\in(1,+\infty)$ 时,$\alpha(x-1)>x^\alpha-1$。
  \end{tasks}
  \item 设函数 $f(x) = a\ln x+bx^2+x$ 在 $x_1=1$ 及 $x_2=2$ 时有极值。试定出 $a$ 与 $b$ 之值,并问 $f(x)$ 在 $x_1$ 与 $x_2$ 是取极大值还是取极小值?
  \item 设函数 $y=\dfrac{ax+b}{x^2+a}$ 在 $x=2$ 时的极大值为 1:
  \begin{tasks}(2)
    \task 确定 $a$、$b$ 之值;
    \task 画出函数的图象。
  \end{tasks}
  \item\label{ex:3t-23} 设抛物线 $y=x^2-1$($x>0$)上点 $P\,(t,t^2-1)$ 的切线与 $x$ 轴、$y$ 轴分别交于 $A$、$B$ 两点,原点为 $O$,\par\noindent
  \begin{minipage}{0.55\linewidth}
  \begin{tasks}
    \task 将 $\triangle OAB$ 的面积用 $t$ 表示出来;
    \task 求面积 $S$ 的最小值;
    \task 求这时 $P$ 点的坐标。
  \end{tasks}
\end{minipage}\hfill
\begin{minipage}{0.4\linewidth}
  \begin{figurehere}
    \includegraphics{3t-23.pdf}
    \caption*{(第 \ref{ex:3t-23} 题图)}
  \end{figurehere}
\end{minipage}
  \item 讨论下列函数的极值及其曲线的凸向与拐点:
  \begin{tasks}(2)
    \task $f(x)=(x+2)^4+2x+1$;
    \task $y(x)=\upe^{\frac{1}{x}}$。
  \end{tasks}
  \item 问 $a$ 和 $b$ 为何值时,点 $(1,3)$ 为曲线 $y=ax^3+bx^2$ 的拐点?
  \item 函数 $y=x^4+ax^3+3ax^2+1$ 的图象有拐点,试求 $a$ 值的范围。
  \item 绘制下列各函数的图象:
  \begin{tasks}[before-skip=10pt,after-skip=10pt,after-item-skip=7pt](2)
    \task $f(x)=\dfrac{1}{5}x^5-4x^2$;
    \task $f(x)=x+\dfrac{4}{x^2}$;
    \task $g(x)=\upe^{-\frac{1}{x}}$;
    \task $g(x)=\dfrac{x^2+3x}{x-1}$。
  \end{tasks}
\end{question}