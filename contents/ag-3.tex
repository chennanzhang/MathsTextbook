\chapter{坐标变换}
\section{平移和旋转}
\subsection{坐标轴的平移}
点的坐标和曲线的方程是对一定的坐标系来说的。
例如,\cref{fig:3-1} 中 $\odot O'$ 的圆心 $O'$,在坐标系 $xOy$ 中的坐标是 $(3,2)$,$\odot O'$ 的方程是 $(x-3)^2+(y-2)^2=5^2$;如果取坐标系 $x'O'y'$($O'x'\parallel Ox,\ O'y'\parallel Oy$),那么在这个坐标系中,它们就分别变成 $(0,0)$ 和 $x^2+y^2=5^2$。
\begin{figure}
  \caption{}\label{fig:3-1}
\end{figure}

这就是说,对于同一点或者同一曲线,由于选取的坐标系不同,点的坐标或曲线的方程也不同。
从上面例子我们看出,把一个坐标系变换为另一个适当的坐标系,可以使曲线的方程简化,便于我们研究曲线的性质。

坐标轴的方向和长度单位都不改变,只改变原点的位置,这种坐标系的变换叫做\Concept{坐标轴的平移}。
简称\Concept{移轴}。

下面研究在平移情况下,同一个点在两个不同的坐标系中坐标之间的关系。

设 $O'$ 在原坐标系 $xOy$ 中的坐标为 $(h,k)$ ,以 $O'$ 为原点平移坐标轴,建立新坐标系 $x'O'y'$。
平面内 任意一点 $M$ 在原坐标系中的坐标为 $(x,y)$,在新坐标系中的坐标为 $(x',y')$,点 $M$ 到 $x$ 轴、$y$ 轴的垂线的垂足分别是 $M_1$、$M_2$。从\cref{fig:3-2} 可以看出,
\begin{figure}
  \caption{}\label{fig:3-2}
\end{figure}


\begin{Practice}
  \begin{question}
    \item 
    \item 
  \end{question}
\end{Practice}
\subsection{利用坐标轴的平移化简二元二次方程}
\begin{Practice}
  \begin{question}
    \item 
    \item 
  \end{question}
\end{Practice}
\begin{Exercise}
  \begin{question}
    \item 
    \item 
    \item 
    \item 
    \item 
  \end{question}
\end{Exercise}
\subsection{坐标轴的旋转}
\begin{Practice}
  \begin{question}
    \item 
    \item 
    \item 
  \end{question}
\end{Practice}
\subsection{利用坐标轴的旋转化简二元二次方程}
\begin{Practice}
  化简下列方程,并画图形:
  \begin{tasks}
    \task $x^2-2xy+y^2=12$;
    \task $4x^2+8xy-2y^2-7=0$。
  \end{tasks}
\end{Practice}
\begin{Exercise}
  \begin{question}
    \item 
    \item 
    \item 
    \item 
    \item 
    \item 
    \item 
  \end{question}
\end{Exercise}

\section{一般二元二次方程的讨论}
\subsection{化一般二元二次方程为标准式}
\begin{Practice}
  \begin{question}
    \item 化简方程 $3x^2-10xy+3y^2+26x-22y+35=0$。
    \item 化简方程 $4xy-3x^2+4=0$,画出它的图形。
  \end{question}
\end{Practice}
\subsection{一般二元二次方程的讨论}
\begin{Practice}
  判别下列方程的类型:
  \begin{question}
    \item $3x^2-7xy+5y^2+x-3y-3=0$;
    \item $5x^2+12xy+5y^2-18x-18y+9=0$;
    \item $2x^2+2xy+y^2+2x+2y-4=0$;
    \item $x^2+2xy+y^2+2x+2y-4=0$。
  \end{question}
\end{Practice}
\begin{Exercise}
  \begin{question}
    \item 
    \item 
    \item 
    \item 
    \item 
    \item 
  \end{question}
\end{Exercise}
\section*{小结}
\begin{enumerate}[C、,itemindent=4.5em]
  \item 
  \item 
  \item 
\end{enumerate}
\chapter*{复习参考题\chinese{chapter}}
\section*{A 组}
\begin{question}
  \item 
  \item 
  \item 
  \item 
  \item 
  \item 
  \item 
  \item 
  \item 
  \item 
  \item 
  \item 
\end{question}
\section*{B 组}
\begin{question}
  \item 
  \item 
  \item 
  \item 
  \item 
\end{question}