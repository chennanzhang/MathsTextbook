\chapter{极限}
\phantomsection\pdfbookmark[1]{极限}{phantomsection1}
\subsection{数列的极限}
我们来考察下面两个数列:
\begin{gather}
  \label{eq:sequence1} 1,\frac{1}{2},\frac{1}{3},\cdots,\frac{1}{n},\cdots\\
  \label{eq:sequence2} \frac{1}{2},\frac{3}{4},\frac{7}{8},\cdots,1-\frac{1}{2^n},\cdots
\end{gather}

为了直观起见,我们把这两个数列中的前几项分别在数轴上表示出来(\cref{fig:1-1}):
\begin{figure}
  \begin{minipage}{\linewidth}\centering
    \subcaption{}\label{fig:1-1a}
  \end{minipage}
  \begin{minipage}{\linewidth}\centering
    \subcaption{}\label{fig:1-1b}
  \end{minipage}
  \caption{}\label{fig:1-1}
\end{figure}

容易看出,当项数 $n$ 无限增大时,数列 \eqref{eq:sequence1} 中的项无限趋近于 0,数列 \eqref{eq:sequence2} 中的项无限趋近于 1。

事实上,在数列 \eqref{eq:sequence1} 中,各项与 0 的差的绝对值如\cref{tab:1-1} 所示。
\begin{table}
  \caption{数列 \eqref{eq:sequence1} 各项与 0 的差的绝对值}\label{tab:1-1}
  \begin{tblr}{colspec={X[c]X[c]X[5,c]},hline{2}={0.8pt},rowsep=2pt}
    项号 & 项 & 这一项与 0 的差的绝对值\\
    1 &  $1$  & $|0-1|=1$ \\
    2 &  $\dfrac{1}{2}$  & $\left|0-\dfrac{1}{2}\right|=\dfrac{1}{2}$ \\
    3 &  $\dfrac{1}{3}$  & $\left|0-\dfrac{1}{3}\right|=\dfrac{1}{3}$ \\
    4 &  $\dfrac{1}{4}$  & $\left|0-\dfrac{1}{4}\right|=\dfrac{1}{4}$ \\
    5 &  $\dfrac{1}{5}$  & $\left|0-\dfrac{1}{5}\right|=\dfrac{1}{5}$ \\
    6 &  $\dfrac{1}{6}$  & $\left|0-\dfrac{1}{6}\right|=\dfrac{1}{6}$ \\
    7 &  $\dfrac{1}{7}$  & $\left|0-\dfrac{1}{7}\right|=\dfrac{1}{7}$ \\
    $\cdots$ &  $\cdots$  & $\cdots$ \\
  \end{tblr}
\end{table}

我们看到,无论预先指定多么小的一个正数 $\varepsilon$,总能在数列 \eqref{eq:sequence1} 中找到这样一项,使得这一项后面的所有项与 0 的差的绝对值都小于 $\varepsilon$。
例如,如果取 $\varepsilon=\dfrac{1}{5}$,那么数列 \eqref{eq:sequence1} 中第 5 项后面所有的项与 0 的差的绝对值都小于 $\varepsilon$。
如果取 $\varepsilon = \dfrac{1}{100}$ ,那么数列 \eqref{eq:sequence1} 中第 100 项后面所有的项与 0 的差的绝对值都小于 $\varepsilon$。
在这种情况下,我们就说数列 \eqref{eq:sequence1} 的极限是 0。

同样,对于数列 \eqref{eq:sequence2},我们也可以列成\cref{tab:1-2}。
\begin{table}
  \caption{数列 \eqref{eq:sequence2} 各项与 0 的差的绝对值}\label{tab:1-2}
  \begin{tblr}{colspec={X[c]X[c]X[5,c]},hline{2}={0.8pt},rowsep=2pt}
    项号 & 项 & 这一项与 1 的差的绝对值\\
    1 &  $\dfrac{1}{2}    $ & $\left|\dfrac{1}{2}    -1\right|=\dfrac{1}{2}=0.5$ \\
    2 &  $\dfrac{3}{4}    $ & $\left|\dfrac{3}{4}    -1\right|=\dfrac{1}{4}=0.25$ \\
    3 &  $\dfrac{7}{8}    $ & $\left|\dfrac{7}{8}    -1\right|=\dfrac{1}{8}=0.125$ \\
    4 &  $\dfrac{15}{16}  $ & $\left|\dfrac{15}{16}  -1\right|=\dfrac{1}{16}=0.0625$ \\
    5 &  $\dfrac{31}{32}  $ & $\left|\dfrac{31}{32}  -1\right|=\dfrac{1}{32}=0.03125$ \\
    6 &  $\dfrac{63}{64}  $ & $\left|\dfrac{63}{64}  -1\right|=\dfrac{1}{64}=0.015625$ \\
    7 &  $\dfrac{127}{128}$ & $\left|\dfrac{127}{128}-1\right|=\dfrac{1}{128}=0.0078125$ \\
    $\cdots$ &  $\cdots$  & $\cdots$ \\
  \end{tblr}
\end{table}

可以看出,如果取 $\varepsilon =0.1$,那么数列 \eqref{eq:sequence2} 中第 3 项后面所有的项与 1 的差的绝对值都小于 $\varepsilon$;如果取 $\varepsilon =0.01$,那么第 6 项后面所有的项与 1 的差的绝对值都小于 $\varepsilon$。
就是说,无论预先指定多么小的一个正数 $\varepsilon$,总能在数列 \eqref{eq:sequence2} 中找到这样一项,使得这一项后面的所有项与 1 的差的绝对值都小于 $\varepsilon$。
这时,我们说数列 \eqref{eq:sequence2} 的极限是 1。

一般地,对于一个无穷数列 $\{a_n\}$,如果存在一个常数 $A$,无论预先指定多么小的正数 $\varepsilon$,都能在数列中找到一项 $a_N$,使得这一项后面所有的项与 $A$ 的差的绝对值都小于 $\varepsilon$(即当 $n>N$ 时,$|a_n-A|<\varepsilon$ 恒成立),就把常数 $A$ 叫做\Concept{数列 $\{a_n\}$ 的极限},记作
\[ \lim_{n\to\infty}a_n=A.\footnotemark[1] \]
\footnotetext[1]{$\lim$ 是拉丁文 limis(极限)一词的前三个字母,一般按英文 limit (极限)一词读音。$\lim\limits_{n\to\infty }a_n=A$ 也可读作 “limit $a_n$ 当 $n$ 趋于无穷大时等于 $A$”。}

这个式子读作 “当 $n$ 趋向于无穷大时,$a_n$ 的极限等于 $A$”。
“$\to$” 表示 “趋向于”,“$\infty$” 表示 “无穷大”,“$n\to\infty$” 表示 “$n$ 趋向于无穷大”,也就是 $n$ 无限增大的意思。

$\lim\limits_{n\to\infty}a_n=A$ 有时也可记作
\[ \text{当}\ n\to\infty\text{ 时,} a_n\to A.\]

从数列极限的定义可以看出,数列 $\{a_n\}$ 以 $A$ 为极限,是指当 $n$ 无限增大时,数列 $\{a_n\}$ 中的项 $a_n$ 无限趋近于常数 $A$。
\begin{example}
  已知数列
  \[1,-\frac{1}{2},\frac{1}{3},-\frac{1}{4},\cdots,(-1)^{n+1}\frac{1}{n},\cdots\]
  \begin{enumerate}
    \item 写出这个数列的各项与 0 的差的绝对值。
    \item 第几项后面所有的项与 0 的差的绝对值都小于 0.1 ?都小于 0.001 ?都小于 0.0003?
    \item 第几项后面所有的项与 0 的差的绝对值都小于任何预先指定的正数 $\varepsilon$?
    \item 0 是不是这个数列的极限?
  \end{enumerate}
\end{example}
\begin{solution}
  这个数列的项在数轴上的表示如\cref{fig:1-2}:
  \begin{figure}
    \caption{}\label{fig:1-2}
  \end{figure}
  \begin{enumerate}
    \item 这个数列的各项与 0 的差的绝对值依次是
    \[ 1,\frac{1}{2},\frac{1}{3},\cdots,\frac{1}{n},\cdots\]
    \item 要使 $\frac{1}{n}<0.1$,只要 $n>10$ 就行了。这就是说,第 10 项后面所有的项与 0 的差的绝对值都小于 0.1。
    
    要使 $\frac{1}{n}<0.001$,只要 $n>1000$ 就行了。这就是说,第 1000 项后面所有的项与 0 的差的绝对值都小于 0.001。

    要使 $\frac{1}{n}<0.0003$,只要 $n>3333\frac{1}{3}$ 就行了。这就是说,第 3333 项后面所有的项与 0 的差的绝对值都小于 0.0003。
    \item 
    \item 
  \end{enumerate}
\end{solution}
\begin{Practice}
  \begin{question}
    \item 
    \item 
  \end{question}
\end{Practice}
\subsection{数列极限的四则运算}
\begin{Practice}
  \begin{question}
    \item 
    \item 
    \item 
  \end{question}
\end{Practice}

\begin{Exercise}
  \begin{question}
    \item 
    \item 
    \item 
    \item 
    \item 
    \item 
    \item 
    \item 
    \item 
    \item 
    \item 
    \item 
    \item 
    \item 
    \item 
  \end{question}
\end{Exercise}

\subsection{函数的极限}

\begin{Practice}
  \begin{question}
    \item 
    \item 
    \item 
    \item 
    \item 
  \end{question}
\end{Practice}
\subsection{函数极限的四则运算法则}
\begin{Practice}
  \begin{question}
    \item 
    \item 
  \end{question}
\end{Practice}

\begin{Exercise}
  \begin{question}
    \item 
    \item 
    \item 
    \item 
  \end{question}
\end{Exercise}

\subsection{函数的连续性}
\begin{Practice}
  \begin{question}
    \item 
    \item 
  \end{question}
\end{Practice}


\begin{Practice}
  \begin{question}
    \item 
    \item 
    \item 
    \item 
  \end{question}
\end{Practice}

\subsection{两个重要的极限}
\begin{Practice}
  \begin{question}
    \item ;
    \item 。
  \end{question}
\end{Practice}

\begin{Exercise}
  \begin{question}
    \item 
    \item 
    \item 
    \item 
    \item 
    \item 
    \item 
  \end{question}
\end{Exercise}

\section*{小结}
\begin{enumerate}[C、,itemindent=4.5em]
  \item 
  \item 
  \item 
  \item 
\end{enumerate}
\chapter*{复习参考题\chinese{chapter}}
\section*{A 组}
\begin{question}
  \item 
  \item 
  \item 
  \item 
  \item 
  \item 
  \item 
  \item 
  \item 
  \item 
  \item 
  \item 
\end{question}
\section*{B 组}
\begin{question}
  \item 
  \item 
  \item 
  \item 
  \item 
  \item 
  \item 
  \item 
  \item 
\end{question}