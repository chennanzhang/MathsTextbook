\chapter{前言}

这一套中学数学实验教材,内容的选取原则是精简实用,教材的处理力求深入浅出,顺理成章,尽量作到使人人能懂,到处有用。

本教材适用于重点中学,侧重在满足学生将来从事理工方面学习和工作的需要。

本教材的教学目的是:使学生切实学好从事现代生产、特别是学习现代科学技术所必需的数学基础知识;通过对数学理论、应用、思想和方法的学习,培养学生运算能力,思维能力,空间想象力,从而逐步培养运用数学的思想和方法去分析和解决实际问题的能力;通过数学的教学和学习,培养学生良好的学习习惯,严谨的治学态度和科学的思想方法,逐步形成辩证唯物主义世界观。

根据上述教学目的,本教材精选了传统数学那些普遍实用的最基础的部分,这就是在理论上、应用上和思想方法上都是基本的、长远起作用的通性、通法。比如,代数中的数系运算律,式的运算,解代数方程,待定系数法;几何中的图形的基本概念和主要性质,向量,解析几何;分析中的函数,极限,连续,微分,积分;概率统计以及逻辑、推理论证等知识。对于那些理论和应用上虽有一定作用,但发展余地不大,或没有普遍意义和实用价值,或不必要的重复和过于繁琐的内容,如立体几何中的空间作图,几何体的体积、表面积计算,几何难题,因式分解,对数计算等作了较大的精简或删减。

全套教材共分六册。第一册是代数。在总结小学所学自然数、小数、分数基础上,明确提出运算律,把数扩充到有理数和实数系。灵活运用运算律解一元一次、二次方程,二元、三元一次方程组,然后进一步系统化,引进多项式运算,综合除法,辗转相除,余式定理及其推论,学到根式、分式、部分分式。第二册是几何。由直观几何形象分析归纳出几何基本概念和基本性质,通过集合术语、简易逻辑转入欧氏推理几何,处理直线形,圆、基本轨迹与作图,三角比与解三角形等基本内容。第三册是函数。数形结合引入坐标,研究多项式函数,指数、对数、三角函数,不等式等。第四册是代数。把数扩充到复数系,进一步加强多项式理论,方程式论,讲线性方程组理论,概率(离散的)统计的初步知识。第五册是几何。引进向量,用向量和初等几何方法综合处理几何问题,坐标化处理直线、圆、锥线,坐标变换与二次曲线讨论,然后讲立体几何,并引进空间向量研究空间解析几何初步知识。第六册是微积分初步。突出逼近法,讲实数完备性,函数,极限,连续,变率与微分,求和与积分。

本教材基本上采取代数、几何、分析分科,初中、高中循环排列的安排体系。教学可按初一、初二代数、几何双科并进,初三学分析,高一、高二代数(包括概率统计)、几何双科并进,高三学微积分的程序来安排。

本教材的处理力求符合历史发展和认识发展的规律,深入浅出,顺理成章。突出由算术到代数,由实验几何到论证几何,由综合几何到解析几何,由常量数学到变量数学等四个重大转折,着力采取措施引导学生合乎规律地实现这些转折,为此,强调数系运算律,集合逻辑,向量和逼近法分别在实现这四个转折中的作用。这样既遵循历史发展的规律,又突出了几个转折关头,缩短了认识过程,有利于学生掌握数学思想发展的脉络,提高数学教学的思想性。

这一套中学数学实验教材是教育部委托北京师范大学、中国科学院数学研究所、人民教育出版社、北京师范学院、北京景山学校等单位组成的领导小组组织“中学数学实验教材编写组”,根据美国加州大学伯克利分校数学系项武义教授的《关于中学实验数学教材的设想》编写的。第一版印出后,由教育部实验研究组和有关省市实验研究组指导在北京景山学校、北京师院附中、上海大同中学、天津南开中学、天津十六中学、广东省实验中学、华南师院附中、长春市实验中学等校试教过两遍,在这个基础上编写组吸收了实验学校老师们的经验和意见,修改成这一版《中学数学实验教材》,正式出版,内部发行,供中学选作实验教材,教师参考书或学生课外读物。在编写和修订过程中,项武义教授曾数次详细修改过原稿,提出过许多宝贵意见。

本教材虽然试用过两遍,但是实验基础仍然很不够,这次修改出版,目的是通过更大范围的实验研究,逐步形成另一套现代化而又适合我国国情的中学数学教科书。在实验过程中,我们热忱希望大家多提意见,以便进一步把它修改好。

\begin{flushright}
    中学数学实验教材编写组\\
    一九八一年三月
\end{flushright}

% \CInscribe[1981-03-01]{中学数学实验教材编写组}






