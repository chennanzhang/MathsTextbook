\chapter{概率}
\phantomsection\pdfbookmark[1]{概率}{probability}
\subsection{随机事件的概率}
在实际生活中,我们会碰到许多事件。
有些事件,例如“在标准大气压下,水的温度达到 \qty{100}{\celsius} 时沸腾”,“抛一石块,下落”等,在一定的条件下是必然要发生的,这种在一定的条件下必然要发生的事件,叫做\Concept{必然事件}。
有些事件,例如“在标准大气压下且温度低于 \qty{0}{\celsius}时,冰融化”,“在常温下,焊锡熔化”等,在一定的条件下是不可能发生的,这种在一定的条件下不可能发生的事件,叫做\Concept{不可能事件}。

此外,还有一些事件,它们在一定的条件下可能发生也可能不发生,例如:某人射击一次,可能中靶,也可能不中靶;掷一枚硬币,可能出现正面,也可能出现反面;检验某件产品,可能合格,也可能不合格;某地五月一日,可能下雨,也可能不下雨等等。
这就是说,“某人射击一次,中靶”,“掷一枚硬币,出现正面”,“检验某件产品,合格”,“某地五月一日,下雨"等事件在一定的条件下是否发生,不能事先确定。
这种在一定的条件下可能发生也可能不发生的事件,叫做\Concept{随机事件}。

随机事件在一次试验\footnote{一次试验就是将事件的条件实现一次。例如对“掷一枚硬币,出现正面”这个事件来说,作一次试验就是将硬币掷一次。}中是否发生虽然不能事先确定,但是在大量重复试验的情况下,它的发生呈现出一定的规律性。

例如,对生产的一批乒乓球进行抽查,结果如\cref{tab:3-1} 所示:
\begin{table}
  \caption{一批乒乓球抽查结果}\label{tab:3-1}
  \begin{tblr}{colspec={X[3,c]*{6}{X[c]}},vline{2}=0.8pt,row{3}={ht=6ex}}
    抽取球数 $n$ & 50 & 100 & 200 & 500 & 1000 & 2000 \\
    优等品数 $m$ & 45 & 92 & 194 & 470 & 954 & 1902 \\
    优等品频率 $\dfrac{m}{n}$ & 0.9 & 0.92 & 0.97 & 0.94 & 0.954 & 0.951 \\
  \end{tblr}
\end{table}

我们看到,当抽查的球数很多时,抽到优等品的频率 $\dfrac{m}{n}$(优等品的个数 $m$ 与抽取的球数 $n$ 的比)接近于常数 0.95,在它附近摆动。

又如,在相同条件下对某种油菜籽进行发芽试验,结果如\cref{tab:3-2} 所示:
\begin{table}
  \caption{某种油菜籽发芽试验结果}\label{tab:3-2}
  \begin{tblr}{colspec={X[c]*{10}{c}},vline{2}=0.8pt,row{3}={ht=6ex}}
    每批试验粒数 $n$ & 2 & 5 & 10 & 70 & 130 & 310 & 700 & 1500 & 2000 & 3000\\
    发芽的粒数 $m$ & 2 & 4 & 9 & 60 & 116 & 282 & 639 & 1339 & 1806 & 2715\\
    发芽的频率 $\dfrac{m}{n}$ & 1 & 0.8 & 0.9 & 0.857 & 0.892 & 0.910 & 0.913 & 0.893 & 0.903 & 0.905\\
  \end{tblr}
\end{table}

我们看到,当试验的油菜籽的粒数很多时,油菜籽发芽的频率接近常数 0.9,在它附近摆动。

一般地,\emph{在大量重复进行同一试验时,事件 $A$ 发生的频率 $\dfrac{m}{n}$ 总是接近于某个常数,在它附近摆动},这时就把这个常数叫做\Concept{事件 $A$ 的概率},记作 $P(A)$\footnote{$P$ 是英文 Probability(概率)的第一个字母。}。根据这个定义,求一个事件的概率的基本方法,是通过大量的重复试验,用这个事件发生的频率近似地作为它的概率。概率从数量上反映了一个事件发生的可能性的大小。在上面的例子中,抽查乒乓球得到优等品的概率是 0.95,就是说,从一批乒乓球中抽取一个,取到优等品的可能性是 95\%;油菜籽发芽的概率是 0.9,就是说,从进行发芽实验的一批油菜籽中任选一粒,它发芽的可能性是 90\%。

由于任何事件 $A$ 发生的次数 $m$ 不会是负数,也不可能大于试验次数 $n$,事件 $A$ 的概率满足
\[\tcbhighmath{0\leqslant P(A)\leqslant 1.}\]

很明显,必然事件的概率是 1,不可能事件的概率是 0。

\begin{Practice}
  \begin{question}
    \item 指出下列事件是必然事件,不可能事件,还是随机事件。
    \begin{tasks}
      \task 如果 $a,b$ 都是实数,那么 $a+b=b+a$;
      \task 从分别标有号数 $1,2,3,4,5,6,7,8,9,10$ 的十张号签中任取一张,得到 4 号签;
      \task 没有水分,种籽发芽;
      \task 某电话总机在一分钟内接到至少 15 次呼唤。
    \end{tasks}
    \item 某射手在同一条件下进行设计,结果如下表所示:
    \begin{tablehere}
      \begin{minipage}{\linewidth}
        \begin{tblr}{colspec={X[4,c]*{6}{X[c]}},vline{2}=0.8pt,row{3}={ht=6ex}}
          射击次数 $n$ & 10 &  20 & 50 &  100 & 200 & 500 \\
          击中靶心次数 $m$ & 8 &  19 & 44 &  92 & 178 & 455 \\
          击中靶心频率 $\dfrac{m}{n}$ &   &    &   &   &   &   \\
        \end{tblr}
      \end{minipage}
    \end{tablehere}
    \begin{tasks}
      \task 计算表中击中靶心的各个概率;
      \task 这个射手射击一次,击中靶心的概率约是多少?
    \end{tasks}
  \end{question}
\end{Practice}

\subsection{等可能性事件的概率}
随机事件的概率,一般可以通过大量重复试验求得其近似值,但对于某些随机事件,也可以不通过重复试验,而只通过对一次试验中可能出现的结果的分析来计算其概率。

例如,掷一枚均匀的硬币,它要么出现正面,要么出现反面,出现这两种结果的可能性是相等的。因此,可以认为出现正面的概率是 $\frac12$,出现反面的概率也是 $\frac12$。这和大量重复试验的结果是一致的。有人做过掷一枚均匀硬币的大量重复试验,结果硬币出现正面的频率总是接近于 $\frac12$,在它附近摆动。其中当掷币 \num{24000} 次时,硬币出现正面 \num{12012} 次,其频率为 0.5005。

又如,有 10 个型号相同的杯子,其中一等品 6 个,二等品 3 个,三等品 1 个。从中任取 1 个,取到各个杯子的可能性是相等的。由于是从 10 个杯子中任取 1 个,共有 10 种等可能的结果。又由于其中有 6 个一等品,从这 10 个杯子中取到一等品的结果有 6 种。因此,可以认为取到一等品的概率是 $\dfrac{6}{10}$。同理,可以认为取到二等品的概率是 $\dfrac{3}{10}$,取到三等品的概率是 $\dfrac{1}{10}$。这和大量重复试验的结果也是一致的。

一般地,\emph{如果一次试验种共有 $n$ 种等可能出现的结果,其中事件 $A$ 包含的结果有 $m$ 种,那么事件 $A$ 的概率 $P(A)$ 是 $\dfrac{m}{n}$}。

\begin{example}\label{exp:3-1}
  先后抛掷两枚均匀的硬币,计算:
  \begin{tasks}
    \task 两枚都出现正面的概率;
    \task 一枚出现正面、一枚出现反面的概率。
  \end{tasks}
\end{example}
\begin{analyze}
  抛掷一枚硬币,可能出现正面或反面这两种结果。因而先后抛掷两枚硬币可能出现的结果数,可根据乘法原理得出。由于硬币是均匀的,所有结果出现的可能性都相等。又在所有等可能的结果中,两枚都出现正面这一事件包含的结果数是可以知道的,从而可以求出这个事件的概率。同样,一枚出现正面、一枚出现反面这一事件包含的结果数是可以知道的,从而也可求出这个事件的概率。
\end{analyze}\par\medskip
\begin{solution}
  由乘法原理,先后抛掷两枚硬币可能出现的结果共有 $2\times 2=4$ 种(\cref{fig:3-1}),且这 4 种结果出现的可能性都相等。
\begin{figure}
  \includegraphics{3-1.pdf}
  \caption{}\label{fig:3-1}
\end{figure}
\begin{enumerate}
  \item 记“抛掷两枚硬币,都出现正面”为事件 $A$,那么在上面 $4$ 种结果中,事件 $A$ 包含的结果有 1 种,因此事件 $A$ 的概率
  \[ P(A)=\frac14.\]

  答:两枚都出现正面的概率是 $\dfrac14$。
  \item 记“抛掷两枚硬币,一枚出现正面、一枚出现反面”为事件 $B$,那么事件 $B$ 包含的结果有 2 种,因此事件 $B$ 的概率
  \[ P(B)=\frac24=\frac12.\]

  答:一枚出现正面、一枚出现反面的概率是 $\dfrac12$。
\end{enumerate}
\end{solution}

\begin{example}
  在 100 件产品种,有 95 件合格品,5 件次品。从中任取 2 件,计算:
  \begin{tasks}
    \task 2 件都是合格品的概率;
    \task 2 件都是次品的概率;
    \task 1 件是合格品,1 件是次品的概率。
  \end{tasks}
\end{example}
\begin{analyze}
  从 100 件产品中任取 2 件可能出现的结果数,就是从 100 个元素中任取 2 个的组合数。由于是任意抽取,这些结果出现的可能性都相等。又由于在所有产品中有 95 件合格品、5 件次品,取到 2 件合格品的结果数,就是从 95 个元素中任取 2 个的组合数;取到 2 件次品的结果数,就是从 5 个元素中任取 2 个的组合数;取到 1 件合格品、1 件次品的结果数,就是从 95 个元素中任取 1 个元素的组合数与从 5 个元素中任取 1 个元素的组合数的积,从而可以分别达到所求各个事件的概率。
\end{analyze}\par\medskip
\begin{solution}
  \begin{enumerate}[itemsep=5pt]
    \item 从 100 件产品中任取 2 件,可能出现的结果共有 $C_{100}^2$ 种,且这些结果出现的可能性都相等。又在 $C_{100}^2$ 种结果中,取到 2 件合格品的结果有 $C_{95}^2$ 种。记“任取 2 件,都是合格品”为事件 $A$,那么事件 $A$ 的概率。
    \[P(A)=\frac{C_{95}^2}{C_{100}^2}=\frac{893}{990}.\]

    答:2 件都是合格品的概率为 $\dfrac{893}{990}$。
    \item 记“任取 2 件,都是次品”为事件 $B$。由于在 $C_{100}^2$ 种结果中,取到 2 件次品的结果有 $C_5^2$ 种,事件 $B$ 的概率
    \[ P(B)=\frac{C_5^2}{C_{100}^2}=\frac{1}{495}.\]

    答:2 件都是次品的概率为 $\dfrac{1}{495}$。
    \item 记“任取 2 件,1 件是合格品、1 件是次品”为事件 $C$。由于在 $C_{100}^2$ 种结果中,取到 1 件合格品,1 件次品的结果有 $C_{95}^1\cdot C_5^1$ 种,事件 $C$ 的概率
    \[P(C)=\frac{C_{95}^1\cdot C_5^1}{C_{100}^2}=\frac{19}{198}.\]

    答:1 件是合格品、1 件是次品的概率为 $\dfrac{19}{198}$。
  \end{enumerate}
\end{solution}

\begin{example}
  某号码锁有 6 个拨盘,每个拨盘上有从 0 到 9 共十个数字。当 6 个拨盘上的数字组成某一个六位数字号码(开锁号码)时,锁才能打开。如果不知道开锁号码,试开一次就把锁打开的概率是多少?
\end{example}
\begin{analyze}
  号码锁每个拨盘上的数字,从 0 到 9 共有十个。6 个拨盘上的各一个数字排在一起,就是一个六位数字号码。根据乘法原理,这种号码共有 $10^6$ 个。由于不知道开锁号码,试开时采用每一个号码的可能性都相等。又开锁号码只有一个,从而可以求出试开一次就把锁打开的概率。
\end{analyze}

\begin{solution}
  号码锁每个拨盘上的数字有 10 种可能的取法,根据乘法原理,6 个拨盘上的数字组成的六位数字号码共有 \num{e6} 个。又试开时采用每一个号码的可能性都相等,且开锁号码只有一个,所以试开一次就把锁打开的概率
  \[ P=\frac{1}{10^6}.\]

  答:试开一次就把锁打开的概率是 $\dfrac{1}{10^6}$。
\end{solution}

\begin{Practice}
  \begin{question}
    \item (口答)在 40 根纤维中,有 12 根的长度超过 \qty{30}{mm}。从中任取 1 根,取到长度超过 \qty{30}{mm} 的纤维的概率是多少?
    \item 在 10 支铅笔中,有 8 支正品和 2 支副品。从中任取 2 支,恰好都取到正品的概率是多少?
    \item 对于\cpageref{exp:3-1}\cref{exp:3-1},有人说,先后抛掷两枚硬币,共出现“两枚都是正面”,“两枚都是反面”,“一枚正面、一枚反面”等 3 种结果,因此,“两枚都出现正面”这一事件的概率是 $\dfrac13$。这种说法错在哪里?
  \end{question}
\end{Practice}

\begin{Exercise}
  \begin{question}
    \item 从生产的一批螺钉中抽取 1000 个进行检查,结果有 4 个是次品。那么从这批螺钉中任取一个螺钉,取到次品的概率约是多少?
    \item 在第 $1,3,5,8$ 路公共汽车都要停靠的一个站(假定这个站只能停靠一辆汽车),有一位乘客等候第 1 路或第 3 路汽车。假定当时各路汽车首先到站的可能性相等,求首先到站正好是这位乘客索要乘的汽车的概率。
    \item 有 100 张已编号的卡片(从 1 号到 100 号),从中任取 1 张,计算:
    \begin{tasks}
      \task 卡片号是奇数的概率;
      \task 卡片号是 7 的倍数的概率。
    \end{tasks}
    \item 一个均匀材料做的正方体玩具,各个面上分别标以数 $1,2,3,4,5,6$。
    \begin{tasks}
      \task 将这个玩具抛掷 1 次,朝上的一面出现奇数的概率是多少?
      \task 将这个玩具抛掷 2 次,朝上的一面的数之和为 7 的概率是多少?
    \end{tasks}
    \item 将一枚硬币连掷 3 次,出现“2 个正面、1 个反面”和“1 个正面、2 个反面”的概率各是多少?
    \item 一个口袋内装有大小相同的 7 个白球和 3 个黑球,从中任意摸出 2 个,得到 1 个白球和 1 个黑球的概率是多少?
    \item 在 7 张数的卡片中,有 4 张正数卡片和 3 张负数卡片。从中任取 2 张作乘法联系,其积为正数的概率是多少?
    \item 某种产品 90 件,其中甲等品 40 件,乙等品 30 件,丙等品 20 件。在运送这些产品的路上损坏了 3 件。如果每件产品被损坏的可能性相同,计算着三等产品中恰好各损坏 1 件的概率。
    \item 在 80 件产品中,有 50 件一等品,20 件二等品,10 件三等品。从中任取 3 件,计算:
    \begin{tasks}
      \task 3 件都是一等品的概率;
      \task 2 件是一等品、1 件是二等品的概率;
      \task 一等品、二等品、三等品各有 1 件的概率。
    \end{tasks}
    \item 一套书共有上、中、下三册,将它们任意列到书架的同一层上去,各册自左至右或自右至左恰好成上、中、下的顺序的概率是多少?
    \item 某城市的电话号码由五个数字组成,每个数字可以是从 0 到 9 这十个数字中的任一个,计算电话号码由五个不同数字组成的概率。
    \item 5 个同学任意站成一排,计算:
    \begin{tasks}
      \task 甲恰好站在正中间的概率;
      \task 甲、乙两人恰好站在两端的概率。
    \end{tasks}
    \item 从数字 $1,2,3,4,5$ 中任取 3 个,组成没有重复数字的三位数,计算:
    \begin{tasks}
      \task 这个三位数是 5 的倍数的概率;
      \task 这个三位数是偶数的概率;
      \task 这个三位数大于 400 的概率。
    \end{tasks}
  \end{question}
\end{Exercise}

\subsection[互斥事件有一个发生的概率]{互斥事件\texorpdfstring{\protect\footnote{有的书上也称为\Concept{互不相容事件}。}}{}有一个发生的概率}
在 10 个乒乓球中,有 7 个一等品,2 个二等品,1 个三等品。
我们把从中任取一个,取处一等品叫做事件 $A$,取出二等品叫做事件 $B$,取出三等品叫做事件 $C$。
我们看到,如果取出的乒乓球是一等品,即事件 $A$ 发生,那么事件 $B$ 就不发生;如果取出的是二等品,即事件 $B$ 发生,那么事件 $A$ 就不发生。
也就是说,事件 $A,B$ 不可能同时发生。
这种不可能同时发生的两个事件叫做\Concept{互斥事件}。
同理,事件 $B,C$ 是互斥事件,事件 $A,C$ 是互斥事件。
换句话说,事件 $A,B,C$ 中,任何两个都是互斥事件。
这时我们说事件 $A,B,C$ 彼此互斥。
一般地,如果事件 $A_1,A_2,\cdots,A_n$ 中任何两个都是互斥事件,那么就说事件 $A_1,A_2,\cdots,A_n$ \Concept{彼此互斥}。

在上面的问题里,因为是任取一个,共有 10 种等可能的取法,其中得到一等品,二等品,三等品的取法分别有 7 种,2 种,1 种,因此,$P(A)=\frac{7}{10}$,$P(B)=\frac{2}{10}$,$P(C)=\frac{1}{10}$。

现在问:“任取一个乒乓球,取出一等品或二等品”这一事件的概率是多少?这一事件,我们记作“$A+B$”。因为不论取出一等品还是二等品,都表示这个事件发生,而得到一等品或二等品的取法共有 $7+2$ 种,所以取出一等品或二等品的概率 $P(A+B)=\frac{7+2}{10}$。由 $\frac{7+2}{10}=\frac{7}{10}+\frac{2}{10}$,我们看到
\begin{equation}
  \label{eq:PaPlusPb}
  \tcbhighmath{P(A+B)=P(A)+P(B).}
\end{equation}
它告诉我们:\emph{如果事件 $A,B$ 互斥,那么事件“$A+B$”发生(即 $A,B$ 中有一个发生)的概率,等于事件 $A,B$ 分别发生的概率的和}。

一般地,\emph{如果事件 $A_1,A_2,\cdots,A_n$ 彼此互斥,那么事件“$A_1+A_2+\cdots+A_n$”发生(即 $A_1,A_2,\cdots,A_n$ 中有一个发生)的概率,等于这 $n$ 个事件分别发生的概率的和},即
\begin{equation}
  \label{eq:SumPn}
  P(A_1+A_2+\cdots+A_n)=P(A_1)+P(A_2)+\cdots+P(A_n).
\end{equation}

\begin{example}
  某地区的年降水量,在 \qtyrange{100}{150}{mm}\footnote{在本章内,$a$~$b$ 表示大于或等于 $a$ 而小于 $b$ 的一个实数范围。}范围内的概率是 0.12,在  \qtyrange{150}{200}{mm} 范围内的概率是 0.25,在 \qtyrange{200}{250}{mm} 范围内的概率是 0.16,在 \qtyrange{250}{300}{mm} 范围内的概率是 0.14。计算年降水量在 \qtyrange{100}{200}{mm} 范围内的概率与在 \qtyrange{150}{300}{mm} 范围内的概率。
\end{example}
\begin{solution}
  记这个地区的年降水量在 \qtyrange{100}{150}{mm},\qtyrange{150}{200}{mm},\qtyrange{200}{250}{mm},\qtyrange{250}{300}{mm} 范围内分别为事件 $A,B,C,D$。这四个事件是彼此互斥的。根据\cref{eq:SumPn},年降水量在 \qtyrange{100}{200}{mm} 范围内的概率
\[ P(A+B)=P(A)+P(B)=0.12+0.25=0.37;\]
年降水量在 \qtyrange{150}{300}{mm} 范围内的概率是
\begin{align*}
  P(B+C+D)&=P(B)+P(C)+P(C)\\
          &=0.25+0.16+0.14=0.55.
\end{align*}

答:年降水量在 \qtyrange{100}{200}{mm} 范围内的概率为 0.37,在 \qtyrange{150}{300}{mm} 范围内的概率为 0.55。
\end{solution}

\begin{example}\label{exp:3-5}
  在 20 件产品中,有 15 件一级品,5 件二级品。从中任取 3 件,其中至少有 1 件为二级品的概率是多少?
\end{example}
\begin{solution}
  记从 20 件产品中任取 3 件,其中恰有 1 件二级品为事件 $A_1$,恰有 2 件二级品为事件 $A_2$,3 件全是二级品为事件 $A_3$。这样,事件 $A_1,A_2,A_3$ 的概率分别是
\begin{align*}
  P(A_1)&=\frac{C_5^1\cdot C_{15}^2}{C_{30}^2}=\frac{105}{228};\\[7pt]
  P(A_2)&=\frac{C_5^2\cdot C_{15}^1}{C_{30}^2}=\frac{30}{228};\\[7pt]
  P(A_3)&=\frac{C_5^3}{C_{30}^2}=\frac{2}{228}.
\end{align*}

根据题意,事件 $A_1,A_2,A_3$ 彼此互斥。由\cref{eq:SumPn},3 件产品中至少有 1 件为二级品的概率是 
\begin{align*}
  P(A_1+A_2+A_3)&=P(A_1)+P(A_2)+P(A_3)\\
          &=\frac{105}{228}+\frac{30}{228}+\frac{2}{228}=\frac{137}{228}.
\end{align*}

答:其中至少有一件为二级品的概率是 $\dfrac{137}{228}$。
\end{solution}

\bigskip
在\cref{exp:3-5} 中,从 20 件产品中任取 3 件,或者都是一级品,或者不都是一级品(即其中至少有一件是二级品),这两个互斥事件必有一个发生。这种其中必有一个发生的两个互斥事件叫做\Concept{对立事件}。

一个事件 $A$ 的对立事件通常记作 $\bar{A}$。根据对立事件的意义,$A+\bar{A}$ 是一个必然事件,它的概率等于 1。又由于 $A$ 与 $\bar{A}$ 互斥,我们得到
\begin{equation}
  \label{eq:PaPlusPabar}
  \tcbhighmath{P(A)+P(\bar{A})=P(A+\bar{A})=1.}
\end{equation}
这就是说,\emph{两个对立事件的概率和等于 1}。

从\cref{eq:PaPlusPabar} 还可得到
\begin{equation}
  \label{eq:PaPlusPabar2}
  P(\bar{A})=1-P(A).
\end{equation}

运用\cref{eq:PaPlusPabar2} 计算事件的概率,有时比较简便。如\cref{exp:3-5} 还可以这样来解:

从 20 件产品中任取 3 件,3 件全是一级品(记作事件 $A$)的概率:
\[ P(A)=\frac{C_{15}^3}{C_{30}^3}=\frac{91}{228},\]
由于“任取 3 件,至少有 1 件为二级品”是事件 $A$ 的对立事件 $\bar{A}$,根据\cref{eq:PaPlusPabar2},
\[ P(\bar{A})=1-P(A)=1-\frac{91}{228}=\frac{137}{228}.\]


\begin{Practice}
  \begin{question}
    \item 判别下列每对事件是不是互斥事件,如果是,再判别它们是不是对立事件。

    从一堆产品(其中正品与次品都多于 2 个)中任取 2 件,其中:
    \begin{tasks}
      \task 恰有 1 件次品和恰有 2 件次品;
      \task 至少有 1 件次品和全是次品;
      \task 至少有 1 件正品和至少有 1 件次品;
      \task 至少有 1 件次品和全是正品。
    \end{tasks}
    \item 在某一时期内,一条河流某处的年最高水位在各个范围内的概率如下:
    \begin{tablehere}
      \begin{minipage}{\linewidth}
        \begin{tblr}{colspec={c*{5}{X[c]}},hline{2}=0.8pt}
          年最高水位 & 低于 \qty{10}{m} & \qtyrange{10}{12}{m} & \qtyrange{12}{14}{m} & \qtyrange{14}{16}{m} & 不低于 \qty{16}{m} \\
          概率 & 0.1 & 0.28 & 0.38 & 0.16 & 0.08 \\
        \end{tblr}
      \end{minipage}
    \end{tablehere}
    计算在同一时期内,河流这一处的年最高水位在下列范围内的概率:
    \begin{tasks}(3)
      \task \qtyrange{10}{16}{m};
      \task 低于 \qty{12}{m};
      \task 不低于 \qty{14}{m}。
    \end{tasks}
  \end{question}
\end{Practice}

\subsection{相互独立事件同时发生的概率}
甲坛子里有 6 个白球,4 个黑球,乙坛子里有 3 个白球,6 个黑球,从这两个坛子里分别摸出一个,它们都是白球的概率是多少?

我们把“从甲坛子里摸一个球,得到白球”叫做事件 $A$,把“从乙坛子里莫一个球,得到白球”叫做事件 $B$。很明显,从一个坛子里摸出的是白球还是黑球,对从另一个坛子里摸出白球的概率没有影响。这就是说,事件 $A$(或 $B$)是否发生对事件 $B$(或 $A$)发生的概率没有影响,这样的两个事件叫做\Concept{相互独立事件}。

在上面的问题里,事件 $\bar{A}$ 是指“从甲坛子里摸一个球,得到黑球”,事件 $\bar{B}$ 是指“从乙坛子里摸一个球,得到黑球”。很明显,事件 $A$ 与 $\bar{B}$,$\bar{A}$ 与 $B$,$\bar{A}$ 与 $\bar{B}$ 也都是相互独立的。一般地,如果事件 $A$ 与 $B$ 相互独立,那么 $A$ 与 $\bar{B}$,$\bar{A}$ 与 $B$,$\bar{A}$ 与 $\bar{B}$ 也都是相互独立的。

“从两个坛子里分别摸出一个,都是白球”是一个事件,它的发生,就是事件 $A,B$ 同时发生,我们将它记作“$A\cdot B$”。于是,这里的问题就是要求相互独立事件 $A,B$ 同时发生的概率 $P(A\cdot B)$。

从甲坛子里摸出一个球,有 10 种等可能的结果;从乙坛子里摸出一个球,有 8 种等可能的结果。于是,从两个坛子里分别摸出一个球,共有 $10\times 8$ 种等可能的结果,其中同时摸出白球的结果有 $6\times 3$ 种。因此,从两个坛子里分别摸出一个球,都是白球的概率 $P(A\cdot B)=\dfrac{6\times 3}{10\times 8}=\dfrac{6}{10}\times\dfrac38$。

另一方面,从甲坛子里摸出一个球,得到白球的概率 $P(A)$ 为 $\dfrac{6}{10}$,从乙坛子里摸出一个球,得到白球的概率 $P(B)$ 为 $\dfrac38$,于是我们看到
\begin{equation}
  \label{eq:PaDotPb}
  P(A\cdot B)=P(A)\cdot P(B).
\end{equation}
这就是说,\emph{两个相互独立事件同时发生的概率,等于每个事件发生的概率的积}。

一般地,\emph{如果事件 $A_1,A_2,\cdots,A_n$ 相互独立,那么这 $n$ 个事件同时发生的概率,等于每个事件发生的概率的积},即
\begin{equation}
  \label{eq:ProdP}
  P(A_1\cdot A_2\cdot\cdots\cdot A_n)=P(A_1)\cdot P(A_2)\cdot\cdots\cdot P(A_n).
\end{equation}

\begin{example}
  甲、乙两人各进行一次射击,如果两人击中目标的概率都是 0.6,计算:
  \begin{tasks}
    \task 两人都击中目标的概率;
    \task 其中恰有一人击中目标的概率;
    \task 至少有一人击中目标的概率。
  \end{tasks}
\end{example}
\begin{analyze}
  甲、乙两人各射击一次,甲(或乙)是否击中,对乙(或甲)击中的概率是没有影响的,也就是说,“甲射击一次,击中目标”与“乙射击一次,击中目标”是相互独立事件。根据\cref{eq:PaDotPb},可以求出这两个事件同时发生的概率。同理可以分别求出,甲击中与乙未击中,甲未击中与乙击中,甲未击中与乙未击中同时发生的概率,从而可以得到所求的各个事件的概率。
\end{analyze}\par\medskip
\begin{solution}
  \begin{enumerate}
    \item 记“甲射击一次,击中目标”为事件 $A$,“乙射击一次,击中目标”为事件 $B$。因此,“两人各射击一次,都击中目标”就是事件 $A\cdot B$。又由题意可知,事件 $A$ 与 $B$ 相互独立。根据\cref{eq:PaDotPb},所求的概率是
    \begin{align*}
      P(A\cdot B)&=P(A)\cdot P(B) \\
      &=0.6\times0.6=0.36.
    \end{align*}

    答:两人都击中目标的概率是 0.36。
    \item “两人各射击一次,恰有一人击中目标”包括两种情况:一种是甲击中、乙未击中(事件 $A\cdot\bar{B}$ 发生),另一种是甲未击中、乙击中(事件 $\bar{A}\cdot B$ 发生)。根据题意,这两种情况在各射击一次时不可能同时发生,即事件 $A\cdot\bar{B}$ 与 $\bar{A}\cdot B$ 互斥,根据\cref{eq:PaPlusPb,eq:PaDotPb},所求的概率是
    \begin{align*}
      &P(A\cdot \bar{B})+P(\bar{A}\cdot B)\\
      ={}&P(A)\cdot P(\bar{B})+P(\bar{A})\cdot P(B)\\
      ={}&0.6\times(1-0.6)+(1-0.6)\times 0.6\\
      ={}&0.24+0.24\\
      ={}&0.48.
    \end{align*}

    答:其中恰有一人击中目标的概率是 0.48。
    \item 解法一:“两人各射击一次,至少有一人击中目标”的概率
    \begin{align*}
      P&=P(A\cdot B)+[P(A\cdot\bar{B})+P(\bar{A}\cdot B)]\\
       &=0.36+0.48=0.84.
    \end{align*}

    解法二:两人都未击中目标的概率是
    \begin{align*}
      P(\bar{A}\cdot\bar{B})&=(\bar{A})\cdot P(\bar{B})=(1-0.6)\times(1-0.6)\\
       &=0.4\times 0.4=0.16.
    \end{align*}
    因此,至少有一人击中目标的概率
    \[P=1-P(\bar{A}\cdot\bar{B})=1-0.16=0.84.\]

    答:至少有一人击中目标的概率是 0.84。
  \end{enumerate}
\end{solution}

\begin{example}
  在一段线路中并联着三个自动控制的常开开关,只要其中有一个开关能够闭合,线路就能正常工作。假定在某段时间内每个开关能够闭合的概率都是 0.7,计算在这段时间内线路正常工作的概率。
\end{example}
\par\medskip\noindent
\begin{minipage}{0.65\linewidth}\parindent2em
\begin{analyze}
  根据题意,这段时间内线路正常工作的概率,就是三个开关中至少有一个能闭合的概率,也就是三个开关都不能闭合的对立事件的概率。由于这段时间内三个开关是否能闭合相互之间没有影响,三个开关都不能闭合的概率可根据\cref{eq:ProdP} 求出,从而可得到所求的概率。
\end{analyze}
\end{minipage}\hfill
\begin{minipage}{0.3\linewidth}
  \begin{figurehere}
    \includegraphics{3-2.pdf}
    \caption{}\label{fig:3-2}
  \end{figurehere}
\end{minipage}
\par\bigskip
\begin{solution}
  分别记这段时间内开关 $J_A,J_B,J_C$ 能够闭合为事件 $A,B,C$(\cref{fig:3-2})。根据题意,这段时间内至少有一个开关能够闭合,从而使线路能正常工作的概率是
\begin{align}
     &1-P(\bar{A}\cdot\bar{B}\cdot\bar{C}) \notag \\
  ={}&1-P(\bar{A})\cdot P(\bar{B})\cdot P(\bar{C}) \tag{$\bar{A},\bar{B},\bar{C}$ 相互独立} \\
  ={}&1-[1-P(A)][1-P(B)][1-P(C)] \notag\\
  ={}&1-(1-0.7)(1-0.7)(1-0.7) \notag\\
  ={}&1-0.3^3 \notag\\
  ={}&0.973 \notag
\end{align}

答:在这段时间内线路正常工作的概率是 0.973。
\end{solution}
 
\begin{Practice}
  \begin{question}
    \item 一个口袋内装有 2 个白球和 2 个黑球,把“从中任意摸出一个球,得到白球”叫做事件 $A$,把“从剩下的 3 个球中任意摸出一个球,得到白球”叫做事件 $B$。在先摸出白球后,再摸出白球的概率是多少?在先摸出黑球后,再摸出白球的概率是多少?这里事件 $A$ 与事件 $B$ 是相互独立的吗?
    \item 生产一种零件,甲车间的合格率是 96\%,乙车间的合格率是 97\%,从它们生产的零件中各抽取一件,都抽到合格品的概率是多少?
    \item 有一问题,在半小时内,甲能解决它的概率是 $\dfrac12$,乙能解决它的概率是 $\dfrac13$,如果两人都试图独立地在半小时内解决它,计算:
    \begin{tasks}
      \task 两人都未解决的概率;
      \task 问题得到解决的概率。
    \end{tasks}
    \item 某射手射击一次,击中目标的概率是 0.9。他连续射击 4 次,且各次射击是否击中相互之间没有影响,那么他第 2 次未击中、其他 3 次都击中的概率是多少?
  \end{question}
\end{Practice}

\subsection{独立重复试验}
某射手射击一次,击中目标的概率是 0.9,他射击 4 次恰好击中 3 次的概率是多少?

分别记在第 $1,2,3,4$ 次射击中,这个射手击中目标为事件 $A_1,A_2,A_3,A_4$,未击中目标为事件 $\bar{A}_1,\bar{A}_2,\bar{A}_3,\bar{A}_4$。那么,射击 4 次、击中 3 次共有下面 4 种情况:
\[ A_1A_2A_3\bar{A}_4,\quad A_1A_2\bar{A}_3A_4,\quad A_1\bar{A}_2A_3A_4,\quad \bar{A}_1A_2A_3A_4.\]

上述每一种情况,都可看成是在 4 个位置上取出 3 个写上 $A$,另一个写上 $\bar{A}$,所以这些情况的种数等于从 4 个元素中取出 3 个的组合数 $C_4^3$,即 4 种。

由于各次射击是否击中相互之间没有影响,根据\cref{eq:ProdP},前 3 次击中、第 4 次未击中的概率
\begin{align*}
  P(A_1\cdot A_2\cdot A_3\cdot \bar{A}_4)&=P(A_1)\cdot P(A_2)\cdot P(A_3)\cdot P(\bar{A}_4) \\
  &= 0.9\times 0.9\times 0.9\times (1-0.9)\\
  &=0.9^3\times(1-0.9)^{4-3} 
\end{align*}
同理,
\begin{align*}
  P(A_1\cdot A_2\cdot \bar{A}_3\cdot A_4)&=P(A_1\cdot \bar{A}_2\cdot A_3\cdot A_4) \\
  &= P(\bar{A}_1\cdot A_2\cdot A_3\cdot A_4)\\
  &=0.9^3\times(1-0.9)^{4-3} 
\end{align*}

这就是说,在上面射击 4 次、击中 3 次的 4 种情况中,每一种发生的概率都是 $0.9^3\times(1-0.9)^{4-3}$。因为这 4 种情况彼此互斥,根据\cref{eq:SumPn},射击 4 次、击中 3 次的概率
\begin{align*}
  P={}&P(A_1\cdot A_2\cdot A_3\cdot \bar{A}_4)+P(A_1\cdot A_2\cdot \bar{A}_3\cdot A_4)\\
  &+P(A_1\cdot \bar{A}_2\cdot A_3\cdot A_4)+P(\bar{A}_1\cdot A_2\cdot A_3\cdot A_4)\\
  ={}&C_4^3\times 0.9^3\times(1-0.9)^{4-3}=4\times 0.9^3\times 0.1\approx 0.29. 
\end{align*}

在上面的例子里,4 次射击可以看成是进行 4 次独立重复试验。

一般地,\emph{如果在一次试验中某事件发生的概率是 $P$,那么在 $n$ 次独立重复试验中这个事件恰好发生 $k$ 次的概率}
\begin{equation}
  \label{eq:Pnk}
  \tcbhighmath{P_n(k)=C_n^kP^k(1-P)^{n-k}.}
\end{equation}

\begin{example}
  某气象站天气预报的准确率为 80\%,计算(结果保留两个有效数字):
  \begin{tasks}
    \task 5 次预报中恰有 4 次准确的概率;
    \task 5 次预报中至少有 4 次准确的概率。
  \end{tasks}
\end{example}
\begin{solution}
  \begin{enumerate}
    \item 记“预报一次,结果准确”为事件 $A$。预报 5 次相当于作 5 次独立重复试验,根据\cref{eq:Pnk},5 次预报中恰有 4 次准确的概率
    \begin{align*}
      P_5(4)&=C_5^4\times 0.8^4\times (1-0.8)^{5-4}\\
            &=5\times 0.8^4\times 0.2\approx 0.41.
    \end{align*}

    答:5 次预报中恰好有 4 次准确的概率约为 0.41。
    \item 5 次预报中至少有 4 次准确的概率,就是 5 次预报中恰好有 4 次准确的概率与 5 次预报都准确的概率的和,即
    \begin{align*}
      P&=P_5(4)+P_5(5)\\
       &=C_5^4\times 0.8^4\times (1-0.8)^{5-4}+C_5^5\times 0.8^5\times (1-0.8)^{5-5}\\
       &=5\times 0.8^4\times 0.2+0.8^5\\
       &\approx 0.410+0.328\\
       &\approx 0.74.
    \end{align*}

    答:5 次预报中至少有 4 次准确的概率约为 0.74。
  \end{enumerate}
\end{solution}

\begin{Practice}
  \begin{question}
    \item 生产一种零件,出现次品的概率是 0.04。生产这种零件 4 件,其中恰有 1 件次品,恰有 2 件次品,至多有 1 件次品的概率各是多少?
    \item 在本节开始关于射手 4 次射击的问题中,分别写出射手恰好击中 4 次,3 次,2 次,1 次,0 次的概率的计算式子,并将它们与 $(0.9+0.1)^4$ 的展开式的各项进行比较。
  \end{question}
\end{Practice}

\begin{Exercise}
  \begin{question}
    \item 从一批乒乓球产品中任取一个,如果其重量小于 \qty{2.45}{g} 的概率是 0.22,重量不小于 \qty{2.50}{g} 的概率是 0.20,那么重量在 \qtyrange{2.45}{2.50}{g} 范围内的概率是多少?
    \item 某射手在一次射击中射中 10 环,9 环,8 环的概率分别为 0.24,0.28,0.19,计算这个射手在一次射击中:
    \begin{tasks}
      \task 射中 10 环或 9 环的概率;
      \task 不够 8 环的概率。
    \end{tasks}
    \item 一个箱子内有 9 张票,其号数分别为 $1,2,\cdots,9$。从中任取 2 张,其号数至少有 1 个为奇数的概率是多少?
    \item 在 50 件产品中有 45 件合格品,5 件次品。从中任取 3 件,计算其中有次品的概率。
    \item 从某地区的儿童中预选体操学员。已知这些儿童体型合格的概率为 $\dfrac15$,身体关节构造合格的概率为 $\dfrac14$。从中任挑一个儿童,这个儿童体型和身体关节构造都合格的概率是多少(假定体型与身体关节构造合格与否相互之间没有影响)?
    \item 甲、乙两个气象台同时作天气预报,如果它们预报准确的概率分别是 0.8 与 0.7,那么在一次预报中两个气象台都预报准确的概率是多少?
    \item 将一个硬币连掷 5 次,5 次都出现正面的概率是多少?
    \item 制造一种零件,甲机床的废品率是 0.04,乙机床的废品率是 0.05。从它们制造的产品中各任抽意见,其中恰有一件废品的概率是多少?
    \item 电子设备的某一部件由 9 个元件组成。如果其中有任何一个元件损坏了,这个部件就不能工作。假定每个元件能使用 \qty{3000}{h} 的概率是 0.99,计算这个部件能工作 \qty{3000}{h} 的概率。
    \item 一个工人负责看管四台机床,如果在一小时内这些机床不需要人去照顾的概率,第一台是 0.79,第二台是 0.79,第三台是 0.80,第四台是 0.81。假设个台机床是否需要照顾相互之间没有影响,计算在这个小时内,这四台机床都不需要人去照顾的概率。
    \item 有甲、乙、丙三批罐头,每批 100 个,其中各有 1 个是不合格的。从三批罐头中各抽出 1 个,计算:
    \begin{tasks}
      \task 3 个中恰有一个不合格的概率;
      \task 3 个中至少有 1 个不合格的概率。
    \end{tasks}
    \item 一次测量中出现正误差和负误差的概率都是 $\dfrac12$,在 3 次测量中,恰好出现 2 次正误差的概率是多少?恰好出现 2 次负误差的概率是多少?
    \item 一头病牛服用某药品后被治愈的概率是 95\%,计算符用这种药的 4 头病牛中至少有 3 头被治愈的概率。
    \item 某一批蚕豆种籽,如果每一粒发芽的概率为 90\%,播下 5 粒种籽,计算:
    \begin{tasks}
      \task 其中恰好有 4 粒发芽的概率;
      \task 其中恰好有 2 粒未发芽的概率。
    \end{tasks}
  \end{question}
\end{Exercise}

\section*{小结}
\begin{enumerate}[C、,itemindent=4.5em]
  \item 在这一章中,我们初步介绍了事件的概率的概念及其计算。
  \item 随机事件在现实世界中是广泛存在的。在一次试验中,事件是否发生虽然带有偶然性,但在大量重复试验下,它的发生呈现出一定的规律性,即事件发生的频率总是接近于某个常数,在它附近摆动,这个常数就叫做这一事件的概率。
  \item 对于某些事件,也可以直接通过分析来计算其概率。如果一次试验中共有 $n$ 种等可能出现的结果,其中事件 $A$ 包含的结果有 $m$ 种,那么事件 $A$ 的概率 $P(A)$ 是 $\dfrac{m}{n}$。
  \item 不可能同时发生的两个事件叫做互斥事件。当 $A,B$ 是互斥事件时,
  \[ P(A+B)=P(A)+P(B).\]
  
  如果一个事件是否发生对另一个事件发生的概率没有影响,那么这两个事件叫做相互独立事件。当 $A,B$ 是相互独立事件时,
  \[ P(A \cdot B)=P(A)\cdot P(B).\]
  
  应注意上面两个概念的区别,注意运用上面两个公式的前提条件。

  其中必有一个发生的两个互斥事件叫做对立事件。当 $A,B$ 是对立事件时,$P(B)=1-P(A)$。利用这个公式,常可使概率的计算简化。
  \item 如果事件 $A$ 在一次试验中发生的概率是 $P$,那么它在 $n$ 次独立重复试验中恰好发生 $k$ 次的概率是
  \[P_n(k)=C_n^kP^k(1-P)^{n-k}.\]
\end{enumerate}
\chapter*{复习参考题\chinese{chapter}}
\section*{A 组}
\begin{question}
  \item 把 5 本不同的书任意列到书架的同一层上,计算其中固定的 3 本书放在中间的概率。
  \item 用数字 $1,2,3,5,8$ 任意组成没有重复数字的五位数,计算:
  \begin{tasks}
    \task 它是奇数的概率;
    \task 它小于 \num{23000} 的概率。
  \end{tasks}
  \item 有五根细木棍,它们的长度分别为 \qtylist{1;3;5;7;9}{cm},从中任取三根,它们能搭成一个三角形的概率是多少?
  \item 同时抛掷两个均匀的正方体玩具(各个面上分别标以数 $1,2,3,4,5,6$),计算:
  \begin{tasks}
    \task 朝上的一面的数相同的概率;
    \task 朝上的一面数之积为偶数的概率。
  \end{tasks}
  \item 甲击中目标的概率是 0.5,乙击中目标的概率是 0.4,两人各射击一次,“目标被击中的概率是 $0.5+0.4=0.9$”这种说法对不对?为什么?
  \item 某售货员负责在三个柜面上售货。如果在某一小时内柜面不需要售货员照顾的概率,第一柜面是 0.9,第二柜面是 0.8,第三柜面是 0.7。假定各个柜面是否需要照顾相互之间没有影响,计算在这个小时内,至少有一个柜面需要售货员照顾的概率。
  \item 某仪表内装有 $m$ 个同样的电子元件,其中任一个电子元件损坏时,这个仪表就不能工作。如果在某段时间内每个电子元件损坏的概率是 $P$,计算在这段时间内,这个仪表不能工作的概率。
  \item 两个篮球运动员在罚球线投球的命中率分别是 0.7 与 0.6,每人投球 3 次,计算两人都恰好投进 2 球的概率。
\end{question}
\section*{B 组}
\begin{question}[resume]
  \item 某人的口袋内装有 1 枚伍分的应比、2 枚贰分的硬币以及 3 枚壹分的硬币,他从中任取 3 枚,取出的总钱数不少于伍分的概率是多少?
  \item 8 个篮球队中有 2 个强队,先任意将这 8 个队分成两个组(每组 4 个队)进行比赛,这两个强队被分在一个组内的概率是多少?
  \item 在一副扑克牌(52 张)中,有“黑桃,红心,梅花,方块”这四种花色的牌各 13 张。从中任取 4 张,这 4 张牌的花色相同的概率是多少?这 4 张牌的花色各不相同的概率是多少?
  \item 甲袋子内有 $m$ 个白球,$n$ 个黑球,乙袋子内有 $n$ 个白球,$m$ 个黑球。从两个袋子内各任意摸出一个,得到一个白球、一个黑球的概率是多少?
  \item 某种大炮击中目标的概率是 0.3,只要以多少门这样的大炮同时射击一次,就可以使击中目标的概率超过 95\%。
  \item 一个通讯小组有两套通讯设备,只要其中有一套设备能正常工作,就能进行通讯。每套设备由 3 个部件组成,只要其中有一个部件出故障,这套设备就不能正常工作。如果在某段时间内每个部件不出故障的概率都是 $P$,计算在这段时间内能进行通讯的概率。
\end{question}