\chapter{定积分及其应用}
\section{定积分的概念和计算}

\subsection{定积分的概念}
\begin{Practice}
  \begin{question}
    \item 
    \item 
  \end{question}
\end{Practice}

\subsection{微积分基本公式}
\begin{Practice}
  计算定积分:
  \begin{tasks}(2)
    \task
    \task
    \task
    \task
    \task
    \task
    \task
    \task
    \task
    \task
  \end{tasks}
\end{Practice}

\begin{Exercise}
  计算定积分:
  \begin{tasks}(2)
    \task
    \task
    \task
    \task
    \task
    \task
    \task
    \task
    \task
    \task
    \task
    \task
    \task
    \task
  \end{tasks}
\end{Exercise}

\section{定积分的应用}
\subsection{平面图形的面积}
\begin{Practice}
  求下列曲线围成的图形的面积:
  \begin{tasks}
    \task 
    \task 
    \task 
    \task 
    \task 
  \end{tasks}
\end{Practice}

\subsection{旋转体的体积}
\begin{Practice}
  \begin{question}
    \item 
    \item 
    \item 
  \end{question}
\end{Practice}

\subsection*{平面曲线的弧长}
\begin{Practice}
  \begin{question}
    \item 
    \item 
  \end{question}
\end{Practice}

\subsection*{旋转体的侧面积}
\begin{Practice}
  求曲线 $y^2=x$,直线 $x=0$,$x=6$ 所围图形绕 $x$ 轴旋转所得旋转体的侧面积。
\end{Practice}

\begin{Exercise}
  \begin{question}
    \item 
    \item 
    \item 
    \item 
    \item 
    \item 
    \item 
    \item 
  \end{question}
\end{Exercise}

\section*{小结}
\begin{enumerate}[C、,itemindent=4.5em]
  \item 本章主要内容是定积分的概念、计算及其简单应用。
  \item 定积分的概念是从求曲边梯形的面积、变速直线运动的路程等实际问题引入的。解决这类问题都是通过分割,取近似,最后归结为求一种和式的极限:
  \[ \lim_{n \to \infty}\sum_{i=1}^{n} f(\xi_i)\Delta x. \]
  (其中 $f(x)$ 为区间 $[a,b]$ 上的连续函数,把区间 $[a,b]$ $n$ 等分后,$\Delta x= \dfrac{b-a}{n}$,而 $\xi_i$ 是第 $i$ 个小区间上的任意一点)。这个极限叫做函数 $f(x)$ 在区间 $[a,b]$ 上的定积分,记作
  \[\int_a^b f(x)\dif x=\lim_{n\to\infty}\sum_{i=1}^{n} f(\xi_i)\Delta x.\]
  \item 微积分基本公式是
  \[\int_a^b f(x) \dif x= F(b)-F(a)\]
  其中 $F(x)$ 是函数 $f(x)$ 的任一原函数,即 $F'(x)=f(x)$,就是说,函数 $f(x)$ 在区间 $[a,b]$ 上的定积分 $\int_a^b f(x)\dif x$,等于它的任一原函数 $F(x)$ 在区间 $[a,b]$ 上的改变量 $F(b)-F(a)$。
  这个公式是定积分与原函数之问的关系式,它使定积分的计算大为简化。
  \item 定积分的一些简单应用:
  \begin{enumerate}[1.]
    \item 求曲边梯形的面积,公式是
    \[S=\int_a^b f(x)\dif x;\]
    \item 求旋转体的体积,公式是
    \[V=\uppi\int_a^b [f(x)]^2\dif x;\]
    \item 求平面曲线弧长,公式是
    \[l=\int_a^b \sqrt{1+[f'(x)]^2}\dif x;\]
    \item 求旋转体的侧面积,公式是
    \[S=2\uppi\int_a^b f(x)\sqrt{1+[f'(x)]^2}\dif x.\]
  \end{enumerate}
\end{enumerate}

\chapter*{复习参考题\chinese{chapter}}
\section*{A 组}
\begin{question}
  \item 计算定积分:
  \begin{tasks}(2)
    \task $\displaystyle \int_0^a (3x^2-x+1) \dif x$;
    \task $\displaystyle \int_1^2 \left(x^2+\frac{1}{x^4}\right) \dif x$;
    \task $\displaystyle \int_2^4 \frac{x^3-3x^2+5}{x^2} \dif x$;
    \task $\displaystyle \int_1^3 y^2(y-2) \dif y$;
    \task $\displaystyle \int_{-1}^{1} x(x-3) \dif x$;
    \task $\displaystyle \int_{-2}^{2} (6x^3+x+1) \dif x$;
    \task $\displaystyle \int_{-1}^{1} \left(x^2-\frac{x}{x^2+1}\right) \dif x$;
    \task $\displaystyle \int_{0}^{\frac{1}{3}} \frac{1}{4-3x} \dif x$;
    \task $\displaystyle \int_{\frac{1}{2}}^{1} \sqrt{3-2x} \dif x$.
  \end{tasks}
  \item 计算定积分:
  \begin{tasks}(2)
    \task $\displaystyle \int_0^{\uppi} \sqrt{1-\cos2x} \dif x$;
    \task $\displaystyle \int_{\frac{\uppi}{6}}^{\frac{\uppi}{2}} \cos^2u \dif u$;
    \task $\displaystyle \int_0^{\frac{\uppi}{4}} \tan^2\theta \dif \theta$;
    \task $\displaystyle \int_{\frac{\uppi}{3}}^{\frac{2\uppi}{3}} (2\sin x+\cos x) \dif x$;
    \task $\displaystyle \int_0^{\frac{\uppi}{2}} \sin\varphi\cos^2\varphi \dif \varphi$;
    \task $\displaystyle \int_0^4 \frac{1}{1+\sqrt{x}} \dif x$;
    \task $\displaystyle \int_0^{e-1} \ln(x+1) \dif x$;
    \task $\displaystyle \int_0^1 xe^x \dif x$。
  \end{tasks}
  \item 求下列各曲线围成的图形的面积:
  \begin{tasks}
    \task 曲线 $y=x^3$,$y=x^2$,直线 $x=1$,$x=2$;
    \task 曲线 $y=\sin x$,$y=\cos x$,直线 $x=-\frac{\uppi}{4}$,$x=\frac{\uppi}{4}$;
    \task 曲线 $y=\frac{1}{x}$,直线 $y=x$,$x=2$,$y=0$;
    \task 曲线 $y=x^2$,直线 $y=x$,$y=2x$;
    \task 曲线 $y=x^2-4x+5$,直线 $x=3$,$x=5$,$y=0$;
    \task 曲线 $y=3-2x-x^2$,$y=0$。
  \end{tasks}
  \item 求下列曲线所围图形绕 $x$ 轴旋转而成的旋转体体积:
  \begin{tasks}
    \task $y=x^3$,$x=2$,$y=0$;
    \task $y=\cos x$,$x=-\frac{\uppi}{4}$,$x=\frac{\uppi}{4}$,$y=0$;
    \task $xy=4$,$x=1$,$x=4$,$y=0$;
    \task $x^2-y^2=a^2$,$x=a+h$,($a>0,h>0$);
    \task $y=1+\sqrt{x}$,$y=3$,$x=0$。
  \end{tasks}
  \item 求曲线 $y=\dfrac{x^2}{2}-2$ 与 $x$ 轴交点间的曲线弧长。
  \item 将立方抛物线 $a^2y=x^3$ 由 $x=0$ 到 $x=a$ 的一段弧,绕 $x$ 轴旋转一周,求旋转面的面积。
  \item 星形线 $x^{\frac{2}{3}}+y^{\frac{2}{3}}=a^{\frac{2}{3}}$ 绕 $x$ 轴旋转一周,求所得曲面面积。
\end{question}
\section*{B 组}
\begin{question}[resume]
  \item 计算定积分:
  \begin{tasks}(2)
    \task $\displaystyle \int_0^{2a} (x-a)^3 \dif x$;
    \task $\displaystyle \int_{-2}^{0} x^3(x-a)^2 \dif x$;
    \task $\displaystyle \int_{-a}^{0} \left(\frac{x+a}{a}\right)^2 \dif x$;
    \task $\displaystyle \int_{-\uppi}^{\uppi} \sin 2x\sin 4x \dif x$。
  \end{tasks}
  \item 求抛物线 $y=-x^2+4x-3$ 及其在点 $A\,(0,-3)$ 与点 $B\,(3,0)$ 处的切线所围图形的面积。
  \item \label{exec:t-10}如图,已知曲线方程 $y^2=x^2(1-x^2)$,求图中阴影部分的面积。
  \begin{figurehere}
    \begin{minipage}{\linewidth}\centering
      \caption*{第 \ref{exec:t-10} 题}
    \end{minipage}
  \end{figurehere}
  \item 过椭圆 $\frac{x^2}{5}+y^2=1$ 的两个焦点作 $x$ 轴的垂线,将椭圆的夹在这两条垂线间的部分与这两条垂线及 $x$ 轴所围曲边梯形绕 $x$ 轴旋转,求得到的旋转体的体积。
  \item 求曲线 $9ay^2=x(x-3a)^2$ 由 $x=0$ 到 $x=3a$ 的弧长。
  \item 求 $x^2+(y-b)^2=a^2$($b>a$)绕 $x$ 轴旋转所成的旋转体的表面积。
\end{question}