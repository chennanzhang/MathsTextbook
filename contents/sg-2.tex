\chapter{多面体和旋转体}
\section{多面体}
\subsection{棱柱}
\subsubsection{棱柱的概念和性质}
我们常见的一些物体,例如三棱镜、方砖以及螺杆的头部,它们都呈棱柱的形状(\cref{fig:2-1})。
\begin{figure}
  \includegraphics{2-1.pdf}
  \caption{}\label{fig:2-1}
\end{figure}

\noindent
\begin{minipage}{0.7\linewidth}\parindent2em%
有两个面互相平行,其余各面都是四边形\footnotemark,并且每相邻两个四边形的公共边都互相平行,由这些面所围成的几何体叫做\Concept{棱柱}(\cref{fig:2-2}),两个互相平行的面叫做\Concept{棱柱的底面},其余各面叫做\Concept{棱柱的侧面}。两个侧面的公共边叫做棱柱的侧棱,侧面与底面的公共顶点叫做棱柱的顶点,不在同一个面上的两个顶点的连线叫做棱柱的对角线,两个底面间的距离叫做棱柱的高。如\cref{fig:2-2} 中的棱柱,多边形 $ABCDE$ 和 $A'B'C'D'E'$ 是底面,四边形 $ABB'A'$、$BCC'B'$ 等是侧面,$A'A$、$B'B$ 等是侧棱,$H'H$ 是高。
\end{minipage}\hfill
\begin{minipage}{0.28\linewidth}
  \begin{figurehere}
    \includegraphics{2-2.pdf}
    \caption{}\label{fig:2-2}
  \end{figurehere}
\end{minipage}\par\medskip
\footnotetext[1]{本章所说的多边形,一般包括它内部的平面部分。}

棱柱用表示底面各顶点的字母来表示,如\cref{fig:2-2} 中的棱柱,记作棱柱 $ABCDE$-$A'B'C'D'E'$,或者用表示一条对角线端点的两个字母来表示,例如,棱柱 $AC'$。

侧棱不垂直于底面的棱柱叫做\Concept{斜棱柱}(\cref{fig:2-3a});侧棱垂直于底面的棱柱叫做\Concept{直棱柱}(\cref{fig:2-3b})。底面是正多边形的直棱柱叫做\Concept{正棱柱}(\cref{fig:2-3c})。
\begin{figure}
  \begin{minipage}[b]{0.32\linewidth}\centering
    \includegraphics{2-3a.pdf}
    \subcaption{}\label{fig:2-3a}
  \end{minipage}
  \begin{minipage}[b]{0.32\linewidth}\centering
    \includegraphics{2-3b.pdf}
    \subcaption{}\label{fig:2-3b}
  \end{minipage}
  \begin{minipage}[b]{0.32\linewidth}\centering
    \includegraphics{2-3c.pdf}
    \subcaption{}\label{fig:2-3c}
  \end{minipage}
  \caption{}\label{fig:2-3}
\end{figure}

棱柱的底面可以是三角形、四边形、五边形、……,我们把这样的棱柱分别叫做\Concept{三棱柱}(\cref{fig:2-3a})、\Concept{四棱柱}(\cref{fig:2-3b})、\Concept{五棱柱}(\cref{fig:2-3c})、……。

根据棱柱的定义,容易得到棱柱的一些性质:
\begin{enumerate}
  \item 侧棱都相等,侧面是平行四边形;
  \item 两个底面与平行于底面的截面是全等的多边形(\cref{fig:2-4a});
  \item 过不相邻的两条侧棱的截面是平行四边形(\cref{fig:2-4b})。
\end{enumerate}

\begin{figure}
  \begin{minipage}{0.4\linewidth}\centering
    \includegraphics{2-4a.pdf}
    \subcaption{}\label{fig:2-4a}
  \end{minipage}
  \begin{minipage}{0.4\linewidth}\centering
    \includegraphics{2-4b.pdf}
    \subcaption{}\label{fig:2-4b}
  \end{minipage}
  \caption{}\label{fig:2-4}
\end{figure}

\begin{Practice}
  \begin{question}
    \item 求证:{\em 直棱柱的侧棱长与高相等,侧面及经过不相邻的两条侧棱的截面都是矩形。}
    \item 有一个侧面是矩形的棱柱是不是直棱柱?有两个相邻侧面是矩形的棱柱呢?为什么? 
    \item 斜棱柱、直棱柱和正棱柱的底面、侧面各有什么特点?
  \end{question}
\end{Practice}



\subsubsection{长方体}
现在研究四棱柱的特殊情形。
\begin{figure}
  \begin{minipage}[b]{0.24\linewidth}\centering
    \includegraphics{2-5a.pdf}
    \subcaption{}\label{fig:2-5a}
  \end{minipage}
  \begin{minipage}[b]{0.24\linewidth}\centering
    \includegraphics{2-5b.pdf}
    \subcaption{}\label{fig:2-5b}
  \end{minipage}
  \begin{minipage}[b]{0.24\linewidth}\centering
    \includegraphics{2-5c.pdf}
    \subcaption{}\label{fig:2-5c}
  \end{minipage}
  \begin{minipage}[b]{0.24\linewidth}\centering
    \includegraphics{2-5d.pdf}
    \subcaption{}\label{fig:2-5d}
  \end{minipage}
  \caption{}\label{fig:2-5}
\end{figure}

底面是平行四边形的四棱柱叫做\Concept{平行六面体}(\cref{fig:2-5a})。
侧棱与底面垂直的平行六面体叫做\Concept{直平行六面体}(\cref{fig:2-5b})。
底面是矩形的直平行六面体叫做\Concept{长方体}(\cref{fig:2-5c})。
棱长都相等的长方体叫做\Concept{正方体}(\cref{fig:2-5d})。

长方体的对角线有下面的性质:
\begin{Theorem}{定理}
  长方体一条对角线长的平方等于一个定点上三条棱的长的平方和。
\end{Theorem}
已知:长方体 $AC'$ 中,$B'D$ 是一条对角线(\cref{fig:2-6})。

求证:$B'D^2=AB^2+BC^2+BB'^2$。

\medskip\noindent
\begin{minipage}{0.6\linewidth}\parindent2em
\begin{proof}
  连结 $BD$。
  \begin{align*}
    \because \quad B'B&\perp BD,\\
    \therefore \quad B'D^2&=BD^2+BB'^2.\\
    \text{又}\ \because \quad BD^2&=AB^2+AD^2\\
    &=AB^2+BC^2,\\
    \therefore\quad B'D^2&=AB^2+BC^2+BB'^2.
  \end{align*}
\end{proof}
\end{minipage}\hfill
\begin{minipage}{0.35\linewidth}\centering
\begin{figurehere}
  \includegraphics{2-6.pdf}
  \caption{}\label{fig:2-6}
\end{figurehere}
\end{minipage}\par\medskip

\begin{example}
  长方体的一条对角线与一个顶点上的三条棱所成的角分别是 $\alpha$、$\beta$、$\gamma$。求证:
  \[\cos^2\alpha+\cos^2\beta+\cos^2\gamma=1.\]
\end{example}
\par\smallskip\noindent
\begin{minipage}{0.6\linewidth}\parindent2em
\begin{proof}
连结 $AB'$、$CB'$、$DB'$,则 $\triangle B'DA$、$\triangle B'DC$、$\triangle B'DD'$ 都是直角三角形(\cref{fig:2-7})。因此
\[\cos\alpha=\frac{DA}{DB'},\ \cos\beta=\frac{DC}{DB'},\ \cos\gamma=\frac{DD'}{DB'}.\]
将上面三个等式的两边平方后相加,得
\[\cos^2\alpha+\cos^2\beta+\cos^2\gamma=\frac{DA^2+DC^2+DD'^2}{DB'^2}.\]
\end{proof}
\end{minipage}\hfill
\begin{minipage}{0.35\linewidth}\centering
  \begin{figurehere}
    \includegraphics{2-7.pdf}
    \caption{}\label{fig:2-7}
  \end{figurehere}
\end{minipage}\par\medskip
又 $\because\quad DB'^2=DA^2+DC^2+DD'^2$,

$\therefore\quad \cos^2\alpha+\cos^2\beta+\cos^2\gamma=1$。

\begin{Practice}
  \begin{question}
    \item 平行六面体的各个面是什么样的四边形?直平行六面体、长方体、正方体呢?
    \item 解答:
    \begin{enumerate}[itemindent=2.4em]
      \item 长方体是直四棱柱,直四棱柱是不是长方体?
      \item 正方体是正四棱柱,正四棱柱是不是正方体?
    \end{enumerate}
    \item 四棱柱集合、平行六面体集合、直平行六面体集合、长方体集合、正方体集合之间有怎样的包含关系?用图表示出来。
  \end{question}
\end{Practice}


\subsubsection{直棱柱直观图的画法}
前面已经研究过水平放置的平面图形的直观图的画法。几何体的直观图的画法规则,与平面图形的画法相比,知识多画一个与 $x$ 轴、$y$ 轴都垂直的 $z$ 轴,并且\emph{平行于 $z$ 轴的线段的平行性和长度都不变}。在直观图上,平面 $x'O'y'$ 表示水平平面,平面 $y'O'z'$ 和 $z'O'x'$ 表示直立平面。

我们以正六棱柱为例,说明直棱柱的直观图的画法。

\begin{solution}[画法]
  \begin{enumerate}
    \item 画轴\quad 画 $x'$ 轴、$y'$ 轴、$z'$ 轴,使 $\angle x'O'y'=\ang{45}$(或 \ang{135}),$\angle x'O'z'=\ang{90}$(\cref{fig:2-8a})。
    \begin{figure}
      \begin{minipage}{0.45\linewidth}\centering
        \includegraphics{2-8a.pdf}
        \subcaption{}\label{fig:2-8a}
      \end{minipage}
      \begin{minipage}{0.45\linewidth}\centering
        \includegraphics{2-8b.pdf}
        \subcaption{}\label{fig:2-8b}
      \end{minipage}
      \caption{}\label{fig:2-8}
    \end{figure}
    \item 画底面\quad 按 $x'$ 轴、$y'$ 轴,画正六边形的直观图 $ABCDEF$。
    \item 画侧棱\quad 过 $A$、$B$、$C$、$D$、$E$、$F$ 各点分别作 $z'$ 轴的平行线,并在这些平行线上分别截取 $AA'$、$BB'$、$CC'$、$DD'$、$EE'$、$FF'$ 都等于侧棱长。
    \item 成图\quad 顺次连结 $A'$、$B'$、$C'$、$D'$、$E'$、$F'$,并加以整理(去掉辅助线,将被遮挡的部分改为虚线),就得到正六棱柱的直观图(\cref{fig:2-8b})。
  \end{enumerate}
\end{solution}

\subsubsection{直棱柱的侧面积}
\par\medskip\noindent
\begin{minipage}{0.5\linewidth}\parindent2em
把棱柱的侧面沿一条侧棱剪开后展在一个平面上,展开图的面积就是棱柱的侧面积。

直棱柱的侧面展开图是矩形(\cref{fig:2-9}),这个矩形的长等于直棱柱底面周长 $c$,宽等于直棱柱的高 $h$,由此我们得到下面的定理:
\end{minipage}\hfill
\begin{minipage}{0.45\linewidth}\centering
\begin{figurehere}
  \includegraphics{2-9.pdf}
  \caption{}\label{fig:2-9}
\end{figurehere}
\end{minipage}

\begin{Theorem}{定理}
  如果直棱柱的底面周长是 $c$,高是 $h$,那么它的侧面积是
  \[\tcbhighmath{S_\text{直棱柱侧}=ch.}\]
\end{Theorem}

棱柱的全面积等于侧面积与两底面面积的和。

\begin{example}
  求证:斜棱柱的侧面积等于它的直截面(垂直于侧棱的截面)的周长与侧棱长的乘积。

  已知:如\cref{fig:2-10},斜棱柱 $AC'$ 的侧棱长是 $l$,直截面 $HKLMN$ 的周长是 $c_1$。

  求证:$S_\text{斜棱柱侧}=c_1l$。
\end{example}

\medskip\noindent
\begin{minipage}{0.55\linewidth}\parindent2em
\begin{proof}
  延长侧棱 $AA'$ 到 $H'$,使 $A'H'=AH$。设过 $H'$ 平行于直截面 $HKLMN$ 的平面,与各侧棱的延长线交于 $K'L'M'N'$。这样,就得到一个以斜棱柱的直截面为底,侧棱长为高的直棱柱 $HL'$(\cref{fig:2-10})。

因为底面 $H'L'\parallel\text{底面}\ HL$,它们的公垂线段 $HH'=KK'=LL'=\cdots=NN'=AA'=l$,所以,斜棱柱 $AC'$ 的各侧面的面积与直棱柱 $HL'$ 中对应的侧面面积相等。因此
\[ S_\text{斜棱柱侧}=S_\text{直棱柱侧}=c_1\cdot HH'.\]
即 $S_\text{斜棱柱侧}=c_1l$。
\end{proof}
\end{minipage}\hfill
\begin{minipage}{0.4\linewidth}
\begin{figurehere}
  \includegraphics{2-10.pdf}
  \caption{}\label{fig:2-10}
\end{figurehere}
\end{minipage}\par\bigskip

实际上,在本题的证明中,是把斜棱柱 $AC'$ 的直截面下面的一部分,移动一下位置,与另一部分组成直棱柱 $HL'$。

\begin{Practice}
  \begin{question}
    \item 画一个底面边长是 \qty{3}{cm},高是 \qty{4.5}{cm} 的正三棱柱的直观图(不写画法)。
    \item 已知正六棱柱的高为 $h$,底面边长为 $a$,求全面积。
  \end{question}
\end{Practice}

\begin{Exercise}
  \begin{question}
    \item\label{exec:7-1}用较厚的纸按照图的样子画好剪下,再把它折起来粘好,做成棱柱的模型(选做其中一个)。
    \item\label{exec:7-2}底面是菱形的直棱柱,对角线 $B'D$ 和 $A'C$ 的长分别是 \qty{9}{cm} 和 \qty{15}{cm},侧棱 $AA'$ 的长是 \qty{5}{cm}。求它的底面边长。
    \begin{figurehere}
      \begin{minipage}{\linewidth}\centering
        \includegraphics{ex7-1.pdf}
        \caption*{(第~\ref{exec:7-1}~题图)}
      \end{minipage}
    \end{figurehere}
    
    \begin{figurehere}
      \begin{minipage}[b]{0.45\linewidth}\centering
        \includegraphics{ex7-2.pdf}
        \caption*{(第~\ref{exec:7-2}~题图)}
      \end{minipage}
      \begin{minipage}[b]{0.45\linewidth}\centering
        \includegraphics{ex7-3.pdf}
        \caption*{(第~\ref{exec:7-3}~题图)}
      \end{minipage}
    \end{figurehere}
    \item\label{exec:7-3}如图,正三棱柱的底面边长是 \qty{4}{cm},过 $BC$ 的一个平面与底面成 \ang{30} 的二面角,交侧棱 $AA'$ 于 $D$。求 $AD$ 的长和截面 $\triangle BCD$ 的面积。
    \item 求证:
    \begin{enumerate}[itemindent=2.4em]
      \item 平行六面体的各对角线交于一点,并且在这一点互相平分;
      \item 对角线相等的平行六面体是长方体。
    \end{enumerate}
    \item 有一个长方体,它的三个面的对角线长分别是 $a,b,c$。求它的对角线长。
    \item 长方体的一条对角线与各个面所成的角分别是 $\alpha,\beta,\gamma$。求证:$\cos^2\alpha+\cos^2\beta+\cos^2\gamma=2$。
    \item 已知一个正五棱柱的高是 \qty{4}{cm},底面外接圆的半径是 \qty{2.5}{cm},画它的直观图(不写画法)。
    \item 求证:直三棱柱的两个侧面的面积的和,大于第三个侧面的面积。
    \item 直平行六面体的底面是菱形,过不相邻的两对侧棱的截面的面积是 $Q_1$ 和 $Q_2$,求它的侧面积。
    \item 一个长方体的三条棱长的比是 $1:2:3$,全面积是 \qty{88}{cm^2}。求这三条棱的长。
    \item 除锈滚筒是正六棱柱形(两端是封闭的),筒长 \qty{1.6}{m},底面外接圆半径是 \qty{0.46}{m},制造这个滚筒需要多少平方米铁板?(精确到 \qty{0.1}{m^2})
    \item 在正三棱柱的一条侧棱上,取距离等于 $a$ 的两点,过这两点作两个与所有的侧棱都相交的互相平行的截面,如果正三棱柱的底面边长为 $b$,求棱柱的侧面夹在这两个平行截面间的部分的面积。
    \item\label{exec:7-13}如图,正方体的棱长为 $a$,$C,D$ 分别是两条棱的中点。
    \begin{enumerate}[itemindent=2.4em]
      \item 试证 $A,B,C,D$ 在同一平面内;
      \item 求截面 $ABCD$ 的面积。
    \end{enumerate}
    \begin{figurehere}
      \begin{minipage}{\linewidth}\centering
        \includegraphics{ex7-13.pdf}
        \caption*{(第~\ref{exec:7-13}~题图)}
      \end{minipage}
    \end{figurehere}
  \end{question}
\end{Exercise}

\subsection{棱锥}
\subsubsection{棱锥的概念和性质}
帆布帐篷、金字塔(\cref{fig:2-11})等物体,都给我们以棱锥的形象。
\begin{figure}
  \begin{minipage}[b]{0.7\linewidth}
    \includegraphics[height=3.7cm]{2-11a.jpg}
    \includegraphics[height=3.7cm]{2-11b.jpg}
    \caption{}\label{fig:2-11}
  \end{minipage}
  \begin{minipage}[b]{0.28\linewidth}
    \includegraphics{2-12.pdf}
    \caption{}\label{fig:2-12}
  \end{minipage}
\end{figure}

有一个面是多边形,其余各面是有一个公共顶点的三角形,由这些面所围成的几何体叫做\Concept{棱锥}(\cref{fig:2-12}),这个多边形叫做\Concept{棱锥的底面},其余各面叫做\Concept{棱锥的侧面}。相邻侧面的公共边叫做\Concept{棱锥的侧棱},各侧面的公共顶点叫做\Concept{棱锥的顶点},顶点到底面的距离叫做\Concept{棱锥的高}。如\cref{fig:2-12} 中的棱锥,多边形 $ABCDE$ 是底面,三角形 $SAB$、$SBC$ 等是侧面,$SA$、$SB$ 等是侧棱,$S$ 是顶点,$SO$ 是高。

棱锥用表示顶点和底面各顶点,或者底面一条对角线端点的字母来表示。例如棱锥 $S$-$ABCDE$,或者棱锥 $S$-$AC$。

棱锥的底面可以是三角形、四边形、五边形、……,因此我们把这样的棱锥分别叫做三棱锥(\cref{fig:2-13a})、四棱锥(\cref{fig:2-13b})、五棱锥(\cref{fig:2-13c})、……。

\begin{figure}
  \begin{minipage}[b]{0.32\linewidth}\centering
    \includegraphics{2-13a.pdf}
    \subcaption{}\label{fig:2-13a}
  \end{minipage}
  \begin{minipage}[b]{0.32\linewidth}\centering
    \includegraphics{2-13b.pdf}
    \subcaption{}\label{fig:2-13b}
  \end{minipage}
  \begin{minipage}[b]{0.32\linewidth}\centering
    \includegraphics{2-13c.pdf}
    \subcaption{}\label{fig:2-13c}
  \end{minipage}
  \caption{}\label{fig:2-13}
\end{figure}

如果一个棱锥的底面是正多边形,并且顶点在底面的射影是底面中心,这样的棱锥叫做\Concept{正棱锥}。

正棱锥有下面一些性质:
\begin{enumerate}
  \item 各侧棱相等,各侧面都是全等的等腰三角形。各等腰三角形底边上的高相等,它叫做\Concept{正棱锥的斜高};
  \item 棱锥的高、斜高和斜高在底面上的射影组成一个直角三角形;棱锥的高、侧棱和侧棱在底面上的射影也组成一个直角三角形(\cref{fig:2-14})。
\end{enumerate}

\begin{figure}
  \begin{minipage}[b]{0.4\linewidth}\centering
    \includegraphics{2-14.pdf}
    \caption{}\label{fig:2-14}
  \end{minipage}
  \begin{minipage}[b]{0.4\linewidth}\centering
    \includegraphics{2-15.pdf}
    \caption{}\label{fig:2-15}
  \end{minipage}
\end{figure}

关于一般棱锥,有下面一个重要的性质:
\begin{Theorem}{定理}
  如果棱锥被平行于底面的平面所截,那么截面和底面相似,并且它们面积的比等于截得的棱锥的高和已知棱锥的高的平方比。
\end{Theorem}
已知:如\cref{fig:2-15},在棱锥 $S$-$AC$ 中,$SH$ 是高,截面 $A'B'C'D'E'$ 平行于底面,并与 $SH$ 交于 $H'$。

求证:$\text{截面}\ A'B'C'D'E'\sim\text{底面}\ ABCDE$,并且
\[ \frac{S_{A'B'C'D'E'}}{S_{ABCDE}}=\frac{SH'^2}{SH^2} \]

\begin{proof}
  因为截面平行于底面,所以 $A'B'\parallel AB$,$B'C'\parallel BC$,$C'D'\parallel CD$,……。因而 $\angle A'B'C'=\angle ABC$,$\angle B'C'D'=\angle BCD$,……。又因为过 $SA$、$SH$ 的平面与截面和底面分别交于 $A'H'$ 和 $AH$,

  $\therefore\quad A'H'\parallel AH$,得
  \[\frac{B'C'}{BC}=\frac{SA'}{SA}=\frac{SH'}{SH}.\]

  同理
  \[\frac{B'C'}{BC}=\frac{SH'}{SH},\cdots.\]

  $\therefore\quad \dfrac{A'B'}{AB}=\dfrac{B'C'}{BC}=\cdots=\dfrac{SH'}{SH}$
  
  \medskip
  因此,截面 $A'B'C'D'E'$ $\sim$ 底面 $ABCDE$。

  \[\therefore\quad \frac{S_{A'B'C'D'E'}}{S_{ABCDE}}=\frac{A'B'^2}{AB^2}=\frac{SH'^2}{SH^2} \]
\end{proof}

\begin{example}
如\cref{fig:2-16},已知正三棱锥 $S$-$ABC$ 的高 $SO=h$,斜高 $SM=l$。求经过 $SO$ 的中点平行于底面的截面\footnote{象这样过高的中点平行于底面的截面叫做中截面。}$\triangle A'B'C'$ 的面积。
\end{example}
\begin{solution}
连结 $OM$、$OA$。在 $\mathrm{Rt}\triangle SOM$ 中,$OM=\sqrt{l^2-h^2}$。

因为棱锥 $S$-$ABC$ 是正棱锥,所以点 $O$ 是正三角形 $ABC$ 的中心。
\par\noindent\begin{minipage}{0.5\linewidth}\centering
\begin{align*}
  AB&=2AM=2\cdot OM\cdot \tan\ang{60}\\ 
    &=2\sqrt{3}\cdot\sqrt{l^2-h^2},\\ 
  S_{\triangle ABC}&=\frac{\sqrt{3}}{4}AB^2\\ 
    &=\frac{\sqrt{3}}{4}\times 4\times 3(l^2-h^2)\\
    &=3\sqrt{3}(l^2-h^2).
\end{align*}
\end{minipage}\hfill
\begin{minipage}{0.45\linewidth}
\begin{figurehere}
  \includegraphics{2-16.pdf}
  \caption{}\label{fig:2-16}
\end{figurehere}
\end{minipage}\par\medskip

根据一般棱锥截面的性质,有
\[ \frac{S_{\triangle A'B'C'}}{S_{\triangle ABC}}=\frac{h'^2}{h^2}=\frac14\]

$S_{\triangle A'B'C'}=\dfrac{3\sqrt{3}}{4}(l^2-h^2)$。
\end{solution}

\begin{Practice}
  \begin{question}
    \item 底面是正多边形的棱锥是正棱锥吗?
    \item 求证:正棱锥各侧面与底面所成的二面角都相等。
  \end{question}
\end{Practice}
\subsubsection{正棱锥的直观图的画法}
正棱锥的直观图由底面和顶点所决定。正棱锥底面的画法与直棱柱的底面画法相同,顶点和底面中心的距离,等于它的高。下面以正五棱锥为例,说明正棱锥的直观图的画法。
\begin{example}
画一个底面边长为 \qty{5}{cm},高为 \qty{11.5}{cm} 的正五棱锥的直观图。比例尺是 $\dfrac15$。
\end{example}
\begin{solution}[画法]
\begin{enumerate}
  \item 画轴\quad 画 $x'$ 轴、$y'$ 轴、$z'$ 轴,使 $\angle x'O'y'=\ang{45}$,$\angle x'O'z'=\ang{95}$(\cref{fig:2-17a})。
  \item 画底面\quad 按 $x'$ 轴、$y'$ 轴画正五边形的直观图 $ABCDE$,按比例尺,取边长等于 $5\div 5=1\,(\unit{cm})$,并使正五边形的中心对应于点 $O'$。
  \item 画高线\quad 在 $z'$ 轴上,取 $O'S=11.5\div 5=2.3\,(\unit{cm})$。
  \item 成图\quad 连结 $SA$、$SB$、$SC$、$SD$、$SE$,并加以整理,就得到所要画的正五棱锥的直观图(\cref{fig:2-17b})。
\end{enumerate}
\end{solution}

\begin{figure}
  \begin{minipage}[b]{0.58\linewidth}\centering
    \begin{minipage}[b]{0.5\linewidth}\centering
      \includegraphics{2-17a.pdf}
      \subcaption{}\label{fig:2-17a}
    \end{minipage}
    \begin{minipage}[b]{0.48\linewidth}\centering
      \includegraphics{2-17b.pdf}
      \subcaption{}\label{fig:2-17b}
    \end{minipage}
    \caption{}\label{fig:2-17}
  \end{minipage}
  \begin{minipage}[b]{0.4\linewidth}\centering
    \includegraphics{2-18.pdf}
    \caption{}\label{fig:2-18}
  \end{minipage}
\end{figure}

\subsubsection{正棱锥的侧面积}
棱锥的侧面展开图是由各个侧面组成的,展开图的面积就是棱锥的侧面积。设正棱锥的底面边长为 $a$,周长为 $c$,斜高为 $h'$,则展开图(\cref{fig:2-18})的面积等于 $n\cdot\dfrac12ah'=\dfrac12ch'$。由此得到下面的定理:
\begin{Theorem}{定理}
  如果正棱锥的底面周长是 $c$,斜高是 $h'$,那么它的侧面积是
  \[\tcbhighmath{S_\text{正棱锥侧}=\frac12ch'.}\]
\end{Theorem}

棱锥的全面积等于侧面积与底面积的和。

\begin{example}
设计一个正四棱锥形冷水塔塔顶,高是 \qty{0.85}{m},底的边长是 \qty{1.5}{m},制造这种塔顶需要多少平方米铁板(保留两位有效数字)?
\end{example}
\begin{solution}
  如\cref{fig:2-19},$S$ 表示塔顶的顶点,$O$ 表示底的中心,则 $SO$ 是高,设 $SE$ 是斜高。
  \par\medskip\noindent
  \begin{minipage}{0.55\linewidth}\parindent2em
  在直角三角形 $SOE$ 中,根据勾股定理得
  \begin{align*}
    SE&=\sqrt{\left(\frac{1.5}{2}\right)^2+0.85^2}\\
      &\approx \qty{1.13}{m}.\\
    \therefore\quad  S_\text{正棱锥侧}&=\frac12ch'\\
    &=\frac12(1.5\times 4)\times 1.13\\
     &\approx \qty{3.4}{m^2}.
  \end{align*}

  答:制造这种塔顶需要铁板约 \qty{3.4}{m^2}。
\end{minipage}\hfill
\begin{minipage}{0.4\linewidth}
\begin{figurehere}
  \includegraphics{2-19.pdf}
  \caption{}\label{fig:2-19}
\end{figurehere}
\end{minipage}
\end{solution}\par\medskip

\begin{Practice}
  \begin{question}
    \item 已知正六棱锥的底面边长为 \qty{6}{cm},高为 \qty{15}{cm}。画它的直观图,比例尺为 $\dfrac13$。
    \item 一个正三棱锥的侧面都是直角三角形,底面边长是 $a$。求它的全面积。
  \end{question}
\end{Practice}
\begin{Exercise}
  \begin{question}
    \item 用厚纸做一个正三棱锥或正四棱锥的模型。
    \item 已知底面边长是 $a$,高是 $h$。求下列棱锥的侧棱长和斜高:
    \begin{tasks}(3)
      \task 正三棱锥;
      \task 正四棱锥;
      \task 正六棱锥。
    \end{tasks}
    \item 已知正六棱锥的底面边长是 \qty{4}{cm},侧棱长是 \qty{8}{cm}。求它的侧面和底面所成的二面角。
    \item 已知正三棱锥的底面边长为 $a$,求过各侧棱中点的截面面积。
    \item\label{exec:8-5}求证:平行于三棱锥的两条相对棱的平面截三棱锥所得的截面是平行四边形。
    \begin{figurehere}
      \begin{minipage}{\linewidth}\centering
        \includegraphics{ex8-5.pdf}
        \caption*{(第~\ref{exec:8-5}~题图)}
      \end{minipage}
    \end{figurehere}
    \item 棱锥的底面积是 \qty{150}{cm^2},平行于底面的一个截面面积是 \qty{54}{cm^2},底面和这个截面的距离是 \qty{12}{cm},求棱锥的高。
    \item 画一个底面边长是 \qty{4}{cm},高是 \qty{8}{cm} 的正六棱锥的直观图(选择适当比例尺)。
    \item 正三棱锥的底面边长是 $a$,高是 $2a$,计算它的全面积。
    \item 一座仓库的屋顶呈正四棱锥形,底面的边长 \qty{2.7}{m},侧棱长 \qty{2.3}{m},如果要在屋顶上铺一层油毡纸,需要油毡纸多少平方米?
    \item 要做一个正六棱锥形的铁烟囱帽,底口边长 \qty{40}{cm},高是 \qty{50}{cm},需要多少平方米铁皮?
    \item 一个棱锥所有的侧面与底面所成的二面角都等于 $\alpha$,那么
    \[S_\text{侧}=\frac{S_\text{底}}{\cos\alpha}.\]
  \end{question}
\end{Exercise}

\subsection{棱台}\label{subsec:frustum}
\subsubsection{棱台的概念和性质}
\par\medskip\noindent
\begin{minipage}{0.55\linewidth}\parindent2em
用一个平行于棱锥底面的平面去截棱锥,底面和截面之间的部分叫做\Concept{棱台}。原棱锥的底面和截面叫做\Concept{棱台的下底面}和\Concept{上底面},其他各面叫做\Concept{棱台的侧面}。相邻侧面的公共边叫做\Concept{棱台的侧棱},上、下底面之间的距离叫做\Concept{棱台的高}。如\cref{fig:2-20} 中的棱台,多边形 $A'B'C'D'$ 和 $ABCD$ 是上、下底面,四边形 $ABB'A'$、$BCC'B'$ 等是侧面,$AA'$、$BB'$ 等是侧棱,$OO'$ 是它的高。
\end{minipage}\hfill
\begin{minipage}{0.4\linewidth}
\begin{figurehere}
  \includegraphics{2-20.pdf}
  \caption{}\label{fig:2-20}
\end{figurehere}
\end{minipage}
\par\medskip


棱台用表示上、下底面各顶点的字母来表示,例如,棱台 $ABCD$-$A'B'C'D'$。或者用它的对角线端点字母表示,如棱台 $AC'$。

由三棱锥、四棱锥、五棱锥、……截得的棱台,分别叫做\Concept{三棱台}、\Concept{四棱台}、\Concept{五棱台}、……。

由正棱锥截得的棱台叫做\Concept{正棱台}。

正棱台有下面性质:
\begin{enumerate}
  \item \emph{正棱台的侧棱相等,侧面是全等的等腰梯形}。各等腰梯形的高相等,它叫做\Concept{正棱台的斜高};
  \item \emph{正棱台的两底面以及平行于底面的截面是}\Concept{相似正多边形};
  \item \emph{正棱台的两底面中心连线、相应的边心距和斜高组成一个直角梯形;两底面中心连线、侧棱和两底面相应的半径也组成一个直角梯形}(\cref{fig:2-21})。
\end{enumerate}

\begin{figure}
  \begin{minipage}[b]{0.4\linewidth}\centering
  \includegraphics{2-21.pdf}
    \caption{}\label{fig:2-21}
  \end{minipage}
  \begin{minipage}[b]{0.55\linewidth}\centering
  \includegraphics{2-22.pdf}
    \caption{}\label{fig:2-22}
  \end{minipage}
\end{figure}

\begin{example}
正四棱台 $AC'$ 的高是 \qty{17}{cm},两底面的边长分别是 \qty{4}{cm} 和 \qty{16}{cm}。求这个棱台的侧棱的长和斜高(\cref{fig:2-22})。
\end{example}
\begin{solution}
解:设棱台两底面的中心分别是 $O'$ 和 $O$,$B'C'$ 和 $BC$ 的中点分别是 $E'$ 和 $E$。连结 $O'O$、$E'E$、$O'B'$、$OB$、$O'E'$、$OE$,则 $OBB'O'$ 和 $OEE'O'$ 都是直角梯形(\cref{fig:2-22})。
\begin{align*}
  \because\quad A'B'&=\qty{4}{cm}, & AB&=\qty{16}{cm},\\
  \therefore\quad O'E'&=\qty{2}{cm}, & OE&=\qty{8}{cm},\\
  \therefore\quad O'B'&=2\sqrt{2}\,\unit{cm}, & OE&=8\sqrt{2}\,\unit{cm},\\
\end{align*}

因此 $B'B=\sqrt{17^2+(8\sqrt{2}-2\sqrt{2})^2}=\qty{19}{cm}$,

$E'E=\sqrt{17^2+(8-2)^2}=5\sqrt{13}\,\unit{cm}$。即,这个棱台的侧棱长是 \qty{19}{cm},斜高是 $5\sqrt{13}$\,\unit{cm}。
\end{solution}

\begin{example}\label{exp:2-7}
设棱台的两底面积分别是 $S$、$S'$,它的中截面的面积是 $S_0$。求证:$2\sqrt{S_0}=\sqrt{S}+\sqrt{S'}$(\cref{fig:2-23})。
\end{example}
\begin{proof}
因为棱台的中截面与两底面平行,所以多边形 $ABCDE$、$A_0B_0C_0D_0E_0$、$A'B'C'D'E'$ 相似。因此
\par\medskip\noindent
\begin{minipage}{0.6\linewidth}\parindent2em
\[ \frac{S}{S_0}=\frac{AB^2}{A_0B_0^2},\quad \frac{S'}{S_0}=\frac{A'B'^2}{A_0B_0^2},\]
也就是
\[ \frac{\sqrt{S}}{\sqrt{S_0}}=\frac{AB}{A_0B_0},\quad \frac{\sqrt{S'}}{\sqrt{S_0}}=\frac{A'B'}{A_0B_0}.\]

将上面等式两边分别相加,因为 $A_0B_0$ 是梯形 $ABB'A'$ 的中位线,
\[\therefore\quad \frac{\sqrt{S}+\sqrt{S'}}{\sqrt{S_0}}=\frac{AB+A'B'}{A_0B_0}=\frac{2A_0B_0}{A_0B_0}=2.\]
由此得到
\[2\sqrt{S_0}=\sqrt{S}+\sqrt{S'}.\]
\end{minipage}\hfill
\begin{minipage}{0.35\linewidth}
  \begin{figurehere}
    \includegraphics{2-23.pdf}
    \caption{}\label{fig:2-23}
  \end{figurehere}
\end{minipage}
\end{proof}

\begin{Practice}
  \begin{question}
    \item\label{prac:2-6-1}图中的几何体是不是棱台?为什么?
    \begin{figurehere}
      \begin{minipage}{\linewidth}\centering
        \includegraphics{pr2-6-1.pdf}
        \caption*{(第~\ref{prac:2-6-1}~题图)}
      \end{minipage}
    \end{figurehere}
    \item 一个正四棱台上、下底面的边长分别是 $a$ 和 $b$,高是 $h$。求经过相对的两条侧棱的截面面积。
  \end{question}
\end{Practice}
\subsubsection{正棱台的直观图的画法}
以正四棱台为例,说明正棱台的画法。

\begin{solution}[画法]
  \begin{enumerate}
    \item 画轴\quad 画 $x'$ 轴、$y'$ 轴、$z'$ 轴,使 $\angle x'Oy'=\ang{45}$,$\angle x'O'z'=\ang{90}$(\cref{fig:2-24a})。
    \item 画底图\quad 以 $O'$ 为中心,按 $x'$ 轴、$y'$ 轴画正四棱台下底面正方形的直观图 $ABCD$。在 $z'$ 轴上取线段 $O'O_1$ 等于正四棱台的高。过 $O_1$ 画 $O_1M$、$O_1N$ 分别平行于 $O'x'$、$O'y'$,再以 $O_1$ 为中心,按 $O_1M$、$O_1N$ 画正四棱台上底面正方形的直观图 $A_1B_1C_1D_1$。
    \item 成图\quad 连结 $AA_1$、$BB_1$、$CC_1$、$DD_1$,并加以整理,就得到正四棱台的直观图(\cref{fig:2-24b})。
  \end{enumerate}
\end{solution}

\begin{figure}
  \begin{minipage}[b]{0.48\linewidth}\centering
    \includegraphics{2-24a.pdf}
    \subcaption{}\label{fig:2-24a}
  \end{minipage}
  \begin{minipage}[b]{0.48\linewidth}\centering
    \includegraphics{2-24b.pdf}
    \subcaption{}\label{fig:2-24b}
  \end{minipage}
  \caption{}\label{fig:2-24}
\end{figure}

\subsubsection{正棱台的侧面积}
\par\medskip\noindent
\begin{minipage}{0.55\linewidth}\parindent2em
{\linespread{1.5}\selectfont
棱台的侧面展开图是由各个侧面组成的。展开图的面积就是棱台的侧面积。正棱台的侧面展开图如\cref{fig:2-25} 所示。设它的上、下底面边长是 $a'$、$a$,边数是 $n$,斜高是 $h'$,那么它的侧面积是 $n\cdot\dfrac12(a+a')h'=\dfrac12(na+na')h'$,由于 $na'$、$na$ 分别是上、下底面的周长 $c'$、$c$,我们得到下面的定理:\par}
\end{minipage}\hfill
\begin{minipage}{0.4\linewidth}
\begin{figurehere}
  \includegraphics{2-25.pdf}
  \caption{}\label{fig:2-25}
\end{figurehere}
\end{minipage}\par\medskip

\begin{Theorem}{定理}
  如果正棱台的上、下底面的周长是 $c'$、$c$,斜高是 $h'$,那么它的侧面积是
  \[\tcbhighmath{S_\text{正棱台侧}=\frac12(c+c')h'.}\]
\end{Theorem}
棱台的全面积等于它的侧面积于上、下底面积的和。

\begin{example}
粉碎机上的下料斗是正四棱台形(\cref{fig:2-26}),它的两底面边长分别是 \qty{80}{mm} 和 \qty{440}{mm},高是 \qty{200}{mm}。计算制造这样一个下料斗所需铁板的面积(保留两位有效数字)。
\end{example}
\begin{solution}
上底面周长 $c'=4\times 80=\qty{320}{mm}$,

下底面周长 $c=4\times 440=\qty{1760}{mm}$,
\par\noindent
\begin{minipage}{0.63\linewidth}\parindent2em
\begin{align*}
  \text{斜高}\ h'&=\sqrt{200^2+\left(\frac{440-80}{2}\right)^2}\approx \qty{269}{mm},\\
  \therefore\quad S_\text{正棱台侧}&=\frac12(c+c')h'=\frac12(320+1760)\times 269\\ 
  &\approx \qty{2.8e5}{mm^2}.
\end{align*}
\end{minipage}\hfill
\begin{minipage}{0.35\linewidth}
\begin{figurehere}
  \includegraphics{2-26.pdf}
  \caption{}\label{fig:2-26}
\end{figurehere}
\end{minipage}\par\medskip
答:制造这样一个下料斗需铁板约 \qty{2.8e5}{mm^2}。
\end{solution}

在正棱台的侧面积公式中,如果设 $c'=c$,就可以得到正棱柱的侧面积公式:$S_\text{正棱柱侧}=ch'$(这里 $h'$ 是高)。如果设 $c'=0$,就得到正棱锥的侧面积公式:$S_\text{正棱锥侧}=\dfrac12ch'$。这样,正棱柱、正棱锥、正棱台的侧面积公式之间的关系可表示如下图。
\begin{figurehere}
\includegraphics{2-flowchart1.pdf}
\end{figurehere}

\begin{Practice}
  \begin{question}
    \item 画出上、下底面边长分别是 \qty{2}{cm} 和 \qty{8}{cm},斜高是 \qty{4}{cm} 的正四棱台的直观图。
    \item 一个正三棱台的两个底面的边长分别等于 \qty{8}{cm} 和 \qty{18}{cm},侧棱长等于 \qty{13}{cm}。求它的侧面积。
  \end{question}
\end{Practice}

\subsubsection{多面体}
前面,我们研究过的棱柱、棱锥、棱台,它们都是由一些多边形围成的几何体。由若干个多边形所围成的几何体,叫做\Concept{多面体}。围成多面体的各个多边形叫做\Concept{多面体的面},两个面的公共边叫做\Concept{多面体的棱},若干个面的公共顶点叫做\Concept{多面体的顶点}。许多矿物结晶体,例如食盐、明矾、石膏等都是呈多面体形的(\cref{fig:2-27})。
\begin{figure}
  \includegraphics{2-27.pdf}
  \caption{}\label{fig:2-27}
\end{figure}

把多面体的任何一个面伸展为平面,如果所有其他各面都在这个平面的同侧,这样的多面体叫做\Concept{凸多面体}(\cref{fig:2-28})。

\begin{figure}
  \begin{minipage}[b]{0.48\linewidth}\centering
    \includegraphics{2-28.pdf}
    \caption{}\label{fig:2-28}
  \end{minipage}
  \begin{minipage}[b]{0.48\linewidth}\centering
    \includegraphics{2-29.pdf}
    \caption{}\label{fig:2-29}
  \end{minipage}
\end{figure}

前面研究过的所有的棱柱、棱锥、棱台指的都是凸多面体。\cref{fig:2-29} 中的多面体不是凸多面体。

一个多面体至少有四个面。多面体依照它的面数分别叫做\Concept{四面体}、\Concept{五面体}、\Concept{六面体}等。例如,三棱锥是四面体,三棱柱是五面体,正方体是六面体等。

\begin{Practice}
  \begin{question}
    \item 图示多面体、凸多面体、棱柱、棱锥、棱台、平行六面体各集合的包含关系。
    \item 在学过的一些多面体中,举出五面体、六面体、七面体的例子。除三棱锥外,还有四面体吗?
  \end{question}
\end{Practice}

\begin{Exercise}
  \begin{question}
    \item 用厚纸做一个正四棱台的模型。
    \item 已知上、下底面的边长和侧棱的长分别是 $a$、$b$、$c$。求下面各棱台的高和斜高:
    \begin{tasks}(3)
      \task 正三棱台;
      \task 正四棱台;
      \task 正六棱台。
    \end{tasks}
    \item 正四棱台上下底面的边长分别是 $a$ 和 $b$,侧面和底面成 \ang{45} 的二面角。求它的斜高和侧棱长。
    \item 棱台的上、下底面的面积各是 $Q'$ 和 $Q$。求证:这个棱台的高和截得这个棱台的原棱锥的高的比是 $\dfrac{Q-\sqrt{QQ'}}{Q}$。
    \item 一个正三棱锥底面边长是 \qty{10}{cm},高是 \qty{15}{cm}。中截面把棱锥分成一个小棱锥和一个棱台。选择适当的比例尺,画出它们的直观图。
    \item 一个正三棱台的上、下底面的边长分别是 \qty{3}{cm} 和 \qty{6}{cm},侧面与底面成 \ang{60} 的二面角。求它的全面积。
    \item 正四棱台的高是 \qty{12}{cm},两底面的边长相差 \qty{10}{cm},全面积是 \qty{512}{cm^2}。求两底面的边长。
    \item 已知正六棱台的两底边长分别是 $a$、$2a$,高是 $a$。求这个正六棱台的侧面积以及过相对侧棱的截面面积。
    \item 正四棱台的上、下底面的边长各为 $a$、$b$,侧面积等于两底面积的和。它的高是多少?
    \item 把一个棱锥用平行于底面的平面截成棱台,使棱台上、下底面积的比为 $1:2$,求截平面的位置。
    \item 棱锥的中截面把它截成两部分。求这两部分的侧面积的比。
    \item 在一个正四棱台内有一个以它的上底面为底面,下底面中心为顶点的棱锥。如果棱台的上、下底面边长分别为 \qty{3}{cm} 和 \qty{4}{cm},棱锥与棱台的侧面积相等,求棱台的高。
  \end{question}
\end{Exercise}

\section{旋转体}
\subsection{圆柱、圆锥、圆台}
\subsubsection{圆柱、圆锥、圆台的概念和性质}

圆钢呈圆柱形,铅锤呈圆锥形,粮囤呈圆台形(\cref{fig:2-30}),这样形状的物体是很多的。
\begin{figure}
  \includegraphics{2-30.pdf}
  \caption{}\label{fig:2-30}
\end{figure}

分别以矩形、直角三角形、直角梯形的一边、一直角边、垂直于底边的腰所在的直线为旋转轴,其余各边旋转而形成的曲面所围成的几何体分别叫做\Concept{圆柱}、\Concept{圆锥}、\Concept{圆台}(\cref{fig:2-31})。旋转轴叫做它们的\Concept{轴},在轴上这条边的长度叫做它们的\Concept{高},垂直于轴的边旋转而成的圆面叫做它们的\Concept{底面},不垂直于轴的边旋转而成的曲面叫做它们的\Concept{侧面},无论旋转到什么位置,这条边都叫做\Concept{侧面的母线}。如\cref{fig:2-31} 中,直线 $O'O$、$SO$ 是轴,线段 $O'O$、$SO$ 是高,$AA'$、$BB'$、$SA$、$SB$ 等是母线。

\begin{figure}
  \includegraphics{2-31.pdf}
  \caption{}\label{fig:2-31}
\end{figure}

很明显,圆台也可以看做是用平行于圆锥底面的平面截这个圆锥而得到的。

圆柱、圆锥、圆台用表示它的轴的字母来表示,如圆柱 $OO'$、圆锥 $SO$、圆台 $O'O$。

圆柱、圆锥、圆台有下面的性质:
\begin{enumerate}
  \item 平行于底面的截面都是圆;
  \item 过轴的截面(轴截面)分别是全等的矩形、等腰三角形、等腰梯形。
\end{enumerate}

\begin{example}
  把一个圆锥截成圆台,已知圆台的上、下底面半径的比是 $1:4$,母线长是 \qty{10}{cm},求圆锥的母线长。
\end{example}
\begin{solution}
  设圆锥的母线长为 $y$,圆台上、下底面半径分别是 $x$、$4x$(\cref{fig:2-32}),根据相似三角形的比例关系,得
  \[ (y-10):y=x:4x\]
\par\medskip\noindent
\begin{minipage}{0.65\linewidth}\parindent2em\noindent
也就是
\begin{gather*}
  4(y-10)=y\\ 
  3y=40,\\
  \therefore \quad y=\frac{40}{3}\,(\unit{cm}).
\end{gather*}

因此,圆锥得母线长为 $\dfrac{40}{3}$\,\unit{cm}。
\end{minipage}\hfill
\begin{minipage}{0.3\linewidth}
\begin{figurehere}
  \includegraphics{2-32.pdf}
  \caption{}\label{fig:2-32}
\end{figurehere}
\end{minipage}\par\medskip
\end{solution}



\begin{Practice}
  \begin{question}
    \item 用一张 $4\times 8$(\unit{cm^2})的矩形硬纸卷成圆柱的侧面。求轴截面的面积(接头忽略不计)。
    \item 求证:平行于圆锥底面的截面与底面的面积的比,等于顶点到截面的距离与圆锥的高的平方比。
    \item 圆台侧面的母线长为 $2a$,母线与轴的夹角为 \ang{30},一个底面半径是另一个底面半径的 2 倍。求两底面的半径。
  \end{question}
\end{Practice}

\subsubsection{圆柱、圆锥、圆台的直观图的画法}
圆柱、圆锥、圆台的底面都是圆,圆的直观图,一般不用斜二测画法,而用正等测画法。它的规则是:
\begin{enumerate}
  \item 在已知图形 $\odot O$ 中取互相垂直的轴 $Ox$、$Oy$。画直观图时,把它们画成对应的轴 $O'x'$、$O'y'$,使 $\angle x'O'y'=\ang{120}$(或 \ang{60})。它们确定的平面表示水平平面。
  \item 已知图形上平行于 $x$ 轴或 $y$ 轴的线段,在直观图中,分别画成平行于 $x'$ 轴或 $y'$ 轴的线段。
  \item 平行于 $x$ 轴或 $y$ 轴的线段,长度都不变。
\end{enumerate}

下面举例说明这种画法。

\begin{example}
  画水平放置的圆的直观图。
\end{example}
\begin{solution}[画法]
  \begin{enumerate}
    \item 如\cref{fig:2-33},在 $\odot O$ 上取一对互相垂直的直径 $AB$、$CD$,分别以它们所在的直线作为 $x$ 轴、$y$ 轴。画对应的 $x'$ 轴、$y'$ 轴,使 $\angle x'O'y'=\ang{120}$。
    \item 将 $\odot O$ 的直径 $AB$ 分成 $n$ 等分,过分点画平行于 $y$ 轴的弦 $CD$、$EF$、……。在 $x'$ 轴上以 $O'$ 为中点画线段 $A'B'$,使 $A'B'=AB$,将 $A'B'$ 分成 $n$ 等分,以分点为中点画 $y'$ 轴的平行线段 $C'D'$、$E'F'$、……,使 $C'D'=CD,E'F'=EF$,……。
    \item 用平滑曲线顺次连结 $A',D',F',B',E',C',\dots,A'$,就得到圆的直观图,它是一个椭圆。
  \end{enumerate}
\end{solution}

\begin{figure}
  \includegraphics{2-33.pdf}
  \caption{}\label{fig:2-33}
\end{figure}

我们看到,在这种画法中,圆的中心 $O$,变为椭圆的中心 $O'$,圆的任意一对互相垂直的直径(如 $AB$、$CD$)变为椭圆的一对直径(如 $A'B'$、$C'D'$),它们叫做椭圆的\Concept{共轭直径}。圆的切线(如 $a$)变为椭圆的切线(如 $a'$)。

由于椭圆的这种画法比较麻烦,所以实际上通常不用这种画法,而是经过椭圆的一对共轭直径的端点(或再加一点,如 $E'$)用椭圆模板(\cref{fig:2-34})来画,或用初中学过的方法画近似椭圆(\cref{fig:2-35})。

\begin{figure}
  \begin{minipage}[b]{0.48\linewidth}\centering
    \includegraphics{2-34.pdf}
    \caption{}\label{fig:2-34}
  \end{minipage}
  \begin{minipage}[b]{0.48\linewidth}\centering
    \includegraphics{2-35.pdf}
    \caption{}\label{fig:2-35}
  \end{minipage}
\end{figure}

画圆柱、圆锥、圆台的直观图时,先用上述方法画出底面,其余部分与棱柱、棱锥、棱台直观图的画法类似。下面举例说明它们的画法。

\begin{example}
  一个圆锥的底面半径是 \qty{1.6}{cm},在它的内部有一个底面半径为 \qty{0.7}{cm},高为 \qty{1.5}{cm} 的内接圆柱\footnote{圆柱的下底在圆锥的底面上,上底的圆周在圆锥的侧面上。}。画出它们的直观图。
\end{example}
\begin{solution}[画法]
  \begin{enumerate}
    \item 画轴\quad 取 $x$ 轴、$y$ 轴、$z$ 轴,使它们两两相交成 \ang{120} 角(\cref{fig:2-36a})。
    \item 画底面\quad 以 $O$ 为中心,按 $x$ 轴、$y$ 轴画半径等于 \qty{1.6}{cm} 的圆的直观图。
    \item 画内接圆柱\quad 以 $O$ 为中心,按 $x$ 轴、$y$ 轴画一个半径等于 \qty{0.7}{cm} 的圆的直观图,然后在 $z$ 轴上,取线段 $OO'=\qty{1.5}{cm}$,过点 $O'$ 作 $O'M\parallel x\,\text{轴}$,$O'N\parallel y\,\text{轴}$,再以 $O'$ 为中心,按 $O'M$、$O'N$ 画一个半径相同的圆的直观图。画圆柱的两条母线,使它们与这两个椭圆相切。
    \item 成图\quad 画圆锥的两条母线与椭圆 $ACB$ 和 $A'C'B'$ 相切,再加以整理,就得到所要画的直观图(\cref{fig:2-36b})。
  \end{enumerate}
\end{solution}
\begin{figure}
  \begin{minipage}[b]{0.48\linewidth}\centering
    \includegraphics{2-36a.pdf}
    \subcaption{}\label{fig:2-36a}
  \end{minipage}
  \begin{minipage}[b]{0.48\linewidth}\centering
    \includegraphics{2-36b.pdf}
    \subcaption{}\label{fig:2-36b}
  \end{minipage}
  \caption{}\label{fig:2-36}
\end{figure}

\begin{Practice}
  画一个上底半径为 \qty{1.5}{cm},下底半径为 \qty{2.5}{cm},高为 \qty{4}{cm} 的圆台的直观图(比例尺取 $\dfrac12$,不写画法)。
\end{Practice}

\subsubsection{圆柱、圆锥、圆台的侧面积}
把圆柱、圆锥、圆台的侧面沿着它们的一条母线剪开后展在平面上,展开图的面积就是它们的侧面积。

\begin{figure}
  \includegraphics{2-37.pdf}
  \caption{}\label{fig:2-37}
\end{figure}

\cref{fig:2-37} 是圆柱的侧面展开图,它是一个矩形。这个矩形的长等于圆柱底面周长 $c$,宽等于圆柱侧面的母线长 $l$(也是高)。由此可得:

\begin{Theorem}{定理}
  如果圆柱底面半径是 $r$,周长是 $c$,侧面母线长是 $l$,那么它的侧面积是
  \[\tcbhighmath{S_\text{圆柱侧}=cl=2\uppi rl.}\]
\end{Theorem}

\cref{fig:2-38} 是圆锥的侧面展开图,它是一个扇形。这个扇形的弧长等于圆锥底面的周长 $c$,半径等于圆锥侧面的母线长 $l$,由此可得:

\begin{Theorem}{定理}
  如果圆锥底面半径是 $r$,周长是 $c$,侧面母线长是 $l$,那么它的侧面积是
  \[\tcbhighmath{S_\text{圆锥侧}=\frac12cl=\uppi rl.}\]
\end{Theorem}

\begin{figure}
  \begin{minipage}[b]{0.48\linewidth}\centering
  \includegraphics{2-38.pdf}
    \caption{}\label{fig:2-38}
  \end{minipage}
  \begin{minipage}[b]{0.48\linewidth}\centering
  \includegraphics{2-39.pdf}
    \caption{}\label{fig:2-39}
  \end{minipage}
\end{figure}

\cref{fig:2-39} 是圆台的侧面展开图,通常把这样的图形叫做扇环。由扇环可以求出圆台的侧面积。

设圆台侧面的母线长为 $l$,上、下底面周长分别是 $c'$、$c$,半径分别是 $r'$、$r$,于是
\begin{gather}
  \label{eq:Area_side}
  S_\text{圆台侧}=\frac12c(l+x)-\frac12c'x=\frac12[cl+(c-c')x].\\
  \because\quad \frac{c'}{c}=\frac{x}{x+l}, \notag\\
  \therefore\quad x=\frac{c'l}{c-c'}.\notag
\end{gather}
代入\cref{eq:Area_side},得
\begin{align*}
  S_\text{圆台侧}&=\frac12\left[cl+(c-c')\frac{c'l}{c-c'}\right]\\
  &=\frac12(c+c')l\\
  &=\uppi(r+r')l.
\end{align*}

由此我们得到下面的定理:
\begin{Theorem}{定理}
  如果圆台的上、下底面半径是 $r'$、$r$,周长是 $c'$、$c$,侧面母线长是 $l$,那么它的侧面积是
  \[\tcbhighmath{S_\text{圆台侧}=\frac12(c+c')l=\uppi(r+r')l.}\]
\end{Theorem}

圆柱、圆锥、圆台的全面积,分别等于它们的侧面积与底面积的和。

\begin{example}
  已知一个圆锥的底面半径为 $R$,高为 $H$。在其中有一个高为 $x$ 的内接圆柱。
  \begin{enumerate}
    \item 求圆柱的侧面积;
    \item $x$ 为何值时,圆柱的侧面积最大?
  \end{enumerate}
\end{example}
\begin{solution}
  \begin{enumerate}
    \item 画圆锥及内接圆柱的轴截面(\cref{fig:2-40})。设所求的圆柱的底面半径为 $r$,它的侧面积
    \par\medskip\noindent\begin{minipage}{0.6\linewidth}\parindent2em
    \[S_\text{圆柱侧}=2\uppi rx.\]

    $\because\quad \dfrac{r}{R}=\dfrac{H-x}{H}$,(为什么?)

    $\therefore\quad r=R-\dfrac{R}{H}\cdot x$。

    $\therefore\quad S_\text{圆柱侧}=2\uppi Rx-\dfrac{2\uppi R}{H}\cdot x^2$。
    \end{minipage}\hfill
    \begin{minipage}{0.35\linewidth}
    \begin{figurehere}
      \includegraphics{2-40.pdf}
      \caption{}\label{fig:2-40}
    \end{figurehere}
    \end{minipage}
    \item 因为 $S_\text{圆柱侧}$ 的表达式中 $x^2$ 的系数小于零,所以这个二次函数有最大值。这时圆柱的高是
    \[ x=-\dfrac{2\uppi R}{-2\cdot\dfrac{2\uppi R}{H}}=\frac{H}{2}.\]

    当圆柱的高是已知圆锥的高的一半时,它的侧面积最大。
  \end{enumerate}
\end{solution}

\begin{example}
圆锥的底面半径为 $r$,侧面母线长为 $l$,侧面展开图扇形的圆心角为 $\theta$\unit{\degree}。求证:$\theta=\dfrac{r}{l}\cdot\ang{360}$。
\end{example}
\par\medskip\noindent
\begin{minipage}{0.65\linewidth}\parindent2em
\begin{proof}
  \cref{fig:2-41} 是圆锥侧面展开图。因为扇形的弧长等于圆锥底面的周长,即
  \[\frac{\uppi l\theta}{180}=2\uppi r,\]
  所以
  \[\theta=\dfrac{r}{l}\cdot\ang{360}.\]
\end{proof}
\end{minipage}\hfill
\begin{minipage}{0.3\linewidth}
\begin{figurehere}
  \includegraphics{2-41.pdf}
  \caption{}\label{fig:2-41}
\end{figurehere}
\end{minipage}\par\medskip

在圆台的侧面积公式中,如果设 $c'=c$,就得到圆柱侧面积公式:$S_\text{圆柱侧}=cl$。如果设 $c'=0$,就得到圆锥侧面积公式:$S_\text{圆锥侧}=\dfrac12cl$。这样,圆柱、圆锥、圆台的侧面积公式之间的关系可表示如下图。
\begin{figurehere}
  \includegraphics{2-flowchart2.pdf}
\end{figurehere}

\begin{Practice}
  \begin{question}
    \item 将半径为 $r$ 的薄铁圆板沿三条半径截成全等的三个扇形,做成三个圆锥筒,求圆锥筒的高(不计接头)。
    \item 一个直角梯形的上、下底和高的比为 $1:2:\sqrt{3}$。求它旋转而成的圆台的上底面积、下底面积和侧面积的比。
    \item 把圆柱、圆锥、圆台的侧面积用中截面周长及母线长表示出来。
  \end{question}
\end{Practice}

\begin{Exercise}
  \begin{question}
    \item 圆锥底面半径为 $r$,轴截面是直角三角形,求轴截面面积。
    \item\label{exec:10-2}如图,圆柱的一个内接直三棱柱的一个侧面经过圆柱的轴,求证,这个棱柱其他两个侧面互相垂直。
    \item 经过高为 \qty{20}{cm} 的圆锥的顶点,与底面成 \ang{45} 二面角的平面把圆锥底面周长截去 $\dfrac14$。求截面面积。
    \item\label{exec:10-4}从一个底面半径和高都是 $R$ 的圆柱中,挖去一个以圆柱上底面为底,下底面中心为顶点的圆锥,得到一个如图的几何体。如果用一个与圆柱下底面距离等于 $l$ 并且平行于底面的平面去截它,求截面面积。
    \begin{figurehere}
      \begin{minipage}[b]{0.25\linewidth}\centering
        \includegraphics{ex10-2.pdf}
        \caption*{(第~\ref{exec:10-2}~题图)}
      \end{minipage}
      \begin{minipage}[b]{0.45\linewidth}\centering
        \includegraphics{ex10-4.pdf}
        \caption*{(第~\ref{exec:10-4}~题图)}
      \end{minipage}
      \begin{minipage}[b]{0.25\linewidth}\centering
        \includegraphics{ex10-7.pdf}
        \caption*{(第~\ref{exec:10-7}~题图)}
      \end{minipage}
    \end{figurehere}
    \item 圆台的一个底面周长是另一个底面周长的 3 倍,轴截面的面积等于 \qty{392}{cm^2},母线与底面的夹角是 \ang{45},求这个圆台的高、母线长和两底面半径。
    \item 已知圆柱的底面直径是 \qty{12}{cm},高是 \qty{16}{cm},内有一个以圆柱的底为底,另一个底的中心为顶点的圆锥。选择适当的比例尺,画出它们的直观图(不写画法)。
    \item\label{exec:10-7}要电镀螺杆(尺寸如图,单位:\unit{mm})。如果每平方米用锌 \qty{0.11}{kg},电镀 100 个这样的螺杆需要多少锌?
    \item 一个圆锥的高是 \qty{10}{cm},侧面展开图是半圆,求圆锥的侧面积。
    \item 用油漆涂 100 个圆台形水桶,桶口直径为 \qty{30}{cm},桶底直径为 \qty{25}{cm},母线长是 \qty{27.5}{cm};已知每平方米需要油漆 \qty{150}{g}。共需油漆多少 \unit{kg}?
    \item 圆锥的轴截面是正三角形。求证:它的侧面积是底面面积的 2 倍。
    \item\label{exec:10-11}已知:圆台上下底面面积是 $S'$、$S$,中截面面积为 $S_0$。求证:$2\sqrt{S_0}=\sqrt{S}+\sqrt{S'}$。
    \item 已知圆台的上、下底面半径是 $r'$、$r$,它的侧面积等于两底面面积的和。求圆台的母线长。
    \begin{figurehere}
      \begin{minipage}[b]{0.48\linewidth}\centering
        \includegraphics{ex10-11.pdf}
        \caption*{(第~\ref{exec:10-11}~题图)}
      \end{minipage}
      \begin{minipage}[b]{0.48\linewidth}\centering
        \includegraphics{ex10-13.pdf}
        \caption*{(第~\ref{exec:10-13}~题图)}
      \end{minipage}
    \end{figurehere}
    \item\label{exec:10-13}如图,圆锥形烟囱帽的底的半径是 \qty{40}{cm},高是 \qty{30}{cm}。计算它的侧面展开图的圆心角和面积。
    \item 设圆台的上、下底面半径分别是 $r'$、$r$,母线长是 $l$。圆台侧面展开后所得的扇环的圆心角是 $\theta$。求证:
    \[\theta=\frac{r-r'}{l}\cdot\ang{360}.\]
  \end{question}
\end{Exercise}

\subsection{球}
\subsubsection{球的概念和性质}
\par\medskip\noindent
\begin{minipage}{0.65\linewidth}\parindent2em
常见的排球、足球及滚珠等物体,都呈球形。

半圆以它的直径为旋转轴,旋转所成的曲面叫做\Concept{球面}。球面所围成的几何体叫做\Concept{球体},简称\Concept{球}。半圆的圆心叫做\Concept{球心}。连结球心和球面上任意一点的线段叫做\Concept{球的半径}。连结球面上两点并且经过球心的线段叫做\Concept{球的直径}。如\cref{fig:2-42} 的球中,点 $O$ 是球心,线段 $OC$ 是球的半径,线段 $AB$ 是球的直径。
\end{minipage}\hfill
\begin{minipage}{0.3\linewidth}
\begin{figurehere}
  \includegraphics{2-42.pdf}
  \caption{}\label{fig:2-42}
\end{figurehere}
\end{minipage}\par\medskip

球面也可以看作与定点(球心)的距离等于定长(半径)的所有点的集合(轨迹)。

一个球用表示它的球心的字母来表示,例如球 $O$。

用一个平面去截一个球,截面是圆面。球的截面有下面的性质:
\begin{enumerate}
  \item 球心和截面圆心的连线垂直于截面(\cref{fig:2-43a});
  \item 球心到截面的距离 $d$ 与球的半径 $R$ 及截面的半径 $r$,有下面的关系:
  \[r=\sqrt{R^2-d^2}.\]
\end{enumerate}

当 $d=0$ 时,截面经过球心,$r=R$。这时,球面被截得的圆最大(\cref{fig:2-43b}),这个圆叫\Concept{球的大圆}。不经过球心的截面所截得的圆叫做\Concept{球的小圆}。
\begin{figure}
  \begin{minipage}{0.32\linewidth}\centering
  \includegraphics{2-43a.pdf}
    \subcaption{}\label{fig:2-43a}
  \end{minipage}
  \begin{minipage}{0.32\linewidth}\centering
  \includegraphics{2-43b.pdf}
    \subcaption{}\label{fig:2-43b}
  \end{minipage}
  \begin{minipage}{0.32\linewidth}\centering
  \includegraphics{2-43c.pdf}
    \subcaption{}\label{fig:2-43c}
  \end{minipage}
  \caption{}\label{fig:2-43}
\end{figure}

当 $d=R$ 时 $r=0$,截面缩成一点。这个点是截平面与球的唯一公共点。和球只有一个公共点的平面叫做\Concept{球的切面}(\cref{fig:2-43c})。球与它的切面的公共点叫做\Concept{切点}。

当我们把地球看作一个球时,经线就是球面上从北极到南极的半个大圆。赤道是一个大圆,其余的纬线都是小圆(\cref{fig:2-44})。
\begin{figure}
  \includegraphics{2-44.pdf}
  \caption{}\label{fig:2-44}
\end{figure}

在球面上,两点之间的最短距离,就是经过这两点的大圆在这两点间的一段劣弧的长度。我们把这个弧长叫做\Concept{两点间的球面距离}。例如,\cref{fig:2-45} 中的 $\overparen{PQ}$ 的长度就是 $P$、$Q$ 两点间的球面距离。飞机、轮船都是尽可能以大圆弧为航线航行。

\begin{figure}
  \begin{minipage}{0.45\linewidth}\centering
  \includegraphics{2-45.pdf}
    \caption{}\label{fig:2-45}
  \end{minipage}
  \begin{minipage}{0.45\linewidth}\centering
  \includegraphics{2-46.pdf}
    \caption{}\label{fig:2-46}
  \end{minipage}
\end{figure}

\begin{example}
  我国首都北京靠近北纬 \ang{40}。求北纬 \ang{40} 纬线的长度约为多少 \unit{km}(地球半径约 \qty{6370}{km})。
\end{example}
\begin{solution}
  如\cref{fig:2-46},$A$ 是北纬 \ang{40} 圈上的一点,$AK$ 是它的半径,所以 $OK\perp AK$。设 $c$ 是北纬 \ang{40} 的纬线长。因为 $\angle AOB=\angle OAK=\ang{40}$,所以
  \begin{align*}
    c&=2\uppi\cdot AK\\
     &=2\uppi\cdot OA\cos OAK\\
     &=2\uppi\cdot OA\cos\ang{40}\\
     &\approx 2\times 3.142\times 6370\times 0.7660\\
     &\approx \qty{3.066e4}{km}
  \end{align*}

  答:北纬 \ang{40} 纬线的长度约为 \qty{3.066e4}{km}。
\end{solution}


\subsubsection{球的直观图的画法}
球的直观图,一般采用正等测画法。这时球的轮廓是一个圆。

\begin{example}
  画半径为 $R$ 的球的直观图。
\end{example}
\par\medskip\noindent
\begin{minipage}{0.68\linewidth}\parindent2em
\begin{solution}[画法]
  \begin{enumerate}
    \item 画轴\quad 经过点 $O$ 画 $x$ 轴、$y$ 轴、$z$ 轴,轴间角为 \ang{120}(\cref{fig:2-47})。
    \item 画大圆\quad 以 $O$ 为中心,分别按 $x$ 轴、$y$ 轴,$y$ 轴、$z$ 轴,$z$ 轴、$x$ 轴画半径为 $R$ 的圆的直观图(三个椭圆)。
    \item 成图\quad 以点 $O$ 为圆心画一个圆与三个椭圆都相切。最后经过整理就得到球的直观图。
  \end{enumerate}
\end{solution}
\end{minipage}\hfill
\begin{minipage}{0.3\linewidth}
\begin{figurehere}
  \includegraphics{2-47.pdf}
  \caption{}\label{fig:2-47}
\end{figurehere}
\end{minipage}\par\medskip
\subsubsection{球的表面积}
圆柱、圆锥、圆台的表面积公式,都是利用它们的展开图求出的。由于球面不能展开成平面图形,所以球的表面积公式无法用展开图求出。为了求得球的表面积公式,我们来证明一个预备定理:
\begin{Theorem}{定理}
  球面内接圆台(圆台上、下底面是球的两个平行截面)的高为 $h$,球心到母线的距离为 $p$,那么圆台的侧面积为 $2\uppi ph$。
\end{Theorem}
已知:球面 $O$ 的内接圆台的高 $OO'=h$,球心 $O$ 到母线 $AD$ 的距离 $OE=p$(\cref{fig:2-48a})。求证:$S_\text{圆台侧}=2\uppi ph$。
\begin{figure}
  \begin{minipage}[b]{0.48\linewidth}\centering
    \includegraphics{2-48a.pdf}
    \subcaption{}\label{fig:2-48a}
  \end{minipage}
  \begin{minipage}[b]{0.48\linewidth}\centering
    \includegraphics{2-48b.pdf}
    \subcaption{}\label{fig:2-48b}
  \end{minipage}
  \caption{}\label{fig:2-48}
\end{figure}

\begin{proof}
  过圆台的轴的平面截圆台和球分别得轴截面 $ABCD$ 和球的大圆 $\odot O$。这时 $ABCD$ 是 $\odot O$ 的内接等腰梯形(\cref{fig:2-48b})。

  作 $OE\perp AD$,垂足 $E$ 是 $AD$ 的中点,$OE=p$。再作 $DD'\perp AB$,$EE'\perp O_1O'$,垂足分别是 $D'$,$E'$,那么 $DD'=h$。

  设圆台上、下底面半径为 $r$、$r'$,母线长为 $l$,则
  \[ EE'=\frac12(r'+r).\]

  由于两个直角三角形 $ADD'$ 和 $OEE'$ 的对应角相等,(为什么?)所以 $\triangle ADD'\sim\triangle OEE'$。
  \[ \therefore \quad l:h=p:\frac12(r+r'),\quad l(r+r')=2ph.\]
  代入圆台侧面积公式,得到
  \[ S_\text{圆台侧}=\uppi(r+r')l=2\uppi ph.\]
\end{proof}

\alertwarning{这个结果对于球的内接圆柱、圆锥同样成立。}

现在,我们来求半径为 $R$ 的球的表面积。
\begin{figure}
  \includegraphics{2-49.pdf}
  \caption{}\label{fig:2-49}
\end{figure}

如\cref{fig:2-49},将半球面上的半大圆 $ANB$ 分成 $2n$ 等分,用过各分点平行于半球大圆面的平面将半球分为 $n$ 部分,以截得的圆为底作圆台、圆锥,设它们的高分别是 $h_1,h_2,\dots,h_n$。球心到它们的母线的距离都为 $p$。根据上面的预备定理,这些圆台、圆锥的侧面积的和为
\begin{align*} 
  S&=2\uppi ph_1+2\uppi ph_2+\cdots+2\uppi ph_n\\ 
   &=2\uppi p(h_1+h_2+\cdots+h_n)\\
   &=2\uppi p\cdot ON\\
   &=2\uppi pR.
\end{align*}

如果分点无限增加,侧面就无限地接近于半球面,同时 $p$ 也无限地接近于 $R$。当 $p$ 变为 $R$ 时,侧面积的和 $S$ 变为 $2\uppi R^2$,我们把这个和作为半球面的面积。由此得到下面的定理:
\begin{Theorem}{定理}
  球面面积等于它的大圆面积的 4 倍。即
  \[\tcbhighmath{S_\text{球面}=4\uppi R^2.}\]
\end{Theorem}

\begin{example}
已知:圆柱的底面直径与高都等于球的直径。求证:
\begin{enumerate*}
  \item 球的表面积等于圆柱的侧面积;
  \item 球的表面积等于圆柱全面积的 $\dfrac23$。
\end{enumerate*}
\end{example}

\begin{proof}
\par\medskip\noindent
\begin{minipage}{0.6\linewidth}\parindent2em
\begin{enumerate}
  \item 设球的半径为 $R$,则圆柱的底面半径为 $R$,高为 $2R$,得
  \begin{align*}
    S_\text{球}&=4\uppi R^2,\\
    S_\text{圆柱侧}&=2\uppi R\cdot 2R=4\uppi R^2.\\
    \therefore\quad S_\text{球}&=S_\text{圆柱侧}.
  \end{align*}
  \item 
  \begin{align*}
    \because S_\text{圆柱全}&=4\uppi R^2+2\uppi R^2=6\uppi R^2,\\
    S_\text{球}&=4\uppi R^2,\\
    \therefore\quad S_\text{球}&=\frac23S_\text{圆柱全}.
  \end{align*}
\end{enumerate}
\end{minipage}\hfill
\begin{minipage}{0.35\linewidth}
\begin{figurehere}
  \includegraphics{2-50.pdf}
  \caption{}\label{fig:2-50}
\end{figurehere}
\end{minipage}
\end{proof}

\begin{Practice}
  \begin{question}
    \item 海面上,地球球心角 \ang{;1;} 所对的大圆弧长约为 1 海里,1 海里约是多少千米?
    \item 计算地球表面积是多少 \unit{km^2}。
  \end{question}
\end{Practice}


\subsection{球冠}
\subsubsection{球冠}
观察天象的天文台的屋顶(\cref{fig:2-51}),光学透镜和反射镜的凹、凸面,都是球面的一部分。
\begin{figure}
  \begin{minipage}[b]{0.4\linewidth}\centering
    \includegraphics[height=4cm]{2-51.jpg}
    \caption{}\label{fig:2-51}
  \end{minipage}
  \begin{minipage}[b]{0.55\linewidth}\centering
    \includegraphics{2-52.pdf}
    \caption{}\label{fig:2-52}
  \end{minipage}
\end{figure}

球被平面所截得的一部分叫做\Concept{球冠},截得的圆叫做\Concept{球冠的底}。垂直于截面的直径被截得的一段叫做\Concept{球冠的高}(\cref{fig:2-52})。

球冠也可以看作一段圆弧绕经过它的一个端点的直径旋转所成的曲面。

我们完全可以仿照求半球面面积的方法,求出球冠的面积。将\cref{fig:2-49} 看作是由半径为 $R$ 的球截得的球冠,那么 $ON$ 就是球冠的高,用 $h$ 表示。这样就得到下面的定理:
\begin{Theorem}{定理}
  球冠的面积等于截成它的球面上大圆周长与球冠的高的积。即
  \[\tcbhighmath{S_\text{球冠}=2\uppi Rh.}\]
\end{Theorem}

这个公式,对于小于半球面的球冠,半球面,大于半球面的球冠及整个球面都是适用的。当 $h=2R$ 时,就是球面的面积 $4\uppi R^2$。

\begin{example}
  运油车的油罐是由一个圆筒与两个相同的球冠形部分组成的。油罐的尺寸如\cref{fig:2-53}(单位:\unit{m})。求制造这样一个油罐需要多少平方米钢板(精确到 \qty{0.1}{m^2})。
\end{example}
\par\medskip\noindent
\begin{minipage}{0.45\linewidth}\parindent2em
\begin{solution}
设圆筒半径为 $r$,它也是球冠的底的半径。由平面几何知识可得
\begin{gather*}
  r^2=0.3(2\times 1.5-0.3)\\ 
  \therefore \quad r=\qty{0.9}{m}.
\end{gather*}
\end{solution}
\end{minipage}\hfill
\begin{minipage}{0.5\linewidth}
\begin{figurehere}
  \includegraphics{2-53.pdf}
  \caption{}\label{fig:2-53}
\end{figurehere}
\end{minipage}\par\medskip


油罐圆筒部分面积:
\[S_1=2\uppi rh_1=2\uppi\times 0.9\times 10\approx\qty{56.52}{m^2};\]

两个球冠部分的面积:
\[2S_2=2\times 2\uppi Rh_2=2\times 2\uppi\times 1.5\times 0.3\approx\qty{5.65}{m^2}\]

油罐的总面积:
\[S=S_1+2S_2=56.52+5.65\approx\qty{62.2}{m^2}\]

答:制造这样一个油罐要用钢板约 \qty{62.2}{m^2}。


\begin{example}
  我国土地面积约为 \qty{9.60e6}{km^2},大部分位于地球北温带。求我国领土是北温带面积的百分之几。
\end{example}

\begin{solution}
\cref{fig:2-54} 表示经过地球南、北极的一个截面。$N$ 是北极,点 $G$ 在赤道上,点 $A,B$ 在北回归线上,$C,D$ 在北极圈上。$CD\parallel AB$,$ON\perp AB$,$ON\perp CD$。
\par\medskip\noindent
\begin{minipage}{0.7\linewidth}\parindent2em
北温带的面积
\begin{align*}
  S_\text{北温带}&=S_{\text{球冠}ABN}-S_{\text{球冠}CDN}\\
  &=2\uppi R\cdot EN-2\uppi R\cdot FN\\
  &=2\uppi R(EN-FN)\\
  &=2\uppi R(OF-OE).
\end{align*}
\end{minipage}\hfill
\begin{minipage}{0.3\linewidth}
\begin{figurehere}
  \includegraphics{2-54.pdf}
  \caption{}\label{fig:2-54}
\end{figurehere}
\end{minipage}
\begin{align*}
  \because\quad OF-OE&=OC\sin OCF-OA\sin OAE\\
  &=\num{6.37e3}(\sin\ang{66.5}-\sin\ang{23.5})\\
  &=\num{6.37e3}(0.9171-0.3987)\\
  &\approx \qty{3.302e3}{km}\\
  \therefore\quad S_\text{北温带}&=2\times 3.1427\times\num{6.37e3}\times\num{3.302e3}\\
  &\approx\qty{1.32e8}{km^2}.
\end{align*}
\[\frac{\num{9.60e6}\times 100}{\num{1.32e8}\times 100}\approx 7.27\%\]

答:我国领土约是北温带面积的 7.27\%。
\end{solution}

\begin{Practice}
  \begin{question}
    \item\label{prac:2-13-1}求证:$S_\text{球冠}=\uppi(r^2+h^2)$,其中 $r$ 是球冠的底的半径,$h$ 是球冠的高。
    \begin{figurehere}
      \begin{minipage}{\linewidth}\centering
        \includegraphics{pr2-13-1.pdf}
        \caption*{(第~\ref{prac:2-13-1}~题图)}
      \end{minipage}
    \end{figurehere}
    \item 有一条半径为 $R$ 的弧,度数是 \ang{120},它绕经过弧的中点的直径旋转得到一个球冠。求这个球冠的面积。
  \end{question}
\end{Practice}


\subsubsection{旋转面和旋转体}
前面我们学过的圆柱、圆锥、圆台的侧面,球面及球冠等,都是平面内的一条曲线绕一条定直线旋转而成的。
\par\medskip\noindent
\begin{minipage}{0.65\linewidth}\parindent2em
一条平面曲线(包括直线)绕它所在的平面内的一条定直线旋转所形成的曲面叫做\Concept{旋转面}。这条定直线叫做\Concept{旋转轴}。无论旋转到什么位置,这条曲线都叫做旋转面的\Concept{母线}。\cref{fig:2-55} 中,直线 $a$ 是旋转轴,曲线 $l$(不论旋转到什么位置)是母线。

如果母线是与旋转轴平行的直线,那么形成的旋转面叫做\Concept{圆柱面}(\cref{fig:2-56a})。如果母线是和旋转轴斜交的直线,那么形成的旋转面叫做\Concept{圆锥面}(\cref{fig:2-56b}),这时,母线和轴的交点叫做\Concept{圆锥面的顶点}。以前学过的%圆柱、圆锥、圆台的侧面可分别看成是圆柱面、圆锥面被垂直于轴的平面截得的一部分。
\end{minipage}\hfill 
\begin{minipage}{0.3\linewidth}
\begin{figurehere}
  \includegraphics{2-55.pdf}
  \caption{}\label{fig:2-55}
\end{figurehere}
\end{minipage}

\par\medskip\noindent
% 如果母线是与旋转轴平行的直线,那么形成的旋转面叫做\Concept{圆柱面}(\cref{fig:2-56a})。如果母线是和旋转轴斜
%交的直线,那么形成的旋转面叫做\Concept{圆锥面}(\cref{fig:2-56b}),
% 这时,母线和轴的交点叫做\Concept{圆锥面的顶点}。以前学过
的圆柱、圆锥、圆台的侧面可分别看成是圆柱面、圆锥面被垂直于轴的平面截得的一部分。

如果一个圆,绕同一平面内与它不相交的一条直线旋转,那么形成的旋转面叫做\Concept{环面}(\cref{fig:2-56c})。

\begin{figure}
  \begin{minipage}[b]{0.28\linewidth}\centering
    \includegraphics{2-56a.pdf}
    \subcaption{}\label{fig:2-56a}
  \end{minipage}
  \begin{minipage}[b]{0.28\linewidth}\centering
    \includegraphics{2-56b.pdf}
    \subcaption{}\label{fig:2-56b}
  \end{minipage}
  \begin{minipage}[b]{0.4\linewidth}\centering
    \includegraphics{2-56c.pdf}
    \subcaption{}\label{fig:2-56c}
  \end{minipage}
  \caption{}\label{fig:2-56}
\end{figure}

封闭的旋转面围成的几何体,叫做\Concept{旋转体}。这时,旋转面的轴也叫\Concept{旋转体的轴}。圆柱、圆锥、圆台、球都是旋转体。环面所围成的几何体也是旋转体,它叫做\Concept{环体},简称\Concept{环}。例如充气的车轮内胎就呈环体形。


\begin{Practice}
  \begin{question}
    \item 举出一些旋转面和旋转体的实例。
    \item 圆柱和圆柱面、圆锥和圆锥面有何区别?
  \end{question}
\end{Practice}

\begin{Exercise}
  \begin{question}
    \item 求证:球的任意两个大圆互相平分。
    \item 在半径是 \qty{13}{cm} 的球面上有 $A,B,C$ 三点,$AB=\qty{6}{cm}$,$BC=\qty{8}{cm}$,$CA=\qty{10}{cm}$。求经过这三点的截面和球心 $O$ 的距离。
    \item 在半径是 $r$ 的球面上有两点 $A,B$,半径 $OA$ 和 $OB$ 的夹角是 $n$\unit{\degree}($n<180$)。求 $A,B$ 两点间的球面距离。
    \item 在北纬 \ang{30} 圈上有甲、乙两地,它们的经度相差 \ang{120},计算这两地间的纬度线长。
    \item 在赤道上,东经 \ang{140} 与西经 \ang{130} 的海面上有两点 $A,B$。求 $A,B$ 两点的球面距离是多少海里?
    \item 已知球的大圆的周长是 \qty{80}{cm}。求这个球的表面积。
    \item 在球心的同一侧有相距 \qty{9}{cm} 的两个平行截面,它们的面积各为 $49\uppi$\,\unit{cm^2} 和 $400\uppi$\,\unit{cm^2}。求球的表面积。
    \item 水箱用的胶质浮球,是由两个半球面和一个圆柱筒贴合而成。已知球的半径是 \qty{3}{cm},圆柱筒长 \qty{2}{cm}。要在这样 2500 个浮球上涂一层胶质。如果每平方米需要涂胶 \qty{100}{g},共需胶多少?
    \item 半径是 \qty{4}{cm} 的球面,被一个平面截得的截面半径是 \qty{2}{cm},求所截得的球冠的面积。
    \item\label{exec:11-10}我国第一颗人造地球卫星的远地点距地面 \qty{2384}{km},在这时约有多少平方公里上的人能看到这颗卫星?
    \begin{figurehere}
      \begin{minipage}[b]{0.45\linewidth}\centering
        \includegraphics{ex11-10.pdf}
        \caption*{(第~\ref{exec:11-10}~题图)}
      \end{minipage}
      \begin{minipage}[b]{0.45\linewidth}\centering
        \includegraphics{ex11-12.pdf}
        \caption*{(第~\ref{exec:11-12}~题图)}
      \end{minipage}
    \end{figurehere}
    \item 有直径为 \qty{10}{cm} 的球,以它的一条直径为轴,钻一个直径为 \qty{6}{cm} 的圆孔,求这个球的球面剩余部分的面积。
    \item\label{exec:11-12}球面夹在两个平行截面间的部分叫做\Concept{球带},两个平行截面间的距离叫做\Concept{球带的高}。如果球的半径是 $R$,球带的高是 $h$,求证:$S_\text{球带}=2\uppi Rh$。
    \item 利用上题结果,计算地球上热带的面积。
  \end{question}
\end{Exercise}

\section{多面体和旋转体的体积}
\subsection{体积的概念与公理}
在生产建设和科学实验中,经常会遇到关于物体体积的问题,这些问题与各种几何体的体积有关。这一节我们就来研究这些几何体的体积问题。

几何体占有空间部分的大小叫做它的\Concept{体积}。
\par\medskip\noindent
\begin{minipage}{0.55\linewidth}\parindent2em
同度量长度、面积一样,要度量一个几何体的体积,首先要选取一个单位体积作为标准,然后求出几何体的体积是单位体积的多少倍,这个倍数就是这个几何体的体积的数值。通常取棱长等于单位长度(例如 \qty{1}{cm}、\qty{1}{m} 等)的正方体的体积作为\Concept{体积单位}(\cref{fig:2-57})。
\end{minipage}\hfill
\begin{minipage}{0.4\linewidth}
\begin{figurehere}
  \includegraphics{2-57.pdf}
  \caption{}\label{fig:2-57}
\end{figurehere}
\end{minipage}\par\medskip

作为推算体积的基础,我们把下面的两个事实当作公理。
\begin{Theorem}{公理 5}
  长方体的体积等于它的长、宽、高的积。
  \[\tcbhighmath{V_\text{长方体}=abc.}\]
\end{Theorem}

从这个公理,可以直接得到下面的推论:
\begin{Deduction}{推论 1}
  长方体的体积等于它的底面积 $S$ 和高 $h$ 的积。
  \[V_\text{长方体}=Sh.\]
\end{Deduction}
\begin{Deduction}{推论 2}
  正方体的体积等于它的棱长 $a$ 的立方。
  \[V_\text{正方体}=a^3.\]
\end{Deduction}
\begin{Theorem}{公理 6}
  夹在两个平行平面间的两个几何体,被平行于这两个平面的任意平面所截,如果截得的两个截面的面积总相等,那么这两个几何体的体积相等。
\end{Theorem}

\cref{fig:2-58} 表示夹在平行平面 $\alpha,\beta$ 之间的两个形状不同的几何体,被平行于平面 $\alpha,\beta$ 的任意一个平面所截,如果截面 $P$ 和 $Q$ 的面积总相等,那么它们的体积一定相等。

\begin{figure}
  \includegraphics{2-58.pdf}
  \caption{}\label{fig:2-58}
\end{figure}

例如,取一摞书或一摞纸张堆放在桌面上,将它如\cref{fig:2-59} 那样改变一下形状,这时高度没有改变,每页纸的面积也没有改变,因而这摞书或纸张的体积于变形前相等。

\begin{figure}
  \includegraphics{2-59.pdf}
  \caption{}\label{fig:2-59}
\end{figure}

我国古代数学家祖暅,早在公元五世纪,就在实践的基础上,总结出这个公理,并首先使用这个公理证明了球的体积公式,因而我们把它叫做\Concept{祖暅原理}。在欧洲直到十七世纪,才有意大利的卡发雷利提出这个事实。

用以上两个公理作基础,我们就可以求出柱、锥、台、球等的体积。

\begin{Practice}
  \begin{question}
    \item 用棱长为 1 的正方体的体积作为体积单位,\cref{fig:2-57} 中长方体体积的数值为 24。假如将体积单位改用棱长为 2 的正方体的体积,这个长方体的体积变为多少?为什么?
    \item\label{prac:2-15-2}把夹在两条平行线间的两个平面图形的面积先等的条件,用祖暅原理的形式叙述出来。并根据矩形面积公式,求平行四边形的面积公式。
    \begin{figurehere}
      \begin{minipage}{\linewidth}\centering
        \includegraphics{pr2-15-2.pdf}
        \caption*{(第~\ref{prac:2-15-2}~题图)}
      \end{minipage}
    \end{figurehere}
  \end{question}
\end{Practice}

\subsection{棱柱、圆柱的体积}
设有底面积都等于 $S$,高都等于 $h$ 的任意一个棱柱和一个圆柱,取一个与它们底面积相等、高也相等的长方体,使它们的下底面在同一个平面 $\alpha$ 上。因为它们的上底面和下底面平行,并且高都相等,所以它们的上底面都在和平面 $\alpha$ 平行的同一个平面内(\cref{fig:2-60})。

\begin{figure}
  \includegraphics{2-60.pdf}
  \caption{}\label{fig:2-60}
\end{figure}

用和平面 $\alpha$ 平行的任意平面去截它们时,所得的截面都和它们的底面分别全等,因而这些截面的面积都等于 $S$。根据祖暅原理,它们的体积相等。由于长方体的体积等于它的底面积和高的乘积,于是我们得到下面的定理:
\begin{Theorem}{定理}
  柱体(棱柱、圆柱)的体积等于它的底面积 $S$ 和高 $h$ 的积。即
\[\tcbhighmath{V_\text{柱体}=Sh.}\]
\end{Theorem}
\begin{Deduction}{推论}
  底面半径是 $r$,高是 $h$ 的圆柱的体积是
  \[V_\text{圆柱}=\uppi r^2h.\]
\end{Deduction}
\par\medskip\noindent
\begin{minipage}{0.62\linewidth}\parindent2em
\begin{example}
  有一堆相同规格的六角螺帽毛坯(\cref{fig:2-61})共重 \qty{5.8}{kg}。已知底面六边形的边长是 \qty{12}{mm},高是 \qty{10}{mm},内孔直径是 \qty{10}{mm}。问约有毛坯多少个(铁的比重是 \qty{7.8}{g/cm^3})。
\end{example}
\end{minipage}
\begin{minipage}{0.35\linewidth}
\begin{figurehere}
  \includegraphics{2-61.pdf}
  \caption{}\label{fig:2-61}
\end{figurehere}
\end{minipage}\par\medskip

\begin{solution}
六角螺帽毛坯的体积是一个正六棱柱的体积与一个圆柱的体积的差。
\begin{gather*}
  V_\text{正六棱柱}=\frac{\sqrt{3}}{4}\times12^2\times 6\times 10\approx \qty{3.74e3}{mm^3}\\ 
  V_\text{圆柱}=3.14\times\left(\frac{10}{2}\right)^2\times 10\approx \qty{0.785e3}{mm^3}.
\end{gather*}

毛坯的体积
\begin{gather*}
  V=\num{3.74e3}-\num{0.785e3}\approx \qty{2.96e3}{mm^3}=\qty{2.96}{cm^3}.\\
  \num{5.8e3}\div(7.8\times 2.96)\approx\num{2.5e2}\,\text{(个)}.
\end{gather*}

答:这堆毛坯约有 250 个。
\end{solution}

\begin{example}
  三棱柱的底面是 $\triangle ABC$,$AB=\qty{13}{cm}$,$BC=\qty{5}{cm}$,$CA=\qty{12}{cm}$,侧棱 $AA'$ 的长是 \qty{20}{cm}。如果侧棱 $AA'$ 与底面所成的角是 \ang{60},求这个三棱柱的体积。
\end{example}
\begin{solution}
  设 $A'$ 在平面 $ABC$ 上的射影为 $H$。则 $A'H$ 是棱柱的高,$\angle A'AH=\ang{60}$(\cref{fig:2-62})。
\par\medskip\noindent
\begin{minipage}{0.65\linewidth}\parindent2em
  在 $\mathrm{Rt}\triangle A'AH$ 中,

  $\because\quad AA'=20$,

  $\therefore\quad A'H=AA'\sin\ang{60}=10\sqrt{3}$。

  在 $\triangle ABC$ 中,$AB=13$,$BC=5$,$CA=12$,

  $\because\quad AB^2=BC^2+CA^2$,

  $\therefore\quad \angle C=\ang{90}$。

  $\therefore\quad S_{\triangle ABC}=\dfrac12 BC\cdot CA=\qty{30}{cm^2}$。
  根据柱体的体积公式,得
  \[ V=Sh=30\times10\sqrt{3}=300\sqrt{3}\,\unit{cm^3}.\]
\end{minipage}\hfill
\begin{minipage}{0.3\linewidth}
\begin{figurehere}
  \includegraphics{2-62.pdf}
  \caption{}\label{fig:2-62}
\end{figurehere}
\end{minipage}\par\medskip
\end{solution}

\begin{Practice}
  \begin{question}
    \item 一个正方体和一个圆柱等高,并且侧面积相等。比较它们的体积哪个大?大多少?
    \item 一个直平行六面体的侧棱长 \qty{9}{cm},底面两条相邻边的长是 \qty{7}{cm} 和 \qty{11}{cm},夹角为 \ang{45}。求它的体积。
  \end{question}
\end{Practice}

\begin{Exercise}
  \begin{question}
    \item 已知长方体形的铜块长、宽、高分别是 \qty{2}{cm}、\qty{4}{cm}、\qty{8}{cm},将它熔化后铸成一个正方体形的铜块。求铸成的铜块的棱长(不计损耗)。
    \item 一个长方体的长、宽、高的比为 $1:2:3$,对角线长是 $2\sqrt{14}$\,\unit{cm}。求它的体积。
    \item\label{exec:12-3}如图,将四棱柱底面的边三等分,过三等分点用平行于侧棱的平面截去四个三棱柱,得到一个八棱柱。这个八棱柱的体积是原四棱柱体积的几分之几?
    \begin{figurehere}
      \begin{minipage}[b]{0.3\linewidth}\centering
        \includegraphics{ex12-3.pdf}
        \caption*{(第~\ref{exec:12-3}~题图)}
      \end{minipage}
      \begin{minipage}[b]{0.65\linewidth}\centering
        \includegraphics{ex12-6.pdf}
        \caption*{(第~\ref{exec:12-6}~题图)}
      \end{minipage}
    \end{figurehere}
    \item 将一个正三棱柱形的木块,旋成与它等高并且尽可能大的圆柱形。旋去的部分是三棱柱体积的几分之几?
    \item 求证:经过长方体相对的两个面的中心的任意平面,把长方体分成体积相等的两个柱体。
    \item\label{exec:12-6}要修建铁路,路基如图(单位:\unit{m}),修建每 \qty{1}{km} 铁路需要碎石多少方(\unit{m^3})? 
    \item 在一块平地上,计划修建一条水渠,渠道长 \qty{1.5}{km},渠道断面是梯形,梯形两底分别是 \qty{1.8}{m}、\qty{0.8}{m},高是 \qty{0.6}{m}。如果每人一天挖土 \qty{2}{m^3},完成这条渠道需要多少个工?
    \item 拟修建堤坝 \qty{1.5}{km}。坝的断面是梯形,上底宽 \qty{4}{m},迎水坡宽 \qty{20}{m},迎水坡、背水坡与水平面分别成 \ang{30}、\ang{45} 的二面角。一台推土机每天推土 \qty{80}{m^3},用 5 台推土机几天完成?
    \item 我国万吨水压机上,有四根圆筒形钢柱,高 \qty{18}{m},内径 \qty{0.4}{m},外径 \qty{1}{m}。求这四根钢柱的重量(钢的比重是 \qty{7.8}{g/cm^3})。
    \item 求证:底面是梯形的直棱柱的体积,等于两个平行侧面面积的和与这两个侧面间距离的积的一半。
    \item 已知正六棱柱较长的一条对角线长是 \qty{13}{cm},侧面积是 \qty{180}{cm^2}。求这个棱柱的体积。
    \item 一根圆木料,长 \qty{3.0}{m},直径 \qty{0.8}{m},距圆木的轴 \qty{0.2}{m} 且平行于轴锯去一片,求剩余木料有多少立方米。
  \end{question}
\end{Exercise}
\subsection{棱锥、圆锥的体积}
前一节,我们用祖暅原理求出了柱体(棱柱、圆柱)的体积。为了求出锥体的体积公式,我们首先研究等底面积等高的任意两个锥体体积之间的关系。

取任意两个锥体,设它们的底面积都是 $S$,高都是 $h$(\cref{fig:2-63})。

\begin{figure}
  \includegraphics{2-63.pdf}
  \caption{}\label{fig:2-63}
\end{figure}

把这两个锥体放在同一平面 $\alpha$ 上,这时它们的顶点都在和平面 $\alpha$ 平行的同一个平面内。用平行于平面 $\alpha$ 的任意平面去截它们,截面分别与底面相似。设截面和顶点的距离是 $h_1$,截面面积分别是 $S_1$、$S_2$,那么
\begin{align*}
  \because \qquad \frac{S_1}{S}&=\frac{h_1^2}{h^2}, & \frac{S_2}{S}&=\frac{h_1^2}{h^2}\\
  \therefore \qquad \frac{S_1}{S}&=\frac{S_2}{S}, & S_1&=S_2.
\end{align*}
根据祖暅原理,这两个锥体的体积相等。由此我们得到下面的定理:
\begin{Theorem}{定理}
  等底面积等高的两个锥体的体积相等。
\end{Theorem}

现在,我们来证明三棱锥的体积公式。

\begin{Theorem}{定理}
  如果三棱锥的底面积是 $S$,高是 $h$,那么它的体积是
  \[\tcbhighmath{V_\text{三棱锥}=\frac13Sh.}\]
\end{Theorem}

已知:三棱锥 1($A'\text{-}ABC$)的底面积是 $S$,高是 $h$。求证:$V_\text{三棱锥}=\dfrac13Sh$。

\begin{proof}
把三棱锥 1($A'\text{-}ABC$)以 $\triangle ABC$ 为底面、$AA'$ 为侧棱补成一个三棱柱,然后再把这个三棱柱分割成三个三棱锥,就是三棱锥 1 和另两个三棱锥 2、3(\cref{fig:2-64})。
\begin{figure}
  \includegraphics{2-64.pdf}
  \caption{}\label{fig:2-64}
\end{figure}

三棱锥 1、2 的底 $\triangle ABA'$、$\triangle B'A'B$ 的面积相等,高也相等(顶点都是 $C$);三棱锥 2、3 的底 $\triangle BCB'$、$\triangle C'B'C$ 的面积相等,高也相等(顶点都是 $A'$),
\begin{gather*}
  \therefore \quad V_1=V_2=V_3=\frac13V_\text{三棱柱}.\\
  \because \quad V_\text{三棱柱}=Sh,\\
  \therefore \quad V_\text{三棱锥}=\dfrac13Sh.
\end{gather*}
\end{proof}

最后,因为和一个三棱锥等底面积等高的任何锥体都和这个三棱锥的体积相等,所以我们得到下面的定理:
\begin{Theorem}{定理}
  如果一个锥体(棱锥、圆锥)的底面积是 $S$,高是 $h$,那么它的体积是
  \[\tcbhighmath{V_\text{锥体}=\frac13Sh.}\]
\end{Theorem}
\begin{Deduction}{推论}
  如果圆锥的底面半径是 $r$,高是 $h$,那么它的体积是
  \[V_\text{圆锥}=\frac13\uppi r^2h.\]
\end{Deduction}

\begin{example}
  如\cref{fig:2-65},已知:三棱锥 $A$-$BCD$ 的侧棱 $AD$ 垂直于底面 $BCD$,侧面 $ABC$ 与底面所成的角为 $\theta$。

  求证:$V_\text{三棱锥}=\dfrac13S_{\triangle ABC}\cdot AD\cos\theta$。
\end{example}
\begin{proof}
  在平面 $BCD$ 内,作 $DE\perp BC$,垂足为 $E$,连结 $AE$,$DE$ 就是 $AE$ 在平面 $BCD$ 上的射影。根据三垂线定理,$AE\perp BC$。
  
  $\therefore\quad \angle AED=\theta$。
\par\noindent
\begin{minipage}{0.6\linewidth}\parindent2em
\begin{align*}
  V_\text{三棱锥}&=\frac13S_{\triangle BCD}\cdot AD\\
  &=\frac13\times\frac12BC\cdot ED\cdot AD\\
  &=\frac13\times\frac12BC\cdot AE\cos\theta \cdot AD\\
  &=\frac13S_{\triangle ABC}\cdot AD\cos\theta.
\end{align*}
\end{minipage}\hfill
\begin{minipage}{0.35\linewidth}
  \begin{figurehere}
    \includegraphics{2-65.pdf}
    \caption{}\label{fig:2-65}
  \end{figurehere}
\end{minipage}\par\medskip
\end{proof}

\begin{example}
  一块正方形薄铁板的边长是 \qty{22}{cm},以它的一个顶点为圆心,边长为半径画弧,沿弧剪下一个扇形。用这块扇形铁板围成一个圆锥筒,求它的容积(保留两位有效数字)。
\end{example}
\begin{solution}
  如\cref{fig:2-66},扇形弧长是 $\dfrac14\cdot 44\uppi=11\uppi$。因此,所作的圆锥筒底的周长 $2\uppi r=11\uppi$。解得 $r=5.5$。
  \par\medskip\noindent
  \begin{minipage}{0.55\linewidth}\parindent2em
  因为母线长是 \qty{22}{cm},所以圆锥的高
  \begin{align*}
    h&=\sqrt{22^2-5.5^2}\approx 21.3.\\
    V_\text{圆锥}&=\frac13\uppi\times 5.5^2\times 21.3.\\
    &\approx\qty{6.7e2}{cm^3}.
  \end{align*}

  答:所求圆锥筒的容积约为 \qty{670}{cm^3}。
  \end{minipage}\hfill
  \begin{minipage}{0.4\linewidth}
    \begin{figurehere}
      \includegraphics{2-66.pdf}
      \caption{}\label{fig:2-66}
    \end{figurehere}
  \end{minipage}
\end{solution}\par\medskip




\begin{Practice}
  \begin{question}
    \item\label{prac:2-17-1} 如图,将长方体沿相邻三个面的对角线截去一个三棱锥。这个三棱锥的体积是长方体体积的几分之几?
    \begin{figurehere}
      \begin{minipage}{\linewidth}\centering
        \includegraphics{pr2-17-1.pdf}
        \caption*{(第~\ref{prac:2-17-1}~题图)}
      \end{minipage}
    \end{figurehere}
    \item 已知:圆锥的底面周长是 $c$,高是 $h$。求证:这个圆锥的体积 $V$ 可以近似地表示为 $V\approx\left(\dfrac{c}{6}\right)^2h$。
  \end{question}
\end{Practice}

\begin{Exercise}
  \begin{question}
    \item\label{exec:13-1}从一个正方体中,如图那样截去四个三棱锥后,得到一个正三棱锥 $A$-$BCD$。求它的体积是正方体体积的几分之几?
    \begin{figurehere}
      \begin{minipage}[b]{0.45\linewidth}\centering
        \includegraphics{ex13-1.pdf}
        \caption*{(第~\ref{exec:13-1}~题图)}
      \end{minipage}
      \begin{minipage}[b]{0.45\linewidth}\centering
        \includegraphics{ex13-4.pdf}
        \caption*{(第~\ref{exec:13-4}~题图)}
      \end{minipage}
    \end{figurehere}
    \item 在下列情况下,正棱锥的体积有什么变化?
    \begin{enumerate}[itemindent=2.4em]
      \item 高和底面的边长都增为原来的 $n$ 倍;
      \item 高增为原来的 $n$ 倍,底面边长缩为原来的 $\dfrac1n$。
    \end{enumerate}
    \item 已知下列各正棱锥的底面边长是 $a$,侧棱长是 $b$,求它的体积:
    \begin{tasks}(3)
      \task 正三棱锥;
      \task 正四棱锥;
      \task 正六棱锥。
    \end{tasks}
    \item\label{exec:13-4}在仓库一角有谷一堆,呈 $\dfrac14$ 圆锥形(如图)。量得底面弧长为 \qty{2.8}{m},母线长为 \qty{2.2}{m},这堆谷重约多少(谷的比重:\qty{720}{kg/m^3})?
    \item 有一铜制工件,它的下部呈正四棱柱形,顶部是一个以正四棱柱的上底为底的正四棱锥形。柱的底面边长是 \qty{50}{mm},高是 \qty{40}{mm},锥的侧面呈正三角形。求这个工件的重量(铜的比重是 \qty{8.9}{g/cm^3}) 
    \item 三棱锥的三个侧面互相垂直,它们的面积分别为 \qty{6}{m^2}、\qty{4}{m^2} 和 \qty{3}{m^2}。求它的体积。
    \item 从一块薄铁板上,裁下一个半径为 \qty{24}{cm},圆心角为 \ang{120} 的扇形,再围成一个圆锥筒,求这个圆锥筒的容积(保留两位有效数字)。
    \item 圆锥的体积是 \qty{22.4}{cm^3},轴和母线所成的角是 \ang{48;15}。求它的高(保留三位有效数字)。
    \item 求证:棱锥被平行于底面的平面截得的小棱锥的体积和原来棱锥的体积的比,等于它们的高的立方比。
  \end{question}
\end{Exercise}

\subsection{棱台、圆台的体积}
我们已知,棱台、圆台分别是棱锥、圆锥用平行于底面的平面截去一个锥体得到的。因此,台体的体积可以用两个锥体的差来计算。

\begin{figure}
  \includegraphics{2-67.pdf}
  \caption{}\label{fig:2-67}
\end{figure}

设任意台体(棱台或圆台)的上、下底面的面积分别是 $S'$、$S$,高是 $h$。截得台体时去掉的锥体的高是 $x$,去掉的锥体和原来的锥体的体积分别是 $V'$、$V$(\cref{fig:2-67})。这时,
\[ V'=\frac13S'x,\quad V=\frac13S(h+x),\]
所以台体的体积
\begin{align*}
  V_\text{台体}&=V-V'=\frac13S(h+x)-\frac13S'x\\
  &=\frac13[Sh+(S-S')x].
\end{align*}
因为台体上、下底面相似,所以
\begin{gather*}
\frac{S'}{S}=\frac{x^2}{(h+x)^2},\quad \frac{\sqrt{S'}}{\sqrt{S}}=\frac{x}{h+x}.\\ 
x=\frac{\sqrt{S'}h}{\sqrt{S}-\sqrt{S'}}.
\end{gather*}
代入上式,得
\begin{align*}
  V_\text{台体}&=\frac13h\left[S+(S-S')\frac{\sqrt{S'}}{\sqrt{S}-\sqrt{S'}}\right]\\
  &=\frac13h\left[S+\sqrt{S'}(\sqrt{S}+\sqrt{S'})\right]\\
  &=\frac13h\left[S+\sqrt{SS'}+S'\right].
\end{align*}

由此我们得到下面的定理:
\begin{Theorem}{定理}
  如果台体(棱台、圆台)的上、下底面的面积分别是 $S'$、$S$,高是 $h$,那么它的体积是
  \[\tcbhighmath{V_\text{台体}=\frac13 h(S+\sqrt{SS'}+S').}\]
\end{Theorem}

\begin{Deduction}{推论}
  如果圆台的上、下底面半径分别是 $r'$、$r$,高是 $h$,那么它的体积是
  \[ V_\text{圆台}=\dfrac13\uppi h(r^2+rr'+r'^2).\]
\end{Deduction}

最后,我们注意到,在台体的体积公式重,如果设 $S'=S$,就得到柱体的体积公式 $V_\text{柱体}=Sh$;如果设 $S'=0$,就得到锥体的体积公式 $V_\text{锥体}=\dfrac13Sh$。

这样,柱体、锥体、台体的体积公式之间的关系,可表示如下图:
\begin{figurehere}
  \includegraphics{2-flowchart3.pdf}
\end{figurehere}

\begin{example}
有一个正四棱台形油槽,可以装煤油 \qty{190}{L},假如它的两底面边长分别等于 \qty{60}{cm} 和 \qty{40}{cm}。求它的深度。
\end{example}

\begin{solution}
\begin{gather*}
  \because \quad \text{上底面面积} S'=40^2=1600,\\
  \phantom{\because \quad }\text{下底面面积} S'=60^2=3600,\\
  \sqrt{S\cdot S'}=\sqrt{40^2\times 60^2}=2400,\\
  \therefore \quad V=\frac13h(3600+2400+1600)=\frac{7600}{3}h.
\end{gather*}

由已知 $V=\qty{190}{L}=\qty{190000}{cm^3}$,
\[ \therefore h=\frac{3\times 190000}{7600}=\qty{75}{cm}.\]

答:油槽深度是 \qty{75}{cm}。
\end{solution}

\begin{Practice}
  已知上、下地面边长分别是 $a$、$b$,高是 $h$。求下列正棱台的体积:
  \begin{tasks}(2)
    \task 正四棱台;
    \task 正六棱台。
  \end{tasks}
\end{Practice}

\subsection{拟柱体及其体积}
\par\medskip\noindent
\begin{minipage}{0.65\linewidth}\parindent2em
我们经常遇到堆放整齐的沙石堆、粪堆等体积的计算问题。虽然它们有两个面平行,但一般不是棱台。

所有的顶点都在两个平行平面内的多面体叫\Concept{拟柱体}。它在这两个平面内的面叫做\Concept{拟柱体的底面},其余各面叫做\Concept{拟柱体的侧面}。两底面之间的距离叫做\Concept{拟柱体的高}(\cref{fig:2-68})。
\end{minipage}\hfill
\begin{minipage}{0.3\linewidth}
\begin{figurehere}
  \includegraphics{2-68.pdf}
  \caption{}\label{fig:2-68}
\end{figurehere}
\end{minipage}\par\medskip

显然,拟柱体的侧面是三角形、梯形或平行四边形。

两底面是矩形,并且它们的对应边平行,这样的拟柱体叫\Concept{长方台}(\cref{fig:2-69a})。如果拟柱体的下底面是梯形或平行四边形,上底面变成了与下底面的平行边的线段,这样的拟柱体叫做\Concept{楔体}(\cref{fig:2-69b})。

\begin{figure}
  \begin{minipage}[b]{0.48\linewidth}\centering
    \includegraphics{2-69a.pdf}
    \subcaption{}\label{fig:2-69a}
  \end{minipage}
  \begin{minipage}[b]{0.48\linewidth}\centering
    \includegraphics{2-69b.pdf}
    \subcaption{}\label{fig:2-69b}
  \end{minipage}
  \caption{}\label{fig:2-69}
\end{figure}

利用棱锥体积公式,可以求出拟柱体的体积。
\begin{Theorem}{定理}
  如果拟柱体的上、下底面的面积为 $S'$、$S$,中截面的面积为 $S_0$,高为 $h$,那么它的体积是
  \[\tcbhighmath{V_\text{拟柱体}=\frac16h(S+4S_0+S').}\]
\end{Theorem}
\begin{proof}
  如\cref{fig:2-70},在拟柱体 $ABCDEF\text{-}A'B'C'D'$ 的中截面 $A_1D_1$ 内任取一点 $P$,并且把它和这个拟柱体的各个顶点分别连结起来。这样,把拟柱体分成若干个以 $P$ 为顶点的棱锥,拟柱体的体积等于这些棱锥体积的和。

  我们把这些棱锥分成两类:一类是以拟柱体的底面为底的,一类是以拟柱体的侧面为底的。

  第一类的棱锥有两个,棱锥 $P\text{-}AC$ 和棱锥 $P\text{-}A'C'$。它们的体积分别是:
  \begin{gather*}
    V_{P\text{-}AC}=\frac13S\cdot \frac12h=\frac16hS,\\
    V_{P\text{-}A'C'}=\frac13S'\cdot \frac12h=\frac16hS'.
  \end{gather*}
  
  在第二类棱锥中,我们先求其中的一个,例如棱锥 $P\text{-}CC'$ 的体积。因为棱锥 $P\text{-}C_1D_1B'$ 和棱锥 $P\text{-}CC'$ 的底面在同一个平面上,顶点相同,所以它们的高相等。又因为梯形 $CDC'B'$ 的面积等于 $\triangle C_1D_1B'$ 的面积的 4 倍,(为什么?)所以
  \[ V_{P\text{-}CC'}=4V_{P\text{-}C_1D_1B'}.\]
  \par\medskip\noindent
  \begin{minipage}{0.55\linewidth}\parindent2em
  另一方面,
  \begin{align*}
    V_{P\text{-}C_1D_1B'}&=V_{B'\text{-}PC_1D_1}\\
    &=\frac13\cdot\frac{h}{2}\cdot S_{\triangle PC_1D_1}\\
    &=\frac16hS_{\triangle PC_1D_1},\\
    \therefore \quad V_{P\text{-}CC'}&=\frac16h\cdot 4S_{\triangle PC_1D_1}.
  \end{align*}

  同样可以证明:
  \begin{gather*}
    V_{P\text{-}DD'}=\frac16h\cdot 4S_{\triangle PD_1E_1},\\
    \cdots \cdots
  \end{gather*}
\end{minipage}\hfill
\begin{minipage}{0.4\linewidth}
  \begin{figurehere}
    \includegraphics{2-70.pdf}
    \caption{}\label{fig:2-70}
  \end{figurehere}
\end{minipage}\par\medskip

  把第二类棱锥的体积相加,得
  \[\frac16h(4S_{\triangle PC_1D_1}+4S_{\triangle PD_1E_1}+\cdots)=\frac16h\cdot 4S_0.\]
  \begin{align*}
    V_\text{拟柱体}&=\frac16hS+\frac16h\cdot 4S_0+\frac16hS'\\
    &=\frac16h(S+4S_0+S').
  \end{align*}
\end{proof}


棱柱、棱锥、棱台是特殊的拟柱体,它们的体积公式有下面的关系。

当拟柱体的上、下底面是对应边平行的全等多边形时,它就是棱柱,这时,$S'=S_0=S$,公式变为 $V=Sh$;

当拟柱体的上底面退缩成一点时,它就是棱锥,这时,$S'=0,S_0=\dfrac14S$,公式变为 $V=\dfrac13Sh$;

\medskip
当拟柱体上、下底面是对应边平行的相似多边形时,它就是棱台。根据\cref{subsec:frustum} \cref{exp:2-7} 可知,$2\sqrt{S_0}=\sqrt{S}+\sqrt{S'}$,即 $4S_0=S+2\sqrt{SS'}+S'$,得 $V=\dfrac13h(S+\sqrt{SS'}+S')$。

因此棱柱、棱锥、棱台得体积公式也都可写成
\[V=\frac16h(s'+4S_0+S).\]

\par\medskip\noindent
\begin{minipage}{0.55\linewidth}\parindent2em
\begin{example}
一草垛下部是倒长方台形,上部是以长方台的上底为底的楔体形。已知长方台上底面边长约为 \qty{8.4}{m} 和 \qty{4.2}{m},下底面边长约为 \qty{7.6}{m} 和 \qty{3.0}{m},高是 \qty{2.2}{m}。楔体形上面的棱长约为 \qty{5.8}{m},高约为 \qty{1.5}{m}。求这垛草的重量约多少千克(每立方米的草重约 \qty{150}{kg})。
\end{example}
\end{minipage}\hfill
\begin{minipage}{0.4\linewidth}
\begin{figurehere}
  \includegraphics{2-71.pdf}
  \caption{}\label{fig:2-71}
\end{figurehere}
\end{minipage}\par\medskip

\begin{solution}
长方台的中截面的边长分别是 $\dfrac12(8.4+7.6)$\,\unit{m} 和 $\dfrac12(4.2+3.0)$\,\unit{m}。
\begin{align*}
  \therefore \quad S_0&=\frac12(8.4+7.6)\times\dfrac12(4.2+3.0)\approx28.8\,(\unit{m^2}), \\ 
  V_\text{长方台}&=\dfrac16\times 2.2\times(8.4\times 4.2+4\times 28.8+7.6\times 3.0)\approx 63.5\,(\unit{m^3}).
\end{align*}

楔体是上底面面积为零的拟柱体,它的中截面边长是 $\dfrac12(8.4+5.8)$\,\unit{m} 和 $\dfrac12\times 4.2$\,\unit{m},
\begin{align*}
  \therefore \quad S_0&=\frac12(8.4+5.8)\times\frac12\times 4.2\approx 14.9\,(\unit{m^2}),\\
  V_\text{楔体}&=\frac16\times 1.5\times (4\times 14.9+8.4\times 4.2)\approx 23.7\,(\unit{m^3}).
\end{align*}

草垛体积 $V=63.5+23.7\approx 87\,(\unit{m^3})$。

重量 $W=150\times 87\approx \qty{1.3e4}{kg}$。

答:这垛草重约 \qty{1.3e4}{kg}。
\end{solution}




\begin{Practice}
  已知拟柱体的下底面积为 \qty{20}{cm^2},上底面积为 \qty{6}{cm^2},中截面面积为 \qty{12}{cm^2},高为 \qty{15}{cm}。求这个拟柱体的体积。
\end{Practice}

\begin{Exercise}
  \begin{question}
    \item 已知棱台两底面的面积分别为 \qty{245}{cm^2},\qty{80}{cm^2},截得这个棱台的棱锥的高是 \qty{35}{cm}。求这个棱台的体积。
    \item 两底面边长分别是 \qty{15}{m}、\qty{10}{m} 的正三棱台,它的侧面积等于两底面面积的和。求这个三棱台的体积。
    \item 已知棱台的体积是 \qty{76}{cm^3},高是 \qty{6}{cm},一个底面面积是 \qty{18}{cm^2}。求这个棱台的另一个底面面积。
    \item 棱台的高为 \qty{20}{cm},体积为 \qty{1720}{cm^3},两底面对应边的比值是 $5:8$。求两底面面积。
    \item\label{exec:14-5}已知过三棱台上底面的一边与一条侧棱平行的一个截面,它的两个顶点是下底面两边的中点。求棱台被分成两部分的体积的比。 
    \begin{figurehere}
      \begin{minipage}[b]{0.45\linewidth}\centering
        \includegraphics{ex14-5.pdf}
        \caption*{(第~\ref{exec:14-5}~题图)}
      \end{minipage}
      \begin{minipage}[b]{0.45\linewidth}\centering
        \includegraphics{ex14-12.pdf}
        \caption*{(第~\ref{exec:14-12}~题图)}
      \end{minipage}
    \end{figurehere}
    \item 圆台的高为 3,一个底面半径是另一个底面半径的 2 倍,母线与下底面所成的角为 \ang{45}。求它的体积。
    \item 圆台的体积是 $234\sqrt{3}\uppi$\,\unit{cm^3},侧面展开图是半圆环,它的大半径等于小半径的 3 倍。求这个圆台的上底面半径。
    \item 证明:当上、下底面半径的大小相近时,圆台的体积可近似地表示为 $V\approx\dfrac{1}{12}c^2h$,其中 $c$ 是中截面周长,$h$ 是高。
    \item 有一草垛,上部是圆锥形,下部是圆台形。圆锥高 \qty{0.7}{m},底面周长是 \qty{5.1}{m};圆台高 \qty{1.5}{m};下底面周长是 \qty{4}{m}。如果每立方米草重 \qty{150}{kg},利用上题结果,估算这个草垛地重量(保留一位有效数字)。
    \item 有一沙堆,它的上底面和下底面是互相平行的两个矩形,各侧面都是梯形,上底面的边长约是 \qty{5.2}{m} 和 \qty{3.4}{m},下底面的边长约是 \qty{7.3}{m} 和 \qty{4.2}{m},上下底面距离约是 \qty{2.5}{m}。求这堆沙有多少立方米?
    \item 有一碴石堆,它的下底呈矩形,长宽分别是 $a$\,\unit{m} 和 $b$\,\unit{m},上下底面互相平行,各侧面与地面成 \ang{45} 角,碴石堆高 $h$\,\unit{m}。求它的体积。
    \item\label{exec:14-12}楔体的底面 $ABCD$ 是边长为 \qty{50}{cm} 和 \qty{30}{cm} 的矩形,棱 $EF$ 平行于 $AB$,与底面距离是 \qty{60}{cm},长是 \qty{40}{cm}。求这个楔体的体积。
    \item 长方台的上底面边长是 $a$、$b$,下底面与它们平行的边长是 $a'$、$b'$,高是 $h$。求证:长方台的体积是
    \[ V=\frac16h[2(ab+a'b')+ab'+a'b].\]
  \end{question}
\end{Exercise}

\subsection{球的体积}\label{subsec:volume-ball}
和柱体、锥体一样,也可以应用祖暅原理推出球体的体积公式。

我们先研究半径为 $R$ 的半球,为了应用祖暅原理,需要找到一个能够求体积的几何体,使它和半球可夹在两个平行平面之间,当用平行于这两个平面的任意一个平面去截它们时,截得的截面面积总相等。

为此,我们取一个底面半径和高都等于 $R$ 的圆柱,从圆柱中挖去一个以圆柱的上底面为底面,下底面圆心为顶点的圆锥,把所得的几何体和半球放在同一个平面 $\alpha$ 上(\cref{fig:2-72})。因为圆柱的高等于 $R$,所以这个几何体和半球都夹在两个平行平面之间。

\begin{figure}
  \includegraphics{2-72.pdf}
  \caption{}\label{fig:2-72}
\end{figure}

用平行于平面 $\alpha$ 的任意一个平面去截这两个几何体,截面分别是圆面和圆环面。如果截平面与平面 $\alpha$ 的距离为 $l$,那么圆面半径 $r=\sqrt{R^2-l^2}$,圆环面的大圆半径为 $R$,小圆半径为 $l$(因为 $\triangle O'O_1B$ 是等腰三角形)。因此
\begin{gather*}
  S_\text{圆}=\uppi r^2=\uppi(R^2-l^2),\\ 
  S_\text{圆环}=\uppi R^2-\uppi l^2=\uppi(R^2-l^2),\\ 
  \therefore \quad S_\text{圆}=S_\text{圆环}.
\end{gather*}
根据祖暅原理,这两个几何体的体积相等,即
\begin{align*}
  \frac12V_\text{球}&=\uppi R^2\cdot R-\frac13\uppi R^2\cdot R\\
  &=\frac23\uppi R^3.\\
  \therefore \quad V_\text{球}&=\frac43\uppi R^3.
\end{align*}

由此,我们得到下面定理:
\begin{Theorem}{定理}
  如果球的半径是 $R$,那么它的体积是
  \[\tcbhighmath{V_\text{球}=\frac43\uppi R^3.}\]
\end{Theorem}

\begin{example}
  有一种空心钢球,重 \qty{142}{g},测得外径等于 \qty{5.0}{cm}。求它的内径(钢比重是 \qty{7.9}{g/cm^3})。
\end{example}
\begin{solution}
  设空心钢球的内径为 $2x$\,\unit{cm},那么钢球的重量是
  \begin{gather*}
    7.9\cdot\left[\frac43\uppi\cdot\left(\frac52\right)^3-\frac43\uppi x^3\right]=142,\\ 
    x^3=\left(\frac52\right)^3-\frac{142\times 3}{7.9\times 4\uppi}\approx 11.3.\\
    \therefore \quad x\approx 2.24,\\ 
    2x\approx 4.5\,(\unit{cm}).
  \end{gather*}

  答:空心钢球的内径约为 \qty{4.5}{cm}。
\end{solution}

\begin{Practice}
  \begin{question}
    \item 球面面积膨胀为原来的二倍,计算体积变为原来的几倍。
    \item 一个正方体的顶点都在球面上,它的棱长是 \qty{4}{cm}。求这个球的体积。
  \end{question}
\end{Practice}


\subsection{球缺的体积}
我们常见到钢桥、轮船、锅炉上有圆头的铆钉,铆钉的圆头具有球缺的形象。一个球被平面截下的一部分叫做\Concept{球缺}。截面叫做\Concept{球缺的底面},垂直于截面的直径被截下的线段长叫做\Concept{球缺的高}(\cref{fig:2-73})。
\par\medskip\noindent
\begin{minipage}{0.5\linewidth}\parindent2em
显然,球缺也可以看作是球冠和截面所围成的几何体。球缺的体积公式,可仿球的体积求法推出。设求得半径是 $R$,球缺的高是 $h$。利用\cref{subsec:volume-ball}的\cref{fig:2-72},因为已证明了平行于平面 $\alpha$ 的任意截面面积都分别相等,所以根据祖暅原理,球缺的体积
\end{minipage}\hfill
\begin{minipage}{0.45\linewidth}
\begin{figurehere}
  \includegraphics{2-73.pdf}
  \caption{}\label{fig:2-73}
\end{figurehere}
\end{minipage}\par
\begin{align*}
  V_\text{球缺}&=V_{\text{圆柱}LKNP}-V_{\text{圆柱}LABP}\\
  &=\uppi Rh^2-\frac13\uppi h[R^2+R(R-h)+(R-h)^2]\\
  &=\uppi Rh^2-\frac13\uppi h^3\\
  &=\frac13\uppi h^2(3R-h).
\end{align*}
由此我们得到下面的定理:
\begin{Theorem}{定理}
  如果球的半径是 $R$,球缺的高是 $h$,那么球缺的体积是
  \[\tcbhighmath{V=\dfrac13\uppi h^2(3R-h).}\]
\end{Theorem}
这个公式对于求半球,大于半球的球缺,整个球的体积都适用,例如,用 $R$ 代 $h$,就得到半球的体积 $\dfrac23\uppi R^3$。

\begin{example}
  钢铆钉钉头呈球缺形,钉身呈圆柱形,尺寸如\cref{fig:2-74}(单位:\unit{mm})。已知钢的比重是 \qty{7.8}{g/cm^3},求铆钉的重量(精确到 \qty{1}{g})。
\end{example}
\begin{solution}
  钉头的体积是
  \par\medskip\noindent
  \begin{minipage}{0.65\linewidth}\parindent2em
  \begin{align*}
    V_\text{球缺}&=\frac13\uppi h^2(3R-h)\\
                 &=\frac13\times 3.14\times 6^2\times(3\times 10-6)\\
                 &\approx 904\,(\unit{mm^3}).
  \end{align*}
  \end{minipage}\hfill
  \begin{minipage}{0.35\linewidth}
  \begin{figurehere}
    \includegraphics{2-74.pdf}
    \caption{}\label{fig:2-74}
  \end{figurehere}
  \end{minipage}\par\medskip
  钉身的体积是
  \begin{align*}
    V_\text{圆柱}&=\uppi R_1^2h_1\\
                 &=3.14\times 5^2\times 20\\
                 &\approx 1570\,(\unit{mm^3}).
  \end{align*}
  所以铆钉的重量是 $7.8\times\dfrac{904+1570}{1000}\approx 19\,(\unit{g})$。

  答:铆钉的重量约 \qty{19}{g}。
\end{solution}


\begin{example}
  已知:球缺底面半径是 $r$,高是 $h$。求证:球缺的体积是
  \[ V_\text{球缺}=\frac16\uppi h(3r^2+h^2).\]
\end{example}
\begin{proof}
  设截得球缺的球的半径是 $R$,由直角三角形 $OAH$(\cref{fig:2-75}),
\par\medskip\noindent
\begin{minipage}{0.6\linewidth}\parindent2em
\begin{gather*}
  r^2=R^2-(R-h)^2=2hR-h^2,\\
  R=\frac{r^2}{2h}+\frac{h}{2}.
\end{gather*}

代入球缺体积公式,得
\begin{align*}
  V_\text{球缺}&=\uppi h^2\left(\frac{r^2}{2h}+\frac{h}{2}-\frac{h}{3}\right)\\
    &=\frac16\uppi h(3r^2+h^2).
\end{align*}
\end{minipage}\hfill
\begin{minipage}{0.35\linewidth}
\begin{figurehere}
  \includegraphics{2-75.pdf}
  \caption{}\label{fig:2-75}
\end{figurehere}
\end{minipage}
\end{proof}\par\medskip

在生产和生活中所遇到的物体,形状虽然比较复杂,但是很多都可以看作是由柱体、锥体、台体、球体、球缺等组合(例如铆钉)或切割(例如螺帽)而成的。所以,我们能求这些几何体的体积,就能求那些形状比较复杂的物体的体积。然而,更复杂的体积,例如环体的体积,还要在以后用积分法去解决。

\begin{Practice}
  \begin{question}
    \item 球缺的高是球的直径的 $\dfrac{1}{10}$。求它们体积的比。
    \item 球缺的底面半径是球的半径的 $\dfrac12$。体积是球的几分之几?
  \end{question}
\end{Practice}

\begin{Exercise}
  \begin{question}[itemsep=3pt]
    \item 如果球的大圆面积增为原来的 100 倍,球的体积有什么变化?
    \item 铜球由于热膨胀而使半径增加 $\dfrac{1}{1000}$,它的体积增加几分之几(精确到 0.001)?
    \item 三个球的半径的比是 $1:2:3$。求证:其中最大的一个球的体积是另两个球的体积和的 3 倍。
    \item 火星的直径约是地球的一半。地球体积是火星体积的几倍?地球半径约是 \qty{6370}{km},地球和火星的体积各是多少?
    \item 木星表面积约是地球的 120 倍。它的体积约是地球的多少倍?
    \item 如果一个圆柱和一个圆锥的底面直径和高都与球的直径相等。求证:圆柱、球、圆锥体积的比是 $3:2:1$。
    \item 一个多面体的各面都与一个球相切。求证:多面体的体积等于它的表面积与球的半径的积的 $\dfrac13$。
    \item 球缺的体积是 $\dfrac\uppi3$\,\unit{cm^3},它的高是 $\dfrac12$\,\unit{cm}。求截得球缺的球的半径。
    \item\label{exec:15-9}运油车的油罐如图(单位:\unit{m}),油罐能装油多少吨(油的比重是 \qty{0.85}{g/cm^3})
    \begin{figurehere}
      \begin{minipage}{\linewidth}\centering
        \includegraphics{ex15-9.pdf}
        \caption*{(第~\ref{exec:15-9}~题图)}
      \end{minipage}
    \end{figurehere}
    \item 球的半径为 \qty{3}{cm},在其正中钻一个半径为 \qty{1}{cm} 的圆孔,求剩余部分的体积。
    \item 一个木球浮于水中,在水面上的球缺高为 \qty{2}{cm},底面半径为 \qty{8}{cm}。求这个木球的重量。
  \end{question}
\end{Exercise}

\section*{小结}
\begin{enumerate}[C、,itemindent=4.5em]
  \item 本章的主要内容是多面体和旋转体中常见的柱、锥、台、球的概念、性质、直观图的画法以及面积、体积的计算。重点研究了应用比较广泛的直棱柱、正棱锥、正棱台、圆柱、圆锥、圆台、球和球缺。
  \item 这些几何体的性质都是在第一章线面关系的基础上由定义推出来的。这些性质包括:它们的棱、面的性质;平行于底面的截面的性质;经过侧棱(或高线、轴线)的截面(或它的一部分)的性质。通过这样的研究,我们对这些几何体就有了一个比较全面的认识。
  \item 本章介绍了两种直观图的画法:斜二测和正等测。画图时,可以根据情况任选一种。画多面体时,常用斜二测,画旋转体时,常用正等测。画多面体和旋转体组合图形时,多用正等测。这时,要注意不要两种方法混用。 
  \item 几种多面体和旋转体的表面积,除球面和球冠外,都是通过它们的展开图求得的。这些公式不但互相区别,而且互相联系。除前面讲过的关系外,直棱柱、正棱锥、正棱台、圆柱、圆锥、圆台的侧面积公式,还可以统一写成 $S_\text{侧}=c_0l$,其中 $c_0$ 是中截面周长,$l$ 分别是侧棱、斜高或母线长。球面、球冠、球带的面积,可以统一写成 $S=2\uppi Rh$,其中 $R$ 是球的半径,$h$ 是高(或直径)。
  \item 几种多面体和旋转体的体积公式是在两个体积公理的基础上推导出来的。在这一章里,我们是把柱体、锥体、台体当作不同的几何体定义的。如果把柱体、锥体当作台体的特殊形式,那么它们,甚至包括球体的体积公式,都可以统一写成 $V=\dfrac16h(S+4S_0+S')$,其中 $S$、$S'$ 是上、下底面积,$S_0$ 是中截面面积,$h$ 是高。
  \item 本章公式较多,为了便于记忆和应用,把它们列成公式表,放在本书的附录中。
\end{enumerate}
\chapter*{复习参考题\chinese{chapter}}
\section*{A 组}
\begin{question}
  \item 求证:正 $n$ 棱柱每相邻两个侧面所成的二面角等于 
  \[\dfrac{(n-2)\times\ang{180}}{n}.\]
  \item 经过正四棱柱 $AC'$ 的底面的一条对角线 $AC$ 引一个平面,平行于对角线 $BD'$,交棱 $DD'$ 于 $P$。如果这个正四棱柱底面的边长为 $a$,对角线 $BD'$ 与底面所成的角是 $\theta$,求截面 $ACP$ 的面积。
  \item 棱锥的底面是正方形,有相邻的两个侧面垂直于底面,另外两个侧面与底面成 \ang{45} 角,最长的侧棱长为 \qty{15}{cm}。求这个棱锥的高。
  \item 一个正四棱台的斜高是 \qty{12}{cm},侧棱的长是 \qty{13}{cm},侧面积是 \qty{720}{cm^2}。求它的上、下底面的边长。
  \item 一个正四棱台,它的下底面边长是 \qty{8}{cm},斜高是 \qty{6}{cm},侧面和底面成 \ang{60} 的二面角。画出它的直观图。
  \item 已知:圆柱侧面的展开图是一个正方形。求证:这个圆柱的侧面积等于两底面积和的 $2\uppi$ 倍。
  \item 圆锥的母线长为 $l$,它和底面所成的角为 $\theta$。求这个圆锥的内接正方体的棱长。
  \item 圆台的母线长是 $l$,母线和下底面所成的角是 $\theta$。轴截面的对角线垂直于母线。求证:这个圆台的侧面积是 $\uppi l^2\sin\theta\tan\theta$。
  \item 一个球冠形凹面镜,底的直径是 \qty{180}{mm},高是 \qty{12}{mm}。求截得球冠的球的半径和凹面镜的面积。
  \item 在北纬 \ang{60} 圈上有甲、乙两地,它们的纬度圈上的弧长等于 $\dfrac\uppi2R$($R$ 是地球半径)。求这两地间的球面距离。
  \item 一个长方体 $AC'$ 的对角线 $A'C$ 长为 $l$,这条对角线与一个面 $AC$ 所成的角为 \ang{30},与另一个面 $AD'$ 所成的角为 \ang{45}。求这个长方体的体积。
  \item 在一个平行六面体中,一个顶点上的三条棱长分别是 $a$、$b$、$c$,这三条棱中每两条所成的角是 \ang{60}。求平行六面体的体积。
  \item 一个圆台的母线长为 \qty{5}{cm},两底面半径的比为 $2:5$,侧面展开图的圆心角为 \ang{216}。求这个圆台的侧面积与体积。
  \item 一个直角三角形的两条直角边为 \qty{15}{cm} 和 \qty{20}{cm},以斜边为轴旋转,求这个旋转体的体积。
  \item 一个正方体所有的顶点都在球面上。如果这个球的体积是 $V$,求正方体的棱长。
  \item 一个球的半径为 \qty{7}{cm},用两个平行平面截去两个高为 \qty{3}{cm} 的球缺。求剩余部分(球台)的体积。
\end{question}
\section*{B 组}
\begin{question}[resume]
  \item 斜三棱柱的一个侧面的面积等于 $S$,这个侧面与它所对的棱的距离等于 $a$。求证:这个楞住的体积等于 $\dfrac12Sa$。
  \item\label{exec:2t-18} 图中长方体 $AC'$ 表示一个封闭的水箱。已知:$BB'=\qty{50}{cm}$,$AB=\qty{70}{cm}$,$BC=\qty{40}{cm}$。因为使用过久,在 $BB'$、$CC'$ 和 $AB$ 棱上各有一个小孔 $P$、$Q$、$R$,已量得 $BR=\qty{30}{cm}$,$BP=\qty{20}{cm}$,$CQ'=\qty{10}{cm}$。如果水箱可以任意放置,那么最多能盛多少水?
  \begin{figurehere}
    \begin{minipage}[b]{0.45\linewidth}\centering
      \includegraphics{2t-18.pdf}
      \caption*{(第~\ref{exec:2t-18}~题图)}
    \end{minipage}
    \begin{minipage}[b]{0.45\linewidth}\centering
      \includegraphics{2t-20.pdf}
      \caption*{(第~\ref{exec:2t-20}~题图)}
    \end{minipage}
  \end{figurehere}
  \item 一个棱锥的体积是 $V$,把棱锥的高三等分,过两个分点的平行于底面的截面将这个棱锥分成三部分。求中间一部分的体积。 
  \item\label{exec:2t-20}有一个圆锥如图。它的底面半径为 $r$,母线长为 $l$,且 $l>2r$,在母线 $SA$ 上有一点 $B$,$AB=a$。求由 $A$ 绕圆锥一周到 $B$ 的最短距离是多少?
  \item 如果真四棱锥的侧面是正三角形,求证:它的相邻两个侧面所成的二面角,是侧面和底面所成的二面角的二倍。
  \item 一个外径是 \qty{12}{cm},壁厚为 \qty{0.2}{cm} 的钢球,能否浮在水面上(钢的比重是 \qty{7.8}{g/cm^3})?
  \item 边长为 $a$ 的正六边形,以它的一边为轴旋转,求旋转体的全面积和体积。
\end{question}