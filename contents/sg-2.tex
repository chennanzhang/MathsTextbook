\chapter{多面体和旋转体}
\section{多面体}
\subsection{棱柱}
\subsubsection{棱柱的概念和性质}
我们常见的一些物体,例如三棱镜、方砖以及螺杆的头部,它们都呈棱柱的形状(\cref{fig:2-1})。
\begin{figure}
  % \includegraphics{2-1.pdf}
  \caption{}\label{fig:2-1}
\end{figure}

有两个面互相平行,其余各面都是四边形\footnote{本章所说的多边形,一般包括它内部的平面部分。},并且每相邻两个四边形的公共边都互相平行,由这些面所围成的几何体叫做\Concept{棱柱}(\cref{fig:2-2}),两个互相平行的面叫做\Concept{棱柱的底面},其余各面叫做\Concept{棱柱的侧面}。

\begin{Practice}
  \begin{question}
    \item 求证:{\em 直棱柱的侧棱长与高相等,侧面及经过不相邻的两条侧棱的截面都是矩形。}
    \item 有一个侧面是矩形的棱柱是不是直棱柱?有两个相邻侧面是矩形的棱柱呢?为什么? 
    \item 斜棱柱、直棱柱和正棱柱的底面、侧面各有什么特点?
  \end{question}
\end{Practice}

\subsubsection{长方体}
现在研究四棱柱的特殊情形。
\begin{figure}
  \caption{}\label{fig:2-5}
\end{figure}

长方体的对角线有下面的性质:
\begin{Theorem}{定理}
  长方体一条对角线长的平方等于一个定点上三条棱的长的平方和。
\end{Theorem}
\begin{Practice}
  \begin{question}
    \item 
    \item 
    \item 
  \end{question}
\end{Practice}
\subsubsection{直棱柱直观图的画法}
\subsubsection{直棱柱的侧面积}
\begin{Practice}
  \begin{question}
    \item 
    \item 
  \end{question}
\end{Practice}
\begin{Exercise}
  \begin{question}
    \item 
    \item 
    \item 
    \item 
    \item 
    \item 
    \item 
    \item 
    \item 
    \item 
    \item 
    \item 
    \item 
  \end{question}
\end{Exercise}

\subsection{棱锥}
\subsubsection{棱锥的概念和性质}
\begin{Practice}
  \begin{question}
    \item 
    \item 
  \end{question}
\end{Practice}
\subsubsection{正棱锥的直观图的画法}
\subsubsection{正棱锥的侧面积}
\begin{Practice}
  \begin{question}
    \item 
    \item 
  \end{question}
\end{Practice}
\begin{Exercise}
  \begin{question}
    \item 
    \item 
    \item 
    \item 
    \item 
    \item 
    \item 
    \item 
    \item 
    \item 
    \item 
  \end{question}
\end{Exercise}

\subsection{棱台}
\subsubsection{棱台的概念和性质}
\begin{Practice}
  \begin{question}
    \item 
    \item 
  \end{question}
\end{Practice}
\subsubsection{正棱台的直观图的画法}
\subsubsection{正棱台的侧面积}
\begin{Practice}
  \begin{question}
    \item 
    \item 
  \end{question}
\end{Practice}
\subsubsection{多面体}
\begin{Practice}
  \begin{question}
    \item 
    \item 
  \end{question}
\end{Practice}

\begin{Exercise}
  \begin{question}
    \item 
    \item 
    \item 
    \item 
    \item 
    \item 
    \item 
    \item 
    \item 
    \item 
    \item 
    \item 
  \end{question}
\end{Exercise}

\section{旋转体}
\subsection{圆柱、圆锥、圆台}
\subsubsection{圆柱、圆锥、圆台的概念和性质}
\begin{Practice}
  \begin{question}
    \item 
    \item 
    \item 
  \end{question}
\end{Practice}
\subsubsection{圆柱、圆锥、圆台的直观图的画法}
\begin{Practice}
  画一个上底半径为 \qty{1.5}{cm},下底半径为 \qty{2.5}{cm},高为 \qty{4}{cm} 的圆台的直观图(比例尺取 $\frac{1}{2}$,不写画法)。
\end{Practice}
\subsubsection{圆柱、圆锥、圆台的侧面积}
\begin{Practice}
  \begin{question}
    \item 
    \item 
    \item 
  \end{question}
\end{Practice}

\begin{Exercise}
  \begin{question}
    \item 
    \item 
    \item 
    \item 
    \item 
    \item 
    \item 
    \item 
    \item 
    \item 
    \item 
    \item 
    \item 
    \item 
  \end{question}
\end{Exercise}

\subsection{球}
\subsubsection{球的概念和性质}
\subsubsection{球的直观图的画法}
\subsubsection{球的表面积}
\begin{Practice}
  \begin{question}
    \item 海面上,地球球心角 \ang{;1;} 所对的大圆弧长约为 1 海里,1 海里约是多少千米?
    \item 计算地球表面积是多少 \unit{km^2}。
  \end{question}
\end{Practice}
\subsection{球冠}
\subsubsection{球冠}
\begin{Practice}
  \begin{question}
    \item 
    \item 
  \end{question}
\end{Practice}
\subsubsection{旋转面和旋转体}
\begin{Practice}
  \begin{question}
    \item 举出一些旋转面和旋转体的实例。
    \item 圆柱和圆柱面、圆锥和圆锥面有何区别?
  \end{question}
\end{Practice}

\begin{Exercise}
  \begin{question}
    \item 
    \item 
    \item 
    \item 
    \item 
    \item 
    \item 
    \item 
    \item 
    \item 
    \item 
    \item 
    \item 
  \end{question}
\end{Exercise}

\section{多面体和旋转体的体积}
\subsection{体积的概念与公理}
\begin{Practice}
  \begin{question}
    \item 
    \item 
  \end{question}
\end{Practice}
\subsection{棱柱、圆柱的体积}
\begin{Practice}
  \begin{question}
    \item ;
    \item 。
  \end{question}
\end{Practice}
\begin{Exercise}
  \begin{question}
    \item 
    \item 
    \item 
    \item 
    \item 
    \item 
    \item 
    \item 
    \item 
    \item 
    \item 
    \item 
  \end{question}
\end{Exercise}
\subsection{棱锥、圆锥的体积}
\begin{Practice}
  \begin{question}
    \item ;
    \item 。
  \end{question}
\end{Practice}

\begin{Exercise}
  \begin{question}
    \item 
    \item 
    \item 
    \item 
    \item 
    \item 
    \item 
    \item 
    \item 
  \end{question}
\end{Exercise}

\subsection{棱台、圆台的体积}
\begin{Practice}
  已知上、下地面边长分别是 $a$、$b$,高是 $h$。求下列正棱台的体积:
  \begin{tasks}(2)
    \task 正四棱台;
    \task 正六棱台。
  \end{tasks}
\end{Practice}
\subsection{拟柱体及其体积}
\begin{Practice}
  已知拟柱体的下底面积为 \qty{20}{cm^2},上底面积为 \qty{6}{cm^2},中截面面积为 \qty{12}{cm^2},高为 \qty{15}{cm}。求这个拟柱体的体积。
\end{Practice}
\begin{Exercise}
  \begin{question}
    \item 
    \item 
    \item 
    \item 
    \item 
    \item 
    \item 
    \item 
    \item 
    \item 
    \item 
    \item 
    \item 
  \end{question}
\end{Exercise}

\subsection{球的体积}
\begin{Practice}
  \begin{question}
    \item 球面面积膨胀为原来的二倍,计算体积变为原来的几倍。
    \item 一个正方体的顶点都在球面上,它的棱长是 \qty{4}{cm}。求这个球的体积。
  \end{question}
\end{Practice}
\subsection{球缺的体积}
\begin{Practice}
  \begin{question}
    \item 
    \item 
  \end{question}
\end{Practice}
\begin{Exercise}
  \begin{question}
    \item 
    \item 
    \item 
    \item 
    \item 
    \item 
    \item 
    \item 
    \item 
    \item 
    \item 
  \end{question}
\end{Exercise}

\section*{小结}
\begin{enumerate}[C、,itemindent=4.5em]
  \item 
  \item 
  \item 
  \item 
  \item 
  \item 
\end{enumerate}
\chapter*{复习参考题\chinese{chapter}}
\section*{A 组}
\begin{question}
  \item 
  \item 
  \item 
  \item 
  \item 
  \item 
  \item 
  \item 
  \item 
  \item 
  \item 
  \item 
  \item 
  \item 
  \item 
  \item 
\end{question}
\section*{B 组}
\begin{question}
  \item 
  \item 
  \item 
  \item 
  \item 
  \item 
  \item 
\end{question}