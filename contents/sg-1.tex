\chapter{直线和平面}
\section{平面}
\subsection{平面}
常见的桌面、黑板面、平静的水面以及纸板等,都给我们以平面的形象。
几何里所说的平面就是从这样的一些物体抽象出来的。
但是,几何里的平面是无线延展的。

当我们从适当的角度和距离观察桌面或黑板面时,感到它们都很像平行四边形。
因此,在立体几何中,通常画平行四边形来表示平面(\cref{fig:1-1})。
当平面是水平放置的时候,通常把平行四边形的锐角画成 \ang{45},横边画成等于邻边的两倍。
当一个平面的一部分被另一个平面遮住时,应把被遮部分的线段画成虚线或不画(\cref{fig:1-2})。
这样看起来立体感强一些。
\begin{figure}
  \begin{minipage}[b]{0.33\linewidth}\centering
    \caption{}\label{fig:1-1}
  \end{minipage}
  \begin{minipage}[b]{0.6\linewidth}\centering
    \caption{}\label{fig:1-2}
  \end{minipage}
\end{figure}

平面通常用一个希腊字母 $\alpha$、$\beta$、$\gamma$ 等来表示,如平面 $\alpha$、平面 $\beta$、平面 $\gamma$ 等,也可以用表示平行四边形的两个相对顶点的字母来表示,如平面 $AC$(\cref{fig:1-1})。
\begin{Practice}
  \begin{question}
    \item 能不能说一个平面长 \qty{4}{m},宽 \qty{2}{m}?为什么?
    \item \label{prac:1-2}观察 中甲乙两个图形,用模型来说明它们的位置有什么不同。并用字母来表示各平面
    \begin{figurehere}
      \begin{minipage}{\linewidth}
        \caption*{(第 \ref{prac:1-2} 题)}
      \end{minipage}
    \end{figurehere}
  \end{question}
\end{Practice}

\subsection{平面的基本性质}
在生产与生活中,人们经过长期的观察与实践,总结出关于平面的三个基本性质。
我们把它们当作公理,作为进一步推理的基础。
\begin{Theorem}{公理 1}
  如果一条直线上的两点在一个平面内,那么这条直线上所有的点都在这个平面内(\cref{fig:1-3})。
\end{Theorem}
这时,我们说直线在平面内,或者说平面经过直线。
\begin{figure}
  \caption{}\label{fig:1-3}
\end{figure}

例如,把一根支持边缘上的任意两点放在平的桌面上,可以看到直尺边缘就落在桌面上。

点 $A$ 在直线 $a$ 上,记作 $A\in a$;点 $A$ 在直线 $a$ 外,记作 $A \not\in a$;
点 $A$ 在平面 $\alpha$ 内,记作 $A\in\alpha$;点 $A$ 在平面 $\alpha$ 外,记作 $A \not\in\alpha$;直线 $a$ 在 平面 $\alpha$ 内,记作 $a\subset \alpha$。

\begin{Theorem}{公理 2}
  如果两个平面有一个公共点,那么它们有且只有一条通过这个点的公共直线(\cref{fig:1-4})。
\end{Theorem}
\begin{figure}
  \caption{}\label{fig:1-4}
\end{figure}

例如,教室内相邻的墙角,在墙角处交于一个点,它们就交于过这个点的一条直线。


如果两个平面 $\alpha$ 和 $\beta$ 有一条公共直线 $a$,就说平面 $\alpha$ 和 $\beta$ 相交,交线是 $a$,记作 $\alpha \cap \beta =a$。

\begin{Theorem}{公理 3}
  经过不在同一直线上的三点,有且仅有一个平面(\cref{fig:1-5})。
\end{Theorem}
\begin{figure}
  \caption{}\label{fig:1-5}
\end{figure}
例如,一扇门用两个合页和一把锁就可以固定了。

过 $A$、$B$、$C$ 三点的平面又可记作“平面 $ABC$”。

根据上述公理,可以得出下面的推论:
\begin{Deduction}{推论 1}
  经过不在同一直线上的三点,有且仅有一个平面(\cref{fig:1-6a})。
\end{Deduction}
\begin{figure}
  \begin{minipage}{0.32\linewidth}\centering
    \subcaption{}\label{fig:1-6a}
  \end{minipage}
  \begin{minipage}{0.32\linewidth}\centering
    \subcaption{}\label{fig:1-6b}
  \end{minipage}
  \begin{minipage}{0.32\linewidth}\centering
    \subcaption{}\label{fig:1-6c}
  \end{minipage}
  \caption{}\label{fig:1-6}
\end{figure}

$A$ 是直线 $a$ 外的一点,在 $a$ 上任取两点 $B$、$C$,根据公理3,经过不共线的三点 $A$、$B$、$C$ 有一个平面 $\alpha$。因为 $B$、$C$ 都在平面 $\alpha$ 内,所以根据公理1,直线 $a$ 在平面 $\alpha$ 内。即平面 $\alpha$ 是经过直线 $a$ 和点 $A$ 的平面。

因为 $B$、$C$ 在直线 $a$ 上,所以经过直线 $a$ 和点 $A$ 的平面一定经过 $A$、$B$、$C$。又根据公理3,经过不共线的三点的平面只有一个,所以经过直线 $a$ 和点 $A$ 的平面只有一个。

类似地,可以得出下面两个推论:
\begin{Deduction}{推论2}
  经过两条相交直线,有且仅有一个平面(\cref{fig:1-6b})。
\end{Deduction}
\begin{Deduction}{推论3}
  经过两条平行直线,有且仅有一个平面(\cref{fig:1-6c})。
\end{Deduction}

“有且仅有一个平面”,我们也说“确定一个平面”

注意:在立体几何里,平面几何中的定义、公理、定理等,对于同一个平面内的图形仍然成立。

\begin{example}
  两两相交且不过同一个点的三条直线比在同一个平面内。

  已知:直线 $AB$、$BC$、$CA$ 两两相交,焦点分别为 $A$、$B$、$C$(\cref{fig:1-7})。求证:直线 $AB$、$BC$、$CA$ 共面\footnote{空间的几个点和几条直线,如果都在同一个平面内,可以简单地说它们“共面”,否则说它们“不共面”。}。
\end{example}

\begin{proof}
  $\because\quad$ 直线 $AB$ 和 $AC$ 相交于点 $A$,

  $\therefore\quad$ 直线 $AB$ 和 $AC$ 确定一个平面 $\alpha$(推论2)。

  $\because\quad B\in AB,\quad C\in AC$,

  $\therefore\quad B\in \alpha,\quad C\in \alpha$。

  $\therefore\quad BC\subset \alpha$(公理1)。

  因此,直线 $AB$、$BC$、$CA$ 都在平面 $\alpha$ 内,即它们共面。
\end{proof}

\begin{figure}
  \caption{}\label{fig:1-7}
\end{figure}

\begin{Practice}
  \begin{question}
    \item 填空
    \begin{tasks}
      \task \CJKunderline[hidden]{不在一条直线上}的三点确定一个平面;
      \task 
      \task 
    \end{tasks}
    \item 用符号表示下列语句:
    \begin{tasks}
      \task 点 $A$ 在 平面 $\alpha$ 内,但在平面 $\beta$ 外;
      \task 
      \task 
      \task 
    \end{tasks}
    \item 
  \end{question}
\end{Practice}
\subsection{水平放置的平面图形的直观图的画法}
\begin{Practice}
  \begin{question}
    \item 画出水平放置的正方形、正三角形的直观图。
    \item 图中所给出的 $x$ 轴、$y$ 轴经过正五边形中心,画这个正五边形的直观图。
  \end{question}
\end{Practice}
\begin{Exercise}
  \begin{question}
    \item 下面的说法正确吗?为什么?
    \begin{tasks}
      \task 线段 $AB$ 在平面 $\alpha$ 内,直线 $AB$ 不全在平面 $\alpha$ 内;
      \task 平面 $\alpha$ 和 $\beta$ 只有一个公共点。
    \end{tasks}
    \item 为什么有的自行车后轮旁只安装一只撑脚?
    \item 三角形、梯形是否一定是平面图形?为什么?
    \item 解答:
    \begin{tasks}
      \task 不共面的四点可以确定几个平面?
      \task 三条直线两两平行,但不共面,它们可以确定几个平面?
      \task 共点的三条直线可以确定几个平面?
    \end{tasks}
    \item 一条直线经过平面内的一点与平面外的一点,它和这个平面有几个公共点?为什么?
    \item 一条直线与两条平行直线都相交,证明:这三条直线在同一个平面内。 
    \item 过已知直线外一点与这条直线上的三点分别画三条直线,证明:这三条直线在同一个平面内。
    \item 四条线段顺次首尾连接,所得的图形一定是平面图形吗?为什么?
    \item 怎样用两根细绳来检查一张桌子的四条腿下端是否在同一个平面内?
    \item 画出图中水平放置的四边形 $OABC$ 的直观图。
    \item 画水平放置的等腰梯形和平行四边形的直观图。
  \end{question}
\end{Exercise}
\section{空间两条直线}
\subsection{两条直线的位置关系}
\begin{Practice}
  \begin{question}
    \item 在教室中找出几对异面直线的离子。
    \item 解答:
    \begin{tasks}
      \task 没有公共点的两条直线叫做平行直线,对吗?
      \task 分别在两个平面内的两条直线一定是异面直线吗?为什么?
    \end{tasks}
    \item 说出正方体中各对线段的位置关系
    \begin{tasks}(3)
      \task $AB$ 和 $CC_1$;
      \task $A_1C$ 和 $BD_1$;
      \task $A_1A$ 和 $CB_1$;
      \task $A_1C_1$ 和 $CB_1$;
      \task $A_1B_1$ 和 $DC$;
      \task $BD_1$ 和 $DC$。
    \end{tasks}
  \end{question}
\end{Practice}
\subsection{平行直线}
\begin{Practice}
  \begin{question}
    \item 把一张长方形的纸对折两次,打开后如图那样,说明为什么这些折痕是互相平行的。
    % \item 已知:如图,$AA'$、$BB'$、$CC'$ 不共面,且 $BB'\paralleleq AA'$
  \end{question}
\end{Practice}
\subsection{两条异面直线所成的角}
\begin{Practice}
  \begin{question}
    \item 
    \item 
    \item 
  \end{question}
\end{Practice}
\begin{Exercise}
  \begin{question}
    \item 什么叫平行直线?什么叫异面直线,说出它们的共同点和区别。
    \item 
    \item 
    \item 
    \item 
    \item 
    \item 
    \item 
    \item 
    \item 
    \item 
  \end{question}
\end{Exercise}

\section{空间直线和平面}
\subsection{直线和平面的位置关系}
\begin{Practice}
  \begin{question}
    \item 观察图中的吊桥,说出立柱和桥面、水面,铁轨和桥面、水面的位置关系。
    \item 举出直线和平面三种位置关系的实例。
  \end{question}
\end{Practice}
\subsection{直线和平面平行的判定与性质}
直线和平面平行,除可根据定义判定外,还有以下的判定定理:
\begin{Theorem}{直线和平面平行的判定定理}
  如果平面外一条直线和这个平面内的一条直线平行,那么这条直线和这个平面平行。
\end{Theorem}

\begin{Theorem}{直线和平面平行的性质定理}
  如果一条直线和一个平面平行,经过这条直线的平面和这个平面相交,那么这条直线就和交线平行。
\end{Theorem}
\begin{Practice}
  \begin{question}
    \item ;
    \item ;
    \item 。
  \end{question}
\end{Practice}
\begin{Exercise}
  \begin{question}
    \item 
    \item 
    \item 
    \item 
    \item 
    \item 
    \item 
    \item 
    \item 
    \item 
  \end{question}
\end{Exercise}
\subsection{直线和平面垂直的判定与性质}
\begin{Practice}
  \begin{question}
    \item ;
    \item ;
    \item ;
    \item 。
  \end{question}
\end{Practice}
\subsection{斜线在平面上的射影、直线和平面所成的角}
\begin{Practice}
  \begin{question}
    \item ;
    \item ;
    \item 。
  \end{question}
\end{Practice}
\subsection{三垂线定理}
\begin{Practice}
  \begin{question}
    \item ;
    \item 。
  \end{question}
\end{Practice}
\begin{Exercise}
  \begin{question}
    \item 
    \item 
    \item 
    \item 
    \item 
    \item 
    \item 
    \item 
    \item 
    \item 
  \end{question}
\end{Exercise}

\section{空间两个平面}
\subsection{两个平面的位置关系}
\begin{Practice}
  \begin{question}
    \item 举出两个平面平行和相交的一些实例。
    \item 画两个平行平面和分别在这两个平面内的两条平行直线,再画一个经过这两条平行直线的平面。
  \end{question}
\end{Practice}
\subsection{两个平面平行的判定和性质}
\begin{Practice}
  \begin{question}
    \item 
    \item 
    \item 
  \end{question}
\end{Practice}
\begin{Exercise}
  \begin{question}
    \item 
    \item 
    \item 
    \item 
    \item 
    \item 
    \item 
    \item 
    \item 
    \item 
  \end{question}
\end{Exercise}

\subsection{二面角}
\begin{Practice}
  \begin{question}
    \item 
    \item 
    \item 
    \item 
  \end{question}
\end{Practice}

\subsection{两个平面垂直的判定和性质}
\begin{Practice}
  \begin{question}
    \item 
    \item 
    \item 
  \end{question}
\end{Practice}
\begin{Exercise}
  \begin{question}
    \item 
    \item 
    \item 
    \item 
    \item 
    \item 
    \item 
    \item 
    \item 
    \item 
    \item 
    \item 
    \item 
  \end{question}
\end{Exercise}

\section*{小结}
\begin{enumerate}[C、,itemindent=4.5em]
  \item 本章的主要内容是有关空间的直线与直线、直线与平面以及平面与平面的位置关系和有关图形的画法。着重研究的是它们之间的平行与垂直关系。
  \item 本章的四个公理是这一章内容的基础。此外,平面几何里的定义、定理等,对于空间的任何平面内的平面图形仍然适用;但对于非平面图形,则需要经过证明才能应用。在解决立体几何的问题时,常把它转化为平面几何的问题来解决。
  \item 空间两条之间的位置关系由“平行”、“相交”、“异面”三种;空间一条直线和一个平面的位置关系有“直线在平面内”、“平行”、“相交”三种;两个平面的位置关系有“平行”、“相交”两种。
  \item 关于空间的直线与直线,直线与平面、平面与平面的平行与垂直关系的性质定理与判定定理是本章的中心问题。应用这些定理时,要弄清定理的题设和结论。判定定理的题设是结论成立的充分条件,性质定理的结论是题设成立的必要条件。学完全章后,判定上述的平行和垂直关系的途径就更为广泛。例如,也可以用“垂直于同一个平面的两条直线必平行”去判定两条直线平行;用“如果两个平面垂直,那么在一个平面内垂直于它们交线的直线垂直于另一个平面”去判定一条直线与一个平面垂直。
  \item 两条异面直线所成的角、直线与平面所成的角以及二面角都是通过平面几何中的角来定义的,因而,它们都可以看作是平面几何中角的概念在空间的拓广。
  
  两条异面直线所成的角和二面角的定义都以定理“两边分别平行且方向相同的两个角相等”为基础。而斜线和平面所成的角实际是用这条斜线和平面内的直线所成的角中最小的角来定义的。

  两条异面直线的距离、直线和平面间的距离以及两个平行平面间的距离,都分别是它们的两点的距离中最小。
\end{enumerate}
\chapter*{复习参考题\chinese{chapter}}
\section*{A 组}
\begin{question}
  \item 下面的说法正确吗?为什么?
  \begin{tasks}
    \task 两条直线确定一个平面;
    \task 如果两个平面有三个公共点,那么这两个平面重合。
  \end{tasks}
  \item 解答:
  \begin{tasks}
    \task 求证:两两相交且不共点的四条直线共面;
    \task 已知四个点不共面,证明它们中任何三点都不在同一条直线上,逆命题正确吗?
  \end{tasks}
  \item 用斜二测画法画出下列水平放置的图形的直观图。
  \item 解答:
  \begin{tasks}
    \task 已知 a 和 b 是异面直线,a 和 c 是异面直线,那么 b 和 c 也是异面直线吗?
    \task 在一个平面内,经过一条直线外一点有几条直线和这条直线垂直?在空间呢?
    \task 在一个平面内,经过一条直线外一点有几条直线和这条直线平行?在空间呢?
  \end{tasks}
  \item 
  \item 
  \item 
  \item 
  \item 
  \item 
  \item 
  \item 
  \item 
  \item 
\end{question}
\section*{B 组}
\begin{question}[resume]
  \item $a$、$b$ 是异面直线,平面 $\alpha$ 经过直线 $b$ 与直线 $a$ 平行,平面 $\beta$ 经过直线 $a$ 与平面 $\alpha$ 相交于直线 $c$。求证:
  \begin{tasks}
    \task 直线 $b$、$c$ 所夹的不大于直角的角就是异面直线 $a$、$b$ 所成的角;
    \task 如果 $\alpha \perp \beta$,$b\cap c=A$,在平面 $\beta$ 内,作 $AB\perp c$ 交直线 $a$ 于点 $B$,那么线段 $AB$ 就是异面直线 $a$、$b$ 的公垂线,直线 $a$ 与平面 $\alpha$ 的距离就是异面直线 $a$、$b$ 的距离。
  \end{tasks}
  \item 两个不全等的三角形不在同一平面内,它们的边两两对应平行。证明:
  \begin{tasks}
    \task 三条对应顶点的连线交于一点;
    \task 这两个三角形相似。
  \end{tasks}
  \item 直线 $a$ 与 $b$ 不平行,如果 $\alpha\perp a$, $\beta\perp b$,那么平面 $\alpha$ 与 $\beta$ 必定相交,并且交线必垂直于直线 $a$、$b$。
  \item 解答:
  \begin{tasks}
    \task 由平面 $\alpha$ 外一点 $P$ 引平面的三条相等的斜线段,斜足分别为 $A$、$B$、$C$,$O$ 为 $\triangle ABC$ 的外心,求证:$OP\perp \alpha$。
    \task 平面 $ABC$ 外一点 $P$ 到 $\triangle ABC$ 三边的距离相等,$O$ 是 $\triangle ABC$ 内一点,且 $OP\perp \text{平面}ABC$。求证:$O$ 是 $\triangle ABC$ 的内心。
  \end{tasks}
  \item 夹在互相垂直的两个平面之间长为 $2a$ 的线段,和这两个平面所成的叫分别为 \ang{45}、\ang{30},过这条线段的两个端点分别在这两个平面内作交线的垂线,求两垂足的距离。
  \item 平面 $\alpha$ 过 $\triangle ABC$ 的重心 $G$。求证:在平面 $\alpha$ 同侧的两个顶点到平面 $\alpha$ 的距离的和,等于另一顶点到平面的距离。
  \item 已知:平面 $\alpha$ 和空间两点 $A$、$B$。在平面 $\alpha$ 内找一点 $C$,使 $AC+BC$ 最小。
\end{question}