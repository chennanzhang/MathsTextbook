\chapter{直线和平面}
\section{平面}
\subsection{平面}
\begin{Practice}
  \begin{question}
    \item 
    \item 
  \end{question}
\end{Practice}
\subsection{平面的基本性质}
\begin{Practice}
  \begin{question}
    \item 
    \item 
    \item 
  \end{question}
\end{Practice}
\subsection{水平放置的平面图形的直观图的画法}
\begin{Practice}
  \begin{question}
    \item 
    \item 
  \end{question}
\end{Practice}
\begin{Exercise}
  \begin{question}
    \item 
    \item 
    \item 
    \item 
    \item 
    \item 
    \item 
    \item 
    \item 
    \item 
    \item 
  \end{question}
\end{Exercise}
\section{空间两条直线}
\subsection{两条直线的位置关系}
\begin{Practice}
  \begin{question}
    \item 
    \item 
    \item 
  \end{question}
\end{Practice}
\subsection{平行直线}
\begin{Practice}
  \begin{question}
    \item 
    \item 
  \end{question}
\end{Practice}
\subsection{两条异面直线所成的角}
\begin{Practice}
  \begin{question}
    \item 
    \item 
    \item 
  \end{question}
\end{Practice}
\begin{Exercise}
  \begin{question}
    \item 
    \item 
    \item 
    \item 
    \item 
    \item 
    \item 
    \item 
    \item 
    \item 
    \item 
  \end{question}
\end{Exercise}

\section{空间直线和平面}
\subsection{直线和平面的位置关系}
\begin{Practice}
  \begin{question}
    \item 
    \item 举出直线和平面三种位置关系的实例。
  \end{question}
\end{Practice}
\subsection{直线和平面平行的判定与性质}
\begin{Practice}
  \begin{question}
    \item ;
    \item ;
    \item 。
  \end{question}
\end{Practice}
\begin{Exercise}
  \begin{question}
    \item 
    \item 
    \item 
    \item 
    \item 
    \item 
    \item 
    \item 
    \item 
    \item 
  \end{question}
\end{Exercise}
\subsection{直线和平面垂直的判定与性质}
\begin{Practice}
  \begin{question}
    \item ;
    \item ;
    \item ;
    \item 。
  \end{question}
\end{Practice}
\subsection{斜线在平面上的射影、直线和平面所成的角}
\begin{Practice}
  \begin{question}
    \item ;
    \item ;
    \item 。
  \end{question}
\end{Practice}
\subsection{三垂线定理}
\begin{Practice}
  \begin{question}
    \item ;
    \item 。
  \end{question}
\end{Practice}
\begin{Exercise}
  \begin{question}
    \item 
    \item 
    \item 
    \item 
    \item 
    \item 
    \item 
    \item 
    \item 
    \item 
  \end{question}
\end{Exercise}

\section{空间两个平面}
\subsection{两个平面的位置关系}
\begin{Practice}
  \begin{question}
    \item 举出两个平面平行和相交的一些实例。
    \item 画两个平行平面和分别在这两个平面内的两条平行直线,再画一个经过这两条平行直线的平面。
  \end{question}
\end{Practice}
\subsection{两个平面平行的判定和性质}
\begin{Practice}
  \begin{question}
    \item 
    \item 
    \item 
  \end{question}
\end{Practice}
\begin{Exercise}
  \begin{question}
    \item 
    \item 
    \item 
    \item 
    \item 
    \item 
    \item 
    \item 
    \item 
    \item 
  \end{question}
\end{Exercise}

\subsection{二面角}
\begin{Practice}
  \begin{question}
    \item 
    \item 
    \item 
    \item 
  \end{question}
\end{Practice}

\subsection{两个平面垂直的判定和性质}
\begin{Practice}
  \begin{question}
    \item 
    \item 
    \item 
  \end{question}
\end{Practice}
\begin{Exercise}
  \begin{question}
    \item 
    \item 
    \item 
    \item 
    \item 
    \item 
    \item 
    \item 
    \item 
    \item 
    \item 
    \item 
    \item 
  \end{question}
\end{Exercise}

\section*{小结}
\begin{enumerate}[C、,itemindent=4.5em]
  \item 
  \item 
  \item 
  \item 
  \item 
\end{enumerate}
\chapter*{复习参考题\chinese{chapter}}
\section*{A 组}
\begin{question}
  \item 
  \item 
  \item 
  \item 
  \item 
  \item 
  \item 
  \item 
  \item 
  \item 
  \item 
  \item 
  \item 
  \item 
\end{question}
\section*{B 组}
\begin{question}
  \item 
  \item 
  \item 
  \item 
  \item 
  \item 
  \item 
\end{question}